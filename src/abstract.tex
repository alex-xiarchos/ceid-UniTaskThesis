\begin{abstract}
    Η διαχείριση εργασιών αποτελεί σημαντική πρόκληση για τους σύγχρονους φοιτητές, οι οποίοι καλούνται να ισορροπήσουν μεταξύ ακαδημαϊκών, κοινωνικών και επαγγελματικών υποχρεώσεων. Η έλλειψη αποτελεσματικής οργάνωσης μπορεί να οδηγήσει σε άγχος, αναβλητικότητα και μειωμένη απόδοση, επηρεάζοντας τη συνολική φοιτητική εμπειρία. Ωστόσο, πολλοί φοιτητές είτε δεν χρησιμοποιούν οργανωμένες μεθόδους είτε βρίσκουν τις υπάρχουσες αναποτελεσματικές.

    Ο χαμηλός κώδικας (low-code) αποτελεί μια σύγχρονη προσέγγιση ανάπτυξης λογισμικού που επιτρέπει την ταχύτερη υλοποίηση εφαρμογών μέσω γραφικών διεπαφών και προκαθορισμένων λειτουργικών στοιχείων. Οι πλατφόρμες χαμηλού κώδικα μειώνουν την ανάγκη για εκτενή προγραμματισμό, επιτρέποντας στους προγραμματιστές να εστιάσουν περισσότερο στη σχεδίαση και βελτιστοποίηση των λειτουργιών της εφαρμογής.

    Η παρούσα διπλωματική εργασία εστιάζει στην ανάπτυξη μιας εφαρμογής διαχείρισης εργασιών για φοιτητές, αξιοποιώντας την ευελιξία και την ταχύτητα υλοποίησης των πλατφορμών χαμηλού κώδικα. Η προτεινόμενη εφαρμογή στοχεύει στη βελτίωση της παραγωγικότητας των φοιτητών, παρέχοντας ένα εύχρηστο και αποτελεσματικό εργαλείο για την καλύτερη οργάνωση των υποχρεώσεών τους. Η αποδοτικότητα και η χρηστικότητά της θα αξιολογηθούν μέσω δοκιμών με πραγματικούς χρήστες, ώστε να διαπιστωθεί η πραγματική της αξία ως εργαλείο υποστήριξης της ακαδημαϊκής διαχείρισης.

%   \begin{keywords}
%   \end{keywords}
\end{abstract}


\begin{abstracteng}
    Task management is a significant challenge for modern students, who must balance academic, social, and professional obligations. The lack of effective organization can lead to stress, procrastination, and decreased performance, ultimately affecting the overall student experience. However, many students either do not use structured methods or find existing solutions ineffective.

    Low-code development is a modern software approach that enables faster application implementation through graphical interfaces and pre-configured functional components. Low-code platforms reduce the need for extensive coding, allowing developers to focus more on designing and optimizing application functionalities.

    This thesis focuses on developing a task management application for students, leveraging the flexibility and rapid implementation capabilities of low-code platforms. The proposed application aims to enhance student productivity by providing a user-friendly and efficient tool for better task organization. Its effectiveness and usability will be evaluated through real-user testing to assess its actual value as a student task management support tool.

%   \begin{keywordseng}
%   \end{keywordseng}
\end{abstracteng}