\documentclass[11pt,a4paper,english,greek,oneside]{template}
\usepackage[main=greek,english]{babel}
\usepackage[utf8]{inputenc}
\usepackage{epstopdf}
\usepackage{indentfirst}
\usepackage{verbatim}
\usepackage{amssymb}
\usepackage{latexsym}
\usepackage{hyphenat}
\usepackage{makeidx}
\usepackage{algpseudocode}
\usepackage{algorithm}
\usepackage[hyphens]{url}
%\usepackage[hyphenbreaks]{breakurl}
\usepackage{enumitem}
\usepackage{xspace}
\usepackage{booktabs}
\usepackage{multirow}
\usepackage{subfig}
\usepackage{tabularx}
\usepackage{listings}
\usepackage{xcolor}
\usepackage{bbding}
\usepackage{footmisc}
\usepackage{fontspec}
\usepackage{hyperref}
\usepackage{csquotes}
\usepackage{dirtytalk} % Say
\captionsetup{belowskip=10pt,aboveskip=15pt}
\addto\captionsgreek{%
  \renewcommand{\indexname}{Ευρετήριο όρων}%
}
\makeindex

\graphicspath{ {./img/} }

% Γραμματοσειρές
\setmainfont{GFS Didot}
\setmonofont[Scale=0.9]{Courier New}

% 1.5 spacing
\renewcommand{\baselinestretch}{1.2}

\lstdefinestyle{mystyle}{
    backgroundcolor=\color{gray!10},
    keywordstyle=\bf\ttfamily,
    numberstyle=\tiny\color{darkgray},
    basicstyle=\ttfamily\footnotesize,
    breakatwhitespace=false,
    breaklines=true,
    captionpos=b,
    showstringspaces=false,
    keepspaces=true,
    numbers=left,
    numbersep=5pt,
}
\lstset{style=mystyle}

\makeatother

\newtheorem{proposition}{Πρόταση}
\newtheorem{theorem}{Θεώρημα}
\newtheorem{corollary}{Συμπέρασμα}
\newtheorem{lemma}{Λήμμα}
\newtheorem{example}{Παράδειγμα}
\newtheorem{remark}{Σημείωση}
\newtheorem{notation}{Συμβολισμός}
\newtheorem{law}{Νόμος}
%\renewcommand{\thedefinition}{\arabic{chapter}.\arabic{definition}}
%\renewcommand{\theproposition}{\arabic{chapter}.\arabic{proposition}}
%\renewcommand{\thetheorem}{\arabic{chapter}.\arabic{theorem}}
%\renewcommand{\thecorollary}{\arabic{chapter}.\arabic{corollary}}
%\renewcommand{\thelemma}{\arabic{chapter}.\arabic{lemma}}
%\renewcommand{\theexample}{\arabic{chapter}.\arabic{example}}
%\newcommand{\set}[1]{\left\{#1\right\}}
%\newcommand{\To}{\Longrightarrow}

\renewcommand\lstlistingname{\tg{Κώδικας}}
\renewcommand\lstlistlistingname{\tg{Παραδείγματα Κώδικα}}
\renewcommand{\listalgorithmname}{Λίστα Αλγορίθμων}

% Εισαγωγή βιβλιογραφίας
\usepackage[backend=biber]{biblatex}
\addbibresource{References.bib}
%\bibliographystyle{ieeetr}

% Αρίθμηση subsubsections
\setcounter{secnumdepth}{3}

\hyphenation{ο-ποί-α}


%%%%%%%%%%%%%%%%%%%%%%%%%%%%%%%%%%%%%%%%%%%%%%%%%%%%%
%% THESIS INFO 
%%
%
% Τίτλος Πτυχιακής Εργασίας
	\title{Ανάπτυξη εφαρμογής σε \en{Low Code} περιβάλλον: \mbox{Σχεδιασμός Εφαρμογής για τον Προγραμματισμό} \\
	και την Παρακολούθηση Εργασιών}
% "του" ή "της", ανάλογα με το φύλο του σπουδαστή
	\edef\toutis{του}
% Ονοματεπώνυμο σπουδαστή (ΚΕΦΑΛΑΙΑ, γενική πτώση)
	\edef\authorNameCapital{ΑΛΕΞΑΝΔΡΟΥ ΞΙΑΡΧΟΥ}
% Ονοματεπώνυμο σπουδαστή (πεζά, ονομαστική πτώση)
	\author{Αλέξανδρος Ξιάρχος}
% Ονοματεπώνυμο Επιβλέποντα Καθηγητή
	\supervisor{Μιχάλης Ξένος}
% Τίτλος Επιβλέποντα Καθηγητή (πχ καθηγητής, λέκτορας κτλ)
	\edef\supervisorTitle{Καθηγητής}
%% "Επιβλέπων" ή "Επιβλέπουσα", ανάλογα με το φύλο του Επιβλέποντα Καθηγητή
	\edef\supervisorMaleFemale{Επιβλέπων}
% Ονοματεπώνυμο Συνεπιβλέποντα Καθηγητή
	\covisor{Ονοματεπώνυμο}
% Τίτλος Επιβλέποντα Καθηγητή
	\edef\covisorTitle{Τίτλος}
% "Συνεπιβλέπων" ή "Συνεπιβλέπουσα", ανάλογα με το φύλο του Καθηγητή
	\edef\covisorMaleFemale{Συνεπιβλέπων}
% Ονοματεπώνυμο Συνεπιβλέποντα Καθηγητή
	\covisorS{Ονοματεπώνυμο}
% Τίτλος Συνεπιβλέποντα Καθηγητή
	\edef\covisorSTitle{Τίτλος}
% "Συνεπιβλέπων" ή "Συνεπιβλέπουσα", ανάλογα με το φύλο του Καθηγητή
	\edef\covisorSMaleFemale{Συνεπιβλέπων}
% Τόπος, μήνας και έτος
	\edef\thesisPlaceDate{Πάτρα, Οκτώβριος 2024}
% Ημερομηνία Εξέτασης
	\edef\examinationDate{Ημερομηνία Εξέτασης}
% Έτος Copyright
	\edef\copyrightYear{2024}
% Ονοματεπώνυμο 1ου εξεταστή
	\epitropiF{Ονοματεπώνυμο!}
% Τίτλος 1ου εξεταστή
	\edef\epitropiFTitle{Τίτλος}
