\chapter{Συμπεράσματα και μελλοντικές προεκτάσεις}
    Έχοντας αναφερθεί σε έννοιες που αφορούν τη διαχείριση εργασιών, τον χαμηλό κώδικα, τις πλατφόρμες χαμηλού κώδικα όπως το Mendix, και μετά από την πραγματοποίηση έρευνας προτίμησης και της πειραματικής διαδικασίας, φτάνουμε στον επίλογο της συγκεκριμένης διπλωματικής εργασίας, όπου πλέον θα αναλυθούν τα συμπεράσματα που βγάλαμε από ολόκληρη την υλοποίηση της διπλωματικής όπως επίσης και μελλοντικές προεκτάσεις της.

    \section{Σύνοψη και συμπεράσματα}
        Ο στόχος αυτής της διπλωματικής εργασίας είναι η ανάπτυξη μιας εφαρμογής σε Low-Code περιβάλλον για τον προγραμματισμό και την παρακολούθηση εργασιών. Για να επιτευχθεί αυτό, ήταν αναγκαίο να αποσαφηνίσουμε τι είναι η διαχείριση εργασιών. Πρόκειται για μια διαδικασία που, ιστορικά, αποτελεί ένα ζωτικό στοιχείο της κοινωνίας μας και του πολιτισμού μας. Από τα αρχαία χρόνια έως σήμερα, οι άνθρωποι χρησιμοποιούσαν διάφορα μέσα για να οργανώνουν τις εργασίες τους, από ημερολόγια και ρολόγια μέχρι σύγχρονες ψηφιακές εφαρμογές. Μάλιστα, πέρα από την οργάνωση, λόγω της βαρύτητας που έχει η διαχείριση εργασιών στην καθημερινότητά των ανθρώπων υπήρξε η ανάγκη για την περαιτέρω βελτιστοποίησή της, με μεθοδολογίες όπως η Kanban και τα διαγράμματα Gantt και PERT. Αυτές οι μεθοδολογίες επιτρέπουν την αποτελεσματική οργάνωση των εργασιών, την εκτίμηση του χρόνου και των πόρων που απαιτούνται για την ολοκλήρωσή τους, και την παρακολούθηση της προόδου τους.

        Όσον αφορά το θεματικό πλαίσιο της εφαρμογής, επιλέχθηκε το ακαδημαϊκό περιβάλλον καθώς πρόκειται για ένα περιβάλλον αγχώδες και απαιτητικό, με πολλαπλές υποχρεώσεις και ανάγκη για ιεράρχηση χρόνου, άρα με κατεξοχήν ανάγκη για συστηματική οργάνωση και διότι έτσι ήταν ευκολότερη η διεξαγωγή των ερευνητικών και πειραματικών διαδικασιών.

        Η ίδια η εφαρμογή αναπτύχθηκε στην πλατφόρμα χαμηλού κώδικα, Mendix. Ο χαμηλός κώδικας θεωρείται το μέλλον της ανάπτυξης λογισμικού, καθώς ενοποιεί τους δύο πιο σημαντικούς πυλώνες της: την ανάγκη αυτοματοποίησης επαναλαμβανόμενων ενεργειών (όπως τα CASE περιβάλλοντα και η MDA αρχιτεκτονική), και την αύξηση του επιπέδου αφαίρεσης των γλωσσών προγραμματισμού διευκολύνοντας τη χρήση τους από μη εξειδικευμένους χρήστες. Οι πλατφόρμες ανάπτυξης λογισμικού σε χαμηλό κώδικα επιτρέπουν την ανάπτυξη εφαρμογών με τη χρήση γραφικού περιβάλλοντος και drag-and-drop στοιχείων. Παρέχουν γρήγορη μοντελοποίηση των δεδομένων, διαγράμματα ροής για τη λογική, χρήση προκατασκευασμένων στοιχείων και άλλα.

        Για τη δημιουργία της εφαρμογής χρησιμοποιήθηκαν υπάρχουσες βιβλιογραφικές έρευνες όπου εξερευνούσαν τα προβλήματα διαχείρισης των φοιτητών και τα χαρακτηριστικά που αυτοί θα επιθυμούσαν σε μια εφαρμογή, αλλά επιπλέον διεξήχθη και μια πρωτογενής έρευνα προτίμησης μεταξύ 14 φοιτητών. Η έρευνα έδειξε πως οι φοιτητές αντιμετωπίζουν προβλήματα στη διαχείριση των υποχρεώσεών τους με προβλήματα στην τήρηση προθεσμιών, ενώ θεωρούν ανεπαρκή τον τρόπο οργάνωσης που ακολουθούν.

        Η ανάπτυξη της εφαρμογής βασίστηκε στα αποτελέσματα της έρευνας, με έμφαση στα χαρακτηριστικά που οι φοιτητές θεώρησαν πιο χρήσιμα. Στόχος ήταν η δημιουργία μιας εφαρμογής που θα είναι αποτελεσματική, εύχρηστη και φιλική προς τον χρήστη και μπορεί όντως να αποτελέσει μια επιλογή διαχείρισης εργασιών και πέρα από τα πλαίσια αυτής της διπλωματικής εργασίας. Η εφαρμογή περιλαμβάνει τη δυνατότητα δημιουργίας εργασιών, την οργάνωσή τους σε διάφορες κατηγορίες και προτεραιότητες, την παρακολούθηση της προόδου τους, τη χρήση μηνυμάτων και ειδοποιήσεων για την ενημέρωση του χρήστη μέχρι τη λήξη τους, ενώ περιλαμβάνονται διαφορετικές σελίδες και προβολές (όπως πίνακας Kanban και ημερολόγιο) για την ευκολότερη παρακολούθηση των εργασιών.

        Τέλος, το έργο της διπλωματικής αξιολογήθηκε μέσω πειράματος συμμετοχής χρηστών σε ένα SUS ημερολόγιο όπου 23 συμμετέχοντες διαφορετικών ακαδημαϊκών επιπέδων και εξειδικεύσεων εκτέλεσαν ένα τυπικό σενάριο χρήσης και αξιολόγησαν την εφαρμογή με βάση την ευχρηστία της. Τα αποτελέσματα ανέδειξαν την ευχρηστία της εφαρμογής, με χαμηλές τυπικές αποκλίσεις και με μέσο όρο 83.59\% στις μέσες τιμές. Αυτό αποδεικνύει ότι η εφαρμογή υλοποιήθηκε με επιτυχία, είναι αποτελεσματική και μπορεί να αποτελέσει μια επιλογή για τη διαχείριση εργασιών σε ακαδημαϊκό περιβάλλον.


    \section{Μελλοντικές προεκτάσεις}
        Βάσει της ανάγκης για μια πιο ολοκληρωμένη λύση, η εφαρμογή μπορεί να επεκταθεί με νέες λειτουργίες και βελτιώσεις στις υπάρχουσες:

        Αρχικά, η εφαρμογή μπορεί να ενσωματώσει τη δυνατότητα συγχρονισμού με εξωτερικές εφαρμογές και υπηρεσίες όπως ημερολόγια (Google Calendar) ή άλλες task-management εφαρμογές (Trello, Notion) με τη χρήση REST APIs. Η αξιοποίηση των Java actions και των microflows στο Mendix μπορεί να χρησιμοποιηθεί για την υλοποίηση διαδικασιών εξουσιοδότησης μέσω OAuth2, εξασφαλίζοντας την ασφαλή ανταλλαγή δεδομένων μεταξύ της εφαρμογής και των εξωτερικών συστημάτων. Έτσι, ο χρήστης θα μπορεί να διαχειρίζεται τις εργασίες του από μία κεντρική εφαρμογή, ενώ θα μπορεί να ενημερώνεται για τις εργασίες του από διάφορες πηγές.

        Επίσης, μπορούν να προστεθούν collaboration εργαλεία που θα επιτρέπουν τη δυνατότητα συνεργασίας μεταξύ χρηστών/φοιτητών μέσω κοινών εργασιών, σχολίων ή ανταλλαγής μηνυμάτων. Πατώντας πάνω σε αυτή την ιδέα μπορεί η εφαρμογή να εξελιχθεί σε μία πλατφόρμα κοινωνικής δικτύωσης περιλαμβάνοντας προσωπικά προφίλ, δημόσιες και ιδιωτικές συζητήσεις της πανεπιστημιακής κοινότητας όπου οι φοιτητές θα μπορούν να ανταλλάσσουν πληροφορίες και μηνύματα σε πραγματικό χρόνο και να συνεργάζονται σε κοινά projects.

        Επιπλέον, για την ενοποίηση της εφαρμογής με το πανεπιστημιακό περιβάλλον, μπορεί να προστεθεί η δυνατότητα ενσωμάτωσης δεδομένων από το Ψηφιακό Άλμα / e-class, όπως τα μαθήματα, οι βαθμοί και οι ανακοινώσεις, κάτι που θα επιτρέψει την απεικόνιση ακριβών και ενημερωμένων ακαδημαϊκών πληροφοριών, δημιουργώντας έτσι έναν ολοκληρωμένο κόμβο πληροφοριών για τους φοιτητές.

        Επίσης, η αξιοποίηση αλγορίθμων μηχανικής μάθησης θα μπορούσε να επιτρέψει την αυτόματη δημιουργία εργασιών κάνοντας scrape τις ανακοινώσεις των μαθημάτων από το API του e-class. Κάτι τέτοιο μπορεί να επιτευχθεί απευθείας χρησιμοποιώντας Python και NLP τεχνικές και τεχνικές κατηγοριοποίησης δεδομένων χρησιμοποιώντας βιβλιοθήκες όπως Sklearn ή Tensorflow. Ο Python κώδικας μπορεί να συνδεθεί στο Mendix με modules στο Marketplace όπως το AWS Lambda Connector ή σε server ξεχωριστά (Flask, FastAPI κ.α.), δημιουργώντας endpoints σε REST APIs που καλεί το Mendix.

        Έτσι, η εφαρμογή θα μπορεί να αναλύει νέες ανακοινώσεις χρησιμοποιώντας προηγούμενα δεδομένα, να τις κατηγοριοποιεί και δημιουργεί αυτοματοποιημένα νέες επόμενες εργασίες για τους χρήστες, παρέχοντας εξατομικευμένες προτάσεις για τον προγραμματισμό των υποχρεώσεών τους. Με αυτό τον τρόπο, η εφαρμογή θα μπορούσε να λειτουργεί ως ένας προσωπικός βοηθός για τους φοιτητές, προτείνοντάς τους εργασίες που πρέπει να ολοκληρώσουν και προτείνοντας τους τρόπους για την εκτέλεσή τους, κάτι πολύ σημαντικό δεδομένου ότι οι φοιτητές συχνά αντιμετωπίζουν προβλήματα με την οργάνωση του χρόνου τους και την προθεσμία των εργασιών τους.

        Επιπλέον, το σύστημα επιβράβευσης μπορεί να εξελιχθεί περαιτέρω σε ένα gamification σύστημα, με τη δημιουργία διαφορετικών επιπέδων, badges, πόντων και επιτευγμάτων ως ανταμοιβή για την ολοκλήρωση εργασιών, με παρόμοια λογική όπως η εφαρμογή Duolingo. Οι φοιτητές θα μπορούν να ανταγωνίζονται μεταξύ τους για την απόκτηση badges και την αύξηση των πόντων τους, ενθαρρύνοντάς τους να ολοκληρώνουν τις εργασίες τους εγκαίρως και να βελτιώνουν την απόδοσή τους.

        Τέλος, η εφαρμογή είναι κατασκευασμένη χρησιμοποιώντας Web-based εργαλεία στο Mendix. Μπορεί ταυτόχρονα, μέσω του Mendix Native Mobile να δημιουργηθεί μια καθαρά mobile εφαρμογή, θα εξασφαλίσει την απρόσκοπτη πρόσβαση και χρήση των λειτουργιών της εφαρμογής από κινητές συσκευές, επιτρέποντας την υλοποίηση push notifications και offline λειτουργικότητας.

        Έτσι, η εφαρμογή θα μπορούσε να μετατραπεί σε ένα ενοποιημένο σύστημα πληροφόρησης και οργάνωσης της ακαδημαϊκής ζωής των φοιτητών, ώστε να μπορούν να αντιμετωπίσουν τις προκλήσεις της ακαδημαϊκής ζωής με περισσότερη αυτοπεποίθηση και αυτονομία, ενώ ταυτόχρονα θα μπορούν να απολαμβάνουν την εμπειρία της μάθησης με περισσότερη οργάνωση και αποτελεσματικότητα.
