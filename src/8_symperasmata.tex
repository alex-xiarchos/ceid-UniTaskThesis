\chapter{Συμπεράσματα - Προεκτάσεις}
    Στην παρούσα διπλωματική εργασία πραγματοποιήθηκε η ανάπτυξη μιας εύχρηστης εφαρμογής διαχείρισης εργασιών για φοιτητές, αξιοποιώντας την ευελιξία και την ταχύτητα υλοποίησης που προσφέρουν οι πλατφόρμες χαμηλού κώδικα. Η εφαρμογή σχεδιάστηκε ως μια ολοκληρωμένη λύση στο πρόβλημα της ανεπαρκούς οργάνωσης και διαχείρισης των ακαδημαϊκών υποχρεώσεων των φοιτητών, προσφέροντας ένα φιλικό και αποτελεσματικό εργαλείο που τους βοηθά να βελτιώσουν τη διαχείριση του χρόνου και των εργασιών τους.

    Η αξιολόγηση της εφαρμογής από πραγματικούς χρήστες κατέδειξε θετικά αποτελέσματα, επιβεβαιώνοντας την ευχρηστία και τη λειτουργικότητά της. Ωστόσο, όπως κάθε σύστημα, υπάρχουν περιθώρια βελτίωσης και περαιτέρω εξέλιξης, προκειμένου η εφαρμογή να ανταποκριθεί ακόμα καλύτερα στις μεταβαλλόμενες ανάγκες των χρηστών της.

    Βάσει της ανάγκης για μια ακόμα πιο ολοκληρωμένη λύση, η εφαρμογή μπορεί να επεκταθεί με νέες λειτουργίες και τη βελτίωση των υπαρχουσών. Ενδεικτικές προτάσεις για μελλοντικές επεκτάσεις περιλαμβάνουν τη δυνατότητα συγχρονισμού με εργαλεία όπως το Google Calendar ή άλλα APIs, τη δυνατότητα συνεργασίας μεταξύ χρηστών μέσω κοινών πλάνων εργασίας, καθώς και την ενσωμάτωση δεδομένων από το Ψηφιακό Άλμα / e-class, όπως τα μαθήματα, οι βαθμοί και οι ανακοινώσεις. Επιπλέον, η αξιοποίηση αλγορίθμων μηχανικής μάθησης θα μπορούσε να επιτρέψει την αυτόματη δημιουργία εργασιών βάσει ανακοινώσεων ή προγραμματισμένων προθεσμιών, ενώ η ενσωμάτωση ενός συστήματος κοινωνικής δικτύωσης θα διευκόλυνε την ανταλλαγή πληροφοριών και την επίλυση αποριών μεταξύ των φοιτητών.
