\chapter{Εισαγωγή}
	\section{Σημασία του προβλήματος}
		Σε έναν κόσμο όπου η πολυπλοκότητα και η συνεχής ροή καθηκόντων αποτελούν καθημερινή πραγματικότητα, το σύγχρονο ακαδημαϊκό σύστημα δε θα μπορούσε να αποτελεί εξαίρεση. Ο σημερινός φοιτητής χαρακτηρίζεται από έντονο φόρτο εργασίας, όπου οι πολλαπλές υποχρεώσεις --ακαδημαϊκές, κοινωνικές και, συχνά, επαγγελματικές-- επιβάλλουν την αναγκαιότητα για αποτελεσματική διαχείριση χρόνου και εργασιών. Αυτή η πολυπλοκότητα μπορεί να οδηγήσει σε σύγχυση, υπερφόρτωση πληροφοριών και δυσκολία στον αποτελεσματικό προγραμματισμό, επηρεάζοντας άμεσα την απόδοσή του.

		Η έλλειψη συστηματικής οργάνωσης μπορεί να οδηγήσει σε αυξημένα επίπεδα άγχους και αναβλητικότητας, δημιουργώντας έναν φαύλο κύκλο όπου η καθυστέρηση εκτέλεσης των εργασιών επιδεινώνει το αίσθημα πίεσης. Ειδικά σε περιόδους εξεταστικής ή προθεσμιών, οι φοιτητές συχνά δυσκολεύονται να ιεραρχήσουν τις προτεραιότητές τους και να καταμερίσουν σωστά τον διαθέσιμο χρόνο τους. Ταυτόχρονα, παρατηρείται πως οι φοιτητές δε χρησιμοποιούν κάποια οργανωμένη μέθοδο διαχείρισης των υποχρεώσεών τους, ή αν χρησιμοποιούν, χρησιμοποιούν μεθόδους όπως σημειώσεις ή ημερολόγια τις οποίες θεωρούν αναποτελεσματικές.

		Παρόλα αυτά, η διαχείριση εργασιών δεν είναι κάτι καινούργιο· υπήρξε θεμελιώδης για την πρόοδο της ανθρωπότητας, από τα αρχαία χρόνια έως σήμερα, συμβάλλοντας στην οργάνωση της εργασίας, την παραγωγικότητα και την ανάπτυξη των κοινωνιών. Στη βάση αυτής της ανάγκης, καθίσταται σαφές πως οι φοιτητές, ως ομάδα με ιδιαίτερα υψηλές απαιτήσεις διαχείρισης χρόνου και εργασιών, θα ωφελούνταν από ένα εξειδικευμένο σύστημα σχεδιασμένο να τους υποστηρίζει.

		Ταυτόχρονα, οι μεταβαλλόμενες ανάγκες του περιβάλλοντος επηρεάζουν και τη μεθοδολογία ανάπτυξης λογισμικού. Η αυξανόμενη ζήτηση για γρήγορες, ευέλικτες και προσαρμοστικές λύσεις οδήγησε τους μηχανικούς λογισμικού να απομακρυνθούν από τα παραδοσιακά μοντέλα υψηλού κώδικα και να στραφούν σε νέες προσεγγίσεις, όπως ο χαμηλός κώδικας. Ο χαμηλός κώδικας αποτελεί απάντηση στην ανάγκη για γρηγορότερη παραγωγή ποιοτικών εφαρμογών, μάλιστα επιτρέποντας σε χρήστες με περιορισμένες ή μηδενικές γνώσεις προγραμματισμού να συμμετέχουν ενεργά στη διαδικασία ανάπτυξης.

		Ως εκ τούτου, η χρήση μιας πλατφόρμας χαμηλού κώδικα για την ανάπτυξη μιας εφαρμογής διαχείρισης εργασιών για φοιτητές αποτελεί μια ιδανική λύση, συνδυάζοντας την ανάγκη για οργάνωση με την ευελιξία και την ταχύτητα ανάπτυξης λογισμικού.


	\section{Στόχοι της εργασίας}
		Επομένως, η παρούσα διπλωματική εργασία έχει ως στόχο τη διερεύνηση των δυσκολιών που αντιμετωπίζουν οι φοιτητές στη διαχείριση των ακαδημαϊκών τους υποχρεώσεων, καθώς και την ανάπτυξη μιας εφαρμογής που θα προσφέρει μια ολοκληρωμένη και αποτελεσματική λύση σε αυτό το πρόβλημα. Για την επίτευξη αυτού του στόχου, θα εξεταστεί σε βάθος η εξέλιξη της διαχείρισης εργασιών μέσα στον χρόνο, καθώς και οι σύγχρονες ανάγκες των φοιτητών. Η ανάπτυξη της εφαρμογής θα πραγματοποιηθεί σε περιβάλλον χαμηλού κώδικα, επιτρέποντας την ταχύτερη και πιο αποδοτική υλοποίησή της, ενώ παράλληλα θα δοθεί ιδιαίτερη έμφαση στη φιλικότητα προς τον χρήστη και την ευχρηστία του συστήματος.

		Τέλος, η χρηστικότητα της εφαρμογής θα αξιολογηθούν μέσω πειραματικής διαδικασίας, η οποία θα περιλαμβάνει δοκιμές από πραγματικούς χρήστες. Η ανάλυση αυτή θα συμβάλει στην αξιολόγηση της συνολικής επίδρασης της εφαρμογής και στην τεκμηρίωση της αξίας της ως εργαλείου διαχείρισης εργασιών στο ακαδημαϊκό περιβάλλον.


	\section{Μεθοδολογία προσέγγισης}
		Για την επίτευξη των παραπάνω στόχων η εργασία χωρίζεται σε τέσσερις φάσεις:

		Η πρώτη φάση επικεντρώνεται στη βιβλιογραφική ανασκόπηση των εννοιών που σχετίζονται με τη διαχείριση εργασιών και την ανάπτυξη λογισμικού σε περιβάλλον χαμηλού κώδικα. Αναλύονται βασικές θεωρητικές αρχές, υφιστάμενες μεθοδολογίες και υπάρχουσες εφαρμογές στον χώρο της διαχείρισης εργασιών, με σκοπό την κατανόηση των βέλτιστων πρακτικών. Επιπλέον, εξετάζονται διεξοδικά τα πλεονεκτήματα του χαμηλού κώδικα και οι δυνατότητες που παρέχουν οι πλατφόρμες χαμηλού κώδικα.

		Στη δεύτερη φάση, πραγματοποιείται ερευνητική μελέτη για την καταγραφή των αναγκών και των προτιμήσεων των φοιτητών αναφορικά με τις εφαρμογές διαχείρισης εργασιών. Τα δεδομένα αυτά αξιοποιούνται για τη διαμόρφωση των βασικών λειτουργικών απαιτήσεων της εφαρμογής.

		Η τρίτη φάση περιλαμβάνει τη σχεδίαση και υλοποίηση της εφαρμογής στην πλατφόρμα Mendix, ακολουθώντας τις ανάγκες και τις προτιμήσεις των φοιτητών με στόχο να αποτελεί μια εύχρηστη λύση.

		Η τελευταία φάση αφορά την πειραματική αξιολόγηση της εφαρμογής όπου πραγματοποιείται δοκιμή με πραγματικούς χρήστες, οι οποίοι καλούνται να χρησιμοποιήσουν την εφαρμογή σε πραγματικές συνθήκες και να παρέχουν ανατροφοδότηση σχετικά με τη χρηστικότητά της.


	\section{Διάρθρωση της διπλωματικής εργασίας}
		Η εργασία είναι οργανωμένη σε οκτώ κεφάλαια:

		Στο Κεφάλαιο 2 γίνεται μια εκτενής αναφορά στη διαδικασία της διαχείρισης εργασιών, περιλαμβάνοντας τον ορισμό της, μια ιστορική αναδρομή στην εξέλιξή της από παραδοσιακές μεθόδους, όπως συσκευές με πολύχρωμες χάντρες ή τα ηλιακά ρολόγια, έως τα σύγχρονα εργαλεία. Παρουσιάζονται επίσης διάφορες μέθοδοι για την αποτελεσματική διαχείριση εργασιών όπως το διάγραμμα Γκαντ.

		Στο Κεφάλαιο 3 παρουσιάζονται τόσο υπάρχουσες έρευνες όσο και μια πρωτογενής μελέτη που διεξήχθη στο πλαίσιο της παρούσας εργασίας, με στόχο να αναλυθούν οι προκλήσεις που αντιμετωπίζουν οι φοιτητές στη διαχείριση των ακαδημαϊκών και εξωπανεπιστημιακών υποχρεώσεών τους. Τα ευρήματα αυτά παρέχουν πολύτιμα δεδομένα που αξιοποιούνται στον σχεδιασμό και την ανάπτυξη της προτεινόμενης εφαρμογής, με σκοπό τη βελτίωση της παραγωγικότητας και της συνολικής φοιτητικής εμπειρίας.

		Το Κεφάλαιο 4 εξετάζει την έννοια του χαμηλού κώδικα, αναλύοντας τον ορισμό, τις αρχές, τα χαρακτηριστικά και τα πλεονεκτήματά του. Επίσης, παρουσιάζεται η ιστορική εξέλιξη των εργαλείων και των μεθοδολογιών αυτοματοποίησης που οδήγησαν στη δημιουργία λογισμικών όπως οι Πλατφόρμες Ανάπτυξης Λογισμικού σε Low-Code.

		Το Κεφάλαιο 5 επικεντρώνεται σε μία από αυτές τις πλατφόρμες, το Mendix, που επιλέχθηκε για την ανάπτυξη της εφαρμογής. Παρουσιάζονται τα βασικά χαρακτηριστικά της πλατφόρμας, καθώς και τα εργαλεία και οι λειτουργίες που προσφέρει, ώστε να υπάρχει μια σαφής κατανόηση των δυνατοτήτων της για την υλοποίηση της εφαρμογής.

		Το Κεφάλαιο 6 παρουσιάζει την υλοποιημένη εφαρμογή. Αναλύεται η σχεδιαστική προσέγγιση που ακολουθήθηκε, βασισμένη στα ευρήματα των ερευνών, ενώ παράλληλα παρουσιάζεται ένα demo της εφαρμογής. Στη συνέχεια, γίνεται λεπτομερής ανάλυση όλων των τεχνικών χαρακτηριστικών και λειτουργιών της.

		Το Κεφάλαιο 7 περιγράφει τη διαδικασία των πειραματικών δοκιμών που διεξήχθησαν για την αξιολόγηση της εφαρμογής. Παρουσιάζονται τα αποτελέσματα των χρηστών, καθώς και η ανάλυση των δεδομένων που συλλέχθηκαν, προκειμένου να αξιολογηθεί η αποτελεσματικότητα και η χρηστικότητα του συστήματος.

		Τέλος, στο Κεφάλαιο 8 διατυπώνονται τα συμπεράσματα της εργασίας και παρουσιάζονται προτάσεις για μελλοντικές επεκτάσεις και βελτιώσεις της εφαρμογής, με στόχο τη διαρκή εξέλιξή της.
