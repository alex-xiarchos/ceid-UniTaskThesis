\chapter{Διαχείριση Εργασιών}
    Όπως αναφέρθηκε στην εισαγωγή, η παρούσα διπλωματική εργασία επικεντρώνεται στην ανάπτυξη μιας εφαρμογής για τον προγραμματισμό και την παρακολούθηση εργασιών. Οι διαδικασίες σχεδιασμού, υλοποίησης και παρακολούθησης αυτών των εργασιών εντάσσονται στο ευρύτερο πλαίσιο της \textbf{διαχείρισης εργασιών} (task management). Για τον λόγο αυτό, στο παρόν κεφάλαιο κρίνεται απαραίτητο να παρουσιαστεί μια πιο αναλυτική ερμηνεία του όρου, καθώς και οι θεμελιώδεις αρχές και μέθοδοι που σχετίζονται με τη συγκεκριμένη έννοια.

    Η διαχείριση εργασιών αποτελεί ουσιώδες στοιχείο της καθημερινότητας, τόσο σε προσωπικό όσο και σε επαγγελματικό επίπεδο. Με την αυξανόμενη πολυπλοκότητα των σύγχρονων υποχρεώσεων και την ανάγκη για αποτελεσματικό συντονισμό πολλαπλών δραστηριοτήτων, αναδύονται νέες προκλήσεις που καθιστούν τη διαδικασία αυτή ολοένα και πιο απαιτητική.

    Στο πλαίσιο του κεφαλαίου αυτού, θα εξεταστεί αναλυτικά η διαδικασία διαχείρισης εργασιών, η ιστορική της εξέλιξη, οι μεθοδολογίες που βελτιώνουν την αποδοτικότητά της και ο ρόλος που διαδραματίζει στην πανεπιστημιακή κοινότητα. Ο στόχος είναι να αναδειχθεί η σημασία της έννοιας αυτής για την καθημερινότητα, με ιδιαίτερη έμφαση στους φοιτητές, καθώς αποτελεί τη βάση της λογικής για την ανάπτυξη της εφαρμογής που θα παρουσιαστεί στη συνέχεια.

    \section{Το πρόβλημα της διαχείρισης εργασιών}
        Στην παρούσα ενότητα θα οριστεί η έννοια της διαχείρισης εργασιών, εστιάζοντας στις διαφορές της από τη διαχείριση έργου. Παράλληλα, θα αναλυθούν τα κύρια πεδία εφαρμογής της, προκειμένου να αποκτηθεί μια ολοκληρωμένη κατανόηση του όρου, ο οποίος αποτελεί το επίκεντρο της ανάλυσης στο παρόν κεφάλαιο.

        \subsection{Ορισμός της εργασίας}
            \textbf{Εργασία} (task, project) στη διαχείριση εργασιών ονομάζεται μια προσωρινή δραστηριότητα που αναλαμβάνεται για τη δημιουργία ενός μοναδικού αποτελέσματος (προϊόντος, υπηρεσίας, αναφοράς ή κάποιου άλλου παραδοτέου). Η εργασία πραγματοποιείται σε μια προκαθορισμένη χρονική περίοδο και τερματίζεται όταν έχει γίνει η επίτευξη των στόχων, όταν δεν υπάρχει πλέον ανάγκη για την επίτευξη των στόχων ή όταν υπάρχει η σαφής εκτίμηση ότι οι στόχοι δεν μπορούν να επιτευχθούν. \cite{PMBOK}

        \subsection{Ορισμός της διαχείρισης εργασιών}
            Η \textbf{διαχείριση εργασιών} (task management) ορίζεται ως η διαδικασία οργάνωσης, ιεράρχησης και παρακολούθησης των επιμέρους εργασιών καθ’ όλη τη διάρκειά τους, από τον αρχικό σχεδιασμό έως την ολοκλήρωσή τους, με στόχο τη διασφάλιση της αποδοτικής και αποτελεσματικής εκτέλεσής τους.

            Η διαδικασία αυτή αποτελεί θεμελιώδες στοιχείο για τη βελτίωση της παραγωγικότητας, τόσο σε ατομικό όσο και σε συλλογικό επίπεδο. Αν και παραδοσιακά η διαχείριση εργασιών πραγματοποιούνταν χειροκίνητα, η ραγδαία ανάπτυξη της τεχνολογίας έχει καταστήσει τα ψηφιακά εργαλεία τον κύριο τρόπο υλοποίησής της, προσφέροντας αυξημένη ευελιξία και ακρίβεια.

        \subsection{Πεδίο εφαρμογής διαχείρισης εργασιών}
            Η διαχείριση εργασιών διαθέτει ένα ευρύ πεδίο εφαρμογής, καλύπτοντας δραστηριότητες που κυμαίνονται από απλές καθημερινές υποχρεώσεις έως τη διαχείριση σύνθετων και απαιτητικών εργασιών.

            Μια από τις πιο άμεσες και προσιτές εφαρμογές της συναντάται στον καθημερινό προσωπικό προγραμματισμό. Σε αυτό το πλαίσιο, περιλαμβάνει τη χρήση εργαλείων όπως λίστες υποχρεώσεων (to-do lists), ημερολόγια ή ψηφιακές εφαρμογές (π.χ. Todoist \cite{Todoist}, Notion \cite{Notion}, Google Keep), τα οποία διευκολύνουν την οργάνωση προσωπικών υποχρεώσεων, τον προγραμματισμό ραντεβού ή δραστηριοτήτων αναψυχής, καθώς και τη διαχείριση στόχων. Μέσα από τέτοιες πλατφόρμες, δίνεται η δυνατότητα τόσο για βραχυπρόθεσμο όσο και για μακροπρόθεσμο σχεδιασμό, ενώ παρέχεται και η ευκολία παρακολούθησης της προόδου.

            \begin{figure}[h!] \noindent \centering
                \includegraphics[width=0.8\textwidth]{Todoist}
                \caption{Η εφαρμογή Todoist.}
            \end{figure}

            Στα επαγγελματικά περιβάλλοντα, η διαχείριση εργασιών διαδραματίζει καθοριστικό ρόλο, συμβάλλοντας στη βελτίωση της συνεργασίας και της συνολικής αποδοτικότητας. Κεντρικός στόχος της είναι η ορθολογική κατανομή του φόρτου εργασίας, η διασφάλιση του συντονισμού μεταξύ των μελών μιας ομάδας και η οργάνωση των καθημερινών δραστηριοτήτων.  Μέσω της διαχείρισης εργασιών καθίσταται δυνατή η διευθέτηση παράλληλων εργασιών, η ιεράρχησή τους με βάση την προτεραιότητα (επείγουσες ή μη), καθώς και η δίκαιη και στοχευμένη ανάθεση καθηκόντων στους εργαζομένους. Ένα από τα πλέον δημοφιλή εργαλεία που χρησιμοποιούνται σε ομαδικά επαγγελματικά περιβάλλοντα για τον σκοπό αυτό είναι το Trello \cite{Trello}, το οποίο παρέχει ευελιξία και οπτική οργάνωση των εργασιών.

            \begin{figure}[h!] \noindent \centering
                \includegraphics[width=0.9\textwidth]{Trello}
                \caption{Ένας kanban πίνακας της εφαρμογής Trello.}
            \end{figure}

            Τέλος, η ραγδαία εξέλιξη της τεχνολογίας έχει οδηγήσει στη δημιουργία προηγμένων πλατφορμών που αξιοποιούν σύγχρονες τεχνολογίες, όπως η τεχνητή νοημοσύνη και η ανάλυση δεδομένων, για τη βελτιστοποίηση της διαχείρισης εργασιών. Αυτές οι πλατφόρμες υπερβαίνουν τις παραδοσιακές μεθόδους οργάνωσης καθώς είναι σε θέση να αναγνωρίζουν πρότυπα, να προβλέπουν χρονικές απαιτήσεις, να ανιχνεύουν πιθανές συγκρούσεις στον χρονοπρογραμματισμό και να προτείνουν την ιδανική σειρά εκτέλεσης των εργασιών.

        \subsection{Διαφορά με διαχείριση έργου}
            Συχνά η έννοια της διαχείρισης εργασιών (task management) συγχέεται με αυτή της διαχείρισης έργου (project management). Η αλήθεια είναι πως πρόκειται για έννοιες που όντως συσχετίζονται αλλά στην πραγματικότητα η καθεμία εστιάζει σε διαφορετικά αντικείμενα.

            Η \textbf{διαχείριση εργασιών}, όπως έχει αναφέρθηκε, αφορά την παρακολούθηση διαφορετικών, μεμονομένων δραστηριοτήτων οι οποίες χρειάζεται να ολοκληρωθούν. Η διαχείριση εργασιών επικεντρώνεται στο \textit{μικροεπίπεδο}, στη διαχείριση καθημερινών υποχρεώσεων, στα διαφορετικά deadlines που μπορεί να υπάρχουν, την εξέλιξή τους ανά το χρόνο κ.α. Τα εργαλεία που αφορούν τη διαχείριση εργασιών περιλαμβάνουν ημερολόγια, υπενθυμίσεις ή χρονοδιαγράμματα.

            Αντίθετα, η \textbf{διαχείριση έργου} περιγράφει τον σχεδιασμό, την εκτέλεση και την ολοκλήρωση ενός ολόκληρου έργου. Ένα έργο αποτελείται και αυτό από διαφορετικές εργασίες, οι οποίες όμως είναι οργανωμένες προς έναν ευρύτερο στόχο. Επομένως, η έννοια της διαχείρισης έργου \textit{συμπεριλαμβάνει} τη διαχείριση εργασιών, αλλά επίσης προϋποθέτει επιπλέον απαιτήσεις όπως τη σωστή κατανομή πόρων (resource allocation) ή την αξιολόγηση κινδύνου (risk assessment). Τα λογισμικά διαχείρισης έργου έχουν λειτουργικότητες όπως διαγράμματα Γκαντ ή παρακολούθηση εξαρτήσεων.

            Στην παρούσα διπλωματική εργασία για λόγους πληρότητας θα αναλυθούν και κάποιες έννοιες που αφορούν τη διαχείριση έργου, έχοντας όμως υπόψην ότι η υλοποίηση της εφαρμογής αφορά τη διαχείριση εργασιών.

            %Εισήγαγε έναν πίνακα που συγκρίνει παραδείγματα εργαλείων για τη διαχείριση εργασιών και τη διαχείριση έργων, με τα χαρακτηριστικά τους.
            %Πρόσθεσε συγκεκριμένα case studies ή παραδείγματα εφαρμογών εργαλείων διαχείρισης σε πραγματικά σενάρια (π.χ., χρήση του Trello σε μια startup).
            %Ίσως είναι χρήσιμο να ολοκληρώσεις το κεφάλαιο με ένα σύντομο συμπέρασμα για τη σημασία της διαχείρισης εργασιών και έργων.
            %Ολοκλήρωσε τις παραπομπές και πρόσθεσε περισσότερες εάν γίνεται.

    \section{Ιστορική αναδρομή}
        Η διαχείριση και ο προγραμματισμός εργασιών αποτελούν έννοιες που υπήρχαν ήδη από την αρχαιότητα, πολύ πριν την ανάπτυξη ψηφιακών εργαλείων και αυτοματισμών, με τις πρώτες μορφές οργάνωσης να βασίζονται κυρίως στον προφορικό λόγο και την ανθρώπινη μνήμη. Με την ανάγκη για καταγραφή και παρακολούθηση, αναπτύχθηκαν και χρησιμοποιήθηκαν διάφορες μνημονικές συσκευές, οι οποίες προσέφεραν στους πολιτισμούς της εποχής τρόπους οργάνωσης και αποθήκευσης κρίσιμων πληροφοριών.

        Στην ενότητα αυτή, θα ασχοληθούμε με τις πρώτες καταγεγγραμμένες μορφές διαχείρισης εργασιών, την εξέλιξή τους μέχρι σήμερα όπως επίσης και τις σημαντικές καινοτομίες και μεθοδολογίες που αναπτύχθηκαν στην πορεία της ιστορίας, όπως το διάγραμμα Gantt και άλλες οργανωτικές τεχνικές, οι οποίες εξέλιξαν τη διαχείριση και τον προγραμματισμό των εργασιών.

%        Η διαχείριση εργασιών ήταν πάντα καθοριστική για την επιτυχή ολοκλήρωση των μεγάλων έργων στην ιστορία του ανθρώπου. Κλασικά παραδείγματα τέτοιων έργων είναι οι πυραμίδες της Αιγύπτου, το Στόουνχεντζ και το Σινικό τείχος, τα οποία απαιτούσαν εξαιρετική οργάνωση, συντονισμό και προγραμματισμό για να ολοκληρωθούν.

%        Η έννοια της διαχείρισης έργων επέτρεψε στους ηγέτες της αρχαιότητας να αναλάβουν φιλόδοξα και μεγαλεπήβολα έργα, καταφέρνοντας να οργανώσουν αποτελεσματικά τους διαθέσιμους πόρους — τόσο ανθρώπινους όσο και υλικούς — και να τηρήσουν αυστηρά χρονοδιαγράμματα.

        \subsection{Προφορικότητα και μνημονικές συσκευές}
            Στην αρχαιότητα, η διαχείριση των εργασιών βασιζόταν κυρίως στον προφορικό λόγο, ο οποίος αποτελούσε το κύριο μέσο μετάδοσης πληροφοριών και οδηγιών. Έτσι οι εργασίες αναθέτονταν μέσω προφορικών οδηγιών, ενώ η ακρίβεια της εκτέλεσης εξαρτιόταν σε μεγάλο βαθμό από την αξιοπιστία της ανθρώπινης μνήμης. Αυτό το σύστημα, αν και ήταν η μόνη επιλογή που υπήρχε εκείνη την εποχή, παρουσίαζε σοβαρά μειονεκτήματα, καθώς η μνήμη είναι επιρρεπής σε λάθη, ειδικά σε καταστάσεις που απαιτούν τη διαχείριση πολλαπλών ή σύνθετων εργασιών. Με λίγα λόγια, η εξάρτηση από την ανθρώπινη μνήμη περιόριζε την ακρίβεια και τη δυνατότητα αποτελεσματικής οργάνωσης, ιδίως σε περιπτώσεις που οι εργασίες ήταν περίπλοκες, έπρεπε να εκτελεστούν σε μεγάλα χρονικά διαστήματα ή αφορούσαν πολλά άτομα. \cite{Goody2013}

            Αυτή η αναγκαιότητα οδήγησε στην ανάπτυξη τεχνικών απομνημόνευσης και συστημάτων καταγραφής, που είχαν ως στόχο τη βελτίωση της διαχείρισης πληροφοριών και την αύξηση της αξιοπιστίας. Τέτοιες τεχνικές περιλάμβαναν τη χρήση επαναλαμβανόμενων φράσεων και ρυθμικών μοτίβων για την ευκολότερη απομνημόνευση οδηγιών. Επιπλέον, η δημιουργία χειροκίνητων συσκευών, όπως το \textit{λουκάσα} (lukasa) από τους Λούμπα του Κονγκό και το \textit{κουίπου} (quipu) από τους Ίνκα, εισήγαγε συστήματα που υποκαθιστούσαν εν μέρει την ανθρώπινη μνήμη, παρέχοντας μια οπτικοποιημένη αναπαράσταση πληροφοριών. Αυτές οι συσκευές δεν ήταν απλώς εργαλεία καταγραφής, αλλά αποτελούσαν καινοτόμες λύσεις διαχείρισης εργασιών, ενισχύοντας την ικανότητα ανάκλησης και οργάνωσης πληροφοριών.

            \begin{figure}[h!] \noindent \centering
                \includegraphics[width=0.6\textwidth]{Lukasa.jpg}
                \caption{Η συσκευή λουκάσα}
            \end{figure}

            \begin{figure}[h!] \noindent \centering
                \includegraphics[width=0.5\textwidth]{Quipo.jpg}
                \caption{Η συσκευή κουίπου}
            \end{figure}

            Το λουκάσα αποτελούνταν από πολύχρωμες χάντρες τοποθετημένες σε συγκεκριμένες θέσεις πάνω σε ξύλινες ή δερμάτινες επιφάνειες, προσφέροντας στους χειριστές έναν τρόπο να αποθηκεύουν, να οργανώνουν και να ανακαλούν πληροφορίες. \cite{Lukasa} Το κουίπου ήταν μια κατασκευή με χορδές από βαμβάκι ή μαλλί. Οι χορδές ήταν πολύχρωμες με κόμπους, επιτρέποντας έτσι την κατηγοριοποίηση και αποθήκευση πληροφοριών βάσει χρώματος, διάταξης και αριθμού. Οι Ίνκα δημιουργούσαν κόμπους στις χορδές και τις χρησιμοποιούσαν για τη συλλογή και παρακολούθηση των υποχρεώσεών τους ή και για την αποθήκευση άλλων πληροφοριών όπως δεδομένα απογραφής, φορολογικών υποχρεώσεων και άλλα. \cite{Quipu}

            Η δημιουργία τέτοιων τεχνικών και συστημάτων καταγραφής αναδεικνύει την ανθρώπινη ικανότητα να προσαρμόζεται σε πρακτικές ανάγκες και να δημιουργεί καινοτόμες λύσεις. Αυτά τα πρώιμα μέσα διαχείρισης εργασιών όχι μόνο κάλυψαν τις απαιτήσεις της εποχής, αλλά έθεσαν τα θεμέλια για τη μεταγενέστερη ανάπτυξη γραπτών και, τελικά, ψηφιακών συστημάτων, που επανακαθόρισαν τον τρόπο με τον οποίο οργανώνουμε και εκτελούμε εργασίες στις μέρες μας.

            % Προσθήκη σύγκρισης των μεθόδων των Λούμπα και των Ίνκα με άλλους πολιτισμούς, όπως οι Αιγύπτιοι, που βασίζονταν σε εγχάρακτες επιγραφές ή πάπυρους. Αυτό θα δώσει έναν ευρύτερο πολιτισμικό ορίζοντα.

        \subsection{Πρώτα ημερολόγια}
            Κατασκευές όπως τα ηλιακά ρολόγια αποτέλεσαν ένα από τα πρώτα εργαλεία που έδωσαν στους ανθρώπους τη δυνατότητα να διαιρέσουν την ημέρα σε διακριτά τμήματα. Η ανακάλυψη αυτών των εργαλείων αποτέλεσε καθοριστικό βήμα στην κατανόηση του χρόνου ως δομημένου και μετρήσιμου πόρου, επιτρέποντας στους πληθυσμούς να οργανώσουν καλύτερα τις καθημερινές τους δραστηριότητες. Η δυνατότητα αυτή οδήγησε σε μια σαφή διάκριση μεταξύ υποχρεώσεων και ελεύθερου χρόνου, καθώς οι άνθρωποι άρχισαν να προγραμματίζουν τις ώρες της ημέρας με μεγαλύτερη ακρίβεια. Αυτός ο διαχωρισμός ήταν θεμελιώδης για την ανάπτυξη πιο σύνθετων συστημάτων διαχείρισης του χρόνου, καθώς οι κοινότητες αντιλήφθηκαν τη σημασία του χρονοπρογραμματισμού για τη βελτιστοποίηση των συλλογικών τους δραστηριοτήτων.

            \begin{figure}[h!] \noindent \centering
                \includegraphics[width=0.4\textwidth]{Ancient-egyptian-sundial.jpg}
                \caption{\centering Το παλαιότερο γνωστό ηλιακό ρολόι από τους Αιγυπτίους· \\ χρησιμοποιούνταν για να μετράει τις ώρες εργασίας τους}
            \end{figure}

            Παράλληλα με τα ηλιακά ρολόγια, οι πρώιμες προσπάθειες δημιουργίας ημερολογίων συνέβαλαν καθοριστικά στην εξέλιξη της διαχείρισης εργασιών και χρόνου. Πολιτισμοί όπως οι Αιγύπτιοι, οι Ρωμαίοι και οι Μάγια ανέπτυξαν πολύπλοκα συστήματα ημερολογίων που βασίζονταν στις κινήσεις του ήλιου, της σελήνης και των άστρων. Αυτά τα ημερολόγια όχι μόνο προσδιόριζαν τον χρόνο για γεωργικές δραστηριότητες, όπως η σπορά και η συγκομιδή, αλλά χρησίμευαν και ως οδηγός για θρησκευτικές τελετές, κοινωνικές εκδηλώσεις και άλλες τελετουργίες. Μέσω αυτών των εργαλείων, έγινε εφικτός ο διαχωρισμός του χρόνου προσφέροντας μια πρώιμη μορφή συστηματικής οργάνωσης που συνέβαλε στην ανάπτυξη των κοινωνιών. \cite{Richards_2000}

            Η τεχνολογική πρόοδος έφερε σημαντικές εξελίξεις στον τρόπο μέτρησης και διαχείρισης του χρόνου. Τα ηλιακά ρολόγια, τα οποία ήταν εξαρτώμενα από την παρουσία του ήλιου, σταδιακά εξελίχθηκαν σε μηχανικά ρολόγια, τα οποία μπορούσαν να λειτουργούν ανεξάρτητα από τις καιρικές συνθήκες ή την ώρα της ημέρας. Αυτή η μετάβαση στα μηχανικά ρολόγια σηματοδότησε μια νέα εποχή για τον χρονοπρογραμματισμό. Τα μηχανικά ρολόγια προσέφεραν μεγαλύτερη ακρίβεια και έθεσαν τα θεμέλια για την ανάπτυξη πιο σύνθετων εργαλείων διαχείρισης εργασιών, που θα μπορούσαν να εξυπηρετήσουν τις αυξανόμενες απαιτήσεις των κοινωνιών.

        \subsection{Σύγχρονη εποχή}
            Η οργανωμένη διαχείριση εργασιών έχει τις ρίζες της βαθιά μέσα στην ιστορία, καθώς οι άνθρωποι πάντα αναζητούσαν τρόπους να οργανώσουν καλύτερα τις δραστηριότητές τους. Ωστόσο, οι πρώτες προσπάθειες τυποποίησης αυτής της διαδικασίας εντοπίζονται στον 18ο αιώνα, όταν η ανάγκη για μια συστηματική προσέγγιση έγινε πιο έντονη λόγω της ανάπτυξης των κοινωνιών και της αυξανόμενης πολυπλοκότητας των έργων. Στα τέλη του 19ου αιώνα, η βιομηχανική επανάσταση επέφερε τεράστιες αλλαγές στον τρόπο παραγωγής και κατασκευής. Τα έργα μεγάλης κλίμακας, όπως σιδηροδρομικά δίκτυα, γέφυρες και εργοστάσια, απαιτούσαν πιο οργανωμένες προσεγγίσεις στη διαχείριση ανθρώπινου δυναμικού και πόρων. Αυτές οι νέες απαιτήσεις οδήγησαν στην ανάγκη για πιο λεπτομερή και αποτελεσματική διαχείριση των εργασιών. Όμως η οργάνωση χιλιάδων εργατών, η διαχείριση μεγάλων ποσοτήτων πρώτων υλών και η τήρηση αυστηρών χρονοδιαγραμμάτων ήταν προκλήσεις που δε θα μπορούσαν να αντιμετωπιστούν με τις παραδοσιακές τεχνικές.

            Μηχανικοί όπως ο Henry Gantt εισήγαγαν πρωτοποριακές μεθόδους οργάνωσης, όπως το \textbf{διάγραμμα Γκαντ}. Το διάγραμμα Γκαντ είναι ένα εργαλείο που παρέχι οπτικοποίηση, αναπαριστώντας όλες τις επιμέρους εργασίες ενός έργου κατά μήκος ενός χρονικού άξονα, παρέχοντας μια καθαρή εικόνα των φάσεων υλοποίησής του. Με αυτόν τον τρόπο, οι υπεύθυνοι έργων μπορούσαν να παρακολουθούν την πρόοδο κάθε φάσης, να εντοπίζουν πιθανές καθυστερήσεις και να αναπροσαρμόζουν τον προγραμματισμό όπου ήταν απαραίτητο. Ένα από τα μεγαλύτερα πλεονεκτήματα του διαγράμματος Γκαντ ήταν η δυνατότητα προσδιορισμού της κρίσιμης διαδρομής του έργου, δηλαδή της αλληλουχίας των εργασιών που πρέπει να ολοκληρωθούν εντός συγκεκριμένων χρονικών ορίων για να διασφαλιστεί η έγκαιρη ολοκλήρωση του έργου. Θα αναφερθούμε με μεγαλύτερη λεπτομέρεια στο διάγραμμα Γκαντ στην ενότητα \ref{sec:methodologies}. Πάντως, το εργαλείο αυτό ήταν καθοριστικό σε μεγάλα έργα υποδομών, όπως η κατασκευή της Διώρυγας του Παναμά, που αποτέλεσε ένα από τα πιο φιλόδοξα και απαιτητικά έργα της εποχής, καθώς και το φράγμα Χούβερ, το οποίο απαίτησε σχολαστικό σχεδιασμό και συντονισμό πόρων σε πρωτόγνωρη κλίμακα. \cite{strefapmiHooverGreatest}

            \begin{figure}[h!] \noindent \centering
                \includegraphics[width=0.5\textwidth]{Hoover}
                \caption{\centering Το φράγμα Χούβερ \cite{britannicaHoover}}
            \end{figure}


            Η εισαγωγή μεθοδολογικών εργαλείων όπως το διάγραμμα Γκαντ δεν περιορίστηκε μόνο στη βιομηχανία και τα έργα υποδομών· αποτέλεσε την έμπνευση για νέες έρευνες και πρακτικές που επεκτάθηκαν σε διάφορους τομείς. Ένα από τα πιο χαρακτηριστικά παραδείγματα είναι το \textbf{Πρότζεκτ Μανχάταν} (Manhattan Project), το οποίο σχεδιάστηκε κατά τη διάρκεια του Β' Παγκοσμίου Πολέμου για το σχεδιασμό πυρηνικών όπλων. Αυτό το ιδιαίτερα απαιτητικό έργο προκάλεσε την ανάπτυξη δύο νέων μοντέλων διαχείρισης, του PERT (Program Evaluation and Review Technique) και του CPM (Critical Path Method). Το PERT σχεδιάστηκε για να αντιμετωπίσει την αβεβαιότητα στις εκτιμήσεις του χρόνου υλοποίησης των εργασιών, ενώ το CPM επικεντρώθηκε στην ανάλυση και τη βελτιστοποίηση της κρίσιμης διαδρομής του έργου. \cite{SaylorAcademyProjectManagement}. Θα αναφερθούμε και σε αυτά τα μοντέλα στην ενότητα \ref{sec:methodologies}.


    \section{Η συνδρομή της τεχνολογίας}
        Από τη δεκαετία του '60 και έπειτα, οι επιχειρήσεις άρχισαν να αναγνωρίζουν την αξία της συστηματικής και μεθοδικής οργάνωσης της εργασίας. Η ψηφιακή επανάσταση που ακολούθησε δε θα μπορούσε παρά να γιγαντώσει αυτή τη νέα πραγματικότητα. Η είσοδο των υπολογιστών, επέφερε και νέες δυνατότητες αποθήκευσης και ανάλυσης δεδομένων, οι οποίες άλλαξαν ριζικά τη διαχείριση έργων και εργασιών. Οι διαδικασίες που προηγουμένως απαιτούσαν χρονοβόρα χειρωνακτική εργασία και εκτεταμένη χρήση χαρτιού, έγιναν πλέον πιο γρήγορες και πιο ακριβείς. Οι επιχειρήσεις είχαν τώρα τη δυνατότητα να παρακολουθούν την πρόοδο των έργων τους σε πραγματικό χρόνο, να αναλύουν τα δεδομένα με μεγαλύτερη ακρίβεια και να λαμβάνουν στρατηγικές αποφάσεις που βασίζονταν σε σαφείς και τεκμηριωμένες πληροφορίες. Αυτή η συνειδητοποίηση οδήγησε στην υιοθέτηση εξειδικευμένων μεθόδων και εργαλείων, τα οποία στόχευαν στη μείωση των καθυστερήσεων, στη μεγιστοποίηση της απόδοσης και στη διασφάλιση της ποιότητας.

        Επιπλέον, η τεχνολογική πρόοδος έφερε εργαλεία και λογισμικά που ενίσχυσαν τη συνεργασία μεταξύ ομάδων και τμημάτων. Οι εφαρμογές διαχείρισης έργων, όπως το Microsoft Project και αργότερα οι πλατφόρμες συνεργασίας τύπου Trello και Asana, επέτρεψαν σε ομάδες διαφορετικών γεωγραφικών περιοχών να συνεργάζονται απρόσκοπτα, μειώνοντας τα εμπόδια επικοινωνίας. Παράλληλα, η δυνατότητα κατανομής εργασιών, η παρακολούθηση του χρονοδιαγράμματος και η ανάλυση των αποτελεσμάτων έγιναν πιο προσβάσιμες από ποτέ. Η τεχνολογία όχι μόνο βελτίωσε τη λειτουργικότητα των εργαλείων διαχείρισης αλλά τα έκανε επίσης πιο προσιτά σε μικρότερες επιχειρήσεις, που προηγουμένως δεν είχαν τη δυνατότητα να επενδύσουν σε τέτοιες λύσεις.

        Η συνεχής βελτίωση της τεχνολογίας συνέβαλε επίσης στη βελτιστοποίηση της λήψης αποφάσεων. Με τη χρήση αλγορίθμων και μοντέλων πρόβλεψης, οι επιχειρήσεις μπορούν να εντοπίζουν πιθανά προβλήματα πριν αυτά προκύψουν, να αξιολογούν εναλλακτικά σενάρια και να λαμβάνουν μέτρα προληπτικά. Αυτή η μεταστροφή προς την προληπτική διαχείριση, αντί της αντιδραστικής, αύξησε την αποτελεσματικότητα και μείωσε τους κινδύνους, καθιστώντας τις επιχειρήσεις πιο ανταγωνιστικές στο σύγχρονο παγκοσμιοποιημένο περιβάλλον.

        Η τεχνολογική εξέλιξη επαναπροσδιόρισε τις δυνατότητες οργάνωσης και παρακολούθησης εργασιών, παρέχοντας όχι μόνο εργαλεία που ενισχύουν τη συνεργασία αλλά και την απαραίτητη ευελιξία ώστε να προσαρμόζονται στις μεταβαλλόμενες ανάγκες της αγοράς.


        \subsection{Ψηφιακά εργαλεία}
            Με την εξέλιξη της τεχνολογίας, οι σημειώσεις και η οργάνωση μεταφέρθηκε από τις χειρόγραφες σημειώσεις, τα ημερολόγια και τα έγγραφα των γραφομηχανών σε ψηφιακά εργαλεία όπως υπολογιστικά φύλλα και προγράμματα σαν το Microsoft Project, σχεδιασμένα αποκλειστικά με σκοπό την αποτελεσματική διαχείριση έργων.

            \subsubsection{Microsoft Project}
                \begin{figure}[H] \noindent \centering
                    \includegraphics[width=0.7\textwidth]{MicrosoftProject3.png}
                    \caption{\centering Στιγμιότυπο από το Microsoft Project 3.0 (σε DOS) \cite{WinWorld}}
                \end{figure}

                \begin{figure}[H] \noindent \centering
                    \includegraphics[width=0.7\textwidth]{MicrosoftProject2000.png}
                    \caption{\centering Στιγμιότυπο από το Microsoft Project 2000 \cite{WinWorld}}
                \end{figure}

                Πρόκειται για ένα από τα πρώτα λογισμικά διαχείρισης έργων, σχεδιασμένα για το κοινό. Η ιδέα για την δημιουργία του προήλθε από μια φάρσα του Ron Bredehoeft, ο οποίος ήθελε να εκφράσει την συνταγή για τα αυγά μπένεντικτ σε όρους διαχείρισης έργων. Παρουσιάστηκε για πρώτη φορά το 1984 ως μια DOS εφαρμογή και πλέον έχει γίνει ένα καθιερωμένο εργαλείο σε όλες τις βιομηχανίες για την οργάνωση, τον προγραμματισμό και την παρακολούθηση της προόδου των έργων.

                Η κεντρική οθόνη του Microsoft Project χωρίζεται σε δύο περιοχές: το διάγραμμα Γκαντ και τον πίνακα που εισάγονται οι εργασίες (input table). Υπάρχει η δυνατότητα ιεράρχησης των εργασιών με την τοποθέτηση εσοχών (indents), η δημιουργία εξαρτήσεων με το καθορισμό προκατόχων (predecessors -- μια εργασία μπορεί να ξεκινήσει να εκτελείται μόνο όταν τελειώσει ο προκάτοχός της) και η δυνατότητα αυτόματου προγραμματισμού των εργασιών, λαμβάνοντας υπόψιν τις εξαρτήσεις τους. Επιπλέον μπορούν να δημιουργηθούν αλυσιδωτές εργασίες (η μια εργασία εκτελείται μετά την άλλη), ενώ επίσης μπορούν να ανατεθούν πόροι (resources) για κάθε εργασία. Κάθε χαρακτηριστικό μπορεί να τροποποιηθεί δυναμικά, είναι εφικτή η αποτύπωση των εργασιών πέρα από το διάγραμμα Γκαντ και σε μορφή ημερολογίου, φύλλου εργασίας (task sheet) κ.α., όπως επίσης και η δημιουργία στατιστικών.


    \section{Μεθοδολογίες} \label{sec:methodologies}

        \subsection{Διάγραμμα Γκαντ}
            Το \textbf{διάγραμμα Γκαντ} (Gantt chart) είναι μια δισδιάστατη γραφική απεικόνιση ενός έργου, με τον οριζόντιο άξονα να αποτελεί τον χρόνο (συχνά χωρισμένο σε διαστήματα ημερών, μηνών, χρόνων) και τον κατακόρυφο άξονα να αποτελεί τις διαφορετικές εργασίες που απαρτίζουν το έργο.

            Πρόκειται για ένα πολύ σημαντικό εργαλείο καθώς δείχνει οπτικά τον χρόνο που εκτιμάται ότι θα χρειαστεί κάθε τμήμα ενός έργου, επομένως μπορεί εύκολα να χρησιμοποιηθεί για την παρακολούθηση της προόδου όλων των επιμέρους εργασιών. Έτσι, αν κάποια ξεφεύγει από το ορισμένο χρονοδιάγραμμα, μπορούν άμεσα να γίνουν οι απαραίτητες ενέργειες που χρειάζονται. \cite{Xenos}

            Για τον σχεδιασμό ενός διαγράμματος Γκαντ είναι απαραίτητος ο αρχικός διαχωρισμός των επιμέρους εργασιών, όπως επίσης και μια εκτίμηση της χρονικής διάρκειάς τους. Στην συνέχεια οι εργασίες τοποθετούνται συνήθως με σειρά ώστε αυτές που τελειώνουν νωρίτερα να βρίσκονται ψηλότερα.

            Γενικά είναι μια εύκολη και γρήγορη κατασκευή που απεικονίζει με σαφήνεια τη χρονική διάρκεια και την αλληλουχία των εργασιών, αλλά από την άλλη δεν μπορεί να απεικονίσει τις εξαρτήσεις μεταξύ των επιμέρους εργασιών. Έτσι δεν είναι εμφανές ποιες εργασίες πρέπει να ολοκληρωθούν πρώτα ώστε να είναι εφικτή η εκτέλεση μιας επόμενης εργασίας, και επίσης δεν αναπαρίσταται η επίδραση μιας καθυστέρησης σε κάποια φάση του έργου. Τέλος, λόγω της στατικής του δομής, δεν δύναται να αναπροσαρμοστεί σε μεταβολές στη χρονική διάρκεια εκτέλεσης κάποιας εργασίας.

        \subsection{Program evaluation and review technique (PERT)}


        \subsection{Μέθοδος κρίσιμης διαδρομής (Critical Path Method -- CPM)}


        \subsection{Agile και Kanban}


    \section{Διαχείριση έργων στο πανεπιστήμιο}
        \subsection{Προβλήματα διαχείρισης που αντιμετωπίζουν οι φοιτητές}
            % Εισαγωγική παράγραφος

            Σε έρευνα \cite{Fukuzawa2015} που διεξήχθη στο Πανεπιστήμιο του Τσουκούμπα της Ιαπωνίας, η οποία διερευνούσε τη διαχείριση του προγραμματισμού των εργασιών από τη πλευρά των φοιτητών, παρατηρήθηκε πως η πλειοψηφία τους αντιμετωπίζει δυσκολίες στην εκκίνηση μιας νέας εργασίας με βασικούς λόγους: α) την έλλειψη χρόνου (26,9\%), β) την αγνόησή της επειδή τη θεωρούσαν ελάσσονος σημασίας (15,7\%), γ) επειδή την ξέχασαν (12,3\%), δ) λόγω κακής συνεργασίας (11,2\%) και ε) επειδή ήταν κουραστική (8,9\%). Παρατηρούμε πως οι τρεις πρώτοι λόγοι --που καλύπτουν το μεγαλύτερο ποσοστό (54,9\%) των λόγων-- αφορούν θέματα κακής οργάνωσης από την πλευρά των φοιτητών.

            Σε διαφορετική έρευνα \cite{Trujillo2020}, πάλι παρουσιάζεται πως το κυριότερο πρόβλημα που αντιμετωπίζουν οι φοιτητές είναι η σωστή δόμηση του προγράμματός τους. Συνήθως ο τρόπος διαβάσματός τους καθοδηγείται από τις ίδιες τις εργασίες που έχουν να κάνουν, μιας και μόνο αυτές έχουν καταληκτικές ημερομηνίες παράδοσης, και έτσι παραμελούν τα υπόλοιπα καθήκοντα που έχουν, όπως το να παρακολουθούν τις διαλέξεις.

            Όλα αυτά μας οδηγούν στο συμπέρασμα πως είναι απαραίτητος ένας αποτελεσματικός τρόπος προγραμματισμού και διαχείρισης των εργασιών τους.

        \subsection{Χαρακτηριστικά που οι φοιτητές θα επιθυμούσαν σε μια εφαρμογή}
            Σε έρευνα \cite{Trujillo2020} που πραγματοποιήθηκε στο τμήμα Πληροφορικής του Πανεπιστημίου του Εδιμβούργου, διαπιστώθηκε πως η πλειοψηφία της ακαδημαϊκής κοινότητας επιθυμεί μια εφαρμογή διαχείρισης εργασιών να διαθέτει ημερολόγιο (θεώρησαν σημαντικό να είναι καταγραμμένες οι ημερομηνίες έναρξης/λήξης για κάθε εργασία για τη σωστή οργάνωση, όπως επίσης και χρωματική ταξινόμηση (color-coding) των εργασιών), ειδοποιήσεις / γνωστοποιήσεις για τις εργασίες και to-do λίστες (με ιεράρχιση, ομαδοποίηση και δυνατότητα εμφάνισης μπάρας προόδου). Επίσης εκφράστηκε ενδιαφέρον για τη δημιουργία ενός συστήματος ανταμοιβής, με σκοπό την ενθάρρυνση των φοιτητών να ολοκληρώνουν εργασίες.