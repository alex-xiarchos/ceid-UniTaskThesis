\chapter{Διαχείριση Εργασιών}
    \begin{displayquote} \centering
        \say{\textit{Plans are nothing; planning is everything}} \\
        \hspace*{\fill}---Dwight D. Eisenhower
    \end{displayquote}
    \vspace{1em}

    Όπως αναφέρθηκε στην εισαγωγή, η παρούσα διπλωματική εργασία επικεντρώνεται στην ανάπτυξη μιας εφαρμογής για τον προγραμματισμό και την παρακολούθηση εργασιών. Οι διαδικασίες σχεδιασμού, υλοποίησης και παρακολούθησης αυτών των εργασιών εντάσσονται στο ευρύτερο πλαίσιο της \textbf{διαχείρισης εργασιών} (task management). Για τον λόγο αυτό, στο παρόν κεφάλαιο κρίνεται απαραίτητο να παρουσιαστεί μια πιο αναλυτική ερμηνεία του όρου, καθώς και οι θεμελιώδεις αρχές και μέθοδοι που σχετίζονται με τη συγκεκριμένη έννοια.

    Η διαχείριση εργασιών αποτελεί αναπόσπαστο στοιχείο της καθημερινότητας, επηρεάζοντας τη συνολική παραγωγικότητα και αποδοτικότητα τόσο σε ατομικό όσο και σε συλλογικό επίπεδο. Με την αυξανόμενη πολυπλοκότητα των σύγχρονων υποχρεώσεων και την ανάγκη για αποτελεσματικό συντονισμό πολλαπλών δραστηριοτήτων, η εφαρμογή αποτελεσματικών πρακτικών διαχείρισης καθίσταται επιτακτική

    Στο πλαίσιο του κεφαλαίου αυτού, θα εξεταστεί αναλυτικά η διαδικασία διαχείρισης εργασιών, η ιστορική της εξέλιξη, οι μεθοδολογίες που βελτιώνουν την αποδοτικότητά της και ο ρόλος που διαδραματίζει στην πανεπιστημιακή κοινότητα. Ο στόχος είναι να αναδειχθεί η σημασία της έννοιας αυτής για την καθημερινότητα, με ιδιαίτερη έμφαση στους φοιτητές, καθώς αποτελεί τη βάση της λογικής για την ανάπτυξη της εφαρμογής που θα παρουσιαστεί στη συνέχεια.

    \section{Το πρόβλημα της διαχείρισης εργασιών}
        Στην παρούσα ενότητα θα οριστεί η έννοια της διαχείρισης εργασιών, εστιάζοντας στις διαφορές της από τη διαχείριση έργου. Παράλληλα, θα αναλυθούν τα κύρια πεδία εφαρμογής της, προκειμένου να αποκτηθεί μια ολοκληρωμένη κατανόηση του όρου.

        \subsection{Ορισμός της εργασίας}
            \textbf{Εργασία} (task, project) στη διαχείριση εργασιών ορίζεται ως μια προσωρινή δραστηριότητα που αναλαμβάνεται για τη δημιουργία ενός μοναδικού αποτελέσματος, όπως ένα προϊόν, μια υπηρεσία, μια αναφορά ή κάποιο άλλο παραδοτέο. Η εκτέλεση μιας εργασίας πραγματοποιείται εντός μιας προκαθορισμένης χρονικής περιόδου και ολοκληρώνεται όταν επιτευχθούν οι στόχοι της, όταν δεν υπάρχει πλέον ανάγκη υλοποίησής της ή όταν εκτιμάται ότι οι στόχοι δεν μπορούν να επιτευχθούν \cite{PMBOK}.

        \subsection{Ορισμός της διαχείρισης εργασιών}
            \textbf{Διαχείριση εργασιών} (task management) ορίζεται ως η διαδικασία οργάνωσης, ιεράρχησης και παρακολούθησης των επιμέρους εργασιών καθ’ όλη τη διάρκειά τους, από τον αρχικό σχεδιασμό έως την ολοκλήρωσή τους, με στόχο τη διασφάλιση της αποδοτικής και αποτελεσματικής εκτέλεσής τους \cite{PMBOK}.

            Η διαδικασία αυτή αποτελεί θεμελιώδη παράγοντα για τη βελτίωση της παραγωγικότητας, τόσο σε ατομικό όσο και σε συλλογικό επίπεδο. Ενώ παραδοσιακά η διαχείριση εργασιών πραγματοποιούνταν χειροκίνητα, η τεχνολογική πρόοδος έχει καταστήσει τα ψηφιακά εργαλεία τον κύριο τρόπο υλοποίησής της, προσφέροντας αυξημένη ευελιξία και ακρίβεια.

        \subsection{Πεδίο εφαρμογής διαχείρισης εργασιών}
            Η διαχείριση εργασιών καλύπτει ένα ευρύ πεδίο εφαρμογών, από την προσωπική οργάνωση έως τη διαχείριση σύνθετων επαγγελματικών διαδικασιών.

            Σε προσωπικό επίπεδο, περιλαμβάνει τη χρήση εργαλείων όπως λίστες υποχρεώσεων (to-do lists), ημερολόγια και ψηφιακές εφαρμογές\footnote{Θα αναφερθούμε πιο εκτεταμένα σε ψηφιακά εργαλεία στην ενότητα \ref{sec:digitaltools}} (π.χ. Todoist \cite{Todoist}, Notion \cite{Notion}, Google Keep). Αυτές οι πλατφόρμες διευκολύνουν την οργάνωση υποχρεώσεων, τον προγραμματισμό δραστηριοτήτων και την παρακολούθηση της προόδου.

            Σε επαγγελματικό επίπεδο, η διαχείριση εργασιών διαδραματίζει καθοριστικό ρόλο, συμβάλλοντας στη βελτίωση της συνεργασίας και της συνολικής αποδοτικότητας. Κεντρικός στόχος της είναι η ορθολογική κατανομή του φόρτου εργασίας, η διασφάλιση του συντονισμού μεταξύ των μελών μιας ομάδας και η οργάνωση των καθημερινών δραστηριοτήτων.  Μέσω της διαχείρισης εργασιών καθίσταται δυνατή η διευθέτηση παράλληλων εργασιών, η ιεράρχησή τους με βάση την προτεραιότητα (επείγουσες ή μη), καθώς και η δίκαιη και στοχευμένη ανάθεση καθηκόντων στους εργαζομένους. Ένα από τα πλέον δημοφιλή εργαλεία που χρησιμοποιούνται σε ομαδικά επαγγελματικά περιβάλλοντα για τον σκοπό αυτό είναι το Trello \cite{Trello}, το οποίο παρέχει ευελιξία και οπτική οργάνωση των εργασιών.

            Τέλος, η ραγδαία εξέλιξη της τεχνολογίας έχει οδηγήσει στη δημιουργία προηγμένων πλατφορμών που αξιοποιούν σύγχρονες τεχνολογίες, όπως η τεχνητή νοημοσύνη και η ανάλυση δεδομένων, για τη βελτιστοποίηση της διαχείρισης εργασιών. Αυτές οι πλατφόρμες υπερβαίνουν τις παραδοσιακές μεθόδους οργάνωσης καθώς είναι σε θέση να αναγνωρίζουν πρότυπα, να προβλέπουν χρονικές απαιτήσεις, να ανιχνεύουν πιθανές συγκρούσεις στον χρονοπρογραμματισμό και να προτείνουν την ιδανική σειρά εκτέλεσης των εργασιών.

        \subsection{Διαφορά με διαχείριση έργου}
            Η διαχείριση εργασιών (task management) συχνά συγχέεται με τη διαχείριση έργου (project management). Αν και οι δύο έννοιες σχετίζονται, έχουν διακριτές διαφορές.

            Η \textbf{διαχείριση εργασιών}, όπως έχει αναφέρθηκε, εστιάζει στην παρακολούθηση και εκτέλεση διαφορετικών, μεμονωμένων δραστηριοτήτων, διασφαλίζοντας την έγκαιρη ολοκλήρωσή τους. Η διαχείριση εργασιών επικεντρώνεται στο \textit{μικροεπίπεδο}, στη διαχείριση καθημερινών υποχρεώσεων, στα διαφορετικά deadlines που μπορεί να υπάρχουν, την εξέλιξή τους ανά το χρόνο κ.α. Τα εργαλεία που αφορούν τη διαχείριση εργασιών περιλαμβάνουν ημερολόγια, υπενθυμίσεις ή χρονοδιαγράμματα.

            Αντίθετα, η \textbf{διαχείριση έργου} περιγράφει τον σχεδιασμό, την εκτέλεση και την ολοκλήρωση ενός ολόκληρου έργου. Ένα έργο αποτελείται και αυτό από διαφορετικές εργασίες, οι οποίες όμως είναι οργανωμένες προς έναν ευρύτερο στόχο. Επομένως, η έννοια της διαχείρισης έργου \textit{συμπεριλαμβάνει} τη διαχείριση εργασιών, αλλά επίσης προϋποθέτει επιπλέον απαιτήσεις όπως τη σωστή κατανομή πόρων (resource allocation) ή την αξιολόγηση κινδύνου (risk assessment). Τα λογισμικά διαχείρισης έργου έχουν λειτουργικότητες όπως διαγράμματα Γκαντ ή παρακολούθηση εξαρτήσεων.

            Στην παρούσα διπλωματική εργασία, για λόγους πληρότητας, θα αναλυθούν και ορισμένες έννοιες διαχείρισης έργου, με την υλοποίηση της εφαρμογής ωστόσο να επικεντρώνεται αποκλειστικά στη διαχείριση εργασιών.


    \section{Ιστορική αναδρομή}
        Η διαχείριση και ο προγραμματισμός εργασιών αποτελούν έννοιες που υπήρχαν ήδη από την αρχαιότητα, πολύ πριν την ανάπτυξη ψηφιακών εργαλείων και αυτοματισμών, με τις πρώτες μορφές οργάνωσης να βασίζονται κυρίως στον προφορικό λόγο και την ανθρώπινη μνήμη. Με την ανάγκη για καταγραφή και παρακολούθηση, αναπτύχθηκαν και χρησιμοποιήθηκαν διάφορες μνημονικές συσκευές, οι οποίες προσέφεραν στους πολιτισμούς της εποχής τρόπους οργάνωσης και αποθήκευσης κρίσιμων πληροφοριών.

        Στην ενότητα αυτή, θα ασχοληθούμε με τις πρώτες καταγεγραμμένες μορφές διαχείρισης εργασιών, την εξέλιξή τους μέχρι σήμερα, καθώς και τις σημαντικές καινοτομίες και μεθοδολογίες που αναπτύχθηκαν στην πορεία της ιστορίας, όπως το διάγραμμα Γκαντ και άλλες οργανωτικές τεχνικές, οι οποίες διαμόρφωσαν τη σύγχρονη διαχείριση και τον προγραμματισμό των εργασιών.

        \subsection{Προφορικότητα και μνημονικές συσκευές}
            Στην αρχαιότητα, η διαχείριση των εργασιών βασιζόταν κυρίως στον προφορικό λόγο, ο οποίος αποτελούσε το κύριο μέσο μετάδοσης πληροφοριών και οδηγιών. Έτσι οι εργασίες αναθέτονταν μέσω προφορικών οδηγιών, ενώ η ακρίβεια της εκτέλεσης εξαρτιόταν σε μεγάλο βαθμό από την αξιοπιστία της ανθρώπινης μνήμης. Αυτό το σύστημα, αν και ήταν η μόνη επιλογή που υπήρχε εκείνη την εποχή, παρουσίαζε σοβαρά μειονεκτήματα, καθώς η μνήμη είναι επιρρεπής σε λάθη, ειδικά σε καταστάσεις που απαιτούν τη διαχείριση πολλαπλών ή σύνθετων εργασιών. Με λίγα λόγια, η εξάρτηση από την ανθρώπινη μνήμη περιόριζε την ακρίβεια και τη δυνατότητα αποτελεσματικής οργάνωσης, ιδίως σε περιπτώσεις που οι εργασίες ήταν περίπλοκες, έπρεπε να εκτελεστούν σε μεγάλα χρονικά διαστήματα ή αφορούσαν πολλά άτομα \cite{Goody2013}.

            Αυτή η αναγκαιότητα οδήγησε στην ανάπτυξη τεχνικών απομνημόνευσης και συστημάτων καταγραφής, που είχαν ως στόχο τη βελτίωση της διαχείρισης πληροφοριών και την αύξηση της αξιοπιστίας. Τέτοιες τεχνικές περιλάμβαναν τη χρήση επαναλαμβανόμενων φράσεων και ρυθμικών μοτίβων για την ευκολότερη απομνημόνευση οδηγιών. Επιπλέον, η δημιουργία χειροκίνητων συσκευών, όπως το \textit{λουκάσα} (lukasa) από τους Λούμπα του Κονγκό και το \textit{κουίπου} (quipu) από τους Ίνκα, εισήγαγε συστήματα που υποκαθιστούσαν εν μέρει την ανθρώπινη μνήμη, παρέχοντας μια οπτικοποιημένη αναπαράσταση πληροφοριών. Αυτές οι συσκευές δεν ήταν απλώς εργαλεία καταγραφής, αλλά αποτελούσαν καινοτόμες λύσεις διαχείρισης εργασιών, ενισχύοντας την ικανότητα ανάκλησης και οργάνωσης πληροφοριών.

            \begin{figure}[h!] \noindent \centering
                \includegraphics[width=0.6\textwidth]{Lukasa.jpg}
                \caption{Η συσκευή λουκάσα}
            \end{figure}

            \begin{figure}[h!] \noindent \centering
                \includegraphics[width=0.5\textwidth]{Quipo.jpg}
                \caption{Η συσκευή κουίπου}
            \end{figure}

            Το λουκάσα αποτελούνταν από πολύχρωμες χάντρες τοποθετημένες σε συγκεκριμένες θέσεις πάνω σε ξύλινες ή δερμάτινες επιφάνειες, προσφέροντας στους χειριστές έναν τρόπο να αποθηκεύουν, να οργανώνουν και να ανακαλούν πληροφορίες \cite{Lukasa}. Το κουίπου ήταν μια κατασκευή με χορδές από βαμβάκι ή μαλλί. Οι χορδές ήταν πολύχρωμες με κόμπους, επιτρέποντας έτσι την κατηγοριοποίηση και αποθήκευση πληροφοριών βάσει χρώματος, διάταξης και αριθμού. Οι Ίνκα δημιουργούσαν κόμπους στις χορδές και τις χρησιμοποιούσαν για τη συλλογή και παρακολούθηση των υποχρεώσεών τους ή και για την αποθήκευση άλλων πληροφοριών όπως δεδομένα απογραφής, φορολογικών υποχρεώσεων και άλλα \cite{Quipu}.

            Η δημιουργία τέτοιων τεχνικών και συστημάτων καταγραφής αναδεικνύει την ανθρώπινη ικανότητα να προσαρμόζεται σε πρακτικές ανάγκες και να δημιουργεί καινοτόμες λύσεις. Αυτά τα πρώιμα μέσα διαχείρισης εργασιών όχι μόνο κάλυψαν τις απαιτήσεις της εποχής, αλλά έθεσαν τα θεμέλια για τη μεταγενέστερη ανάπτυξη γραπτών και, τελικά, ψηφιακών συστημάτων, που επανακαθόρισαν τον τρόπο με τον οποίο οργανώνουμε και εκτελούμε εργασίες στις μέρες μας.

        \subsection{Πρώτα ημερολόγια}
            Κατασκευές όπως τα ηλιακά ρολόγια αποτέλεσαν ένα από τα πρώτα εργαλεία που έδωσαν στους ανθρώπους τη δυνατότητα να διαιρέσουν την ημέρα σε διακριτά τμήματα. Η ανακάλυψη αυτών των εργαλείων αποτέλεσε καθοριστικό βήμα στην κατανόηση του χρόνου ως δομημένου και μετρήσιμου πόρου, επιτρέποντας στους πληθυσμούς να οργανώσουν καλύτερα τις καθημερινές τους δραστηριότητες. Η δυνατότητα αυτή οδήγησε σε μια σαφή διάκριση μεταξύ υποχρεώσεων και ελεύθερου χρόνου, καθώς οι άνθρωποι άρχισαν να προγραμματίζουν τις ώρες της ημέρας με μεγαλύτερη ακρίβεια. Αυτός ο διαχωρισμός ήταν θεμελιώδης για την ανάπτυξη πιο σύνθετων συστημάτων διαχείρισης του χρόνου, καθώς οι κοινότητες αντιλήφθηκαν τη σημασία του χρονοπρογραμματισμού για τη βελτιστοποίηση των συλλογικών τους δραστηριοτήτων.

            \begin{figure}[h!] \noindent \centering
                \includegraphics[width=0.4\textwidth]{Ancient-egyptian-sundial.jpg}
                \caption{\centering Το παλαιότερο γνωστό ηλιακό ρολόι από τους Αιγυπτίους· \\ χρησιμοποιούνταν για να μετράει τις ώρες εργασίας τους}
            \end{figure}

            Παράλληλα με τα ηλιακά ρολόγια, οι πρώιμες προσπάθειες δημιουργίας ημερολογίων συνέβαλαν καθοριστικά στην εξέλιξη της διαχείρισης εργασιών και χρόνου. Πολιτισμοί όπως οι Αιγύπτιοι, οι Ρωμαίοι και οι Μάγια ανέπτυξαν πολύπλοκα συστήματα ημερολογίων που βασίζονταν στις κινήσεις του ήλιου, της σελήνης και των άστρων. Αυτά τα ημερολόγια όχι μόνο προσδιόριζαν τον χρόνο για γεωργικές δραστηριότητες, όπως η σπορά και η συγκομιδή, αλλά χρησίμευαν και ως οδηγός για θρησκευτικές τελετές, κοινωνικές εκδηλώσεις και άλλες τελετουργίες. Μέσω αυτών των εργαλείων, έγινε εφικτός ο διαχωρισμός του χρόνου, προσφέροντας μια πρώιμη μορφή συστηματικής οργάνωσης που συνέβαλε στην ανάπτυξη των κοινωνιών \cite{Richards_2000}.

            Η τεχνολογική πρόοδος έφερε σημαντικές εξελίξεις στον τρόπο μέτρησης και διαχείρισης του χρόνου. Τα ηλιακά ρολόγια, τα οποία ήταν εξαρτώμενα από την παρουσία του ήλιου, σταδιακά εξελίχθηκαν σε μηχανικά ρολόγια, τα οποία μπορούσαν να λειτουργούν ανεξάρτητα από τις καιρικές συνθήκες ή την ώρα της ημέρας. Αυτή η μετάβαση στα μηχανικά ρολόγια σηματοδότησε μια νέα εποχή για τον χρονοπρογραμματισμό. Τα μηχανικά ρολόγια προσέφεραν μεγαλύτερη ακρίβεια και έθεσαν τα θεμέλια για την ανάπτυξη πιο σύνθετων εργαλείων διαχείρισης εργασιών, που θα μπορούσαν να εξυπηρετήσουν τις αυξανόμενες απαιτήσεις των κοινωνιών.

        \subsection{Σύγχρονη εποχή}
            Η οργανωμένη διαχείριση εργασιών έχει τις ρίζες της βαθιά μέσα στην ιστορία, καθώς οι άνθρωποι πάντα αναζητούσαν τρόπους να οργανώσουν καλύτερα τις δραστηριότητές τους. Ωστόσο, οι πρώτες προσπάθειες τυποποίησης αυτής της διαδικασίας εντοπίζονται στον 18ο αιώνα, όταν η ανάγκη για μια συστηματική προσέγγιση έγινε πιο έντονη λόγω της ανάπτυξης των κοινωνιών και της αυξανόμενης πολυπλοκότητας των έργων. Στα τέλη του 19ου αιώνα, η βιομηχανική επανάσταση επέφερε τεράστιες αλλαγές στον τρόπο παραγωγής και κατασκευής. Τα έργα μεγάλης κλίμακας, όπως σιδηροδρομικά δίκτυα, γέφυρες και εργοστάσια, απαιτούσαν πιο οργανωμένες προσεγγίσεις στη διαχείριση ανθρώπινου δυναμικού και πόρων. Αυτές οι νέες απαιτήσεις οδήγησαν στην ανάγκη για πιο λεπτομερή και αποτελεσματική διαχείριση των εργασιών. Όμως η οργάνωση χιλιάδων εργατών, η διαχείριση μεγάλων ποσοτήτων πρώτων υλών και η τήρηση αυστηρών χρονοδιαγραμμάτων ήταν προκλήσεις που δε θα μπορούσαν να αντιμετωπιστούν με τις παραδοσιακές τεχνικές.

            Μηχανικοί όπως ο Henry Gantt εισήγαγαν πρωτοποριακές μεθόδους οργάνωσης, όπως το \textbf{διάγραμμα Γκαντ}. Το διάγραμμα Γκαντ είναι ένα εργαλείο που παρέχει οπτικοποίηση, αναπαριστώντας όλες τις επιμέρους εργασίες ενός έργου κατά μήκος ενός χρονικού άξονα, παρέχοντας μια καθαρή εικόνα των φάσεων υλοποίησής του. Με αυτόν τον τρόπο, οι υπεύθυνοι έργων μπορούσαν να παρακολουθούν την πρόοδο κάθε φάσης, να εντοπίζουν πιθανές καθυστερήσεις και να αναπροσαρμόζουν τον προγραμματισμό όπου ήταν απαραίτητο. Ένα από τα μεγαλύτερα πλεονεκτήματα του διαγράμματος Γκαντ ήταν η δυνατότητα προσδιορισμού της κρίσιμης διαδρομής του έργου, δηλαδή της αλληλουχίας των εργασιών που πρέπει να ολοκληρωθούν εντός συγκεκριμένων χρονικών ορίων για να διασφαλιστεί η έγκαιρη ολοκλήρωση του έργου. Θα αναφερθούμε με μεγαλύτερη λεπτομέρεια στο διάγραμμα Γκαντ στην ενότητα \ref{sec:methodologies}. Πάντως, το εργαλείο αυτό ήταν καθοριστικό σε μεγάλα έργα υποδομών, όπως η κατασκευή της Διώρυγας του Παναμά, που αποτέλεσε ένα από τα πιο φιλόδοξα και απαιτητικά έργα της εποχής, καθώς και το φράγμα Χούβερ, το οποίο απαίτησε σχολαστικό σχεδιασμό και συντονισμό πόρων σε πρωτόγνωρη κλίμακα \cite{strefapmiHooverGreatest}.

            \begin{figure}[h!] \noindent \centering
                \includegraphics[width=0.5\textwidth]{Hoover}
                \caption{\centering Το φράγμα Χούβερ \cite{britannicaHoover}}
            \end{figure}

            Η εισαγωγή μεθοδολογικών εργαλείων όπως το διάγραμμα Γκαντ δεν περιορίστηκε μόνο στη βιομηχανία και τα έργα υποδομών· αποτέλεσε την έμπνευση για νέες έρευνες και πρακτικές που επεκτάθηκαν σε διάφορους τομείς. Ένα από τα πιο χαρακτηριστικά παραδείγματα είναι το Πρότζεκτ Μανχάταν (Manhattan Project), το οποίο σχεδιάστηκε κατά τη διάρκεια του Β' Παγκοσμίου Πολέμου για το σχεδιασμό πυρηνικών όπλων. Αυτό το ιδιαίτερα απαιτητικό έργο οδήγησε στην ανάπτυξη δύο νέων μοντέλων διαχείρισης, του PERT (Program Evaluation and Review Technique) και του CPM (Critical Path Method). Το PERT σχεδιάστηκε για να αντιμετωπίσει την αβεβαιότητα στις εκτιμήσεις του χρόνου υλοποίησης των εργασιών, ενώ το CPM επικεντρώθηκε στην ανάλυση και τη βελτιστοποίηση της κρίσιμης διαδρομής του έργου \cite{SaylorAcademyProjectManagement}. Θα αναφερθούμε και σε αυτά τα μοντέλα στην ενότητα \ref{sec:methodologies}.


    \section{Η συνδρομή της τεχνολογίας}
        Από τη δεκαετία του '60 και έπειτα, οι επιχειρήσεις άρχισαν να αναγνωρίζουν την αξία της συστηματικής και μεθοδικής οργάνωσης της εργασίας. Η ψηφιακή επανάσταση που ακολούθησε δε θα μπορούσε παρά να γιγαντώσει αυτή τη νέα πραγματικότητα. Η είσοδο των υπολογιστών, επέφερε και νέες δυνατότητες αποθήκευσης και ανάλυσης δεδομένων, δυνατότητες που άλλαξαν ριζικά τη διαχείριση των εργασιών. Έτσι, διαδικασίες που προηγουμένως απαιτούσαν χρονοβόρα χειρωνακτική εργασία και εκτεταμένη χρήση χαρτιού, έγιναν πλέον πιο γρήγορες και πιο ακριβείς.

        Η τεχνολογική πρόοδος έφερε νέα εργαλεία και λογισμικά που ενίσχυσαν τη συνεργασία μεταξύ ομάδων και τμημάτων, όπως το Microsoft Project και αργότερα οι πλατφόρμες συνεργασίας τύπου Trello και Asana, επέτρεψαν σε ομάδες διαφορετικών γεωγραφικών περιοχών να συνεργάζονται απρόσκοπτα, μειώνοντας τα εμπόδια επικοινωνίας, ενώ πρόσθεσε και αυτοματισμούς στην κατανομή εργασιών και στη δημιουργία χρονοδιαγραμμάτων. Η τεχνολογία όχι μόνο βελτίωσε τη λειτουργικότητα των εργαλείων διαχείρισης αλλά τα έκανε επίσης πιο προσιτά σε μικρότερες επιχειρήσεις, που προηγουμένως δεν είχαν τη δυνατότητα να επενδύσουν σε τέτοιες λύσεις.

        \subsection{Ψηφιακά εργαλεία} \label{sec:digitaltools}
            Με την εξέλιξη της τεχνολογίας, οι σημειώσεις και η οργάνωση μεταφέρθηκε από τις χειρόγραφες σημειώσεις, τα ημερολόγια και τα έγγραφα των γραφομηχανών σε  ψηφιακά εργαλεία. Παραδείγματα αυτών σε ατομικό επίπεδο είναι το Todoist ή το Notion και σε επαγγελματικό επίπεδο προγράμματα σαν το Microsoft Project.

            \subsubsection{Todoist}
                Αν και έχει χάσει πλέον την πρωτοκαθεδρία του, το Todoist \cite{Todoist} παραμένει μια από τις πιο δημοφιλείς εφαρμογές διαχείρισης εργασιών, σχεδιασμένη για να βοηθάει τόσο απλούς χρήστες όσο και επαγγελματίες να οργανώνουν τις υποχρεώσεις τους και να βελτιώνουν την παραγωγικότητά τους. Η εφαρμογή επιτρέπει τη δημιουργία λιστών εργασιών με δυνατότητα ομαδοποίησης σε έργα, υποέργα ή ετικέτες (tags). Οι χρήστες μπορούν να καθορίσουν ημερομηνίες και προθεσμίες καθώς και να ρυθμίσουν υπενθυμίσεις, καταφέροντας έτσι την αποδοτικότερη οργάνωση του χρόνου τους, ενώ οι ειδοποιήσεις τους διασφαλίζουν ότι δε θα παραλείψουν καμία σημαντική εργασία. Επιπλέον, περιλαμβάνει σύστημα επιβράβευσης (\say{Karma}) ενθαρρύνοντας τη συνέπεια και την ολοκλήρωση των εργασιών, ενώ υπάρχει η δυνατότητα ενσωμάτωσης με εξωτερικές εφαρμογές όπως το Google Calendar.

                \begin{figure}[h!] \noindent \centering
                    \includegraphics[width=0.7\textwidth]{Todoist}
                    \caption{Η εφαρμογή Todoist.}
                \end{figure}

%                \begin{figure}[h!] \noindent \centering
%                    \includegraphics[width=0.8\textwidth]{TodoistCalendar.png}
%                    \caption{\centering Στιγμιότυπο του Todoist}
%                \end{figure}

            \subsubsection{Notion}
                Το Notion \cite{Notion} είναι αυτή τη στιγμή ίσως η πιο δημοφιλής πλατφόρμα προσωπικής οργάνωσης. Συνδυάζει στοιχεία για task-management, note-taking, βάσεις δεδομένων και συνεργατικότητας σε μία ενιαία εφαρμογή, καθιστώντας το ιδανικό για χρήστες που επιζητούν έναν κεντρικό χώρο για τη διαχείριση της εργασιακής ή προσωπικής τους ζωής. Το κύριο χαρακτηριστικό του Notion είναι η δυνατότητα δημιουργίας προσαρμοσμένων σελίδων χρησιμοποιώντας Markdown γλώσσα, στις οποίες οι χρήστες μπορούν να μορφοποιήσουν το κείμενο, να προσθέσουν πίνακες ή να ενσωματώσουν διάφορα στοιχεία όπως λίστες εργασιών, πίνακες Kanban, ημερολόγια, checklists, ή ακόμη και embedded αρχεία. Αυτή η ευελιξία επιτρέπει τη δημιουργία εξατομικευμένων συστημάτων διαχείρισης, προσαρμοσμένων στις μοναδικές ανάγκες του κάθε χρήστη, λειτουργώντας ως μια προσωπική εγκυκλοπαίδεια.

                \begin{figure}[h!] \noindent \centering
                    \includegraphics[width=0.8\textwidth]{Notion Screen.png}
                    \caption{\centering Στιγμιότυπο του Notion}
                \end{figure}

                Επίσης, υπάρχει η δυνατότητα δημιουργίας βάσεων δεδομένων (οι οποίες μπορούν να λειτουργήσουν ως λίστες εργασιών, παρακολούθησης προόδου για έργα κ.α.) τις οποίες μπορούν να φιλτράρουν, να ταξινομούν και να χειρίζονται όπως θέλουν. Τέλος, υπάρχει δυνατότητα για συνεργατική κοινή χρήση σελίδων, την ενσωμάτωση με άλλα εργαλεία όπως το Google Drive ή το Slack.

            \subsubsection{Microsoft Project}
                \begin{figure}[h!] \noindent \centering
                    \includegraphics[width=0.7\textwidth]{MicrosoftProject3.png}
                    \caption{\centering Στιγμιότυπο από το Microsoft Project 3.0 (σε DOS) \cite{WinWorld}}
                \end{figure}

                Πρόκειται για ένα από τα πρώτα λογισμικά διαχείρισης έργων, σχεδιασμένα για το κοινό. Η ιδέα για τη δημιουργία του προήλθε από μια φάρσα του Ron Bredehoeft, ο οποίος, θέλοντας να αναπαραστήσει τη διαδικασία παρασκευής αυγών μπένεντικτ σε όρους διαχείρισης έργων, ανέπτυξε την ιδέα για ένα εργαλείο που θα μπορούσε να χρησιμοποιηθεί για την οργάνωση και τον προγραμματισμό οποιουδήποτε έργου. Το Microsoft Project παρουσιάστηκε για πρώτη φορά το 1984 ως εφαρμογή για DOS, κερδίζοντας αμέσως την προσοχή των επαγγελματιών. Σήμερα, αποτελεί ένα από τα πιο καθιερωμένα εργαλεία για την οργάνωση, τον χρονοπρογραμματισμό και την παρακολούθηση έργων, με εφαρμογές σε πλήθος βιομηχανιών, από την κατασκευή μέχρι την πληροφορική και τη διαχείριση ανθρώπινων πόρων.

                \begin{figure}[h!] \noindent \centering
                    \includegraphics[width=0.7\textwidth]{MicrosoftProject2000.png}
                    \caption{\centering Στιγμιότυπο από το Microsoft Project 2000 \cite{WinWorld}}
                \end{figure}

                Η κεντρική οθόνη του λογισμικού χωρίζεται σε δύο βασικές περιοχές: το διάγραμμα Γκαντ (Gantt chart) και τον πίνακα εισαγωγής εργασιών (input table). Ο πίνακας εισαγωγής επιτρέπει την εισαγωγή λεπτομερών πληροφοριών σχετικά με κάθε εργασία, όπως η διάρκεια, οι εξαρτήσεις και οι πόροι.

                Παρέχονται λειτουργικότητες όπως η δυνατότητα ιεράρχησης των εργασιών μέσω της τοποθέτησης εσοχών (indents) (δημιουργώντας μιας σαφούς δομής του έργου, διευκολύνοντας τη διαχείριση πολύπλοκων έργων με πολλές υποκατηγορίες), η δημιουργία εξαρτήσεων μεταξύ των εργασιών (predecessors) (για παράδειγμα, μια εργασία μπορεί να προγραμματιστεί να ξεκινήσει μόνο όταν ολοκληρωθεί μια άλλη), η δυνατότητα αυτόματου προγραμματισμού των εργασιών (auto-scheduling), η δυνατότητα δημιουργίας αλυσιδωτών εργασιών (linked tasks), στις οποίες μια εργασία εκτελείται αμέσως μετά την ολοκλήρωση μιας άλλης, διευκολύνοντας τον προγραμματισμό μεγάλων και σύνθετων έργων. Επιπλέον, το λογισμικό προσφέρει ευελιξία στην προβολή των δεδομένων, επιτρέποντας την αποτύπωση των εργασιών πέρα από το διάγραμμα Γκαντ σε μορφή ημερολογίου, φύλλου εργασίας (task sheet) ή ακόμη και η δημιουργία στατιστικών αναφορών. Έτσι, για παράδειγμα μια ομάδα μπορεί να επιλέξει το ημερολόγιο για να βλέπει τις ημερήσιες υποχρεώσεις της, ενώ ένας διευθυντής μπορεί να επιλέξει τις στατιστικές αναφορές για να αξιολογήσει τη συνολική πρόοδο.

            \subsubsection{Trello}
                \begin{figure}[h!] \noindent \centering
                    \includegraphics[width=0.8\textwidth]{Trello}
                    \caption{Στιγμιότυπο του Trello}
                \end{figure}

                Το Trello \cite{Trello} είναι μια πλατφόρμα διαχείρισης εργασιών που βασίζεται στους πίνακες Kanban (θα αναλυθούν στην ενότητα \ref{sec:methodologies}), επιτρέποντας μια εύληπτη οργάνωση της καθημερινότητας. Με τη χρήση πινάκων, λιστών και καρτών, οι χρήστες μπορούν συνεργατικά να παρακολουθούν την πρόοδο των εργασιών τους, ιδιαίτερα για εργασίες που χρειάζονται ομαδική συνεργασία, καθιστώντας το ιδιαίτερα δημοφιλές σε ομάδες όλων των μεγεθών και σε διάφορους τομείς. Κάθε πίνακας αντιπροσωπεύει ένα πρότζεκτ, ενώ οι λίστες μπορούν να χρησιμοποιηθούν για την κατηγοριοποίηση των εργασιών σε στάδια (\say{To Do}, \say{In Progress}, \say{Done}). Οι κάρτες, που τοποθετούνται μέσα στις λίστες, αντιπροσωπεύουν συγκεκριμένες εργασίες που πρέπει να ολοκληρωθούν. Οι χρήστες μπορούν να προσθέτουν περιγραφές, checklists, προθεσμίες, συνημμένα αρχεία, και ετικέτες (labels) στις κάρτες, επιτρέποντας την εξατομίκευση και την οργάνωση κάθε εργασίας σύμφωνα με τις ανάγκες τους. Προσφέρονται εργαλεία αυτοματοποίησης (λειτουργία \say{Butler}) και υπάρχουν δυνατότητες ενσωμάτωσης με άλλες εφαρμογές.


    \section{Μεθοδολογίες} \label{sec:methodologies}
        Στις προηγούμενες ενότητες, αναφερθήκαμε στην ιστορική εξέλιξη της οργάνωσης των εργασιών και του χρόνου μας και το πως η ραγδαία πρόοδος της τεχνολογίας έθεσαν τα θεμέλια για τις σύγχρονες προσεγγίσεις που ακολουθούμε σήμερα στη διαχείριση εργασιών. Επίσης, έγινε αναφορά σε ψηφιακά εργαλεία, όπως το Notion, το Trello και το Todoist, και το πώς παρέχουν λειτουργικότητα που διευκολύνει τη ροή των εργασιών και ενισχύει την παραγωγικότητα των ομάδων.

        Παράλληλα με την ανάπτυξη των εργαλείων, ωστόσο, αναπτύχθηκαν και υιοθετήθηκαν και αντίστοιχες μεθοδολογίες, οι οποίες παρέχουν βασικές αρχές για την αποτελεσματική διαχείριση σύνθετων έργων. Παραδείγματα αυτών των μεθοδολογιών που θα αναλυθούν είναι το διάγραμμα Γκαντ, το PERT και το Kanban.

        Στο πλαίσιο της παρούσας διπλωματικής, η οποία επικεντρώνεται στην ανάπτυξη μιας εφαρμογής διαχείρισης εργασιών που θα περιλαμβάνει και πίνακα Kanban, η αναφορά στις μεθοδολογίες αυτές είναι απαραίτητη για λόγους πληρότητας ώστε να αναδειχθούν οι αρχές που καθοδηγούν τον σχεδιασμό τέτοιων εργαλείων.

        \subsection{Διάγραμμα Γκαντ}
            Το \textbf{διάγραμμα Γκαντ} (Gantt chart) έχει καθιερωθεί ως ένα από τα πιο χρήσιμα εργαλεία στη διαχείριση έργων, καθώς παρέχει μια εύληπτη γραφική αναπαράσταση των επιμέρους εργασιών ενός έργου. Στον οριζόντιο άξονα απεικονίζεται ο χρόνος, ενώ στον κατακόρυφο άξονα παρατίθενται οι διαφορετικές εργασίες που συνθέτουν το έργο.  Για τη δημιουργία ενός διαγράμματος Γκαντ, είναι απαραίτητος ο αρχικός καταμερισμός του έργου σε επιμέρους εργασίες. Αυτό περιλαμβάνει την αναλυτική καταγραφή κάθε δραστηριότητας που πρέπει να ολοκληρωθεί και την εκτίμηση της χρονικής διάρκειας που θα απαιτηθεί για την ολοκλήρωσή της. Αφού γίνει ο καταμερισμός, οι εργασίες τοποθετούνται στο διάγραμμα με τέτοιο τρόπο ώστε αυτές που ολοκληρώνονται νωρίτερα να βρίσκονται ψηλότερα, διατηρώντας μια σαφή δομή που διευκολύνει την ανάγνωση και την κατανόηση του χρονοδιαγράμματος.

            Η ευκολία και η ταχύτητα με την οποία μπορεί να κατασκευαστεί ένα διάγραμμα Γκαντ αποτελούν έναν από τους κύριους λόγους για τη δημοτικότητά του. Παρέχει μια σαφή απεικόνιση της χρονικής διάρκειας και της αλληλουχίας των εργασιών, κάνοντας τη χρήση του προσιτή ακόμη και για άτομα που δεν έχουν εξειδικευμένες γνώσεις στη διαχείριση έργων.

            Παρόλα αυτά, το διάγραμμα Γκαντ έχει και ορισμένους περιορισμούς. Ένας από αυτούς είναι η αδυναμία του να αποτυπώσει τις εξαρτήσεις μεταξύ των εργασιών και την επίδραση της καθυστέρισης μιας εργασίας στο συνολικό έργο. Για παράδειγμα, δεν είναι εμφανές ποιες εργασίες πρέπει να ολοκληρωθούν πριν ξεκινήσει μια άλλη, κάτι που μπορεί να οδηγήσει σε παρανοήσεις ή καθυστερήσεις αν δεν υπάρχει κατάλληλος συντονισμός. Επιπλέον, η στατική του φύση περιορίζει τη δυνατότητα αναπροσαρμογής όταν οι συνθήκες αλλάξουν, όπως σε περιπτώσεις μεταβολής της διάρκειας μιας εργασίας ή προσθήκης νέων δραστηριοτήτων. Αυτό σημαίνει ότι, ενώ είναι εξαιρετικό για την αρχική φάση σχεδιασμού και την παρακολούθηση, μπορεί να μην επαρκεί σε δυναμικά περιβάλλοντα όπου απαιτείται συνεχής αναπροσαρμογή. Παρόλα αυτά, παραμένει ένα από τα πιο χρήσιμα εργαλεία για την κατανόηση της χρονικής διάστασης ενός έργου και τη συνολική εποπτεία της προόδου του \cite{Xenos}.

            \begin{figure}[h!] \noindent \centering
                \includegraphics[width=0.9\textwidth]{GanttReddit}
                \caption{Παράδειγμα διαγράμματος Γκαντ}
            \end{figure}

        \subsection{Program evaluation and review technique (PERT)}
            Το \textbf{Program Evaluation and Review Technique (PERT)} συνδυαστικά και με τη μέθοδο κρίσιμης διαδρομής (Critical Path Method -- CPM), είναι μια μεθοδολογία προγραμματισμού και ελέγχου έργων που επικεντρώνεται στον υπολογισμό και την αξιολόγηση του χρόνου ολοκλήρωσης ενός έργου, ενώ παράλληλα παρέχει σαφή εικόνα των σχέσεων και των εξαρτήσεων μεταξύ των δραστηριοτήτων του. Αναπτύχθηκε τη δεκαετία του 1950 για την υποστήριξη σύνθετων έργων με υψηλή αβεβαιότητα, όπως ο προγραμματισμός στρατιωτικών προγραμμάτων.

            Σε ένα διάγραμμα PERT (διάγραμμα αξιολόγησης έργου), το έργο αναλύεται σε επιμέρους δραστηριότητες, καθεμία από τις οποίες απεικονίζεται ως κόμβος σε ένα γράφημα. Αυτή η δομή επιτρέπει την οπτικοποίηση των σχέσεων και των εξαρτήσεων μεταξύ των δραστηριοτήτων, καθιστώντας σαφές ποιες πρέπει να ολοκληρωθούν πρώτα και ποιες μπορούν να εκτελούνται παράλληλα.

            Το PERT βασίζεται στην εκτίμηση του χρόνου για κάθε δραστηριότητα χρησιμοποιώντας τρεις παραμέτρους: τον αισιόδοξο χρόνο (ο ελάχιστος χρόνος ολοκλήρωσης), τον πιο πιθανό χρόνο και τον απαισιόδοξο χρόνο (ο μέγιστος χρόνος ολοκλήρωσης). Με βάση αυτές τις εκτιμήσεις, υπολογίζεται ο αναμενόμενος χρόνος για κάθε δραστηριότητα, λαμβάνοντας υπόψη την αβεβαιότητα.

            \begin{figure}[h!] \noindent \centering
                \includegraphics[width=0.5\textwidth]{XenosPert1}
                \caption{Παράδειγμα κόμβου σε διάγραμμα PERT \cite{Xenos}}
            \end{figure}

            Μια σημαντική έννοια στο PERT είναι ο υπολογισμός της κρίσιμης διαδρομής, δηλαδή της αλληλουχίας δραστηριοτήτων που καθορίζει τη συνολική διάρκεια του έργου. Οι δραστηριότητες που βρίσκονται στην κρίσιμη διαδρομή δεν επιτρέπουν καθυστερήσεις, καθώς οποιαδήποτε καθυστέρηση σε αυτές θα παρατείνει το συνολικό χρονοδιάγραμμα του έργου. Αντίθετα, οι μη κρίσιμες δραστηριότητες έχουν ένα χρονικό περιθώριο (slack), το οποίο τους επιτρέπει κάποιες καθυστερήσεις χωρίς να επηρεαστεί το συνολικό έργο.

            \begin{figure}[h!] \noindent \centering
                \includegraphics[width=0.7\textwidth]{Pert_Chart_Example}
                \caption{\centering Παράδειγμα διαγράμματος PERT \\ (με κόκκινο εμφανίζεται η κρίσιμη διαδρομή)}
            \end{figure}

            Η μεθοδολογία PERT παρέχει σημαντικά εργαλεία για την ανάλυση του χρόνου ολοκλήρωσης του έργου, όπως ο υπολογισμός του νωρίτερου και του αργότερου χρόνου για κάθε γεγονός και δραστηριότητα. Με τη βοήθεια αυτών των υπολογισμών, οι διαχειριστές έργων μπορούν να προβλέψουν την ελάχιστη και μέγιστη διάρκεια του έργου, να αναγνωρίσουν κρίσιμα σημεία και να σχεδιάσουν κατάλληλα τις ενέργειες τους για την αντιμετώπιση πιθανών καθυστερήσεων.

            Το PERT, παρόλο που είναι ιδιαίτερα χρήσιμο σε έργα με πολλές αβεβαιότητες, παρουσιάζει ορισμένους περιορισμούς. Οι εκτιμήσεις χρόνου συχνά βασίζονται σε υποκειμενικά δεδομένα, γεγονός που μπορεί να οδηγήσει σε αποκλίσεις. Παρόλα αυτά, η εφαρμογή του PERT συμβάλλει σημαντικά στη διαχείριση έργων, διευκολύνοντας τη λήψη αποφάσεων και την κατανομή πόρων με τρόπο αποτελεσματικό \cite{ProjectManagement21stCentury}.

%        \subsection{Ευέλικτη (Agile) μεθοδολογία}
%            Το Agile είναι ένας συνδυασμός εργαλείων, διαδικασιών και νοοτροπίας που βοηθά τις ομάδες να σκέφτονται πιο αποτελεσματικά, να εργάζονται με μεγαλύτερη αποδοτικότητα και να λαμβάνουν καλύτερες αποφάσεις. Αυτή η προσέγγιση πρωτοεμφανίστηκε το 2001 με τη δημοσίευση του Agile Manifesto \cite{AgileManifesto}, το οποίο θέτει τέσσερις κεντρικές αξίες που εστιάζουν στους ανθρώπους και τις αλληλεπιδράσεις, τη λειτουργικότητα του προϊόντος, τη συνεργασία με τους πελάτες και την προσαρμογή στις αλλαγές. Οι Agile μεθοδολογίες καλύπτουν παραδοσιακούς τομείς της μηχανικής λογισμικού, όπως η διαχείριση έργων, ο σχεδιασμός και η αρχιτεκτονική λογισμικού, αλλά και η βελτίωση διαδικασιών.
%
%            Κάθε μεθοδολογία του Agile περιλαμβάνει πρακτικές που είναι σχεδιασμένες ώστε να είναι όσο το δυνατόν πιο απλές στην υιοθέτηση, ενώ το σύνολο αυτών ενισχύει την αποτελεσματικότητα μιας ομάδας. Ωστόσο, το Agile δεν περιορίζεται μόνο στις τεχνικές· αποτελεί επίσης έναν τρόπο σκέψης. Αυτή η νοοτροπία επιτρέπει στις ομάδες να μοιράζονται πληροφορίες και να λαμβάνουν αποφάσεις συλλογικά, αντί να εξαρτώνται αποκλειστικά από τη διαχείριση ενός ατόμου.
%
%            Η νοοτροπία του Agile προάγει τη συνεργασία και τη συμμετοχή όλων των μελών της ομάδας στον σχεδιασμό, την υλοποίηση και τη βελτίωση των διαδικασιών. Μέσα από αυτήν την προσέγγιση, κάθε μέλος της ομάδας έχει ισότιμο λόγο στη λήψη αποφάσεων, ενώ διασφαλίζεται ότι όλοι εργάζονται με κοινή κατανόηση των στόχων και των προτεραιοτήτων.
%
%            Η βασική αρχή του Agile είναι η διαίρεση της εργασίας σε μικρότερες, διαχειρίσιμες επαναλήψεις, γνωστές ως sprints. Κάθε sprint έχει συγκεκριμένη διάρκεια, συνήθως δύο έως τέσσερις εβδομάδες, και στο τέλος του παραδίδεται λειτουργικό λογισμικό ή προϊόν, το οποίο μπορεί να αξιολογηθεί από τον πελάτη. Αυτός ο τρόπος εργασίας επιτρέπει την ενσωμάτωση των σχολίων και τη γρήγορη προσαρμογή στις νέες απαιτήσεις. Η συνεργασία είναι κεντρικό στοιχείο του Agile, με τις ομάδες να εργάζονται στενά και διαρκώς με τους πελάτες και τα ενδιαφερόμενα μέρη για την επίτευξη του καλύτερου δυνατού αποτελέσματος.
%
%            Μια από τις πιο δημοφιλείς πρακτικές που υιοθετούνται στο πλαίσιο του Agile είναι το Scrum, το οποίο οργανώνει την εργασία μέσω συγκεκριμένων ρόλων, συναντήσεων και εργαλείων, διατηρώντας μια σαφή εικόνα της προόδου. Επίσης, το Extreme Programming (XP) και το Lean Software Development είναι άλλες μέθοδοι που βασίζονται στις αρχές του Agile, εστιάζοντας στην ποιότητα του κώδικα και την αποδοτικότητα αντίστοιχα \cite{Agile_Stellmano}.

        \subsection{Kanban} \label{subsec:Kanban}
            Το Kanban είναι μια μέθοδος διαχείρισης εργασιών που αρχικά αναπτύχθηκε από την Toyota στον τομέα της παραγωγής τη δεκαετία του 1940, με σκοπό τη βελτίωση της ροής εργασιών και την ελαχιστοποίηση των καθυστερήσεων και στη συνέχεια εξελίχθηκε και υιοθετήθηκε ευρέως στον τομέα της ανάπτυξης λογισμικού και άλλων έργων, για να βελτιώσει την αποτελεσματικότητα των ομάδων και την παράδοση των έργων.

            Η κεντρική ιδέα του Kanban είναι η οπτικοποίηση των εργασιών μέσω ενός πίνακα (Kanban board). Στον πίνακα, οι εργασίες αναπαρίστανται ως κάρτες που μετακινούνται μεταξύ διαφορετικών σταδίων, όπως \say{To Do}, \say{In Progress} και \say{Done}. Αυτή η οπτική αναπαράσταση της διαδικασίας επιτρέπει στα μέλη της ομάδας να παρακολουθούν την πρόοδο των εργασιών σε πραγματικό χρόνο και να εντοπίζουν εύκολα τα τμήματα του έργου που καθυστερούν ή χρειάζονται προσοχή. Στη σύγχρονη πρακτική, τα Kanban boards συνήθως υποστηρίζονται από ψηφιακά εργαλεία όπως το Trello \cite{Trello}, το Jira \cite{Jira} και το Asana \cite{Asana}, τα οποία προσφέρουν επιπλέον δυνατότητες όπως η αυτόματη ενημέρωση και η συνεργασία σε πραγματικό χρόνο.

            \begin{figure}[h!] \noindent \centering
                \includegraphics[width=0.9\textwidth]{Kanban}
                \caption{Ένας Kanban πίνακας στο Jira}
            \end{figure}

            \begin{figure}[h!] \noindent \centering
                \includegraphics[width=0.8\textwidth]{Kanban_Bottleneck}
                \caption{Ένα παράδειγμα συμφορήσεων \cite{Agile_Stellman}}
            \end{figure}

            Η βασική αρχή του Kanban είναι ο περιορισμός της εργασίας που βρίσκεται σε εξέλιξη (\say{Work In Progress, WIP}). Έτσι οι ομάδες επικεντρώνονται σε συγκεκριμένες εργασίες και τις ολοκληρώνουν προτού ξεκινήσουν μια νέα. Με αυτόν τον τρόπο, το Kanban διασφαλίζει ότι η ομάδα δεν αναλαμβάνει περισσότερες εργασίες από όσες μπορεί να διαχειριστεί, ενισχύοντας τη ροή της εργασίας και μειώνοντας τις καθυστερήσεις. Επιπλέον, μπορούν εύκολα να εντοπίζουν συμφορήσεις (bottlenecks) στη ροή εργασιών. Οι συμφορήσεις αυτές προκύπτουν όταν μια συγκεκριμένη φάση ή εργασία καθυστερεί και εμποδίζει την πρόοδο των υπόλοιπων εργασιών.

            Η εφαρμογή του Kanban έχει αποδειχθεί αποτελεσματική στην αύξηση της παραγωγικότητας και της ποιότητας των έργων, καθώς ενθαρρύνει τη συνεχιζόμενη βελτίωση και την ανάλυση της ροής εργασίας. Με την εφαρμογή της μεθόδου, οι ομάδες μπορούν να παρακολουθούν τις καθυστερήσεις, να μειώσουν τις χρόνιες καθυστερήσεις και να βελτιώσουν τις επικοινωνιακές ροές, ενισχύοντας τη συνεργασία και τη διαφάνεια \cite{AndersonKanban} \cite{Agile_Stellman}.


    \section{Διαχείριση εργασιών στο πανεπιστήμιο}
        Σε πολλές περιπτώσεις, η αποτελεσματικότητα των φοιτητών καθορίζεται κυρίως από την οργάνωσή τους και τον σωστό προγραμματισμό τους, τόσο στις ακαδημαϊκές όσο και στις προσωπικές τους υποχρεώσεις. Η σωστή διαχείριση χρόνου και προτεραιοτήτων είναι καθοριστική για την αποφυγή άγχους, την αύξηση της παραγωγικότητας και την επίτευξη των στόχων τους. Παρόλα αυτά, οι φοιτητές συχνά έρχονται αντιμέτωποι με προκλήσεις, όπως η αναβλητικότητα και η έλλειψη σωστής οργάνωσης για την ολοκλήρωση των καθημερινών τους καθηκόντων. Οι ερευνητικές προσπάθειες στον τομέα αυτό αναδεικνύουν τα εμπόδια που αντιμετωπίζουν οι φοιτητές και προσφέρουν πολύτιμες πληροφορίες για τη βελτίωση των δεξιοτήτων διαχείρισης εργασιών.

        \subsection{Προβλήματα διαχείρισης που αντιμετωπίζουν οι φοιτητές}
            Σε έρευνα \cite{Fukuzawa2015} που διεξήχθη στο Πανεπιστήμιο του Τσουκούμπα της Ιαπωνίας, η οποία διερευνούσε τη διαχείριση του προγραμματισμού των εργασιών από την πλευρά των φοιτητών, παρατηρήθηκε πως η πλειοψηφία τους αντιμετωπίζει δυσκολίες στην εκκίνηση μιας νέας εργασίας. Οι βασικοί λόγοι που καταγράφηκαν περιλαμβάνουν: α) την έλλειψη χρόνου (26,9\%), β) την αγνόηση της εργασίας επειδή τη θεωρούσαν ελάσσονος σημασίας (15,7\%), γ) επειδή τη ξέχασαν (12,3\%), δ) λόγω κακής συνεργασίας (11,2\%) και ε) επειδή ήταν κουραστική (8,9\%). Οι τρεις πρώτοι λόγοι, που καλύπτουν το μεγαλύτερο ποσοστό (54,9\%), αφορούν σε θέματα κακής οργάνωσης από την πλευρά των φοιτητών, υποδεικνύοντας την ανάγκη για αποτελεσματικότερα εργαλεία χρονοπρογραμματισμού.

            Παράλληλα, μια διαφορετική έρευνα \cite{Trujillo2020} καταδεικνύει πάλι πως το κυριότερο πρόβλημα που αντιμετωπίζουν οι φοιτητές είναι η σωστή δόμηση του προγράμματός τους. Παρατηρήθηκε ότι ο τρόπος με τον οποίο οργανώνουν το διάβασμά τους καθοδηγείται κυρίως από τις καταληκτικές ημερομηνίες των εργασιών τους, με αποτέλεσμα να παραμελούν άλλες σημαντικές ακαδημαϊκές δραστηριότητες, όπως η παρακολούθηση διαλέξεων. Αυτό οδηγεί σε ανισορροπία μεταξύ των ακαδημαϊκών τους υποχρεώσεων, επηρεάζοντας την απόδοσή τους συνολικά.

            Από τις παραπάνω έρευνες προκύπτουν ορισμένα σημαντικά συμπεράσματα. Πρώτον, η έλλειψη οργανωτικών δεξιοτήτων παραμένει ένας από τους κύριους παράγοντες που εμποδίζουν την αποτελεσματική διαχείριση των εργασιών από τους φοιτητές. Οι λόγοι αυτοί συνδέονται συχνά με την αναβλητικότητα και την έλλειψη εργαλείων που θα μπορούσαν να βοηθήσουν στην αποδοτικότερη οργάνωση των καθημερινών τους υποχρεώσεων. Δεύτερον, η ανάγκη για ένα δομημένο σύστημα προγραμματισμού είναι εμφανής, καθώς θα μπορούσε να παρέχει σαφείς προτεραιότητες και να συμβάλει στη μείωση του άγχους που προκαλείται από τις καταληκτικές ημερομηνίες. Συνεπώς, ένα αποτελεσματικό σύστημα διαχείρισης και προγραμματισμού των εργασιών, προσαρμοσμένο στις ανάγκες των φοιτητών, μπορεί να λειτουργήσει ως βασικό εργαλείο για την ενίσχυση της παραγωγικότητας και της επιτυχίας τους.

        \subsection{Χαρακτηριστικά που οι φοιτητές θα επιθυμούσαν σε μια εφαρμογή} \label{sec:student_preferences}
            Σε έρευνα \cite{Trujillo2020} που πραγματοποιήθηκε στο τμήμα Πληροφορικής του Πανεπιστημίου του Εδιμβούργου, αναδείχθηκαν κάποιες σημαντικές προτιμήσεις και απαιτήσεις της ακαδημαϊκής κοινότητας σχετικά με τη λειτουργικότητα των εφαρμογών διαχείρισης εργασιών. Η πλειοψηφία των συμμετεχόντων τόνισε τη σημασία της ύπαρξης ενός ενσωματωμένου ημερολογίου, το οποίο θα παρέχει τη δυνατότητα καταγραφής των ημερομηνιών έναρξης και λήξης για κάθε εργασία. Αυτή η λειτουργία θεωρήθηκε κρίσιμη για τη σωστή οργάνωση και τον προγραμματισμό των υποχρεώσεων, καθώς επιτρέπει στους χρήστες να έχουν σαφή εικόνα των προθεσμιών τους. Παράλληλα, υπογραμμίστηκε η αξία της χρωματικής ταξινόμησης (color-coding), η οποία διευκολύνει τη διάκριση μεταξύ διαφορετικών κατηγοριών ή τύπων εργασιών, ενισχύοντας τη διαφάνεια και την ευκολία χρήσης της εφαρμογής.

            Επιπλέον, οι συμμετέχοντες επισήμαναν την ανάγκη για ειδοποιήσεις/γνωστοποιήσεις, οι οποίες θα λειτουργούν ως υπενθυμίσεις για τις επερχόμενες προθεσμίες ή τις εκκρεμότητες που απαιτούν άμεση προσοχή. Εξίσου σημαντική θεωρήθηκε η ύπαρξη to-do λιστών, οι οποίες θα διαθέτουν λειτουργίες όπως ιεράρχηση των εργασιών, ομαδοποίηση σε κατηγορίες, και δυνατότητα εμφάνισης μπάρας προόδου. Αυτές οι δυνατότητες συμβάλλουν στην αποτελεσματικότερη παρακολούθηση της προόδου των εργασιών και στη δημιουργία μιας αίσθησης ολοκλήρωσης και επίτευξης στόχων.

            Ιδιαίτερο ενδιαφέρον παρουσίασε η πρόταση για την ενσωμάτωση ενός συστήματος ανταμοιβής, το οποίο στοχεύει στην ενθάρρυνση των φοιτητών να ολοκληρώνουν τις εργασίες τους εγκαίρως. Ένα τέτοιο σύστημα θα μπορούσε να περιλαμβάνει την εμφάνιση γραφικών στοιχείων όπως κομφετί ή εικονικά μετάλλια, ή την ανταμοιβή πόντων, συμβάλλοντας στη δημιουργία κινήτρων. Η εισαγωγή αυτού του συστήματος έχει σκοπό να ενισχύσει τη δέσμευση και την παραγωγικότητα των χρηστών, προσφέροντας έναν πιο διαδραστικό και ελκυστικό τρόπο διαχείρισης εργασιών. Με την ενσωμάτωση χαρακτηριστικών όπως τα προαναφερθέντα, τέτοιες εφαρμογές μπορούν να βελτιώσουν ουσιαστικά την παραγωγικότητα και την αποτελεσματικότητα, προσφέροντας παράλληλα μια ευχάριστη εμπειρία χρήσης.