\chapter{Διαχείριση Εργασιών}

    \section{<Πρόβλημα διαχείρισης εργασιών>}
        Studies of Task Management Towards the Design of a Personal Task List Manager

    \section{Διαχείριση εργασιών στο πανεπιστήμιο}
        \subsection{Προβλήματα διαχείρισης που αντιμετωπίζουν οι φοιτητές}
            Σε έρευνα \cite{Fukuzawa2015} που διεξήχθει στο Πανεπιστήμιο του Τσουκούμπα της Ιαπωνίας η οποία διερευνούσε τη διαχείριση του προγραμματισμού των εργασιών των φοιτητών,
                παρατηρήθηκε πως η πλειοψηφία των φοιτητών αντιμετωπίζει δυσκολίες στην εκκίνηση μιας νέας εργασίας με βασικούς λόγους:
                α) την έλλειψη χρόνου (26,9\%), β) την αγνόησή της επειδή τη θεωρούσαν ελάσσονος σημασίας (15,7\%), γ) επειδή την ξέχασαν (12,3\%),
                δ) λόγω κακής συνεργασίας (11,2\%) και ε) επειδή ήταν κουραστική (8,9\%).
            Παρατηρούμε πως οι τρείς πρώτοι λόγοι --που καλύπτουν το μεγαλύτερο ποσοστό (54,9\%) των λόγων-- αφορούν θέματα κακής οργάνωσης από την πλευρά των φοιτητών.

            Σε διαφορετική έρευνα \cite{Trujillo2020}, πάλι παρουσιάζεται πως το κυριότερο πρόβλημα που αντιμετωπίζουν οι φοιτητές είναι η σωστή δόμηση του προγράμματός τους.
            Συνήθως ο τρόπος διαβάσματός τους καθοδηγείται από τις ίδιες τις εργασίες που έχουν να κάνουν, μιας και μόνο αυτές έχουν καταληκτικές ημερομηνίες παράδοσης,
                και έτσι παραμελούν τα υπόλοιπα καθήκοντα που έχουν, όπως το να παρακολουθούν τις διαλέξεις.

            Όλα αυτά μας οδηγούν στο συμπέρασμα πως είναι απαραίτητος ένας αποτελεσματικός τρόπος προγραμματισμού και διαχείρισης των εργασιών τους.

        \subsection{Χαρακτηριστικά που οι φοιτητές θα επιθυμούσαν σε μια εφαρμογή}
            Σε έρευνα \cite{Trujillo2020} που πραγματοποιήθηκε στο τμήμα Πληροφορικής του Πανεπιστημίου του Εδιμβούργου,
                διαπιστώθηκε πως η πλειοψηφία της ακαδημαϊκής κοινότητας επιθυμεί μια εφαρμογή διαχείρισης εργασιών να διαθέτει ημερολόγιο
                (θεώρησαν σημαντικό να είναι καταγραμμένες οι ημερομηνίες έναρξης/λήξης για κάθε εργασία για τη σωστή οργάνωση, όπως επίσης και χρωματική ταξινόμηση [color-coding] των εργασιών),
                ειδοποιήσεις / γνωστοποιήσεις για τις εργασίες και to-do λίστες (με προτεραιοποίηση, ομαδοποίηση και δυνατότητα εμφάνισης μπάρας προόδου).
            Επίσης εκφράστηκε ενδιαφέρον για τη δημιουργία ενός συστήματος ανταμοιβής, με σκοπό την ενθάρρυνση των φοιτητών να ολοκληρώνουν εργασίες.