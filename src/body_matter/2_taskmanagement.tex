\chapter{Διαχείριση Εργασιών}

    \section{<Πρόβλημα διαχείρισης εργασιών>}
        Studies of Task Management Towards the Design of a Personal Task List Manager

    \section{<Πανεπιστήμιο>}
    \subsection{Προβλήματα διαχείρισης που αντιμετωπίζουν οι φοιτητές}
        Σε έρευνα \cite{Fukuzawa2015} που διεξήχθει στο Πανεπιστήμιο του Τσουκούμπα της Ιαπωνίας η οποία διερευνούσε τη διαχείριση του προγραμματισμού των εργασιών των φοιτητών,
            παρατηρήθηκε πως η πλειοψηφία των φοιτητών αντιμετωπίζει δυσκολίες στην εκκίνηση μιας νέας εργασίας με βασικούς λόγους:
            α) την έλλειψη χρόνου (26,9\%), β) την αγνόησή της επειδή τη θεωρούσαν ελάσσονος σημασίας (15,7\%), γ) επειδή την ξέχασαν (12,3\%), δ) λόγω κακής συνεργασίας (11,2\%) και ε) επειδή ήταν κουραστική (8,9\%).

        Παρατηρούμε πως οι τρείς πρώτοι λόγοι --που καλύπτουν το μεγαλύτερο ποσοστό (54,9\%) των λόγων-- αφορούν θέματα κακής οργάνωσης από την πλευρά των φοιτητών,
            κάτι που μας οδηγεί στο συμπέρασμα πως είναι απαραίτητος ένας τρόπος προγραμματισμού και διαχείρισης των εργασιών τους.

        Ε ΚΑΤΙ ΑΚΟΜΗ ΘΕΛΕΙ

    \subsection{<Designing A Time Management App For And With Informatics Students>}
