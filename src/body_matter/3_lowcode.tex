\chapter{Low code}
    Στις μέρες μας, όλα τα υπάρχοντα συστήματα σε κάθε τομέα, από τις τραπεζικές υπηρεσίες ως τα αεροναυπηγικά, ελέγχεται από κάποιο λογισμικό.
    Ως αποτέλεσμα τα λογισμικά γίνονται όλο και πιο περίπλοκα, με την ανάπτυξή τους να απαιτεί υψηλό κόστος και χρόνο.

    Από το 2014 και μετά, αρχίζει και εδραιώνεται μια νέα προσέγγιση ανάπτυξης λογισμικού, που έχει ως άμεσο στόχο την αύξηση της παραγωγικότητας των προγραμματιστών.
    Οι πλατφόρμες αυτές ονομάζονται \textbf{Πλατφόρμες Ανάπτυξης Εφαρμογών σε Low-Code} (Low-Code Application Platforms - LCDP)\footnote{Εναλλακτικές ονομασίες είναι Low-Code Platforms (LCP), Low-Code Development Platforms (LCDP), ενώ η διαδικασία ανάπτυξης αναφέρεται ως Low-Code Software Development (LCSD).}. \cite{Bock2021}

    \section{Ορισμός}
        Μια \textbf{Πλατφόρμα Ανάπτυξης Εφαρμογών σε Low-Code} (LCAP) είναι μια πλατφόρμα ανάπτυξης λογισμικού που υποστηρίζει την ταχεία ανάπτυξη και διαχείριση εφαρμογών.
            Συνήθως είναι Platform-as-a-service (PaaS) cloud μοντέλα, και χρησιμοποιείται ελάχιστος ή και μηδενικός δομημένος προγραμματισμός (structured \linebreak programming).

        Για τον προγραμματισμό παρέχεται γραφικό περιβάλλον με οπτικές αφαιρέσεις, επιτρέποντας χρήστες χωρίς προγραμματιστική εμπειρία (citizen developers) να συνεισφέρουν στην ανάπτυξη του λογισμικού χωρίς είναι αναγκαία η βοήθεια των προγραμματιστών.
        Έτσι οι προγραμματιστές εστιάζουν παραπάνω στη σχεδίαση της εφαρμογής, χωρίς να ξοδεύουν χρόνο άσκοπα σε λεπτομέρειες.

        Με λίγα λόγια, σκοπός είναι η παραγωγική ανάπτυξη λογισμικού με τη λιγότερη δυνατή προσπάθεια και με χαμηλότερο κόστος, και η εύκολη προσαρμογή του λογισμικού στις ταχέως μεταβαλλόμενες συνθήκες των σημερινών λειτουργικών συστημάτων.
        Η χρήση των LCAP έχει τύχει θετικής αποδοχής από τη βιομηχανία και η υιοθέτησή τους αυξάνεται συνεχώς. \cite{Bock2021} \cite{Bucaioni2022} \cite{Sahay2020}

        \begin{displayquote} \justifying
            \say{When you can visually create new business applications with minimal \linebreak hand-coding --when your developers can do more of greater value, faster-- \linebreak that’s low-code.} \cite{Ibm_2024}
        \end{displayquote}

    \section{<Ιστορία>}
        γλώσσες βασιζόμενες σε μοντέλα\footnote{Οι αρχιτεκτονικές που βασίζονται σε μοντέλα (model-driven architecture) διαχωρίζουν τη λειτουργικότητα μιας εφαρμογής από την υλοποίησή της σε μια συγκεκριμένη πλατφόρμα, προσφέροντας ένα υψηλότερο επίπεδο αφαίρεσης. Στόχος είναι οι προγραμματιστές να εστιάζουν περισσότερο στον σχεδιασμό και λιγότερο στο να λύνουν θέματα που αφορούν την πλατφόρμα υλοποίησης. \cite{MDAFAQ}},
