\chapter{Πειραματική διαδικασία και αποτελέσματα}
    Με στόχο τη διερεύνηση της ευχρηστίας της εφαρμογής, πραγματοποιήθηκε πειραματική διαδικασία με τη συμμετοχή φοιτητών του Πανεπιστημίου Πατρών. Ο στόχος της μελέτης ήταν να αξιολογηθεί η εμπειρία χρήσης, να εντοπιστούν πιθανά προβλήματα χρηστικότητας και να συλλεχθεί ανατροφοδότηση σχετικά με τις λειτουργίες της εφαρμογής. Μέσω αυτής της διαδικασίας, επιδιώξαμε να κατανοήσουμε τον τρόπο με τον οποίο οι φοιτητές αλληλεπιδρούν με την εφαρμογή και πώς αυτή μπορεί να βελτιώσει τη διαχείριση των καθημερινών τους εργασιών.

    \section{Πειραματική διαδικασία}
        Το πείραμα σχεδιάστηκε ώστε να προσομοιώσει ένα ρεαλιστικό σενάριο χρήσης της εφαρμογής. Οι συμμετέχοντες κλήθηκαν να ολοκληρώσουν συγκεκριμένες εργασίες μέσα στην εφαρμογή, ενώ στο τέλος της διεξαγωγής του πειράματος καταγράφτηκαν οι εντυπώσεις τους.

        Συνολικά συμμετείχαν 12 φοιτητές. Πριν τη διεξαγωγή της πειραματικής διαδικασίας, οι συμμετέχοντες ενημερώθηκαν πλήρως για τους σκοπούς του πειράματος και υπέγραψαν έντυπο συναίνεσης, το οποίο τόνιζε τον εθελοντικό χαρακτήρα της συμμετοχής τους, διασφαλίζοντας ότι οι συμμετέχοντες κατανοούσαν το δικαίωμά τους να αποχωρήσουν από τη μελέτη οποιαδήποτε στιγμή χωρίς συνέπειες. Η φόρμα ενημέρωνε πλήρως τους συμμετέχοντες σχετικά με τον σκοπό, τις διαδικασίες, και τα οφέλη της μελέτης και εξασφαλιζόταν η εμπιστευτικότητα και η προστασία των προσωπικών δεδομένων των συμμετεχόντων.

        Για την προσέλκυση συμμετεχόντων, χρησιμοποιήθηκαν διάφορα μέσα επικοινωνίας, όπως ανακοινώσεις μέσω κοινωνικών δικτύων και πανεπιστημιακά κανάλια επικοινωνίας. Οι ενδιαφερόμενοι συμμετέχοντες έλαβαν email με λεπτομερείς οδηγίες και το έντυπο συναίνεσης όπως επίσης τον σύνδεσμο της εφαρμογής και κωδικούς πρόσβασης.

        \subsection{Οδηγίες}
            Οι οδηγίες που δόθηκαν στους συμμετέχοντες περιελάμβαναν τα εξής:

            \textit{Ο σκοπός αυτής της μελέτης είναι να αξιολογήσουμε την ευχρηστία της εφαρμογής και να εντοπίσουμε πιθανά σημεία που μπορούν να βελτιωθούν. Θα σας ζητηθεί να ολοκληρώσετε μια σειρά από βασικές ενέργειες στην εφαρμογή και στη συνέχεια να συμπληρώσετε ένα ερωτηματολόγιο SUS (System Usability Scale), εκφράζοντας την εμπειρία σας.}

            \textit{Η συμμετοχή σας εκτιμάται να διαρκέσει περίπου 15--20 λεπτά και ευχαριστούμε εκ των προτέρων για τον χρόνο που θα αφιερώσετε για τη συμμετοχή σας. Η συμμετοχή σας είναι εθελοντική και μπορείτε να αποσυρθείτε ανά πάσα στιγμή χωρίς καμία συνέπεια.}

            \textit{Η συμμετοχή σας περιλαμβάνει τα εξής βήματα:}

            \paragraph{\textit{Ανάγνωση και αποδοχή των όρων συμμετοχής.}}
                \textit{Επισυνάπτεται το Έντυπο Συναίνεσης. Παρακαλούμε διαβάστε το προσεκτικά και, εφόσον συμφωνείτε, επιβεβαιώστε τη συμμετοχή σας απαντώντας σε αυτό το email με τη φράση: \say{Αποδέχομαι τη συμμετοχή μου στην έρευνα}.}

            \paragraph{\textit{Πρόσβαση στην εφαρμογή.}}
                \textit{Μπορείτε να αποκτήσετε πρόσβαση στην εφαρμογή μέσω του ακόλουθου συνδέσμου:} \texttt{https://unitask-sandbox.mxapps.io/login.html} \textit{Τα στοιχεία σύνδεσής σας είναι τα εξής:}

            \begin{table}[H] \noindent
                \begin{tabular}{ll}
                    \textit{Όνομα χρήστη:} & \texttt{foithtisX} \\
                    \textit{Κωδικός πρόσβασης}: & \texttt{<password>}
                \end{tabular}
            \end{table}

            \noindent \textit{όπου} \texttt{X} \textit{ένας αριθμός μεταξύ 1 και 12, και} \texttt{<password>} \textit{ένα εξαψήφιο αλφαριθμητικό.}

            \paragraph{\textit{Σενάριο χρήσης.}}
                \textit{Αφού αποκτήσετε πρόσβαση στην εφαρμογή, θα σας ζητηθεί να εκτελέσετε μια σειρά από συγκεκριμένες ενέργειες που έχουν σχεδιαστεί για την αξιολόγηση της ευχρηστίας και της συνολικής εμπειρίας χρήστη.}

            \begin{enumerate}[label=\textit{\arabic*}.]
                \setlength\itemsep{-0.25em}
                \item \textit{Δοκιμάστε να πλοηγηθείτε στο γραφικό περιβάλλον της εφαρμογής: δείτε τις διαφορετικές σελίδες, τη δομή του μενού και τις διαθέσιμες λειτουργίες.}
                \item \textit{Ο στόχος είναι να δημιουργήσετε ένα σύνολο από εργασίες που αντιστοιχούν σε δραστηριότητες που κάνατε ή θα κάνετε στην καθημερινότητά σας. Δημιουργήστε μια ολοκληρωμένη εργασία, μια εργασία σε εξέλιξη και μια επόμενη εργασία. Για την καθεμία θέστε την κατάλληλη προτεραιότητα, χρώμα και δοκιμάστε να αφήσετε μια σύντομη σημείωση.}
                \item \textit{Επεξεργαστείτε τις εργασίες που δημιουργήσατε τροποποιώντας την ημερομηνία λήξης και την κατηγορία τους.}
                \item \textit{Διαγράψτε κάποια από τις εργασίες που δημιουργήσατε.}
                \item \textit{Δοκιμάστε τα παραπάνω και στη σελίδα} {\ZonaSB kanban}.
                \item \textit{Παρατηρήστε τις εργασίες που δημιουργήσατε στη σελίδα} {\ZonaSB ημερολόγιο}.
                \item \textit{Διαγράψτε με μια ενέργεια το σύνολο των εργασιών που δημιουργήσατε.}
            \end{enumerate}

            \textit{Μετά την ολοκλήρωση παρακαλούμε συμπληρώστε το σύντομο ερωτηματολόγιο αξιολόγησης ευχρηστίας.}

        \subsection{Ερωτηματολόγιο}
            Το System Usability Scale (SUS) είναι ένα από τα πιο διαδεδομένα και αξιόπιστα εργαλεία για την αξιολόγηση της χρηστικότητας ενός συστήματος, εφαρμογής ή διεπαφής χρήστη.

            Δημιουργήθηκε από τον John Brooke το 1986 και έχει χρησιμοποιηθεί ευρέως σε έρευνες και βιομηχανικές εφαρμογές για την αξιολόγηση της εμπειρίας χρήστη. Αποτελείται από 10 προτάσεις με τις οποίες οι συμμμετέχοντες καλούνται να τις βαθμολογήσουν σε μια κλίμακα 1 -- 5 με το 1 να αντιστοιχεί σε \say{Διαφωνώ απόλυτα} και το 5 σε \say{Συμφωνώ απόλυτα}. Οι προτάσεις έχουν σχεδιαστεί ώστε να καλύπτουν διάφορες πτυχές της χρηστικότητας, όπως η ευκολία χρήσης, η πολυπλοκότητα, η συνέπεια και η αυτοπεποίθηση των χρηστών κατά τη χρήση του συστήματος \cite{SUS, SUSXenos}.

            Ακολουθεί η λίστα των ερωτήσεων που κλήθηκαν να απαντήσουν οι συμμετέχοντες.

            \begin{table}[H] \noindent\centering\small
                    \begin{tabular}{l|l}
                        \# & \textbf{Ερωτήσεις} \\
                        \midrule
                        01 & Νομίζω ότι θα ήθελα να χρησιμοποιώ αυτή την εφαρμογή συχνά. \\
                        \midrule
                        02 & Βρήκα αυτή την εφαρμογή αδικαιολόγητα περίπλοκη. \\
                        \midrule
                        03 & Σκέφτηκα πως αυτή η εφαρμογή ήταν εύκολη στη χρήση. \\
                        \midrule
                        04 & Νομίζω ότι θα χρειαστώ βοήθεια από κάποιον τεχνικό \\
                          & για να είμαι σε θέση να χρησιμοποίησω αυτή την εφαρμογή. \\
                        \midrule
                        05 & Βρήκα τις διάφορες λειτουργίες σε αυτή την εφαρμογή καλά ολοκληρωμένες. \\
                        \midrule
                        06 & Σκέφτηκα ότι υπήρχε μεγάλη ασυνέπεια σε αυτή την εφαρμογή. \\
                        \midrule
                        07 & Φαντάζομαι ότι οι περισσότεροι άνθρωποι θα μάθουν \\
                          & να χρησιμοποιούν αυτή την εφαρμογή πολύ γρήγορα. \\
                        \midrule
                        08 & Βρήκα αυτή την εφαρμογή πολύ περίπλοκη/δύσκολη στη χρήση. \\
                        \midrule
                        09 & Ένιωσα πολύ σίγουρος/η χρησιμοποιώντας αυτή την εφαρμογή. \\
                        \midrule
                        10 & Χρειάστηκε να μάθω πολλά πράγματα πριν μπορέσω \\
                           & να ξεκινήσω με αυτή την εφαρμογή. \\
                    \end{tabular}
            \end{table}


    \section{Αποτελέσματα}
        Στον πίνακα που ακολουθεί παρουσιάζονται οι μέσες τιμές και οι τυπικές αποκλίσεις των απαντήσεων των συμμετεχόντων στο ερωτηματολόγιο.
        \begin{table}[H] \noindent\centering \small
                \begin{tabular}{c|c|c}
                   \textbf{Ερωτήσεις} & \textbf{Μέση τιμή} & \textbf{Τυπική απόκλιση} \\
                    \midrule
                    Ερώτηση 01  & 4.58 & 0.51 \\
                    Ερώτηση 02  & 1.83 & 0.57 \\
                    Ερώτηση 03  & 4.33 & 0.88 \\
                    Ερώτηση 04  & 1.41 & 0.79 \\
                    Ερώτηση 05  & 4.33 & 0.77 \\
                    Ερώτηση 06  & 1.08 & 0.29 \\
                    Ερώτηση 07  & 4.41 & 0.51 \\
                    Ερώτηση 08  & 1.66 & 0.77 \\
                    Ερώτηση 09  & 4.16 & 0.57 \\
                    Ερώτηση 10 & 1.50 & 0.67 \\
                \end{tabular}
        \end{table}

        Οι ερωτήσεις του SUS διακρίνονται σε θετικές και αρνητικές. Στις θετικές ερωτήσεις, που βρίσκονται σε περιττές θέσεις (1, 3, 5, 7, 9), η βαθμολογία κάθε συμμετέχοντα προκύπτει αφαιρώντας 1 από την αρχική του απάντηση. Για παράδειγμα, αν ένας χρήστης απαντήσει 5, η τιμή μετατρέπεται σε 4, αν απαντήσει 3, μετατρέπεται σε 2 κ.ο.κ. Αντίστοιχα, στις αρνητικές ερωτήσεις, που βρίσκονται σε ζυγές θέσεις (2, 4, 6, 8, 10), το σκορ υπολογίζεται αφαιρώντας την απάντηση από 5. Αυτό σημαίνει πως αν κάποιος απαντήσει 5, η τιμή μετατρέπεται σε 0, αν απαντήσει 3, μετατρέπεται σε 2 κ.ο.κ.

        Μετά την προσαρμογή των απαντήσεων, τα σκορ όλων των ερωτήσεων αθροίζονται και πολλαπλασιάζονται με τον συντελεστή 2.5. Με αυτόν τον τρόπο, το τελικό SUS σκορ κυμαίνεται από 0 έως 100, επιτρέποντας την εύκολη σύγκριση μεταξύ διαφορετικών προϊόντων ή εφαρμογών. Όσο υψηλότερο είναι το σκορ, τόσο καλύτερη θεωρείται η ευχρηστία της υπό εξέταση εφαρμογής.

        Στην παρούσα αξιολόγηση, οι απαντήσεις των χρηστών έδειξαν ότι η εφαρμογή ήταν ιδιαίτερα ευχάριστη στη χρήση, με μέση τιμή 85.82. Αυτή η τιμή υποδηλώνει ότι οι περισσότεροι χρήστες είχαν θετική εμπειρία χρήσης και αξιολόγησαν την εφαρμογή ως εύχρηστη και καλά σχεδιασμένη.