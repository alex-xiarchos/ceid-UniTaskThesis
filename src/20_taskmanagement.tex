\chapter{ΠΡΟΓΡΑΜΜΑΤΙΣΜΟΣ ΕΡΓΑΣΙΩΝ}

    \section{ΠΡΟΒΛΗΜΑΤΑ ΠΡΟΓΡΑΜΜΑΤΙΣΜΟΥ ΠΟΥ ΑΝΤΙΜΕΤΩΠΙΖΟΥΝ ΟΙ ΦΟΙΤΗΤΕΣ}
        Σε έρευνα \cite{Fukuzawa2015} που διεξήχθει στο Πανεπιστήμιο του Τσουκούμπα της Ιαπωνίας η οποία ερευνούσε τη διαχείριση του προγραμματισμού των εργασιών των φοιτητών,
            παρατηρήθηκε πως η πλειοψηφία των φοιτητών αντιμετωπίζει δυσκολίες στην εκκίνηση μιας νέας εργασίας με βασικούς λόγους:
            α) την έλλειψη χρόνου (26,9\%), β) την αγνόησή της επειδή τη θεωρούσαν ελάσσονος σημασίας (15,7\%), γ) την ξέχασαν (12,3\%), δ) κακή συνεργασία (11,2\%) και ε) ήταν κουραστική (8,9\%).

        Παρατηρούμε πως οι τρείς πρώτοι λόγοι --που καλύπτουν το μεγαλύτερο ποσοστό (54,9\%) των λόγων-- αφορούν θέματα κακής οργάνωσης από την πλευρά των φοιτητών,
            κάτι που μας οδηγεί στο συμπέρασμα πως ???????????????