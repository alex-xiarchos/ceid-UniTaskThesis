\documentclass[11pt,a4paper]{template}
\usepackage[main=greek,english]{babel}
\usepackage{float}
\usepackage{epstopdf}
\usepackage{indentfirst}
\usepackage{verbatim}
\usepackage{amsthm}
\usepackage{amssymb}
\usepackage{latexsym}
\usepackage{hyphenat}
\usepackage{makeidx}
\usepackage{algpseudocode}
\usepackage{algorithm}
\usepackage[hyphens]{url}
%\usepackage[hyphenbreaks]{breakurl}
\usepackage{enumitem}
\usepackage{xspace}
\usepackage{booktabs}
\usepackage{multirow}
\usepackage{subfig}
\usepackage{ragged2e} % for \justifying
\usepackage{dirtytalk} % Say
\usepackage{tabularx}
\usepackage{listings}
\usepackage{xcolor}
\usepackage{bbding}
\usepackage{footmisc}
\usepackage{fontspec}
\usepackage{microtype}
\usepackage{subcaption}
\usepackage{xurl}
\usepackage{csquotes}

\emergencystretch=1em
\captionsetup{belowskip=10pt,aboveskip=15pt}
\addto\captionsgreek{%
  \renewcommand{\indexname}{Ευρετήριο όρων}%
}
\makeindex

\usepackage{hyperref}
\hypersetup{
    colorlinks=false,
}

\graphicspath{ {./img/} }

% Γραμματοσειρές
\setmainfont{GFS Didot}
\setmonofont[Scale=0.9]{Courier New}
\newfontfamily\Zona[Scale=0.9]{ZonaPro}
\newfontfamily\ZonaSB[Scale=0.9]{ZonaPro-SemiBold}

% 1.5 spacing
\renewcommand{\baselinestretch}{1.2}

\lstdefinestyle{mystyle}{
    backgroundcolor=\color{gray!10},
    keywordstyle=\bf\ttfamily,
    numberstyle=\tiny\color{darkgray},
    basicstyle=\ttfamily\footnotesize,
    breakatwhitespace=false,
    breaklines=true,
    captionpos=b,
    showstringspaces=false,
    keepspaces=true,
    numbers=left,
    numbersep=5pt,
}
\lstset{style=mystyle}

\newtheorem{proposition}{Πρόταση}
\newtheorem{theorem}{Θεώρημα}
\newtheorem{corollary}{Συμπέρασμα}
\newtheorem{lemma}{Λήμμα}
\newtheorem{example}{Παράδειγμα}
\newtheorem{remark}{Σημείωση}
\newtheorem{notation}{Συμβολισμός}
\newtheorem{law}{Νόμος}
\renewcommand{\thedefinition}{\arabic{chapter}.\arabic{definition}}
\renewcommand{\theproposition}{\arabic{chapter}.\arabic{proposition}}
\renewcommand{\thetheorem}{\arabic{chapter}.\arabic{theorem}}
\renewcommand{\thecorollary}{\arabic{chapter}.\arabic{corollary}}
\renewcommand{\thelemma}{\arabic{chapter}.\arabic{lemma}}
\renewcommand{\theexample}{\arabic{chapter}.\arabic{example}}
\newcommand{\set}[1]{\left\{#1\right\}}
\newcommand{\To}{\Longrightarrow}

\renewcommand\lstlistingname{\tg{Κώδικας}}
\renewcommand\lstlistlistingname{\tg{Παραδείγματα Κώδικα}}
\renewcommand{\listalgorithmname}{Λίστα Αλγορίθμων}

% Εισαγωγή βιβλιογραφίας
\usepackage[backend=biber, style=ieee]{biblatex}
\addbibresource{References.bib}

% Αρίθμηση subsubsections
\setcounter{secnumdepth}{3}
\setcounter{tocdepth}{3}

\emergencystretch=1em


%%%%%%%%%%%%%%%%%%%%%%%%%%%%%%%%%%%%%%%%%%%%%%%%%%%%%
%% THESIS INFO
%%
%
% Τίτλος Πτυχιακής Εργασίας
	\title{Ανάπτυξη εφαρμογής σε Low Code περιβάλλον: \mbox{Σχεδιασμός Εφαρμογής για τον Προγραμματισμό} \\
	και την Παρακολούθηση Εργασιών}
% "του" ή "της", ανάλογα με το φύλο του σπουδαστή
	\edef\toutis{του}
% Ονοματεπώνυμο σπουδαστή (ΚΕΦΑΛΑΙΑ, γενική πτώση)
	\edef\authorNameCapital{ΑΛΕΞΑΝΔΡΟΥ ΞΙΑΡΧΟΥ}
% Ονοματεπώνυμο σπουδαστή (πεζά, ονομαστική πτώση)
	\author{Αλέξανδρος Ξιάρχος}
% Ονοματεπώνυμο Επιβλέποντα Καθηγητή
	\supervisor{Ξένος Μιχάλης}
% Τίτλος Επιβλέποντα Καθηγητή (πχ καθηγητής, λέκτορας κτλ)
	\edef\supervisorTitle{Καθηγητής}
%% "Επιβλέπων" ή "Επιβλέπουσα", ανάλογα με το φύλο του Επιβλέποντα Καθηγητή
	\edef\supervisorMaleFemale{Επιβλέπων}
% Ονοματεπώνυμο Συνεπιβλέποντα Καθηγητή
	\covisor{Μακρής Χρήστος}
% Τίτλος Επιβλέποντα Καθηγητή
	\edef\covisorTitle{Αν. Καθηγητής}
% "Συνεπιβλέπων" ή "Συνεπιβλέπουσα", ανάλογα με το φύλο του Καθηγητή
	\edef\covisorMaleFemale{Συνεπιβλέπων}
% Ονοματεπώνυμο Συνεπιβλέποντα Καθηγητή
	\covisorS{Βογιατζάκη Ελένη}
% Τίτλος Συνεπιβλέποντα Καθηγητή
	\edef\covisorSTitle{ΕΔΙΠ}
% "Συνεπιβλέπων" ή "Συνεπιβλέπουσα", ανάλογα με το φύλο του Καθηγητή
	\edef\covisorSMaleFemale{Συνεπιβλέπουσα}
% Τόπος, μήνας και έτος
	\edef\thesisPlaceDate{Πάτρα, Φεβρουάριος 2025}
% Ημερομηνία Εξέτασης
	\edef\examinationDate{Ημερομηνία Εξέτασης}
% Έτος Copyright
	\edef\copyrightYear{2025}
% Ονοματεπώνυμο 1ου εξεταστή
	\epitropiF{Ονοματεπώνυμο!}
% Τίτλος 1ου εξεταστή
	\edef\epitropiFTitle{Τίτλος}


%%%%%%%%%%%%%%%%%%%%%%%%%%%%%%%%%%%%%%%%%%%%%%%%%%%%%


\begin{document}
\maketitle

%\frontmatter
% Ευχαριστίες
	\begin{acknowledgements}
    Θα ήθελα αρχικά να εκφράσω τις ευχαριστίες μου προς τον επιβλέποντα καθηγητή μου, κ. Μιχάλη Ξένο, για την εμπιστοσύνη που μου έδειξε, αναθέτοντάς μου την εκπόνηση της παρούσας διπλωματικής εργασίας. Η δυνατότητα να ασχοληθώ με το συγκεκριμένο θέμα υπήρξε μια πολύτιμη εμπειρία, και εκτιμώ ιδιαίτερα την καθοδήγηση και τις συμβουλές του στις κρίσιμες φάσεις της εργασίας.

    Επιπλέον, θα ήθελα να ευχαριστήσω θερμά τη Σεμίρα Μαρία Ευαγγέλου για την πολύτιμη βοήθειά της καθ’ όλη τη διάρκεια της εκπόνησης της εργασίας μου. Η διάθεσή της να προσφέρει τη γνώση και την υποστήριξή της υπήρξε ανεκτίμητη.

    Ιδιαίτερα θα ήθελα να ευχαριστήσω τη μητέρα μου, που με την αδιάκοπη στήριξη και κατανόησή της αποτέλεσε το σταθερό θεμέλιο σε κάθε μου βήμα. Η στήριξή της σε κάθε στάδιο αυτής της πορείας ήταν ανεκτίμητη, και χωρίς εκείνη, η ολοκλήρωση των σπουδών μου θα ήταν σίγουρα πιο δύσκολη.

    Τέλος, θα ήθελα να ευχαριστήσω τους φίλους και κοντινούς μου ανθρώπους, που με τη συντροφιά και την κατανόησή τους έκαναν αυτή τη διαδρομή πιο ευχάριστη και λιγότερο απαιτητική.
\end{acknowledgements}


% Περίληψη
	\begin{abstract}
    Η διαχείριση εργασιών αποτελεί σημαντική πρόκληση για τους σύγχρονους φοιτητές, οι οποίοι καλούνται να ισορροπήσουν μεταξύ ακαδημαϊκών, κοινωνικών και επαγγελματικών υποχρεώσεων. Η έλλειψη αποτελεσματικής οργάνωσης μπορεί να οδηγήσει σε άγχος, αναβλητικότητα και μειωμένη απόδοση, επηρεάζοντας τη συνολική φοιτητική εμπειρία. Ωστόσο, πολλοί φοιτητές είτε δεν χρησιμοποιούν οργανωμένες μεθόδους είτε βρίσκουν τις υπάρχουσες αναποτελεσματικές.

    Ο χαμηλός κώδικας (low-code) αποτελεί μια σύγχρονη προσέγγιση ανάπτυξης λογισμικού που επιτρέπει την ταχύτερη υλοποίηση εφαρμογών μέσω γραφικών διεπαφών και προκαθορισμένων λειτουργικών στοιχείων. Οι πλατφόρμες χαμηλού κώδικα μειώνουν την ανάγκη για εκτενή προγραμματισμό, επιτρέποντας στους προγραμματιστές να εστιάσουν περισσότερο στη σχεδίαση και βελτιστοποίηση των λειτουργιών της εφαρμογής.

    Η παρούσα διπλωματική εργασία εστιάζει στην ανάπτυξη μιας εφαρμογής διαχείρισης εργασιών για φοιτητές, αξιοποιώντας την ευελιξία και την ταχύτητα υλοποίησης των πλατφορμών χαμηλού κώδικα. Η προτεινόμενη εφαρμογή στοχεύει στη βελτίωση της παραγωγικότητας των φοιτητών, παρέχοντας ένα εύχρηστο και αποτελεσματικό εργαλείο για την καλύτερη οργάνωση των υποχρεώσεών τους. Η αποδοτικότητα και η χρηστικότητά της θα αξιολογηθούν μέσω δοκιμών με πραγματικούς χρήστες, ώστε να διαπιστωθεί η πραγματική της αξία ως εργαλείο υποστήριξης της ακαδημαϊκής διαχείρισης.

%   \begin{keywords}
%   \end{keywords}
\end{abstract}


\begin{abstracteng}
    Task management is a significant challenge for modern students, who must balance academic, social, and professional obligations. The lack of effective organization can lead to stress, procrastination, and decreased performance, ultimately affecting the overall student experience. However, many students either do not use structured methods or find existing solutions ineffective.

    Low-code development is a modern software approach that enables faster application implementation through graphical interfaces and pre-configured functional components. Low-code platforms reduce the need for extensive coding, allowing developers to focus more on designing and optimizing application functionalities.

    This thesis focuses on developing a task management application for students, leveraging the flexibility and rapid implementation capabilities of low-code platforms. The proposed application aims to enhance student productivity by providing a user-friendly and efficient tool for better task organization. Its effectiveness and usability will be evaluated through real-user testing to assess its actual value as a student task management support tool.

%   \begin{keywordseng}
%   \end{keywordseng}
\end{abstracteng}
% Αφιέρωση
%	\thesisDedication{ }
%	\thesisDedication{\small{στη μητέρα μου}}
% Πίνακας Περιεχομένων
	\tableofcontents
% Κατάλογοι
	\listoffigures
\clearpage


\mainmatter
\cleardoublepage

\pagebreak
	\chapter{Εισαγωγή}

        % Κεντρική ιδέα: Ο άνθρωπος μέσα από το πέρασμα των χρόνων πάντα προσπαθούσε να βελτιώσει τη ζωή του και να κάνει τη δουλειά του αποτελεσματικότερα.

	\section{Σημασία του προβλήματος}

	\section{Στόχοι της εργασίας}

	\section{Μεθοδολογία προσέγγισης}

	\section{Διάρθρωση της διπλωματικής εργασίας}
%            Η εργασία είναι οργανωμένη σε Χ κεφάλαια. Στο Κεφάλαιο 2 γίνεται μια εκτενής αναφορά στη διαδικασία της διαχείρισης έργων (task management), συμπεριλαμβάνοντας τον ορισμό της, μια ιστορική αναδρομή στο πώς μεταβήκαμε από τη συσκευές με πολύχρωμες χάντρες ή χορδές σε εξειδικευμένα προγράμματα με αυτοματισμούς, όπως επίσης και τον ρόλο που αυτή διαδραματίζει στην ακαδημαϊκή κοινότητα.
%            Στο Κεφάλαιο 3 γίνεται ...


%
%The traditional method of app design includes many difficult parts. This includes security, authentication, data loss prevention, and deployment of apps.

%	\chapter{Διαχείριση Εργασιών}
    Καταρχάς είναι σημαντικό να αποσαφηνίσουμε την έννοια του προγραμματισμού και της παρακολούθησης εργασιών, μια διαδικασία που μπορούμε να ονομάσουμε ως διαχείριση εργασιών. Στο κεφάλαιο αυτό θα αναφερθούμε αναλυτικά στη διαδικασία αυτή, θα...
    
    \section{Το πρόβλημα της διαχείρισης εργασιών}

        \subsection{Εργασία}
    
        \subsection{Ορισμός της διαχείρισης εργασιών}
            Ως \textbf{διαχείριση εργασιών} (task management) θα χαρακτηρίζαμε τη διαδικασία οργάνωσης, ιεράρχησης και παρακολούθησης των εργασιών (tasks) καθ' όλη τη διάρκεια μέχρι να πραγματοποιηθούν, με σκοπό τη διασφάλιση της αποδοτικής και αποτελεσματικής εκτέλεσής τους.
            
            Πρόκειται για μια διαδικασία που είναι καθοριστικής σημασίας για τη βελτίωση της παραγωγικότητας, είτε σε ατομικό είτε σε συλλογικό επίπεδο. Μπορεί να γίνει χειροκίνητα, αλλά πλέον συχνά χρησιμοποιούνται ψηφιακά εργαλεία τα οποία αυτοματοποιούν διάφορες πτυχές της.
        
        \subsection{Πεδίο εφαρμογής διαχείρισης εργασιών}
            Η διαδικασία της διαχείρισης εργασιών έχει ένα ευρύ πεδίο εφαρμογής, καθώς εκτίνεται από απλές καθημερινές δραστηριότητες ως τον χειρισμό πολύπλοκων χρονοβόρων εργασιών.

            Η πιο εύληπτη εφαρμογή της αφορά τον καθημερινό προσωπικό προγραμματισμό των υποχρεώσεών μας. Αυτός περιλαμβάνει to-do λίστες, ημερολόγια ή απλές εφαρμογές (όπως Todoist, Notion ή το Google Keep) όπου οργανώνονται οι προσωπικές υποχρεώσεις, προγραμματίζονται ραντεβού ή δραστηριότητες αναψυχής, και εύκολα μπορούν να θέτονται βραχυπρόθεσμοι ή μακροπρόθεσμοι στόχοι και να παρακολουθείται η πρόοδός τους. \cite{Todoist} \cite{Notion}
            % TODO: Βελτιώνεται η ζωή των ατόμων καθώς πετυχαίνουν τις εργασίες τους και μειώνεται το άγχος τους

            Πέρα από το προσωπικό επίπεδο, μέθοδοι διαχείρισης έργων χρησιμοποιούνται ευρέως και σε επαγγελματικά περιβάλλοντα. Στόχος τους είναι η αποτελεσματική κατανομή του φόρτου εργασίας των εργαζομένων και η συντονισμένη συνεργασία τους. Μέσω της διαχείρισης έργων είναι ικανή η διευθέτηση παράλληλων εργασιών ταυτόχρονα, η ιεράρχησή τους σε επείγουσες και μη, η δίκαιη ανάθεση των εργασιών στους εργαζομένους κ.α. Ένα δημοφιλές εργαλείο που χρησιμοποιείται σε αυτά τα ομαδικά περιβάλλοντα είναι το Trello. \cite{Trello}

            Τέλος, με την πρόοδο της τεχνολογίας, έχουν δημιουργηθεί προηγμένες πλατφόρμες στις οποίες μπορεί να γίνει βελτιστοποίηση της ιεράρχησης εργασιών, χρησιμοποιώντας τεχνητή νοημοσύνη και ανάλυση δεδομένων. 

        \subsection{Διαφορά με διαχείριση έργου}
            Συχνά η έννοια της διαχείρισης εργασιών (task management) συγχέεται με αυτή της διαχείρισης έργου (project management). Η αλήθεια είναι πως πρόκειται για έννοιες που όντως συσχετίζονται αλλά στην πραγματικότητα η καθεμία εστιάζει σε διαφορετικά αντικείμενα.

            Η \textbf{διαχείριση εργασιών}, όπως έχει αναφέρθηκε, αφορά την παρακολούθηση διαφορετικών, μεμονομένων δραστηριοτήτων οι οποίες χρειάζεται να ολοκληρωθούν. Με άλλα λόγια εστιάζει περισσότερο στο \textit{μικροεπίπεδο}, στη διαχείριση καθημερινών υποχρεώσεων, στα διαφορετικά deadlines που μπορεί να υπάρχουν, την εξέλιξή τους ανά το χρόνο κ.α. Τα εργαλεία που αφορούν τη διαχείριση εργασιών περιλαμβάνουν ημερολόγια, υπενθυμίσεις ή χρονοδιαγράμματα.

            Αντίθετα, η \textbf{διαχείριση έργου} περιγράφει τον σχεδιασμό, την εκτέλεση και την ολοκλήρωση ενός ολόκληρου έργου. Ένα έργο αποτελείται και αυτό από διαφορετικές εργασίες, οι οποίες όμως είναι οργανωμένες προς έναν ευρύτερο στόχο. Επομένως, η έννοια της διαχείρισης έργου \textit{συμπεριλαμβάνει} τη διαχείριση εργασιών, αλλά επίσης προϋποθέτει επιπλέον απαιτήσεις όπως τη σωστή κατανομή πόρων (resource allocation) ή την αξιολόγηση κινδύνου (risk assessment). Τα λογισμικά διαχείρισης έργου έχουν λειτουργικότητες όπως διαγράμματα Γκαντ, παρακολούθηση εξαρτήσεων κ.α.

            Στην παρούσα διπλωματική εργασία για λόγους πληρότητας θα αναλύσουμε κάποιες έννοιες που αφορούν και τη διαχείριση έργου, έχοντας όμως υπόψην ότι η υλοποίηση αφορά τη διαχείριση εργασιών.

    \section{Ιστορική αναδρομή}
        Είναι προφανές ότι η διαχείριση έργων δεν περιλάμβανε πάντα ψηφιακά εργαλεία και αυτοματισμούς. Πάντα όμως ήταν πρόδηλη σε όλα τα μεγάλα έργα της ιστορίας. Άλλωστε πώς αλλιώς θα μπορούσαν να ολοκληρωθούν τεράστια κατασκευάσματα όπως οι πυραμίδες της Αιγύπτου, το Στόουνχεντζ ή το Σινικό τείχος;
        
        Η έννοια της διαχείρισης έργων έδωσε τη δυνατότητα στους ηγέτες να σχεδιάζουν τολμηρά και ογκώδη έργα και να κατανέμουν σωστά τα διαθέσιμα υλικά και ανθρώπους σε ένα καθορισμένο χρονικό διάστημα.
        
        \subsection{Προφορικότητα και μνημονικές συσκευές}
            Στην αρχαιότητα η διαχείριση των εργασιών γινόταν προφορικά. Οι εργασίες αναθέτονταν και διαχειρίζονταν μέσω προφορικών οδηγιών, κάτι που σήμαινε πως μεγάλο ρόλο έπαιζε η ανθρώπινη μνήμη. \cite{Goody2013}

            Πολιτισμοί όπως οι Λούμπα του Κονγκό χρησιμοποιούσαν χειροκίνητες συσκευές όπως τη \textit{λουκάσα} (lukasa) που περιλάμβαναν πολύχρωμες χάντρες. Οι χάντρες ήταν τοποθετημένες σε συγκεκριμένες θέσεις, οι οποίες βοηθούσαν τη μνήμη την χειριστών ώστε να θυμηθούν την εκάστοτε πληροφορία και να οργανωθούν. \cite{Lukasa}

            \begin{figure}[H] \noindent \centering
                \includegraphics[width=0.6\textwidth]{img/Lukasa.jpg}
                \caption{Η συσκευή λουκάσα}
            \end{figure}
            
            Άλλοι πολιτισμοί όπως οι Ίνκα χρησιμοποιούσαν συσκευές όπως το \textit{κουίπου} \linebreak (quipu), μια κατασκευή με χορδές από βαμβάκι με κατηγοροποιημένες πληροφορίες βάσει χρώματος, διάταξης και αριθμού. Οι Ίνκα δημιουργούσαν κόμπους στις χορδές και τις χρησιμοποιούσαν για τη συλλογή και παρακολούθηση των υποχρεώσεών τους ή και για την αποθήκευση άλλων πληροφοριών όπως δεδομένα απογραφής, φορολογικών υποχρεώσεων και άλλα. \cite{Quipu}

            \begin{figure}[H] \noindent \centering
                \includegraphics[width=0.5\textwidth]{img/Quipo.jpg}
                \caption{Η συσκευή κουίπου}
            \end{figure}

        \subsection{Πρώτα ημερολόγια}
            Κατασκευές όπως τα ηλιακά ρολόγια επέτρεψαν στους πληθυσμούς να διαιρέσουν την ημέρα σε τμήματα, που οδήγησε στον διαχωρισμό μεταξύ υποχρεώσεων και ελεύθερου χρόνου, ενώ με κάποιες πρώιμες προσπάθειες δημιουργίας ημερολογίων (κυρίως από Ρωμαίους, Αιγυπτίους και Μάγια) έγινε εφικτός ο διαχωρισμός του χρόνου σε διαφορετικές περιόδους για την γεωργία, τις θρησκευτικές τελετές και τις υπόλοιπες τελετουργίες, οδηγώντας έτσι σε μια μορφή διαχείρισης έργων και χρόνου. \cite{Richards_2000}

        \subsection{Σύγχρονη εποχή}
            Οι άνθρωποι διαχειρίζονται την εργασία τους εδώ και εκατοντάδες χρόνια, και μπορούμε να εντοπίσουμε τις πρώτες απόπειρες τυποποίησης αυτής της διαχείρισης από τον 18ο αιώνα. Στα τέλη του 19ου αιώνα, ανερχόμενα έργα μεγάλης κλίμακας επέφεραν την ανάγκη για μια πιο λεπτομερή διαχείριση των εργασιών, άμεση απόρροια των αυξανόμενων απαιτήσεων που θα προέκυπταν. Όμως η οργάνωση χιλιάδων εργατών και η επεξεργασία τεράστιων ποσοτήτων από πρώτες ύλες δεν θα μπορούσε να πραγματοποιηθεί επιτυχώς με τις υπάρχουσες τεχνικές.
           
            Μηχανικοί όπως ο Henry Gantt για να αντιμετωπίσουν αυτές τις ανάγκες εισήγαγαν νέες μεθόδους οργάνωσης όπως το \textbf{διάγραμμα Γκαντ}. Πρόκειται για ένα διάγραμμα αναπαράστασης όλων των επιμέρους εργασιών ενός έργου ανά το χρόνο και παρέχει μια σαφή εικόνα όλων των  διαφορετικών φάσεων, επιτρέποντας τον προσδιορισμό της κρίσιμης διαδρομής του. Επομένως είναι μια εξαιρετικά χρήσιμη μέθοδος για τον σωστό χρονοπρογραμματισμό σε τέτοια μεγέθη μεγάλης κλίμακας. Το διάγραμμα Γκαντ χρησιμοποιήθηκαν κατά κόρον σε μεγάλα έργα υποδομών όπως η Διώρυγα του Παναμά ή το φράγμα Χούβερ. \cite{strefapmiHooverGreatest}
            
            Αυτά τα νέα μεθοδολογικά εργαλεία ενέπνευσαν νέες έρευνες όπως το Πρότζεκτ Μανχάταν (Manhattan Project), τις προσπάθειες των Αμερικανών για το σχεδιασμό πυρηνικών όπλων. Για την επίτευξη αυτού του σκοπού αναπτυχτήκαν δύο νέα μοντέλα, το PERT (Program Evaluation and Review Technique) και το CPM (Critical Path Method). \cite{SaylorAcademyProjectManagement}
            
    \section{Η συνδρομή της τεχνολογίας}
        Από την δεκαετία του '60 και μετά, οι επιχειρήσεις άρχισαν να αντιλαμβάνουν τα οφέλη από τη μεθοδική οργάνωση της εργασίας. Η ψηφιακή επανάσταση που ακολούθησε δεν θα μπορούσε παρά να γιγαντώσει αυτή τη νέα πραγματικότητα.
    
        \subsection{Ψηφιακά εργαλεία}
            Με την εξέλιξη της τεχνολογίας, οι σημειώσεις και η οργάνωση μεταφέρθηκε από τις χειρόγραφες σημειώσεις, τα ημερολόγια και τα έγγραφα των γραφομηχανών σε ψηφιακά εργαλεία όπως υπολογιστικά φύλλα και προγράμματα σαν το Microsoft Project, σχεδιασμένα αποκλειστικά με σκοπό την αποτελεσματική διαχείριση έργων. 

            \subsubsection{Microsoft Project}
                \begin{figure}[H] \noindent \centering
                    \includegraphics[width=0.7\textwidth]{img/MicrosoftProject3.png}
                    \caption{\centering Στιγμιότυπο από το Microsoft Project 3.0 (σε DOS) \cite{WinWorld}}
                \end{figure}
                
                \begin{figure}[H] \noindent \centering
                    \includegraphics[width=0.7\textwidth]{img/MicrosoftProject2000.png}
                    \caption{\centering Στιγμιότυπο από το Microsoft Project 2000 \cite{WinWorld}}
                \end{figure}
                
                Πρόκειται για ένα από τα πρώτα λογισμικά διαχείρισης έργων, σχεδιασμένα για το κοινό. Η ιδέα για την δημιουργία του προήλθε από μια φάρσα του Ron Bredehoeft, ο οποίος ήθελε να εκφράσει την συνταγή για τα αυγά μπένεντικτ σε όρους διαχείρισης έργων. Παρουσιάστηκε για πρώτη φορά το 1984 ως μια DOS εφαρμογή και πλέον έχει γίνει ένα καθιερωμένο εργαλείο σε όλες τις βιομηχανίες για την οργάνωση, τον προγραμματισμό και την παρακολούθηση της προόδου των έργων.
                
                Η κεντρική οθόνη του Microsoft Project χωρίζεται σε δύο περιοχές: το διάγραμμα Γκαντ και τον πίνακα που εισάγονται οι εργασίες (input table). Υπάρχει η δυνατότητα ιεράρχησης των εργασιών με την τοποθέτηση εσοχών (indents), η δημιουργία εξαρτήσεων με το καθορισμό προκατόχων (predecessors -- μια εργασία μπορεί να ξεκινήσει να εκτελείται μόνο όταν τελειώσει ο προκάτοχός της) και η δυνατότητα αυτόματου προγραμματισμού των εργασιών, λαμβάνοντας υπόψιν τις εξαρτήσεις τους. Επιπλέον μπορούν να δημιουργηθούν αλυσιδωτές εργασίες (η μια εργασία εκτελείται μετά την άλλη), ενώ επίσης μπορούν να ανατεθούν πόροι (resources) για κάθε εργασία. Κάθε χαρακτηριστικό μπορεί να τροποποιηθεί δυναμικά, είναι εφικτή η αποτύπωση των εργασιών πέρα από το διάγραμμα Γκαντ και σε μορφή ημερολογίου, φύλλου εργασίας (task sheet) κ.α., όπως επίσης και η δημιουργία στατιστικών.

                    
    \section{Μεθοδολογίες}
       
        \subsection{Διάγραμμα Γκαντ}
            Το \textbf{διάγραμμα Γκαντ} (Gantt chart) είναι μια δισδιάστατη γραφική απεικόνιση ενός έργου, με τον οριζόντιο άξονα να αποτελεί τον χρόνο (συχνά χωρισμένο σε διαστήματα ημερών, μηνών, χρόνων) και τον κατακόρυφο άξονα να αποτελεί τις διαφορετικές εργασίες που απαρτίζουν το έργο.

            Πρόκειται για ένα πολύ σημαντικό εργαλείο καθώς δείχνει οπτικά τον χρόνο που εκτιμάται ότι θα χρειαστεί κάθε τμήμα ενός έργου, επομένως μπορεί εύκολα να χρησιμοποιηθεί για την παρακολούθηση της προόδου όλων των επιμέρους εργασιών. Έτσι, αν κάποια ξεφεύγει από το ορισμένο χρονοδιάγραμμα, μπορούν άμεσα να γίνουν οι απαραίτητες ενέργειες που χρειάζονται. \cite{Xenos}

            Για τον σχεδιασμό ενός διαγράμματος Γκαντ είναι απαραίτητος ο αρχικός διαχωρισμός των επιμέρους εργασιών, όπως επίσης και μια εκτίμηση της χρονικής διάρκειάς τους. Στην συνέχεια οι εργασίες τοποθετούνται συνήθως με σειρά ώστε αυτές που τελειώνουν νωρίτερα να βρίσκονται ψηλότερα.
            
            Γενικά είναι μια εύκολη και γρήγορη κατασκευή που απεικονίζει με σαφήνεια τη χρονική διάρκεια και την αλληλουχία των εργασιών, αλλά από την άλλη δεν μπορεί να απεικονίσει τις εξαρτήσεις μεταξύ των επιμέρους εργασιών. Έτσι δεν είναι εμφανές ποιες εργασίες πρέπει να ολοκληρωθούν πρώτα ώστε να είναι εφικτή η εκτέλεση μιας επόμενης εργασίας, και επίσης δεν αναπαρίσταται η επίδραση μιας καθυστέρησης σε κάποια φάση του έργου. Τέλος, λόγω της στατικής του δομής, δεν δύναται να αναπροσαρμοστεί σε μεταβολές στη χρονική διάρκεια εκτέλεσης κάποιας εργασίας.

        \subsection{Program evaluation and review technique (PERT)}


        \subsection{Μέθοδος κρίσιμης διαδρομής (Critical Path Method -- CPM)}


        \subsection{Agile και Kanban}


    \section{Διαχείριση έργων στο πανεπιστήμιο}
        \subsection{Προβλήματα διαχείρισης που αντιμετωπίζουν οι φοιτητές}
            % Εισαγωγική παράγραφος
            
            Σε έρευνα \cite{Fukuzawa2015} που διεξήχθη στο Πανεπιστήμιο του Τσουκούμπα της Ιαπωνίας, η οποία διερευνούσε τη διαχείριση του προγραμματισμού των εργασιών από τη πλευρά των φοιτητών, παρατηρήθηκε πως η πλειοψηφία τους αντιμετωπίζει δυσκολίες στην εκκίνηση μιας νέας εργασίας με βασικούς λόγους: α) την έλλειψη χρόνου (26,9\%), β) την αγνόησή της επειδή τη θεωρούσαν ελάσσονος σημασίας (15,7\%), γ) επειδή την ξέχασαν (12,3\%), δ) λόγω κακής συνεργασίας (11,2\%) και ε) επειδή ήταν κουραστική (8,9\%). Παρατηρούμε πως οι τρεις πρώτοι λόγοι --που καλύπτουν το μεγαλύτερο ποσοστό (54,9\%) των λόγων-- αφορούν θέματα κακής οργάνωσης από την πλευρά των φοιτητών.
           
            Σε διαφορετική έρευνα \cite{Trujillo2020}, πάλι παρουσιάζεται πως το κυριότερο πρόβλημα που αντιμετωπίζουν οι φοιτητές είναι η σωστή δόμηση του προγράμματός τους. Συνήθως ο τρόπος διαβάσματός τους καθοδηγείται από τις ίδιες τις εργασίες που έχουν να κάνουν, μιας και μόνο αυτές έχουν καταληκτικές ημερομηνίες παράδοσης, και έτσι παραμελούν τα υπόλοιπα καθήκοντα που έχουν, όπως το να παρακολουθούν τις διαλέξεις.

            Όλα αυτά μας οδηγούν στο συμπέρασμα πως είναι απαραίτητος ένας αποτελεσματικός τρόπος προγραμματισμού και διαχείρισης των εργασιών τους.

        \subsection{Χαρακτηριστικά που οι φοιτητές θα επιθυμούσαν σε μια εφαρμογή}
            Σε έρευνα \cite{Trujillo2020} που πραγματοποιήθηκε στο τμήμα Πληροφορικής του Πανεπιστημίου του Εδιμβούργου, διαπιστώθηκε πως η πλειοψηφία της ακαδημαϊκής κοινότητας επιθυμεί μια εφαρμογή διαχείρισης εργασιών να διαθέτει ημερολόγιο (θεώρησαν σημαντικό να είναι καταγραμμένες οι ημερομηνίες έναρξης/λήξης για κάθε εργασία για τη σωστή οργάνωση, όπως επίσης και χρωματική ταξινόμηση (color-coding) των εργασιών), ειδοποιήσεις / γνωστοποιήσεις για τις εργασίες και to-do λίστες (με ιεράρχιση, ομαδοποίηση και δυνατότητα εμφάνισης μπάρας προόδου). Επίσης εκφράστηκε ενδιαφέρον για τη δημιουργία ενός συστήματος ανταμοιβής, με σκοπό την ενθάρρυνση των φοιτητών να ολοκληρώνουν εργασίες.
%	\chapter{Ανάλυση προτιμήσεων φοιτητών}
    \section{Υπάρχουσες έρευνες και σχετική βιβλιογραφία}
        Σε πολλές περιπτώσεις, η αποτελεσματικότητα των φοιτητών καθορίζεται κυρίως από την οργάνωσή τους και τον σωστό προγραμματισμό τους, τόσο στις ακαδημαϊκές όσο και στις προσωπικές τους υποχρεώσεις. Η σωστή διαχείριση χρόνου και προτεραιοτήτων είναι καθοριστική για την αποφυγή άγχους, την αύξηση της παραγωγικότητας και την επίτευξη των στόχων τους. Παρόλα αυτά, οι φοιτητές συχνά έρχονται αντιμέτωποι με προκλήσεις, όπως η αναβλητικότητα και η έλλειψη σωστής οργάνωσης για την ολοκλήρωση των καθημερινών τους καθηκόντων. Οι ερευνητικές προσπάθειες στον τομέα αυτό αναδεικνύουν τα εμπόδια που αντιμετωπίζουν οι φοιτητές και προσφέρουν πολύτιμες πληροφορίες για τη βελτίωση των δεξιοτήτων διαχείρισης εργασιών.

        \subsection{Προβλήματα διαχείρισης που αντιμετωπίζουν οι φοιτητές} \label{sec:student_problems}
            Σε έρευνα \cite{Fukuzawa2015} που διεξήχθη στο Πανεπιστήμιο του Τσουκούμπα της Ιαπωνίας, η οποία διερευνούσε τη διαχείριση του προγραμματισμού των εργασιών από την πλευρά των φοιτητών, παρατηρήθηκε πως η πλειοψηφία τους αντιμετωπίζει δυσκολίες στην εκκίνηση μιας νέας εργασίας. Οι βασικοί λόγοι που καταγράφηκαν περιλαμβάνουν: α) την έλλειψη χρόνου (26,9\%), β) την αγνόηση της εργασίας επειδή τη θεωρούσαν ελάσσονος σημασίας (15,7\%), γ) επειδή τη ξέχασαν (12,3\%), δ) λόγω κακής συνεργασίας (11,2\%) και ε) επειδή ήταν κουραστική (8,9\%). Οι τρεις πρώτοι λόγοι, που καλύπτουν το μεγαλύτερο ποσοστό (54,9\%), αφορούν σε θέματα κακής οργάνωσης από την πλευρά των φοιτητών, υποδεικνύοντας την ανάγκη για αποτελεσματικότερα εργαλεία χρονοπρογραμματισμού.

            Παράλληλα, μια διαφορετική έρευνα \cite{Trujillo2020} καταδεικνύει πάλι πως το κυριότερο πρόβλημα που αντιμετωπίζουν οι φοιτητές είναι η σωστή δόμηση του προγράμματός τους. Παρατηρήθηκε ότι ο τρόπος με τον οποίο οργανώνουν το διάβασμά τους καθοδηγείται κυρίως από τις καταληκτικές ημερομηνίες των εργασιών τους, με αποτέλεσμα να παραμελούν άλλες σημαντικές ακαδημαϊκές δραστηριότητες, όπως η παρακολούθηση διαλέξεων. Αυτό οδηγεί σε ανισορροπία μεταξύ των ακαδημαϊκών τους υποχρεώσεων, επηρεάζοντας την απόδοσή τους συνολικά.

            Οι παραπάνω έρευνες οδήγησαν σε συγκεκριμένα συμπεράσματα. Πρώτον, η έλλειψη οργανωτικών δεξιοτήτων παραμένει ένας από τους κύριους παράγοντες που εμποδίζουν την αποτελεσματική διαχείριση των εργασιών από τους φοιτητές. Οι λόγοι αυτοί συνδέονται συχνά με την αναβλητικότητα και την έλλειψη εργαλείων που θα μπορούσαν να βοηθήσουν στην αποδοτικότερη οργάνωση των καθημερινών τους υποχρεώσεων. Δεύτερον, η ανάγκη για ένα δομημένο σύστημα προγραμματισμού είναι εμφανής, καθώς θα μπορούσε να παρέχει σαφείς προτεραιότητες και να συμβάλει στη μείωση του άγχους που προκαλείται από τις καταληκτικές ημερομηνίες. Συνεπώς, ένα αποτελεσματικό σύστημα διαχείρισης και προγραμματισμού των εργασιών, προσαρμοσμένο στις ανάγκες των φοιτητών, μπορεί να λειτουργήσει ως βασικό εργαλείο για την ενίσχυση της παραγωγικότητας και της επιτυχίας τους.

        \subsection{Χαρακτηριστικά που οι φοιτητές θα επιθυμούσαν σε μια εφαρμογή} \label{sec:student_preferences}
            Σε έρευνα \cite{Trujillo2020} που πραγματοποιήθηκε στο τμήμα Πληροφορικής του Πανεπιστημίου του Εδιμβούργου, αναδείχθηκαν κάποιες σημαντικές προτιμήσεις και απαιτήσεις της ακαδημαϊκής κοινότητας σχετικά με τη λειτουργικότητα των εφαρμογών διαχείρισης εργασιών. Η πλειοψηφία των συμμετεχόντων τόνισε τη σημασία της ύπαρξης ενός ενσωματωμένου ημερολογίου, το οποίο θα παρέχει τη δυνατότητα καταγραφής των ημερομηνιών έναρξης και λήξης για κάθε εργασία. Αυτή η λειτουργία θεωρήθηκε κρίσιμη για τη σωστή οργάνωση και τον προγραμματισμό των υποχρεώσεων, καθώς επιτρέπει στους χρήστες να έχουν σαφή εικόνα των προθεσμιών τους. Παράλληλα, υπογραμμίστηκε η αξία της χρωματικής ταξινόμησης (color-coding), η οποία διευκολύνει τη διάκριση μεταξύ διαφορετικών κατηγοριών ή τύπων εργασιών, ενισχύοντας τη διαφάνεια και την ευκολία χρήσης της εφαρμογής.

            Επιπλέον, οι συμμετέχοντες επισήμαναν την ανάγκη για ειδοποιήσεις/γνωστοποιήσεις, οι οποίες θα λειτουργούν ως υπενθυμίσεις για τις επερχόμενες προθεσμίες ή τις εκκρεμότητες που απαιτούν άμεση προσοχή. Εξίσου σημαντική θεωρήθηκε η ύπαρξη to-do λιστών, οι οποίες θα διαθέτουν λειτουργίες όπως ιεράρχηση των εργασιών, ομαδοποίηση σε κατηγορίες, και δυνατότητα εμφάνισης μπάρας προόδου. Αυτές οι δυνατότητες συμβάλλουν στην αποτελεσματικότερη παρακολούθηση της προόδου των εργασιών και στη δημιουργία μιας αίσθησης ολοκλήρωσης και επίτευξης στόχων.

            Ιδιαίτερο ενδιαφέρον παρουσίασε η πρόταση για την ενσωμάτωση ενός συστήματος ανταμοιβής, το οποίο στοχεύει στην ενθάρρυνση των φοιτητών να ολοκληρώνουν τις εργασίες τους εγκαίρως. Ένα τέτοιο σύστημα θα μπορούσε να περιλαμβάνει την εμφάνιση γραφικών στοιχείων όπως κομφετί ή εικονικά μετάλλια, ή την ανταμοιβή πόντων, συμβάλλοντας στη δημιουργία κινήτρων. Η εισαγωγή αυτού του συστήματος έχει σκοπό να ενισχύσει τη δέσμευση και την παραγωγικότητα των χρηστών, προσφέροντας έναν πιο διαδραστικό και ελκυστικό τρόπο διαχείρισης εργασιών. Με την ενσωμάτωση χαρακτηριστικών όπως τα προαναφερθέντα, τέτοιες εφαρμογές μπορούν να βελτιώσουν ουσιαστικά την παραγωγικότητα και την αποτελεσματικότητα, προσφέροντας παράλληλα μια ευχάριστη εμπειρία χρήσης.

    \section{Πρωτογενής έρευνα}
        Πέρα από τις υπάρχουσες έρευνες, στο πλαίσιο της διπλωματικής εργασίας πραγματοποιήθηκε νέα έρευνα με στόχο την κατανόηση των δυσκολιών που αντιμετωπίζουν οι φοιτητές στο καθημερινό προγραμματισμό των εργασιών τους, καθώς και την καταγραφή των χαρακτηριστικών που θεωρούν πιο χρήσιμα σε μια εφαρμογή διαχείρισης εργασιών. Η συλλογή αυτών των δεδομένων αποτέλεσε κρίσιμο βήμα για τη διαμόρφωση των απαιτήσεων της υπό ανάπτυξη εφαρμογής, διασφαλίζοντας ότι το τελικό προϊόν θα είναι όσο το δυνατόν πιο λειτουργικό και προσαρμοσμένο στις ανάγκες των φοιτητών.

        \subsection{Μεθοδολογία και δείγμα}
            Για τη συλλογή των δεδομένων, συγκεντρώθηκαν ποσοτικά στοιχεία μέσω ενός ερωτηματολογίου που δημιουργήθηκε στο Google Forms. Το ερωτηματολόγιο επικεντρωνόταν στους δύο βασικούς άξονες που αναλύθηκαν στις ενότητες \ref{sec:student_problems} και \ref{sec:student_preferences}.

            Η έρευνα πραγματοποιήθηκε σε ένα δείγμα 14 φοιτητών, γεγονός που διασφαλίζει τη συνάφεια των απαντήσεων τους με το αντικείμενο της έρευνας. Η συμμετοχή ήταν εθελοντική και οι φοιτητές απάντησαν στο ερωτηματολόγιο ανώνυμα. Η συλλογή των δεδομένων διήρκεσε μια εβδομάδα και ο σύνδεσμος για την πρόσβαση στο ερωτηματολόγιο διανεμήθηκε μέσω κοινωνικών δικτύων.

        \subsection{Ερωτηματολόγιο}
            Τα ερωτήματα με τις πιθανές απαντήσεις που περιελάμβανε το ερωτηματολόγιο παρουσιάζονται παρακάτω:

            \begin{table}[H] \noindent\centering
                \resizebox{0.7\textwidth}{!}{
                    \begin{tabular}{l|l}
                       \textbf{Ερωτήσεις} & \textbf{Πιθανές απαντήσεις}  \\
                        \midrule
                        Τι ηλικία έχεις; & 18-19 \\& 20-21 \\& 22-24 \\& 25+ \\
                        \midrule
                        Ποιο είναι το ακαδημαϊκό σου επίπεδο; & Προπτυχιακό \\& Μεταπτυχιακό \\& Διδακτορικό \\
                        \midrule
                        Τι σπουδάζεις; & (κείμενο σύντομης απάντησης) \\
                        \midrule
                        Σε ποιο έτος σπουδών είσαι; & 1ο, 2ο, 3ο, 4ο, 5ο, Επί πτυχίο \\
                    \end{tabular}}
            \end{table}

            \begin{table}[H] \noindent\centering
                \resizebox{\textwidth}{!}{
                    \begin{tabular}{l|l}
                       \textbf{Ερωτήσεις} & \textbf{Πιθανές απαντήσεις}  \\
                        \midrule
                        Χρησιμοποιείς κάποιο εργαλείο για τη διαχείριση & Ναι \\
                        του χρόνου σου και των υποχρεώσεών σου; & Όχι, προγραμματίζω στο μυαλό μου \\
                        \midrule
                        Αν ναι, ποιο/α εργαλείο/α χρησιμοποιείς; & Εφαρμογές (π.χ., Ημερολόγιο, Notion, Todoist) \\& Υπενθυμίσεις στο κινητό ή υπολογιστή \\& Ημερολόγιο σε φυσική μορφή \\& Post-it ή άλλες χειρόγραφες σημειώσεις \\& Excel ή άλλα υπολογιστικά φύλλα \\& Άλλο \\
                        \midrule
                        Αν ναι, οι εργασίες που προγραμματίζεις & Υποχρεώσεις σχολής (π.χ., διάβασμα, εργασίες) \\
                        και οργανώνεις αφορούν περισσότερο... & Υποχρεώσεις καθημερινότητας (π.χ., ψώνια) \\
                        & Κοινωνικές δραστηριότητες (π.χ. χόμπι) \\& Προσωπική ανάπτυξη (π.χ., μαθήματα γλωσσών) \\& Προσωπική φροντίδα (π.χ., ραντεβού με γιατρούς) \\& Άλλο \\
                        \midrule
                        Αν ναι, σε κλίμακα από 1 έως 5, πόσο αποτελεσματικές & 1 -- 5 \\
                        θεωρείς τις μεθόδους διαχείρισης εργασιών που χρησιμοποιείς; & \\
                        \midrule
                        Τι δυσκολίες αντιμετωπίζεις & Αναβλητικότητα \\
                        στην διαχείριση των υποχρεώσεών σου; & Δυσκολία στην ιεράρχιση των εργασιών
                       \\& Πάρα πολλές υποχρεώσεις \\& Έλλειψη κινήτρων \\& Αδυναμία τήρησης προγράμματος \\& Απόσπαση προσοχής από social media \\& Έλλειψη χρόνου για σωστό προγραμματισμό \\& Αγωνία ή άγχος σχετικά με τις προθεσμίες \\& Κακή εκτίμηση του απαιτούμενου χρόνου \\
                        \midrule
                        Πόσο συχνά έχει τύχει να χάσεις κάποια προθεσμία (deadline); & Ποτέ, Σπάνια, Μερικές φορές, Συχνά, Πάντα \\
                        \midrule
                        Έχει τύχει να μην έχεις ξεκινήσει καν κάποια εργασία σου; & Ναι, Όχι \\
                        \midrule
                        Αν ναι, ποιοι ήταν οι λόγοι; & Έλλειψη χρόνου \\& Την ξέχασα \\& Κακή συνεργασία ομάδας \\& Ήταν κουραστική \\& Άλλο \\
                        \midrule
                        Πόσο συχνά αισθάνεσαι αγχωμένος/η από τον φόρτο των εργασιών σου; & Ποτέ, Σπάνια, Μερικές φορές, Συχνά, Πάντα \\
                        \midrule
                        Τι θεωρείς χρήσιμο σε μια εφαρμογή διαχείρισης εργασιών για φοιτητές; \\
                        \textit{Ύπαρξη ημερολογίου} & \textit{1 (καθόλου χρήσιμο) -- 5 (πολύ χρήσιμο)} \\
                        \textit{Ιεράρχιση εργασιών (χαμηλή, υψηλή προτεραιότητα)} & \textit{1 (καθόλου χρήσιμο) -- 5 (πολύ χρήσιμο)} \\
                        \textit{Κατηγοροποίηση εργασιών βάσει του χρώματός τους (color-coding)} & \textit{1 (καθόλου χρήσιμο) -- 5 (πολύ χρήσιμο)} \\
                        \textit{Ύπαρξη επαναλαμβανόμενων (recurring) εργασίες} & \textit{1 (καθόλου χρήσιμο) -- 5 (πολύ χρήσιμο)} \\
                        \textit{Σύστημα επιβράβευσης κατά την ολοκλήρωση μιας εργασίας} & \textit{1 (καθόλου χρήσιμο) -- 5 (πολύ χρήσιμο)} \\
                        \textit{Καλαίσθητο, φιλικό για τον χρήστη, γραφικό περιβάλλον} & \textit{1 (καθόλου χρήσιμο) -- 5 (πολύ χρήσιμο)} \\
                        \textit{Προειδοποίηση πριν λήξει η προθεσμία μιας εργασίας} & \textit{1 (καθόλου χρήσιμο) -- 5 (πολύ χρήσιμο)} \\
                        \midrule
                        Σε κλίμακα από 1 έως 5, πόσο πιθανό είναι να χρησιμοποιήσεις  & 1 -- 5 \\
                        μια εφαρμογή σχεδιασμένη για τη διαχείριση εργασιών των φοιτητών;
                    \end{tabular}}
            \end{table}

        \subsection{Αποτελέσματα}
            Εν τέλει το δείγμα αποτελείται από 14 φοιτητές, με το 28.6\% (n = 4) να ανήκει στην ηλικιακή ομάδα 18--19 ετών, το 21.4\% (n = 3) στην ηλικιακή ομάδα 20--21 ετών, το 28.6\% (n = 4) στην ηλικιακή ομάδα 22--24 ετών και το 21.4\% (n = 3) στην ηλικιακή ομάδα 25+.

            Όσον αφορά το ακαδημαϊκό επίπεδο, το 71.4\% (n = 10) των συμμετεχόντων είναι προπτυχιακοί φοιτητές, ενώ το 14.3\% (n = 2) μεταπτυχιακοί και το 14.3\% (n = 2) διδακτορικοί φοιτητές.

            Σχετικά με το αντικείμενο σπουδών, 6 φοιτητές σπουδάζουν στο παρόν τμήμα, οι υπόλοιποι φοιτητές σπουδάζουν Επιστήμη Υλικών, Διοίκηση Επιχειρήσεων, Ιατρική, ΗΜΤΥ, Φιλολογία, και Διατροφολογία. Το 7.1\% (n = 1) των φοιτητών είναι στο 1ο έτος σπουδών, το 28.6\% (n = 4) στο 2ο έτος, το 21.4\% (n = 3) στο 3ο έτος, το 7.1\% (n = 1) στο 4ο έτος, το 14.3\% (n = 2) στο 5ο έτος, και το 21.4\% (n = 3) είναι επί πτυχίο.

            Σχετικά με τη χρήση εργαλείων για τη διαχείριση του χρόνου, το 64.3\% (n = 9) των φοιτητών χρησιμοποιούν κάποιο εργαλείο, ενώ το 35.7\% (n = 5) προγραμματίζουν τις υποχρεώσεις τους στο μυαλό τους. Από τους φοιτητές που χρησιμοποιούν εργαλεία, και οι 9 χρησιμοποιούν κάποια εφαρμογή, ενώ 2 από αυτούς επιπλέον χρησιμοποιούν υπενθυμίσεις ή ημερολόγιο σε φυσική μορφή.

            Όσον αφορά το είδος των εργασιών που προγραμματίζουν, και οι 9 φοιτητές αναφέρουν ότι προγραμματίζουν υποχρεώσεις σχολής και επίσης 4 από αυτούς προγραμματίζουν υποχρεώσεις καθημερινότητας, 3 κοινωνικές δραστηριότητες, 2 προσωπική ανάπτυξη και 3 προσωπική φροντίδα. Σε κλίμακα από 1 έως 5, 5 φοιτητές βαθμολογούν τις μεθόδους διαχείρισης εργασιών που χρησιμοποιούν με βαθμό 4, 4 φοιτητές με βαθμό 3, 3 φοιτητές με βαθμό 2 ενώ ένας φοιτητής με βαθμό 1.

            Σχετικά με τις δυσκολίες που αντιμετωπίζουν στη διαχείριση των υποχρεώσεών τους, 9 φοιτητές αναφέρουν ότι αντιμετωπίζουν απόσπαση προσοχής από social media, 6 φοιτητές αντιμετωπίζουν αναβλητικότητα, 5 φοιτητές αντιμετωπίζουν αγωνία ή άγχος σχετικά με τις προθεσμίες, 4 φοιτητές αντιμετωπίζουν αδυναμία τήρησης προγράμματος, 3 φοιτητές έχουν πάρα πολλές υποχρεώσεις ή έλλειψη κινήτρων, ενώ 2 φοιτητές έχουν δυσκολία στην ιεράρχηση των εργασιών τους, έλλειψη χρόνου για σωστό προγραμματισμό ή κακή εκτίμηση του απαιτούμενου χρόνου για τον προγραμματισμό τους.

            Σχετικά με τις προθεσμίες, το 42.9\% (n = 6) των φοιτητών αναφέρουν ότι μερικές φορές έχουν χάσει κάποια προθεσμία, το 21.4\% (n = 3) συχνά, το 14.3\% (n = 2) πάντα και ποτέ και το 7.1\% (n = 1) σπάνια. Όσον αφορά το αν έχει τύχει να μην έχουν ξεκινήσει κάποια εργασία τους, οι μισοί απάντησαν ότι ναι, ενώ οι υπόλοιποι όχι. Αν ναι, έχουν δοθεί 8 απαντήσεις (μια επιπλέον), με 2 απαντήσεις να αναφέρουν ότι η έλλειψη χρόνου ήταν ο λόγος που δεν ξεκίνησαν την εργασία τους, 2 απαντήσεις να αναφέρουν ότι την ξέχασαν, 2 απαντήσεις να αναφέρουν ότι ήταν κουραστική και μια απάντηση να αναφέρει πως ήταν ο λόγος ήταν ότι ήταν ομαδική και δεν υπήρχε συνεργασία με την υπόλοιπη ομάδα.

            Σχετικά με το αν αγχώνονται οι φοιτητές από τον φόρτο των εργασιών, το 42.9\% (n = 6) των φοιτητών αναφέρουν ότι συχνά αισθάνονται αγχωμένοι, το 28.6\% (n = 4) μερικές φορές, το 14.3\% (n = 2) σπάνια, το 7.1\% (n = 1) ποτέ και το 7.1\% (n = 1) πάντα.

            \begin{figure}[h!] \noindent \centering
                \includegraphics[width=\textwidth]{poll/poll-usefullFeatures}
                \caption{Βαθμολόγηση χαρακτηριστικών που οι φοιτητές θεωρούν χρήσιμα σε μια εφαρμογή διαχείρισης εργασιών}
                \label{fig:poll-usefullFeatures}
            \end{figure}

            Τα χαρακτηριστικά που οι φοιτητές θεωρούν χρήσιμα σε μια εφαρμογή διαχείρισης εργασιών εμφανίζονται στο \ref{fig:poll-usefullFeatures}. Παρατηρούμε πως οι φοιτητές δίνουν περισσότερο βάση στην ύπαρξη ημερολογίου, στην κατηγοριοποίηση βάση χρώματος, και στην προειδοποίηση πριν λήξει μια προθεσμία, ενώ δεν ενδιαφέρονται τόσο για την ύπαρξη επαναλαμβανόμενων εργασιών.

            Τέλος, σε κλίμακα από 1 έως 5, 6 φοιτητές βαθμολογούν με βαθμό 3 την πιθανότητα χρήσης μιας εφαρμογής διαχείρισης εργασιών σχεδιασμένη για φοιτητές, 3 φοιτητές με βαθμούς 2 και 5, και 2 φοιτητές με βαθμό 1.

        \subsection{Συμπεράσματα}
            Η έρευνα ανέδειξε ενδιαφέρουσες πτυχές σχετικά με τις δυσκολίες που αντιμετωπίζουν οι φοιτητές στη διαχείριση των εργασιών τους, καθώς και τα χαρακτηριστικά που θεωρούν ιδιαίτερα χρήσιμα σε μία εφαρμογή οργάνωσης υποχρεώσεων.

            Αρχικά, διαπιστώνεται ότι, παρά την ύπαρξη ψηφιακών εργαλείων διαχείρισης χρόνου, ένα αξιοσημείωτο ποσοστό φοιτητών εξακολουθεί να βασίζεται σε μη συστηματικές μεθόδους προγραμματισμού, όπως η απομνημόνευση των υποχρεώσεων. Επιπλέον, μεταξύ όσων χρησιμοποιούν κάποιο εργαλείο, η συντριπτική πλειοψηφία (100\%) οργανώνει υποχρεώσεις της σχολής και ένα μικρότερο ποσοστό επεκτείνει τη χρήση αυτών των εργαλείων και σε άλλους τομείς, όπως υποχρεώσεις καθημερινότητας (44,4\%), κοινωνικές δραστηριότητες (33,3\%) και προσωπική φροντίδα (33,3\%).

            Ιδιαίτερα σημαντική είναι η ανάδειξη της αναβλητικότητας (42,9\%) και της απόσπασης προσοχής από τα μέσα κοινωνικής δικτύωσης (64,3\%) ως δύο από τα πλέον συχνά προβλήματα στη διαχείριση των εργασιών. Αυτά τα ευρήματα υποδεικνύουν ότι οι φοιτητές δε χρειάζονται απλώς ένα εργαλείο καταγραφής υποχρεώσεων, αλλά μία εφαρμογή που να ενσωματώνει μηχανισμούς οι οποίοι διευκολύνουν τη συγκέντρωση και ενισχύουν τη συνέπεια στην τήρηση του προγράμματος. Η εισαγωγή ενός συστήματος επιβράβευσης, το οποίο βαθμολογείται με 3,9/5 από τους συμμετέχοντες, θα μπορούσε να λειτουργήσει ενισχυτικά στη διατήρηση κινήτρου και στη σταδιακή μείωση της αναβλητικότητας.

            Η δυσκολία τήρησης προθεσμιών αναδεικνύεται ως σημαντική πρόκληση, καθώς 78,6\% των φοιτητών έχει χάσει κάποια προθεσμία τουλάχιστον μία φορά. Επιπλέον, οι μισοί φοιτητές (50\%) έχουν τουλάχιστον μία φορά ξεχάσει ή εγκαταλείψει μία εργασία τους λόγω έλλειψης χρόνου, κούρασης ή κακής συνεργασίας ομάδας. Για να αντιμετωπιστεί αυτή η δυσκολία, η εφαρμογή θα πρέπει να ενσωματώνει υπενθυμίσεις και ειδοποιήσεις που θα προσαρμόζονται στις συνήθειες του χρήστη, επιτρέποντάς του να έχει έγκαιρη ενημέρωση σχετικά με τις προθεσμίες. Ένα σύστημα ειδοποιήσεων που θα ξεκινά με υπενθύμιση αρκετές ημέρες πριν από την προθεσμία θα μπορούσε να βελτιώσει την τήρηση των προθεσμιών.

            Όσον αφορά τα χαρακτηριστικά που θεωρούνται απαραίτητα σε μία εφαρμογή διαχείρισης εργασιών, παρατηρείται έντονο ενδιαφέρον για λειτουργίες όπως το ημερολόγιο, η ιεράρχηση εργασιών βάσει προτεραιότητας, η δυνατότητα κατηγοριοποίησης μέσω χρωμάτων και η αποστολή ειδοποιήσεων πριν από τις προθεσμίες, με όλα αυτά τα χαρακτηριστικά να λαμβάνουν μέσο όρο βαθμολογίας μεταξύ 4,2 και 4,7 σε κλίμακα 1-5. Παράλληλα, δίνεται έμφαση στη σημασία της ευχρηστίας και της αισθητικής της εφαρμογής, καθώς το φιλικό γραφικό περιβάλλον αξιολογείται με μέσο όρο 4,6/5.

            Η πρόθεση χρήσης μιας τέτοιας εφαρμογής δεν είναι απόλυτα εγγυημένη, καθώς ο μέσος όρος στην κλίμακα 1-5 φτάνει το 3,14, κάτι που δείχνει ενδιαφέρον αλλά όχι ενθουσιασμό. Αυτό υποδηλώνει ότι η αποδοχή μιας εφαρμογής εξαρτάται από το αν μπορεί να προσφέρει σημαντική βελτίωση σε σχέση με τα υπάρχοντα εργαλεία, να είναι εύκολη στη χρήση και να παρέχει σαφή οφέλη, χωρίς να προσθέτει επιπλέον γνωστικό φορτίο.

            Τα αποτελέσματα της έρευνας καταδεικνύουν ότι η ανάπτυξη μιας εφαρμογής διαχείρισης εργασιών για φοιτητές πρέπει να εστιάσει όχι μόνο στην καταγραφή υποχρεώσεων, αλλά και στη δημιουργία μηχανισμών που βοηθούν στην πειθαρχία, στη μείωση των περισπασμών και στην ενίσχυση της παραγωγικότητας. Η αποτελεσματική υλοποίηση αυτών των χαρακτηριστικών μπορεί να οδηγήσει σε μία εφαρμογή που όχι μόνο διευκολύνει τον προγραμματισμό, αλλά και βελτιώνει τη συνολική ακαδημαϊκή εμπειρία των χρηστών της.
%	\input{4_lowcode}
%	\chapter{Mendix} \label{ch:mendix}
    Έχοντας πλέον μια καλή εικόνα για τον ορισμό του χαμηλού κώδικα και των Πλατφορμών Ανάπτυξης Λογισμικού σε Low-Code, στο παρόν κεφάλαιο, θα επικεντρωθούμε σε μια από αυτές τις πλατφόρμες, το Mendix, η οποία αποτέλεσε το βασικό εργαλείο για την υλοποίηση της εφαρμογής που αναπτύχθηκε στο πλαίσιο αυτής της διπλωματικής εργασίας.

    Σε αυτό το κεφάλαιο, θα περιγραφεί το γραφικό περιβάλλον του Mendix Studio Pro, και θα αναλυθεί η δομή μιας εφαρμογής που αναπτύσσεται στο Mendix. Θα περιγραφούν έννοιες όπως τα modules, τα έγγραφα, τα widgets, οι σελίδες, τα microflows, τα domain models και άλλα. Ο στόχος είναι να κατανοηθεί πλήρως η δομή μιας εφαρμογής στο Mendix, πριν προχωρήσουμε στη διαδικασία υλοποίησης στο επόμενο κεφάλαιο.

    \section{Τι είναι το Mendix;}
        Το Mendix αποτελεί μία από τις πιο διαδεδομένες πλατφόρμες ανάπτυξης λογισμικού που βασίζεται σε χαμηλό κώδικα. Ιδρύθηκε το 2005 στο Ρότερνταμ της Ολλανδίας με στόχο να παρέχει στους επιχειρηματίες και τους οργανισμούς τη δυνατότητα να αναπτύσσουν, να προσαρμόζουν και να διαχειρίζονται εφαρμογές αποδοτικά με χαμηλό κόστος. Το Mendix περιλαμβάνει όλα τα οφέλη και τα χαρακτηριστικά των LCDP που έχουν περιγραφεί στην ενότητα \ref{sec:LCDP}, συμπεριλαμβάνοντας γραφικό περιβάλλον με WYSIWYG GUI σχεδιαστή, drag-and-drop εργαλεία και έτοιμες βιβλιοθήκες, τη χρήση domain models, το εύκολο deployment της εφαρμογής στο cloud, version control μέσω Git, συνεργασία χρησιμοποιώντας Agile μεθοδολογία και άλλα.

        Το 2018, το Mendix εξαγοράστηκε από τη Siemens, τη μεγαλύτερη βιομηχανική κατασκευαστική εταιρεία στην Ευρώπη, γεγονός που επέφερε σημαντικές εξελίξεις στην πλατφόρμα. Η συγχώνευση αυτή επέτρεψε την ενσωμάτωση προηγμένων βιομηχανικών και IoT (Internet of Things) λύσεων, ενισχύοντας τη θέση του Mendix στην αγορά των λογισμικών σχεδιασμένων για επιχειρήσεις. Έτσι, το Mendix συγκαταλέγεται στις πιο ισχυρές και ευέλικτες λύσεις στην αγορά του low-code προγραμματισμού, προσφέροντας αποτελεσματικότητα, ταχύτητα και καινοτομία στην ανάπτυξη λογισμικού, ενώ παράλληλα ενσωματώνει τις πιο σύγχρονες τεχνολογίες για να καλύψει τις ανάγκες επιχειρήσεων που επιθυμούν να παραμείνουν ανταγωνιστικές στην ψηφιακή εποχή \cite{LowCodeMendix}.

        Η Gartner, μια από τις μεγαλύτερες εταιρείες έρευνας και συμβουλευτικών υπηρεσιών στον κλάδο της τεχνολογίας, χαρακτηρίζει το Mendix ως ηγέτη στην αγορά των πλατφορμών ανάπτυξης λογισμικού για 8 συνεχόμενα χρόνια, (εικόνα \ref{fig:GartnerQuadrant}). Η κατάταξη αυτή επιβεβαιώνει τη δυνατότητα του Mendix να παρέχει λύσεις υψηλής ποιότητας και αξίας στους πελάτες του, καθώς και την ικανότητά του να προσαρμόζεται στις ανάγκες της αγοράς και να προσφέρει συνεχώς καινοτόμες λύσεις \cite{mendixGartnerQuadrant}. Για αυτούς τους λόγους έχει προτιμηθεί για την υλοποίηση της εφαρμογής που θα παρουσιαστεί στο επόμενο κεφάλαιο.

            \begin{figure}[h!] \noindent \centering
                \includegraphics[width=0.5\textwidth]{GartnerQuadrant}
                \caption{\centering Τεταρτημόριο της Gartner με πλατφόρμες ανάπτυξης λογισμικού \cite{mendixGartnerQuadrant}}
                \label{fig:GartnerQuadrant}
            \end{figure}

    \section{Το Mendix Studio}
        Το Mendix Studio Pro αποτελεί το βασικό εργαλείο ανάπτυξης εφαρμογών της πλατφόρμας Mendix, προσφέροντας στους χρήστες τη δυνατότητα να δημιουργήσουν, να προσαρμόσουν, να δοκιμάσουν και να αναπτύξουν εφαρμογές.

        \begin{figure}[h!] \noindent \centering
            \includegraphics[width=\textwidth]{Mendix/MendixStudioOverlay}
            \caption{\centering Στιγμιότυπο του Mendix Studio Pro.}
            \label{fig:MendixStudioOverlay}
        \end{figure}

        \subsection{Περιβάλλον ανάπτυξης}
        Στο \ref{fig:MendixStudioOverlay} παρουσιάζεται το γραφικό περιβάλλον του Mendix Studio Pro με ανοιχτή την εφαρμογή RapidDeveloper\footnote{Η εφαρμογή RapidDeveloper δημιουργήθηκε ως αποτέλεσμα των μαθημάτων (crash courses) που προσφέρονται μέσω του Mendix Academy. Αυτά τα μαθήματα έχουν σχεδιαστεί για να παρέχουν στους χρήστες μια γρήγορη και πρακτική εισαγωγή στις βασικές δυνατότητες της πλατφόρμας Mendix, επιτρέποντάς τους να εξοικειωθούν με τη διαδικασία ανάπτυξης εφαρμογών σε περιβάλλον low-code.}.

        Μπορούμε να διαχωρίσουμε το γραφικό περιβάλλον σε τέσσερα μέρη. Η μαύρη μπάρα στο πάνω μέρος περιλαμβάνει το βασικό μενού της πλατφόρμας, το όνομα της εφαρμογής που αναπτύσσουμε, κουμπιά τα οποία επιτρέπουν είτε την τοπική εκτέλεση της εφαρμογής μέσω localhost ή τη διάθεσή της στο cloud (βλ. ενότητα \ref{sec:MendixDeployment}) και σύνδεσμοι που οδηγούν στο Mendix προφίλ του χρήστη, στο Marketplace κ.α.

        Στο κεντρικό τμήμα της οθόνης βρίσκεται το \textit{Working Area}, ένας WYSIWYG σχεδιαστής όπου μπορούμε να προβάλλουμε και να επεξεργαστούμε τις σελίδες της εφαρμογής μας. Μια σελίδα μπορεί να εμφανιστεί στο Working Area με διαφόρους τρόπους, οι οποίοι θα αναλυθούν στην ενότητα \ref{sec:MendixPageView}.

        Στην αριστερή πλευρά, εντοπίζουμε τον \textit{App Explorer}, ο οποίος περιλαμβάνει τη δομή φακέλων και αρχείων της εφαρμογής, καθώς και τον \textit{Page Explorer}, που καταγράφει όλα τα στοιχεία που έχουν χρησιμοποιηθεί στη σελίδα της εφαρμογής που είναι ανοιχτή. Στη δεξιά πλευρά, βρίσκεται το \textit{Properties} Panel, όπου εμφανίζονται όλες οι ρυθμίσεις και οι παράμετροι του στοιχείου ή της σελίδας που έχουμε επιλεγμένη, και το \textit{Toolbox}, το οποίο περιέχει ένα σύνολο από προδιαμορφωμένα στοιχεία που μπορούν να εισαχθούν στη σελίδα. Στο κάτω μέρος εμφανίζονται panels με τις αλλαγές που πραγματοποιούμε ανά commit, τα σφάλματα αν τυχόν υπάρχουν, logs, κονσόλα και άλλα. Μπορούμε σε οποιαδήποτε πεδίο να προσθέσουμε ή να αφαιρέσουμε Panels από το $ \text{Μενού} \rightarrow \text{View} $ \cite{mendixDoc}.

        \subsubsection{Επιλογές για deployment} \label{sec:MendixDeployment}
            Το Mendix παρέχει διάφορες επιλογές για το deployment των εφαρμογών που αναπτύσσονται στην πλατφόρμα.

            Με το Mendix Free μπορούμε να κάνουμε deploy την εφαρμογή στο διαδίκτυο σε ένα URL της μορφής \texttt{<όνομα εφαρμογής>-sandbox.mxapps.io}, και είναι αυτό που έχει χρησιμοποιηθεί για την υλοποίηση της εφαρμογής μας στο κεφάλαιο \ref{ch:unitask}. Το Mendix Free είναι κατάλληλο για την ανάπτυξη μικρών εφαρμογών και την εξοικείωση με την πλατφόρμα, αλλά δεν προσφέρει την απαιτούμενη ασφάλεια και ευελιξία για την ανάπτυξη μεγάλων επιχειρησιακών εφαρμογών καθώς οι εφαρμογές τίθενται σε κατάσταση αναστολής λειτουργίας (sleep mode) μετά από λίγες ώρες αδράνειας, δεν μπορούν να κλιμακωθούν, υπάρχει όριο στην υπολογιστική ισχύ και το μέγεθος της βάσης δεδομένων, δεν εκτελούνται προγραμματισμένα συμβάντα, δεν υποστηρίζονται custom domains κ.α. Παρόλα αυτά είναι μια πολύ καλή επιλογή που εξυπηρετεί τις ανάγκες των αρχάριων χρηστών και μικρών εφαρμογών. Για χρήστες ή επιχειρήσεις με αυξανόμενες ανάγκες, το Mendix προσφέρει λύσεις επί πληρωμή που άρουν τους περιορισμούς που αναφέρθηκαν νωρίτερα \cite{mendixCloud}.

            Το Mendix προφίλ κάθε χρήστη περιλαμβάνει πίνακες ελέγχου (dashboards) (εικόνα \ref{fig:MendixEnvironments}) για κάθε εφαρμογή που αναπτύσσει όπου περιλαμβάνονται ρυθμίσεις και πληροφορίες όσον αφορά το deployment.

            \begin{figure}[h!] \noindent \centering
                \includegraphics[width=0.9\textwidth]{Mendix/Environments}
                \caption{\centering Σελίδα Environments της εφαρμογής UniTask (κεφ. \ref{ch:unitask})}
                \label{fig:MendixEnvironments}
            \end{figure}


    \section{Δομή εφαρμογών του Mendix}
        Μια εφαρμογή στο Mendix απαρτίζεται από διαφορετικά έγγραφα και modules. Τα modules επιτρέπουν τον διαχωρισμό της εφαρμογής σε αυτόνομα λειτουργικά κομμάτια. Ο τρόπος με τον οποίο πραγματοποιείται ο διαχωρισμός εξαρτάται από την κρίση και τη σχεδιαστική προσέγγιση του μηχανικού λογισμικού.

        \subsection{Modules}
            Μια εφαρμογή αποτελείται από ένα \texttt{App} module, ένα \texttt{System} module, από modules που δημιουργούνται από τους χρήστες, από modules από το marketplace του Mendix ή από modules που καθορίζουν την εμφάνιση της εφαρμογής (UI resources modules). Τα modules του marketplace προσφέρουν έτοιμες λειτουργικότητες κατασκευασμένες από τρίτους ενώ τα UI resources modules εμφανίζονται με πράσινο χρώμα και περιλαμβάνουν πρότυπα σελίδων (page templates) και δομικά στοιχεία (building blocks).

            Για παράδειγμα, στο App Explorer της εφαρμογής RapidDeveloper (εικόνα \ref{fig:MendixStudioOverlay}) παρατηρούμε πως εμφανίζονται τρία διαφορετικά modules: το module \texttt{App}, το module \texttt{System} και το module \texttt{MyFirstModule}. Τα δύο πρώτα είναι modules που δημιουργούνται αυτόματα κατά τη δημιουργία μιας εφαρμογής, ενώ το τρίτο είναι ένα module που δημιουργήθηκε από τον χρήστη.

            Κάθε module περιλαμβάνει ένα domain model, που καθορίζει τη δομή των δεδομένων του. Θα αναφερθούμε αναλυτικά στο domain model στην ενότητα \ref{sec:MendixDomainModel}. Πέρα από το domain models, κάθε module περιλαμβάνει παράθυρα για τις Ρυθμίσεις (Settings) του module και την Ασφάλεια (Security).

            Στις \textbf{Ρυθμίσεις} επιτρέπεται η εξαγωγή του Module ως \textit{App Module} με όλο το πηγαίο κώδικα του ή ως \textit{Add-on Module} με σκοπό να χρησιμοποιηθεί απο άλλους χρήστες, και επίσης εκεί μπορεί να γίνει προσθήκη Java Dependencies στο module. Στην \textbf{Ασφάλεια} μπορεί να καθοριστεί η πρόσβαση όλων των ρόλων χρηστών για κάθε σελίδα, οντότητα ή microflow που υπάρχει στο συγκεκριμένο module.

            \subsubsection{Το module App}
            Κατά τη δημιουργία μιας νέας εφαρμογής, το Mendix παρέχει ένα σύνολο προεγκατεστημένων σελίδων με έτοιμο σχεδιασμό και λειτουργικότητα. Τα αρχεία αυτών των σελίδων βρίσκονται στο module \texttt{App}. Το module \texttt{App} πέρα από τα παράθυρα των Ρυθμίσεων και Ασφάλειας (τα οποία έτσι και αλλιώς υπάρχουν σε όλα τα modules) επιπλέον περιλαμβάνει την Πλοήγηση (Navigation) και τα Κείμενα Συστήματος (System Texts)\footnote{Στο έγγραφο με τα Κείμενα Συστήματος μπορεί να γίνει μετάφραση των μηνυμάτων που παράγονται από τον διακομιστή κατά την εκτέλεση μιας εφαρμογής (για παράδειγμα \say{Password too short}).}. Επιπλέον, περιλαμβάνει ένα φάκελο \texttt{Styling} με \texttt{.js} και \texttt{.css} αρχεία τα οποία χρησιμοποιούνται για το styling της εφαρμογής\footnote{Στα συγκεκριμένα αρχεία μπορούν να επεξεργαστούν μεταβλητές που αφορούν τη σχεδίαση του UI resources module \texttt{Atlas}, το οποίο χρησιμοποιείται ως κεντρικό theme για τις προεγκατεστημένες σελίδες του Mendix. Το \texttt{Atlas} για παράδειγμα περιλαμβάνει έτοιμα color schemes (\texttt{primary}, \texttt{success}, \texttt{warning}, \texttt{danger}, \texttt{info}) τα οποία χρησιμοποιούνται σε όλα τα widgets των σελιδών, όπως επίσης γραμματοσειρές, spacings κτλ. Τα αρχεία λοιπόν αποτελούν έναν από τους τρόπους προσαρμογής της προεπιλεγμένης σχεδίασης του \texttt{Atlas}. Εναλλακτικός τρόπος είναι η δημιουργία ενός custom UI resources module.} και έναν φάκελο \texttt{Marketplace modules} το οποίο περιλαμβάνει εξωτερικά modules που μπορούν να προστεθούν μέσω του Mendix Marketplace.

            Το παράθυρο \textbf{Ρυθμίσεων} του \texttt{App} (εικόνα \ref{fig:MendixAppSettings}) διαφέρει σε σχέση με τα υπόλοιπα modules, καθώς αυτό περιλαμβάνει παραμετροποιήσεις για το runtime περιβάλλον της εφαρμογής, το theme που χρησιμοποιείται, την επιλογή συγκεκριμένων ενεργειών πριν αρχικοποιηθεί η εφαρμογή κατά την εκκίνησή της, επιλογή συγκεκριμένου αλγόριθμου κρυπτογράφησης για το Hashed String τύπο δεδομένων, καθορισμός γλώσσας και άλλα.

            \begin{figure}[h!] \noindent \centering
                \includegraphics[width=0.7\textwidth]{Mendix/AppSettings}
                \caption{\centering Παράθυρο Settings του App}
                \label{fig:MendixAppSettings}
            \end{figure}

            \begin{figure}[h!] \noindent \centering
                \includegraphics[width=0.8\textwidth]{Mendix/AppSecurity}
                \caption{\centering Παράθυρο Security του App}
                \label{fig:MendixAppSecurity}
            \end{figure}

            Το παράθυρο \textbf{Ασφάλεια} του \texttt{App} (εικόνα \ref{fig:MendixAppSecurity}) το οποίο και αυτό διαφέρει σε σχέση με τα αντίστοιχα παράθυρα των υπόλοιπων modules, χρησιμοποιείται για να τροποποιηθεί το \textit{επίπεδο ασφάλειας} (Security level) της εφαρμογής. Συγκεκριμένα μπορεί να επιλεγεί ώστε οι χρήστες να μη χρειάζεται να συνδεθούν για να έχουν πρόσβαση στην εφαρμογή (Security level = Off), να χρειάζεται να συνδεθούν ώστε να μπορέσουν να χρησιμοποιήσουν φόρμες, microflows κτλ (Security level = Prototype/demo), ή να χρειάζεται να συνδεθούν ώστε να έχουν πρόσβαση στην εφαρμογή καθολικά (Security level = Production).

            Επίσης, μπορούν να οριστούν \textit{ρόλοι χρηστών} όπως για παράδειγμα Administrator, User ή Guest. Κατ' αυτόν τον τρόπο καθίσταται δυνατό να διαχωριστούν τα δικαιώματα πρόσβασης των modules ανάλογα με το ποιος είναι ο ρόλος του χρήστη. Για παράδειγμα, ένας Administrator χρήστης μπορεί να έχει πλήρη πρόσβαση στα modules και στις σελίδες της εφαρμογής, ενώ ένας Guest μπορεί να έχει περιορισμένη προβολή.

            Επιπλέον, στην Ασφάλεια μπορούν να οριστούν τα στοιχεία σύνδεσης του διαχειριστή (Administrator) της εφαρμογής (ώστε να μπορέσει να γίνει η πρώτη σύνδεση κατά το deployment), να οριστούν demo χρήστες (ώστε να δοκιμαστούν οι ρόλοι χρηστών) ή ανώνυμοι χρήστες (οι οποίοι έχουν πρόσβαση στην εφαρμογή χωρίς να συνδεθούν) και να ρυθμιστούν κανόνες για τους κωδικούς πρόσβασης.

            \begin{figure}[h!] \noindent \centering
                \includegraphics[width=0.8\textwidth]{Mendix/Navigation}
                \caption{\centering Το έγγραφο Navigation}
                \label{fig:MendixAppNavigation}
            \end{figure}

            Η \textbf{Πλοήγηση} του module \texttt{App} (εικόνα \ref{fig:MendixAppNavigation}) εμφανίζει το σύνολο των σελίδων του κεντρικού μενού της εφαρμογής. Εκεί μπορεί να γίνει η επεξεργασία του μενού με την προσθήκη στοιχείων ή και υποστοιχείων σε στοιχεία. Επίσης, περιλαμβάνονται διαφορετικές προβολές ανάλογα με τον εκάστοτε ρόλο χρήστη, στις οποίες εμφανίζονται μόνο οι σελίδες που είναι προσβάσιμες σε κάθε ρόλο. Στην Πλοήγηση επίσης μπορεί να διαμορφωθεί το όνομα και το εικονίδιο της εφαρμογής, καθώς και να οριστεί η αρχική σελίδα (home page) της, η οποία μάλιστα μπορεί να διαμορφωθεί ώστε να είναι διαφορετική για εκάστοτε ρόλο χρήστη \cite{mendixDoc}.

            \subsubsection{Το module System}
                Το module \texttt{System} είναι ένα προεγκατεστημένο module που περιλαμβάνει τη βασική λειτουργικότητα η οποία χρησιμοποιείται στις ήδη κατασκευασμένες σελίδες και microflows του \texttt{App}. Το συγκεκριμένο module δεν μπορεί να επεξεργαστεί από τον χρήστη, αλλά μπορούν με βάσει αυτό να προστεθούν συσχετίσεις (associations) ή γενικεύσεις (generalizations) τα modules τα οποία κατασκευάζονται από τους χρήστες. Για παράδειγμα, μπορεί να δημιουργηθεί μια οντότητα \texttt{Πελάτης} η οποία θα συσχετίζεται με την οντότητα \texttt{User} του module \texttt{System} και έτσι θα κληρονομεί τις ρυθμίσεις ασφαλείας του \cite{mendixSystemModule}.

        \subsection{Έγγραφα}
            Η Πλοήγηση είναι ένα \textit{έγγραφο} της εφαρμογής. Άλλα παραδείγματα εγγράφων είναι οι \textit{Σελίδες} (Pages) (βλ. ενότητα \ref{sec:MendixPages}), τα \textit{Microflows} (βλ. ενότητα \ref{sec:MendixMicroflows}) και τα \textit{Enumerations}\footnote{Τα enumerations (κατάλογος, απαρίθμηση) καθορίζουν μια λίστα από προκαθορισμένες τιμές. Για παράδειγμα, ένα enumeration μπορεί να χρησιμοποιηθεί για τον καθορισμό της κατάστασης μιας εργασίας ως \texttt{To do}, \texttt{Done} ή \texttt{Doing}.}.

        \subsection{Σελίδες} \label{sec:MendixPages}
            Η σελίδα είναι ο κεντρικός τρόπος αλληλεπίδρασης του χρήστη με την εφαρμογή. Η δημιουργία μιας σελίδας μπορεί να βασιστεί σε \textit{πόρους} που προέρχονται από έγγραφα, όπως οι Εικόνες (Images), τα Layouts τα οποία καθορίζουν τη διάταξη της σελίδας, τα Μενού (Menus) που διαμορφώνουν την πλοήγηση, ή τα Snippets, τα οποία αποτελούν επαναχρησιμοποιήσιμα τμήματα διεπαφής.\footnote{Το Mendix είναι μια Εφαρμογή Μίας Σελίδας (Single-page application -- SPA) που σημαίνει ότι όλη η αλληλεπίδραση πραγματοποιείται σε μία μόνο καρτέλα ή παράθυρο του προγράμματος περιήγησης που φορτώνεται μία φορά και στη συνέχεια ενημερώνεται δυναμικά χωρίς να χρειάζεται να φορτωθεί ξανά. Ως αποτέλεσμα, δεν είναι δυνατό το άνοιγμα νέων σελίδων σε διαφορετική καρτέλα ή παράθυρο.}

                \begin{figure}[h!] \noindent \centering
                    \includegraphics[width=0.7\textwidth]{Mendix/CreateNewPage}
                    \caption{\centering Το παράθυρο δημιουργίας μιας νέας σελίδας}
                    \label{fig:MendixCreateNewPage}
                \end{figure}

            \subsubsection{Layout}
            Κάθε σελίδα στο Mendix βασίζεται σε ένα προκαθορισμένο layout, το οποίο καθορίζει βασικές ιδιότητες της σελίδας, όπως το μήκος, το πλάτος ή για παράδειγμα αν πρόκειται για αναδυόμενη (popup) σελίδα. Επιπλέον, τα layouts επιτρέπουν τον ορισμό στατικών τμημάτων, όπως ένα header ή ένα μενού, που παραμένουν σταθερά σε όλες τις σελίδες που τα χρησιμοποιούν. Για παράδειγμα, θα μπορούσαν να δημιουργηθούν δύο layouts όπου το ένα θα έχει το μενού πλοήγησης ως μπάρα στο πάνω μέρος της οθόνης και το άλλο κάθετη στα αριστερά. Το Mendix προσφέρει επίσης προκαθορισμένα πρότυπα σελίδων (page templates), τα οποία διευκολύνουν τη γρήγορη και απλή δημιουργία σελίδων με προκαθορισμένη δομή και σχεδίαση.

            Η εικόνα \ref{fig:MendixCreateNewPage} απεικονίζει το παράθυρο δημιουργίας νέας σελίδας, όπου ο χρήστης μπορεί να επιλέξει είτε από τα έτοιμα πρότυπα είτε να δημιουργήσει μια κενή σελίδα. Εδώ δίνεται επίσης η δυνατότητα επιλογής του layout (Navigation layout) της σελίδας, το οποίο μπορεί να είναι είτε ένα από τα προεγκατεστημένα layouts του Mendix είτε ένα προσαρμοσμένο layout που έχει δημιουργηθεί από τον χρήστη.

            \subsubsection{Widgets}
                Τα Widgets είναι προδιαμορφωμένα στοιχεία, έτοιμες λειτουργικές μονάδες που μπορούν να προστεθούν απευθείας σε κάθε σελίδα της εφαρμογής. Πρόκειται για εργαλεία που ενσωματώνονται εύκολα μέσω του Toolbox, όπως παρουσιάστηκε στην εικόνα \ref{fig:MendixStudioOverlay}. Ενδεικτικά παραδείγματα περιλαμβάνουν:

                \begin{itemize}[label={\tiny \blacksquare}]
                    \setlength\itemsep{-0.25em}
                    \item \textbf{Data containers} -- δομές δεδομένων που περιέχουν δεδομένα από τη βάση δεδομένων. Παραδείγματα είναι Data view, Data grid, List view κ.α.
                    \item \textbf{Text widgets} -- περιέχουν κείμενο. Παραδείγματα είναι Text, Label, Page Title κ.α.
                    \item \textbf{Structure widgets} -- δομικά στοιχεία που χρησιμοποιούνται για την οργάνωση των widgets στη σελίδα. Παραδείγματα είναι Layout grid, Container, Tab container, Snippet call, Table κ.α.
                    \item \textbf{Input widgets} -- πεδία εισόδου δεδομένων. Παραδείγματα είναι Text box, Text area, Check box, Radio button, Drop-down, Date picker, File uploader κ.α.
                    \item \textbf{Images, Videos ή Files} -- widgets που περιέχουν πολυμέσα.
                    \item \textbf{Buttons} -- widgets που εκτελούν ενέργειες. Το Mendix παρέχει αρκετά buttons με προκαθορισμένες ενέργειες για την εκτέλεση ενεργειών όπως Save, Cancel, Delete, New, Edit, Crate, Call microflow κ.α.
                \end{itemize}

            \begin{figure}[h!] \noindent \centering
                \includegraphics[width=0.4\textwidth]{Mendix/ToolboxPage}
                \caption{\centering Τα Widgets περιλαμβάνονται στο Toolbox panel}
                \label{fig:MendixToolboxPage}
            \end{figure}

                Tο Mendix προσφέρει προκαθορισμένα σύνολα από widgets (εικόνα \ref{fig:MendixToolboxPage}), γνωστά ως \textit{Building blocks}, τα οποία δημιουργούν στοιχεία όπως επικεφαλίδες (headers), φόρμες και ειδοποιήσεις (notifications). Αυτά τα Building blocks διευκολύνουν και επιταχύνουν τη διαδικασία ανάπτυξης, παρέχοντας στους χρήστες έτοιμες λύσεις που μπορούν να ενσωματωθούν απευθείας στις εφαρμογές.

                \begin{figure}[h!] \noindent \centering
                    \includegraphics[width=0.3\textwidth]{Mendix/PropertiesPanel1}
                    \includegraphics[width=0.3\textwidth]{Mendix/PropertiesPanel2}
                    \caption{\centering Το panel Properties}
                    \label{fig:MendixPropertiesPanel}
                \end{figure}

                Τέλος, για κάθε widget (όπως και για κάθε σελίδα) μπορούν να επεξεργαστούν οι ιδιότητές του. Το panel Properties (εικόνα \ref{fig:MendixPropertiesPanel}) είναι ένα δυναμικό panel το οποίο αλλάζει το περιεχόμενό του ανάλογα με το στοιχείο που έχουμε επιλεγμένο. Στο Properties μπορούμε να ορίσουμε συνθήκες όπου θα επιτρέπεται η εμφάνιση ενός στοιχείου, να επιλέξουμε διαφορετικά render styles από τα προδιαμορφωμένα που παρέχει το Mendix, να ορίσουμε events που θα συμβαίνουν όταν ο χρήστης αλληλεπιδρά με το στοιχείο, όπως επίσης και να αλλάξουμε CSS κλάσεις είτε μέσω των παρεχόμενων επιλογών του Mendix είτε μέσω της προσθήκης δικού μας custom CSS κώδικα.

        \subsubsection{Εμφάνιση σελίδων} \label{sec:MendixPageView}
            Το Mendix Studio Pro επιτρέπει την προβολή μιας σελίδας είτε σε Structure Mode είτε σε Design Mode, παρέχοντας διαφορετικές οπτικές για τη διαχείριση και τον σχεδιασμό της.

            \begin{figure}[h!] \noindent \centering
                \includegraphics[width=0.6\textwidth]{Mendix/PageView2} \\
                \includegraphics[width=0.6\textwidth]{Mendix/PageView1} \\
                \includegraphics[width=0.6\textwidth]{Mendix/PageView3}
                \caption{\centering Διαφορετικές εμφανίσεις της ίδιας σελίδας από αριστερά \\ προς τα δεξιά: Structure Mode, Design Mode, X-Ray Mode.}
            \end{figure}

            Στο \textbf{Structure Mode}, παρουσιάζονται με σαφήνεια όλα τα δομικά στοιχεία που συνθέτουν τη σελίδα, δίνοντας έμφαση στη δομή της και επιτρέποντας την εύκολη προσαρμογή και οργάνωση των περιεχομένων. Αυτή η λειτουργία είναι ιδανική για την ανάλυση της λογικής της σελίδας, τη διαχείριση των στοιχείων της και τη διόρθωση πιθανών προβλημάτων διάταξης. Είναι επίσης ο μοναδικός τρόπος προβολής των εφαρμογών κινητών συσκευών (Native mobile).

            Αντίθετα, στο \textbf{Design Mode}, η σελίδα απεικονίζεται όπως ακριβώς θα εμφανίζεται στον τελικό χρήστη. Αυτό προσφέρει μια πιο ρεαλιστική απεικόνιση της εμπειρίας χρήστη. Επιπλέον, το Design Mode περιλαμβάνει τη λειτουργία \textbf{X-Ray Mode}, η οποία συνδυάζει στοιχεία από το Structure Mode και το Design Mode. Με το X-Ray Mode, οι χρήστες μπορούν να δουν τόσο την αισθητική εμφάνιση όσο και τη δομή της σελίδας ταυτόχρονα. Αυτή η λειτουργία επιτρέπει την ακριβή τοποθέτηση και διαχείριση στοιχείων, ενώ ταυτόχρονα προσφέρει τη δυνατότητα άμεσων προσαρμογών σε επίπεδο σχεδιασμού και λογικής.

            Οι διαφορετικές προβολές μπορούν να επιλεγούν από το πάνω μέρος του Working Area, το οποίο επιπλέον περιλαμβάνει και δυνατότητα αλλαγών στο μέγεθος του καμβά της σελίδας ώστε να ελεγχθεί το πως εμφανίζεται η εφαρμογή σε κινητά και τάμπλετ.

        \subsubsection{Παράδειγμα δομής σελίδας}
            \begin{figure}[h!] \noindent \centering
                \includegraphics[width=\textwidth]{Mendix/PageExplorer}
                \caption{\centering Page Explorer και Working area της σελίδας \texttt{Course\_Overview}}
                \label{fig:MendixPageExplorer}
            \end{figure}

            Στην εικόνα \ref{fig:MendixPageExplorer} εμφανίζεται o Page Explorer και το Working area της σελίδας \verb|Course_Overview| της εφαρμογής RapidDeveloper. Παρατηρούμε πως όλη η σελίδα είναι δομημένη γύρω από ένα Layout Grid, το οποίο αποτελείται από δύο Rows (γραμμές). Το \texttt{Row 1} δημιουργεί το Header της σελίδας μαζί με το Supporting text τα οποία είναι τοποθετημένα σε ένα container\footnote{Τα containers χρησιμοποιούνται καθώς είναι ένας βολικός τρόπος ομαδοποίσης κοινών στοιχείων της σελίδας. Έτσι μπορούν να δοθούν εύκολα κανόνες για αυτά τα στοιχεία όπως για παράδειγμα το visability, συγκεκριμένα events ή και custom CSS κλάσεις.}, και το \Texttt{Row 2} δημιουργεί ένα List View το οποίο επαναλαμβάνεται. Το List View αποτελείται και αυτό από ένα Layout Grid το οποίο δημιουργεί τα δύο texts τα οποία βλέπουμε στη λίστα όπως επίσης και ένα κουμπί (το οποίο εμφανίζεται εκτός οθόνης). Η μπλε οριζόντια μπάρα και η μπλε κάθετη μπάρα στα αριστερά αποτελούν κομμάτι του Layout της σελίδας, το οποίο είναι το \verb|Atlas_Default|.


        \subsection{Microflows} \label{sec:MendixMicroflows}
            Τα Microflows (όπως και τα Nanoflows και τα Workflows)\footnote{Η βασική διαφορά μεταξύ Microflows και Nanoflows έγκειται στη λειτουργικότητα και τον τρόπο εκτέλεσής τους. Τα Microflows βασίζονται σε βιβλιοθήκες της Java, εκτελούνται στον διακομιστή (runtime server) και, ως εκ τούτου, δεν είναι διαθέσιμα για εφαρμογές που λειτουργούν εκτός σύνδεσης. Από την άλλη πλευρά, τα Nanoflows χρησιμοποιούν βιβλιοθήκες της JavaScript, εκτελούνται στην πλευρά του client, γεγονός που τα καθιστά εν δυνάμει γρηγορότερα.

            Τα Microflows είναι ιδανικά για την προσκόμιση και την επεξεργασία δεδομένων από τη βάση δεδομένων ή από εξωτερικές πηγές, εξασφαλίζοντας υψηλή αξιοπιστία και συνέπεια. Αντίθετα, τα Nanoflows χρησιμοποιούνται κυρίως για ενέργειες που σχετίζονται με την εμπειρία του χρήστη, όπως η εμφάνιση αναδυόμενων (pop-up) μηνυμάτων, η προβολή progress bars ή η ανταλλαγή cookies.

            Τέλος, τα Workflows ενδείκνυνται για τη διαχείριση σταθερών και επαναλαμβανόμενων διαδικασιών, επιτρέποντας την αυτοματοποίηση και την απλοποίηση της εκτέλεσής τους.} συνιστούν τον βασικό μηχανισμό εκτέλεσης λογικής στις εφαρμογές του Mendix καθώς αναπαριστούν τη λογική της εφαρμογής με έναν οπτικό τρόπο χαμηλού κώδικα. Αυτά τα διαγράμματα ροής (εικόνα \ref{fig:MendixMicroflowExample}) απεικονίζουν τη λογική εκτέλεσης διαφόρων εντολών και αλληλουχιών ενεργειών, επιτρέποντας την κατασκευή σύνθετης λειτουργικότητας χωρίς την ανάγκη γραφής παραδοσιακού κώδικα. Χρησιμοποιούνται ευρέως για ενέργειες όπως η δημιουργία, η ενημέρωση και η διαγραφή δεδομένων, η εμφάνιση σελίδων, το φιλτράρισμα δεδομένων, η εκτέλεση ελέγχων, η είσοδος δεδομένων από εξωτερικές πηγές και άλλα.

            \begin{figure}[h!] \noindent \centering
                \includegraphics[width=0.7\textwidth]{Mendix/microflow-nanoflow-example}
                \caption{\centering Παράδειγμα Microflow. Τα πράσινα και κόκκινα κυκλάκια αναπαριστούν τα events, τα μπλε ορθογώνια τα activities και ο πορτοκαλί ρόμβος το decision. Ως είσοδο έχουμε το parameter AccountPasswordData \cite{mendixDoc}.}
                \label{fig:MendixMicroflowExample}
            \end{figure}

            Τα Microflows αποτελούνται από τα παρακάτω στοιχεία:
                \begin{itemize}[label={\tiny \blacksquare}]
                    \setlength\itemsep{-0.25em}
                    \item \textbf{Events} -- λειτουργούν ως σημεία εκκίνησης και τερματισμού του Microflow. Χρησιμοποιώντας το τελικό event μπορούμε να ορίσουμε την τιμή και το τύπο δεδομένων που επιστρέφει το Microflow, με παρόμοιο τρόπο όπως στις συναρτήσεις ή μεθόδους του υψηλού κώδικα.
                    \item \textbf{Decisions} -- επιτρέπουν την εισαγωγή λογικών συνθηκών. Για παράδειγμα, ένα decision μπορεί να ελέγξει αν μια μεταβλητή έχει τιμή και, ανάλογα με την απάντηση, να κατευθύνει τη ροή σε διαφορετικές ενέργειες. Οι συνθήκες ορίζονται με τη χρήση εκφράσεων (βλ. ενότητα \ref{sec:MendixExpressions}).
                    \item \textbf{Activities} -- αποτελούν τις κύριες ενέργειες που εκτελούνται στη ροή. Παραδείγματα τέτοιων ενεργειών είναι η δημιουργία (ή ενημέρωση ή διαγραφή) αντικειμένων μέσω του activity Create (ή Change ή Delete) object, η εμφάνιση μιας σελίδας στον χρήστη μέσω του Show page, η κλήση ενός άλλου Microflow μέσω του Microflow call, το φιλτράρισμα λιστών (List operation), η σύνδεση με εξωτερικές υπηρεσίες μέσω REST APIs κ.α.
                    \item \textbf{Loops} -- επιτρέπουν την εκτέλεση επαναλαμβανόμενων ενεργειών.
                    \item \textbf{Parameter} -- πρόκειται για τα δεδομένα εισόδου του Microflow, με αντίστοιχη λογική όπως οι συναρτήσεις υψηλού κώδικα.
                \end{itemize}

            \begin{figure}[h!] \noindent \centering
                \includegraphics[width=0.4\textwidth]{Mendix/ToolboxMicroflow}
                \caption{\centering Όταν επεξεργαζόμαστε ένα Microflow, \\ τo Toolbox panel περιλαμβάνει τα διαθέσιμα activities.}
            \end{figure}

            Ένα Microflow μπορεί να εκτελεστεί από διαφορετικά μέρη όπως το Navigation menu, ένα κουμπί, ένα link ή ακόμα και να καλεστεί από ένα άλλο Microflow. Επιπλέον, τα Microflows μπορούν να εκτελεστούν αυτόματα μετά από μια συγκεκριμένη ενέργεια (Event Handlers), όπως η αποθήκευση ενός αντικειμένου ή η ενημέρωση ενός πεδίου.

            Τέλος, κάθε Microflow αποθηκεύεται ως ένα Java αρχείο στον πηγαίο κώδικα της εφαρμογής. Για εξειδικευμένες λειτουργίες, το Mendix παρέχει τη δυνατότητα ενσωμάτωσης προσαρμοσμένου κώδικα Java μέσω του $ \text{App} \rightarrow \text{Deploy to Eclipse} $.

        \subsection{Εκφράσεις} \label{sec:MendixExpressions}
            Οι εκφράσεις (expressions) του Mendix είναι ένας τρόπος ενσωμάτωσης λειτουργικότητας στην εφαρμογή μας. Οι εκφράσεις μπορούν να περιλαμβάνουν σταθερές τιμές, μεταβλητές, συναρτήσεις, λογικές πράξεις, συγκρίσεις, επιλογές κ.α. Για παράδειγμα, μπορεί να οριστεί η εμφάνιση ενός συγκεκριμένου widget μόνο αν ισχύει μια συγκεκριμένη συνθήκη. Οι εκφράσεις μπορούν να χρησιμοποιηθούν σε πολλά σημεία της εφαρμογής, όπως στα Microflows, στις ιδιότητες των widgets κ.α.

            Για παράδειγμα, η έκφραση \verb|if $package/weight < 1.00 then 0.00 else 5.00| ελέγχει το γνώρισμα \texttt{weight} της οντότητας \texttt{package} και επιστρέφει \texttt{0.00} αν το βάρος είναι μικρότερο από \texttt{1.00}, αλλιώς επιστρέφει \texttt{5.00}. Θα δούμε περισσότερες τέτοιες εκφράσεις στην πράξη στο κεφάλαιο \ref{ch:unitask}.

        \subsection{Domain Model} \label{sec:MendixDomainModel}
            Το domain model αναπαριστά τη δομή των δεδομένων κάποιου module στην πλατφόρμα Mendix. Τα δεδομένα που περιγράφονται από το domain model αποθηκεύονται στη συνέχεια σε ένα σχεσιακό σύστημα βάσεων δεδομένων του Mendix.

            Το domain model αποτελεί κεντρικό πυλώνα της αρχιτεκτονικής κάθε εφαρμογής. Κάθε module έχει το δικό του domain model, και όλα τα modules μπορούν να χρησιμοποιούν δεδομένα από όλα τα υπόλοιπα domain modules μέσω συσχετίσεων.

            \begin{figure}[h!] \noindent \centering
                \includegraphics[width=0.8\textwidth]{Mendix/annotated-domain-model}
                \caption{\centering Παράδειγμα από domain model \cite{mendixDoc}}
                \label{fig:MendixDomainModel}
            \end{figure}

            Η εικόνα \ref{fig:MendixDomainModel} είναι ένα παράδειγμα ενός domain model που αναπαριστά πελάτες και παραγγελίες. Οι πελάτες και οι παραγγελίες αποτελούν οντότητες (entities) του domain model. Οι οντότητες συσχετίζονται μεταξύ τους με μια συσχέτιση (association) πολλών-προς-ένα. Κάθε παραγγελία ανήκει σε έναν μόνο πελάτη, ενώ ένας πελάτης μπορεί να σχετίζεται με πολλές παραγγελίες. Φυσικά, το Mendix περιλαμβάνει και άλλες πληθικότητες, όπως συσχετίσεις ένα-προς-ένα όπως επίσης και πολλά-προς-πολλά. Επιπλέον, αν διαγραφτεί κάποια οντότητα μπορεί να ρυθμιστεί τι θα συμβεί με τις συσχετίσεις της (π.χ. να διαγραφούν και αυτές).

            Μέσα στα ορθογώνια που αναπαριστούν τις οντότητες βρίσκονται τα γνωρίσματα, οι ιδιότητες (attributes) των οντοτήτων. Στην παρένθεση κάθε γνωρίσματος καταγράφεται ο τύπος δεδομένων του. Παρατηρούμε πως υπάρχουν ορθογώνια με διαφορετικά χρώματα, κάτι που αντιστοιχεί σε διαφορετικού είδους οντότητες. Τα μπλε ορθογώνια αναπαριστούν οντότητες που αποθηκεύονται στη βάση δεδομένων, με κίτρινο μη-διατηρήσιμες οντότητες (non-persistent entities), δηλαδή οντότητες που δεν αποθηκεύονται στη βάση δεδομένων αλλά αποθηκεύονται προσωρινά στη μνήμη, και τέλος με μωβ οντότητες από εξωτερικές πηγές δεδομένων. Η μπλε ετικέτα \texttt{System.User} που συνοδεύει την οντότητα \texttt{Customer} δηλώνει πως η οντότητα αυτή βασίζεται σε μια άλλη οντότητα, την οντότητα \texttt{User} του module \texttt{System} (Γενίκευση -- Generalization)\footnote{Η έννοια της γενίκευσης στο Mendix βασίζεται σε μια λογική που θυμίζει την κληρονομικότητα (inheritance) στις αντικειμενοστραφείς γλώσσες προγραμματισμού, δηλαδή επιτρέπει σε μία οντότητα (entity) να κληρονομεί τις ιδιότητες και τις συσχετίσεις μιας άλλης υπεροντότητας, χρησιμοποιώντας τα χαρακτηριστικά και τις συσχετίσεις της, ενώ παράλληλα μπορεί να ορίσει πρόσθετες ιδιότητες ή συσχετίσεις που είναι μοναδικές για την ίδια. Η γενίκευση είναι ιδιαίτερα χρήσιμη καθώς συμβάλλει στη διατήρηση της ακεραιότητας και της ασφάλειας των δεδομένων. Για παράδειγμα, οι \say{Customers} που κληρονομούν τα γνωρίσματα της οντότητας \texttt{System.User}, ταυτόχρονα κληρονομούν και όλα τα χαρακτηριστικά ασφάλειας του \texttt{User} που το Mendix έχει προδιαμορφώσει.}. Τέλος, ανάλογα αν στην οντότητα έχει οριστεί κάποιος Event handler ή αν σε κάποιο γνώρισμα υπάρχει κάποιο validation rule, το Mendix το αναγνωρίζει και το αναπαριστά με το αντίστοιχο σύμβολο.

            \subsubsection{Οντότητες}
                \begin{figure}[h!] \noindent \centering
                    \includegraphics[width=0.7\textwidth]{Mendix/EntityProperties}
                    \caption{\centering Ιδιότητες μιας οντότητας Task (βλ. κεφ. \ref{ch:unitask})}
                    \label{fig:MendixEntityProperties}
                \end{figure}

                Στην εικόνα \ref{fig:MendixEntityProperties} παρουσιάζεται το παράθυρο που εμφανίζεται όταν δημιουργούμε ή επεξεργαζόμαστε μια οντότητα. Στο πάνω μέρος ορίζεται αν η οντότητα έχει κάποια γενίκευση και το αν είναι διατηρήσιμη στη βάση δεδομένων.

                Στην καρτέλα \textbf{Attributes} (Γνωρίσματα) καθορίζονται όλα τα γνωρίσματα της οντότητας. Εκεί επιλέγεται ο τύπος δεδομένων του γνωρίσματος. Οι τύποι δεδομένων που υποστηρίζονται από το Mendix είναι οι εξής: \texttt{AutoNumber} (αυτόματα παραγόμενοι αριθμοί, π.χ. IDs), \texttt{Binary}, \texttt{Boolean}, \texttt{Date and time}, \texttt{Decimal}, \texttt{Enumeration} (επιλέγεται κάποιο Enumeration έγγραφο), \texttt{Hashed string}, \texttt{Integer}, \texttt{Long}, \texttt{String}. Ο τύπος ενός γνωρίσματος είναι πιθανό να καθοριστεί αυτόματα από το Mendix βάσει του ονόματος που επιλέγεται κατά τον ορισμό του. Τέλος, μπορεί να οριστεί μια προεπιλεγμένη τιμή του κάθε ορίσματος ή να οριστεί η τιμή να καθορίζεται από κάποιο microflow.

                \begin{figure}[h!] \noindent \centering
                    \includegraphics[width=0.6\textwidth]{Mendix/AddAttribute}
                    \caption{\centering Προσθήκη καινούριου γνωρίσματος σε μια οντότητα}
                \end{figure}

                Η καρτέλα \textbf{Associations} (Συσχετίσεις) είναι ένας τρόπος επεξεργασίας των συσχετίσεων μιας οντότητας. Ένας εναλλακτικός τρόπος είναι απευθείας από το domain model μέσω των συνδέσεων μεταξύ των οντοτήτων.

                \begin{figure}[h!] \noindent \centering
                    \includegraphics[width=0.6\textwidth]{Mendix/ValidationRule}
                    \caption{\centering Προσθήκη κανόνα επικύρωσης (βλ. κεφ. \ref{ch:unitask})}
                \end{figure}

                \begin{figure}[h!] \noindent \centering
                    \includegraphics[width=0.62\textwidth]{Mendix/AccessRuleEntity}
                    \caption{\centering Επεξεργασία κανόνων πρόσβασης της οντότητας Task (βλ. κεφ. \ref{ch:unitask})}
                \end{figure}

                Η καρτέλα \textbf{Validation rules} (Κανόνες Επικύρωσης) δημιουργεί συνθήκες που πρέπει να ικανοποιούνται για να είναι έγκυρα τα γνωρίσματα κάθε οντότητας. Οι κανόνες επικύρωσης μπορούν να είναι απλές συνθήκες, όπως π.χ. ένα πεδίο να μην είναι κενό, ή πιο σύνθετες, όπως π.χ. η τιμή ενός πεδίου να καθορίζεται από ένα εύρος τιμών. Αν η συνθήκη δεν πληρούται, εμφανίζεται αυτόματα μήνυμα σφάλματος κάτω από το πεδίο.

                Η καρτέλα \textbf{Event handlers} (Χειριστές Συμβάντων) είναι ένας τρόπος για να εκτελεστούν συγκεκριμένες ενέργειες (microflows) όταν συμβεί κάποιο συγκεκριμένο συμβάν στην οντότητα. Τα συμβάντα μπορεί να είναι η δημιουργία (create), η ενημέρωση (commit ή save), η διαγραφή (delete) ή η ακύρωση (rollback ή cancel) μιας οντότητας. Μπορεί να επιλεγεί η εκτέλεση του microflow πριν ή μετά την εκτέλεση των παραπάνω συμβάντων.

                Η καρτέλα \textbf{Access Rules} (Κανόνες Πρόσβασης) ορίζει τα δικαιώματα κάθε ρόλου χρήστη όσον αφορά την αλληλεπίδραση με τα γνωρίσματα της οντότητας. Μπορεί να επιλεγεί ποιος ρόλος χρήστη μπορεί να δημιουργήσει ή να διαγράψει στιγμιότυπα από οντότητες και να διαβάσει ή να τροποποιήσει τα γνωρίσματά τους.

                Τέλος, μπορούν να προστεθούν breakpoints στα Microflows ώστε να εντοπιστούν σφάλματα μέσω debugging κατά την εκτέλεσή τους.

            \subsection{Dashboard εφαρμογής}
                Παράλληλα με την ανάπτυξη της εφαρμογής στο Mendix Studio Pro, στο προσωπικό προφίλ κάθε χρήστη υπάρχει ένα dashboard για κάθε εφαρμογή που αναπτύσσουμε όπου παρέχονται project management εργαλεία και μεθοδολογίες ανάπτυξης λογισμικού.

                \begin{figure}[h!] \noindent \centering
                    \includegraphics[width=0.5\textwidth]{Mendix/Scrum}
                    \caption{\centering Πίνακας Kanban στο dashboard \cite{mendixDoc}}
                \end{figure}

%	\chapter{Υλοποίηση εφαρμογής} \label{ch:unitask}
    Σε αυτό το κεφάλαιο παρουσιάζεται η υλοποίηση της εφαρμογής UniTask.  Αρχικά, γίνεται αναφορά στη σχεδιαστική προσέγγιση που ακολουθήθηκε, λαμβάνοντας υπόψη τα αποτελέσματα έρευνας σχετικά με τις προτιμήσεις των φοιτητών. Στη συνέχεια, παρουσιάζεται ένα demo της εφαρμογής με όλες τις δυνατότητές της και στο τέλος εξηγείται η αρχιτεκτονική της εφαρμογής καθώς και τα επιμέρους τεχνικά
    στοιχεία που την απαρτίζουν. Συγκεκριμένα, αναλύονται τα διάφορα modules που χρησιμοποιήθηκαν, τα διαφορετικά layouts που συνθέτουν το περιβάλλον χρήστη, η λειτουργικότητα των διαφορετικών microflows που έχουν σχεδιαστεί και οι ρυθμίσεις της εφαρμογής.

    Στόχος του κεφαλαίου είναι να παρέχει μια ολοκληρωμένη εικόνα της ανάπτυξης του UniTask, παρουσιάζοντας τόσο το frontend όσο και το backend της εφαρμογής, δίνοντας έμφαση στις επιλογές που διασφαλίζουν την ευχρηστία και την αποδοτικότητα της πλατφόρμας.

    \section{Mockups και σχεδιαστική προσέγγιση}
        Στην ενότητα \ref{sec:student_preferences}, παρουσιάστηκε μια έρευνα με τα βασικά χαρακτηριστικά που θεωρήθηκαν απαραίτητα από τους φοιτητές για μια εφαρμογή τους. Λαμβάνοντας υπόψιν τις προτιμήσεις αυτές δόθηκε βάση στην υλοποίησή τους ώστε η εφαρμογή να ανταποκρίνεται στις ανάγκες της ακαδημαϊκής κοινότητας.

        Αρχικά, ως μια εφαρμογή διαχείρισης εργασιών, θα περιλαμβάνει προφανώς ένα σύστημα δημιουργίας, τροποποίησης και διαγραφής εργασιών, καθορισμού του χρόνου έναρξης και λήξεώς τους, και ενός συστήματος κατηγοριοποίησης των εργασιών ανάλογα με το αν έχουν πραγματοποιηθεί, αν πραγματοποιούνται και αν έχουν σκοπό να πραγματοποιηθούν μελλοντικά. Επιπλέον, με βάση την έρευνα, είναι σημαντική η ενσωμάτωση ενός ημερολογίου, η δυνατότητα χρωματικής ταξινόμησης (color-coding) και η υλοποίηση ενός συστήματος ανταμοιβής για την ενίσχυση της παρακίνησης των χρηστών. Επίσης, θα ήταν εξίσου σημαντική η δημιουργία ενός Kanban πίνακα για την άμεση οπτικοποίηση των εργασιών και την ευκολότερη διαχείρισή τους.

        Σχεδιαστικά θεωρείται σημαντική η τήρηση σύγχρονων σχεδιαστικών κανόνων με ένα καθαρό interface και συνοχή στον σχεδιασμό για τη δημιουργία μιας λειτουργικής, αισθητικά ευχάριστης και ευκολόχρηστης εμπειρίας χρήστη ώστε να εξασφαλιστεί ότι η εφαρμογή μπορεί να ανταποκριθεί στις ανάγκες διαφορετικών τύπων χρηστών, αλλά και να παρουσιαστεί ως ένα προϊόν έτοιμο προς υλοποίηση και χρήση και σε πραγματικές συνθήκες.

        Στις εικόνες \ref{fig:unitaskMockupCalendar}, \ref{fig:unitaskMockupDashboard} και \ref{fig:unitaskMockupKanban} παρουσιάζονται κάποια αρχικά mockups που χρησιμοποιήθηκαν για τον σχεδιασμό της εφαρμογής.

        \begin{figure}[h!] \noindent \centering
            \includegraphics[width=0.65\textwidth]{mockups/Calendar}
            \caption{\centering Mockup Calendar σελίδας}
            \label{fig:unitaskMockupCalendar}
        \end{figure}

        \begin{figure}[h!] \noindent \centering
            \includegraphics[width=0.65\textwidth]{mockups/Dashboard}
            \caption{\centering Mockup Dashboard σελίδας}
            \label{fig:unitaskMockupDashboard}
        \end{figure}

        \begin{figure}[h!] \noindent \centering
            \includegraphics[width=0.65\textwidth]{mockups/Kanban}
            \caption{\centering Mockup Kanban σελίδας}
            \label{fig:unitaskMockupKanban}
        \end{figure}

    \pagebreak

    \section{Παρουσίαση της εφαρμογής}
        Κατά την εκτέλεση της εφαρμογής, είτε τοπικά είτε μέσω της απομακρυσμένης πρόσβασης στο cloud, στη διεύθυνση \texttt{https://unitask-sandbox.mxapps.io/}, εμφανίζεται αρχικά η \textbf{σελίδα σύνδεσης}, όπως φαίνεται στην εικόνα \ref{fig:unitask_Login}. Στη συγκεκριμένη σελίδα, οι χρήστες καλούνται να εισάγουν τα στοιχεία σύνδεσής τους για να αποκτήσουν πρόσβαση στις λειτουργίες της εφαρμογής.

        Το σύστημα αναγνωρίζει δύο επίπεδα πρόσβασης, ανάλογα με τα στοιχεία σύνδεσης που εισάγονται: δικαιώματα διαχειριστή (Administrator) και δικαιώματα χρήστη (User). Ο ρόλος του χρήστη (User) αντιστοιχεί σε φοιτητές που κάνουν χρήση της εφαρμογής, ενώ ο ρόλος του διαχειριστή παρέχει επιπλέον λειτουργίες διαχείρισης.

       \begin{figure}[h!] \noindent \centering
            \includegraphics[width=\textwidth]{UniTask/Login}
            \caption{\centering Σελίδα σύνδεσης}
            \label{fig:unitask_Login}
        \end{figure}

        Αρχικά, πραγματοποιείται σύνδεση με τον λογαριασμό διαχειριστή (Administrator) προκειμένου να παρουσιαστούν οι λειτουργίες διαχείρισης χρηστών, συμπεριλαμβανομένης της δυνατότητας προσθήκης νέου χρήστη. Μετά την επιτυχή εισαγωγή των διαπιστευτηρίων του διαχειριστή, τα οποία έχουν οριστεί προκαταβολικά κατά την ανάπτυξη της εφαρμογής (βλ. ενότητα \ref{sec:unitask_mendix}), εμφανίζεται η \textbf{σελίδα διαχείρισης χρηστών}, όπως απεικονίζεται στην εικόνα \ref{fig:unitask_AccountOverview}. Η σελίδα αυτή παρέχει στους διαχειριστές μια ολοκληρωμένη επισκόπηση της λίστας χρηστών της εφαρμογής, καθώς και εργαλεία για τη διαχείρισή τους. Το layout της σελίδας αποτελείται από μια κάθετη μπάρα μενού η οποία περιλαμβάνει τις ίδιες δυνατότητες με τους απλούς χρήστες οι οποίες θα αναλυθούν στη συνέχεια.

       \begin{figure}[h!] \noindent \centering
            \includegraphics[trim={0 12cm 0 0}, clip, width=\textwidth]{UniTask/AccountOverview}
            \caption{\centering Σελίδα διαχείρισης χρηστών}
            \label{fig:unitask_AccountOverview}
        \end{figure}

        Πατώντας στο κουμπί {\Zona New}, εμφανίζεται η αναδυόμενη σελίδα της εικόνας \ref{fig:unitask_NewAccount} με μια \textbf{φόρμα για την προσθήκη νέου χρήστη}. Η φόρμα περιλαμβάνει πεδία για την εισαγωγή του ονόματος χρήστη ({\Zona Username}), του ρόλου του χρήστη ({\Zona User role}) όπου επιλέγεται αν πρόκειται για προσθήκη διαχειριστή ή χρήστη, του κωδικού πρόσβασης ({\Zona New password} και {\Zona Confirm password}). Λόγω του ότι ο χρήστης User έχει κληρονομήσει γνωρίσματα από την κλάση \texttt{System.User} του Mendix, έχουν προστεθεί πεδία όπως το {\Zona Blocked}, η οποία γίνεται αληθής μετά από κάποιες αποτυχημένες προσπάθειες σύνδεσης, το {\Zona Active} που γίνεται αληθές όταν ο χρήστης συνδεθεί, το {\Zona Time zone} όπου ορίζεται η ζώνη ώρας του χρήστη και το {\Zona Language} όπου ορίζεται η γλώσσα του χρήστη.

        \begin{figure}[h!] \noindent \centering
            \includegraphics[width=\textwidth]{UniTask/NewAccount}
            \caption{\centering Φόρμα προσθήκης νέου χρήστη}
            \label{fig:unitask_NewAccount}
        \end{figure}

        Ένας νέος χρήστης δημιουργείται με το όνομα \texttt{Foithths}. Ο χρήστης προστίθεται στη λίστα χρηστών, όπως φαίνεται στην εικόνα \ref{fig:unitask_AccountOverview_WithStudent}. Πατώντας στο όνομά του, εμφανίζεται η σελίδα επεξεργασίας του χρήστη, όπως φαίνεται στην εικόνα \ref{fig:unitask_EditAccount}. Στη λίστα των χρηστών υπάρχει η δυνατότητα αναζήτησης χρηστών βάσει όλων των στοιχείων τους (εικόνα \ref{fig:unitask_SearchAccounts}), όπως επίσης και η δυνατότητα διαγραφής τους.

        \begin{figure}[h!] \noindent \centering
            \includegraphics[trim={0 17cm 0 0}, clip, width=\textwidth]{UniTask/AccountOverview_WithStudent}
            \caption{\centering Λίστα χρηστών με τον χρήστη \texttt{Foithths}}
            \label{fig:unitask_AccountOverview_WithStudent}
        \end{figure}

        \begin{figure}[h!] \noindent \centering
            \includegraphics[trim={0 3cm 0 0}, clip, width=\textwidth]{UniTask/EditAccount}
            \caption{\centering Επεξεργασία στοιχείων χρήστη}
            \label{fig:unitask_EditAccount}
        \end{figure}

        \begin{figure}[h!] \noindent \centering
            \includegraphics[trim={0 7cm 0 0}, clip, width=\textwidth]{UniTask/SearchAccounts}
            \caption{\centering Αναζήτηση χρήστη}
            \label{fig:unitask_SearchAccounts}
        \end{figure}

        Μετά την αποσύνδεση από τον λογαριασμό διαχειριστή και τη σύνδεση ως \texttt{Foithths}, εμφανίζεται η \textbf{αρχική σελίδα} της εφαρμογής (εικόνα \ref{fig:unitask_Home}). Η σελίδα περιλαμβάνει ένα κεντρικό call to action κουμπί ({\Zona όλες οι εργασίες}) το οποίο οδηγεί στο {\Zona dashboard}.

        \begin{figure}[h!] \noindent \centering
            \includegraphics[width=\textwidth]{UniTask/Home}
            \caption{\centering Αρχική σελίδα εφαρμογής}
            \label{fig:unitask_Home}
        \end{figure}

        \begin{figure}[h!] \noindent \centering
            \includegraphics[width=\textwidth]{UniTask/TaskDashboard}
            \caption{\centering Dashboard εργασιών}
            \label{fig:unitask_TaskDashboard}
        \end{figure}

        Στην εικόνα \ref{fig:unitask_TaskDashboard} εμφανίζεται η σελίδα {\ZonaSB dashboard}. Πρόκειται για την κεντρική σελίδα προβολής, δημιουργίας και επεξεργασίας των εργασιών του χρήστη. Περιλαμβάνονται τρεις καρτέλες ({\Zona Επόμενες εργασίες}, {\Zona Εργασίες σε εξέλιξη}, {\Zona Ολοκληρωμένες εργασίες}). Οι επόμενες εργασίες αφορούν εργασίες που έχουν σκοπό να πραγματοποιηθούν στο άμεσο μέλλον αλλά όχι τη δεδομένη χρονική στιγμή, οι εργασίες σε εξέλιξη αφορούν εργασίες που βρίσκονται σε εξέλιξη και οι ολοκληρωμένες εργασίες αφορούν εργασίες που έχουν ολοκληρωθεί.

        Στη σελίδα επίσης περιλαμβάνεται ένα επεξηγηματικό παράθυρο που εμφανίζεται μόνο όταν ο χρήστης δεν έχει δημιουργήσει κάποια εργασία και του εξηγεί το τρόπο λειτουργίας της εφαρμογής. Στο {\Zona dashboard} επίσης περιλαμβάνονται μετρητές για το σύνολο των εργασιών που υπάρχουν ανά κατηγορία, όπως επίσης και ένα κεντρικό κουμπί δημιουργίας εργασιών ({\Zona Νέα εργασία}).

        \begin{figure}[h!] \noindent \centering
            \includegraphics[width=\textwidth]{UniTask/NewTask_Default}
            \includegraphics[width=\textwidth]{UniTask/NewTask_Finished}
            \caption{\centering Δημιουργία νέας εργασίας}
            \label{fig:unitask_NewTask}
        \end{figure}

        Πατώντας το, εμφανίζεται ένα αναδυόμενο παράθυρο (εικόνα \ref{fig:unitask_NewTask}.1) με μια φόρμα για τη δημιουργία μιας νέας εργασίας. Η φόρμα αυτή περιλαμβάνει πεδία για την εισαγωγή του τίτλου της εργασίας ({\Zona Τίτλος}), της ημερομηνίας έναρξης ({\Zona Εκκίνηση}) και λήξης ({\Zona Λήξη}), της κατάστασης της εργασίας ({\Zona Κατάσταση}), της προτεραιότητας της εργασίας ({\Zona Προτεραιότητα}) όπως επίσης και του χρώματος της εργασίας ({\Zona Χρώμα}), όπως θα αποτυπωθεί μετέπειτα στο ημερολόγιο. Στο αναδυόμενο παράθυρο ορίζεται προεπιλεγμένα ως ημερομηνία λήξης της εργασίας μια εβδομάδα μετέπειτα από την ημερομηνία δημιουργίας της, ενώ η κατάσταση της εργασίας έχει προκαθοριστεί (γίνεται να τροποποιηθεί φυσικά) ανάλογα με το ποια καρτέλα ήταν ανοιχτή στο dashboard. Αφού επεξεργαστούμε το παράθυρο όπως επιθυμούμε (εικόνα \ref{fig:unitask_NewTask}.2), πατάμε αποθήκευση για την αποθήκευση της εργασίας.

        Σημειώνεται ότι οι επιλογές για την κατάσταση και την προτεραιότητα της εργασίας είναι χρωματικά κωδικοποιημένες (color-coded) και συνοδεύονται από γραφικά σύμβολα, όπως φαίνεται στην εικόνα \ref{fig:unitask_NewTask_AllOptions} με σκοπό να διευκολύνει τον χρήστη για την άμεση αναγνώριση τους.

        \begin{figure}[h!] \noindent \centering
            \includegraphics[width=0.7\textwidth]{UniTask/NewTask_AllOptions}
            \caption{\centering Επιλογές κατάστασης και προτεραιότητας εργασίας}
            \label{fig:unitask_NewTask_AllOptions}
        \end{figure}

        \begin{figure}[h!] \noindent \centering
            \includegraphics[trim={0 10cm 0 0}, clip, width=\textwidth]{UniTask/TaskDashboard_WithTask}
            \caption{\centering Dashboard με δημιουργημένη εργασία}
            \label{fig:unitask_TaskDashboard_WithTask}
        \end{figure}

        Στη σελίδα πλέον είναι δημιουργημένη η κάρτα με την πρώτη μας εργασία (εικόνα \ref{fig:unitask_TaskDashboard_WithTask}). Το πλαίσιο {\Zona σύντομη σημείωση} επιτρέπει την εισαγωγή μιας συνοπτικής περιγραφής της εργασίας, στα αριστερά εμφανίζεται ο τίτλος της εργασίας, μια μπάρα προόδου (progress bar) η οποία δυναμικά αυξάνεται όσο πλησιάζουμε στη λήξη της εργασίας, όπως επίσης η προτεραιότητα της εργασίας και οι ημερομηνίες και ώρες εκκίνησης και λήξης της εργασίας, ενώ στο δεξί μέρος της κάρτας εμφανίζεται το εικονίδιο για την επεξεργασία της εργασίας.

        Η ταξινόμηση των εργασιών στο dashboard βασίζεται στην προτεραιότητα και στον χρόνο λήξης τους. Έτσι μια εργασία με υψηλή προτεραιότητα θα εμφανίζεται ψηλότερα από μια εργασία με χαμηλή προτεραιότητα που δε λήγει σύντομα.

        \begin{figure}[h!] \noindent \centering
            \includegraphics[width=\textwidth]{UniTask/EditTask}
            \caption{\centering Επεξεργασία εργασίας}
            \label{fig:unitask_EditTask}
        \end{figure}

        Με την επιλογή του εικονιδίου επεξεργασίας, εμφανίζεται το αναδυόμενο παράθυρο (εικόνα \ref{fig:unitask_EditTask}), το οποίο επιτρέπει την τροποποίηση των στοιχείων της εργασίας, συμπεριλαμβανομένου του πλαισίου {\Zona σύντομη σημείωση}. Το ίδιο παράθυρο περιλαμβάνει και το κουμπί διαγραφής της εργασίας.

        Για την καλύτερη διευκόλυνση του χρήστη, την τελευταία εβδομάδα πριν λήξη της προθεσμίας εμφανίζεται ενημερωτικό κείμενο στην κάρτα (εικόνα \ref{fig:unitask_cardInfo}.1), ενώ όταν η εργασία έχει λήξει, εμφανίζεται call-to-action κουμπί που μπορεί να μετακινήσει αυτήν την εργασία στις ολοκληρωμένες εργασίες (εικόνα \ref{fig:unitask_cardInfo}.2).

        \begin{figure}[h!] \noindent \centering
            \includegraphics[width=\textwidth]{UniTask/προθεσμία λήγει}
            \includegraphics[width=\textwidth]{UniTask/προθεσμία έληξε}
            \caption{\centering Ενημερωτικό κείμενο κάρτας εργασίας}
            \label{fig:unitask_cardInfo}
        \end{figure}

        Επίσης, κατά την αλλαγή κατηγορίας της εργασίας σε ολοκληρωμένη, εμφανίζονται κομφετί ως ένα σύστημα ανταμοιβής για τον χρήστη (εικόνα \ref{fig:unitask_confetti}).

        \begin{figure}[h!] \noindent \centering
            \includegraphics[width=0.7\textwidth]{UniTask/κομφετί}
            \caption{\centering Επιβράβευση όταν ολοκληρωθεί μια εργασία}
            \label{fig:unitask_confetti}
        \end{figure}

        \begin{figure}[h!] \noindent \centering
            \includegraphics[trim={0 10cm 0 0}, clip, width=\textwidth]{UniTask/Kanban_WithTask}
            \caption{\centering Σελίδα Kanban}
            \label{fig:unitask_Kanban_WithTask}
        \end{figure}

        Η σελίδα {\ZonaSB Kanban} της εικόνας \ref{fig:unitask_Kanban_WithTask} εμφανίζει μια διαφορετική παρουσίαση στις εργασίες με τον τρόπο που έχει εξηγηθεί στην ενότητα \ref{subsec:Kanban}. Στον πίνακα Kanban οι εργασίες χωρίζονται σε τρεις κατηγορίες: τις εργασίες που έχουν ολοκληρωθεί, τις εργασίες που βρίσκονται σε εξέλιξη και τις εργασίες που έχουν σκοπό να πραγματοποιηθούν στο μέλλον. Κάθε στήλη περιλαμβάνει ένα κουμπί δημιουργίας νέας εργασίας που αυτόματα καθορίζει και την κατηγορία της.

        Οι εργασίες απεικονίζονται ως κάρτες που περιλαμβάνουν τον τίτλο, την προτεραιότητα, καθώς και την ημερομηνία έναρξης και λήξης τους, ενώ πατώντας πάνω στην κάρτα μιας εργασίας, εμφανίζεται το αναδυόμενο παράθυρο επεξεργασίας της εργασίας της εικόνας \ref{fig:unitask_EditTask}.

        Στη σελίδα {\ZonaSB ημερολόγιο} (εικόνα \ref{fig:unitask_Calendar_WithTask}) εμφανίζεται ένα ημερολόγιο με τις εργασίες του χρήστη. Οι εργασίες εμφανίζονται στο ημερολόγιο με το χρώμα που έχει οριστεί στην επιλογή του χρήστη κατά τη δημιουργία της εργασίας, υπάρχουν προβολές ανά ημέρα, εβδομάδα ή μήνα ενώ εμφανίζεται και ένα σύντομο παράρτημα στα δεξιά με τη λίστα των εργασιών.

        \begin{figure}[h!] \noindent \centering
            \includegraphics[width=\textwidth]{UniTask/Calendar_WithTask}
            \caption{\centering Σελίδα Calendar}
            \label{fig:unitask_Calendar_WithTask}
        \end{figure}

        Η σελίδα {\ZonaSB ρυθμίσεις} (εικόνα \ref{fig:unitask_Settings}) παρέχει επιλογές για τη μαζική διαγραφή εργασιών ή την ενεργοποίηση της λειτουργίας γρήγορης διαγραφής. Αυτή η λειτουργία εισάγει ένα κουμπί διαγραφής στις κάρτες εργασιών στο dashboard, επιτρέποντας την άμεση διαγραφή των εργασιών που θέλουμε χωρίς να χρειάζεται να μεταβούμε στη σελίδα επεξεργασίας κάθε εργασίας (εικόνα \ref{fig:unitask_Dashboard_Kanban_Calendar_WithTasks}.1). Τέλος, παρέχεται η δυνατότητα αρχικοποίησης των εργασιών. Στην αρχικοποίηση διαγράφονται οι υπάρχουσες εργασίες του χρήστη και δημιουργούνται κάποιες προκαθορισμένες οι οποίες λειτουργούν ως ένα demo (εικόνα \ref{fig:unitask_Dashboard_Kanban_Calendar_WithTasks}).

        \begin{figure}[h!] \noindent \centering
            \includegraphics[trim={0 20cm 0 0}, clip, width=\textwidth]{UniTask/Settings_1}
            \includegraphics[width=\textwidth]{UniTask/Settings_2}
            \caption{\centering Σελίδα ρυθμίσεων}
            \label{fig:unitask_Settings}
        \end{figure}

        Σε κάθε ρύθμιση εμφανίζονται επιβεβαιωτικά αναδυόμενα μηνύματα προκειμένου να αποφευχθούν ακούσιες διαγραφές εργασιών (εικόνα \ref{fig:unitask_InitializationPopUp}).

        \begin{figure}[h!] \noindent \centering
            \includegraphics[trim={0 10cm 0 0}, clip, width=\textwidth]{UniTask/InitializationPopUp}
            \caption{\centering Σελίδα ρυθμίσεων}
            \label{fig:unitask_InitializationPopUp}
        \end{figure}

        \begin{figure}[p!] \noindent \centering
            \includegraphics[trim={0 5cm 0 0}, clip, width=\textwidth]{UniTask/TaskDashboard_QuickDelete}
            \includegraphics[trim={0 10cm 0 0}, clip, width=\textwidth]{UniTask/Kanban_WithTasks}
            \includegraphics[width=\textwidth]{UniTask/Calendar_WithTasks}
            \caption{\centering Οι σελίδες Dashboard, Kanban και Calendar μετά την αρχικοποίηση των εργασιών. Στο Dashboard φαίνεται η λειτουργία γρήγορης διαγραφής εργασιών.}
            \label{fig:unitask_Dashboard_Kanban_Calendar_WithTasks}
        \end{figure}

    \section{Δομή της εφαρμογής} \label{sec:unitask_mendix}
        Πέρα από τα προκατασκευασμένα modules του Mendix, η λειτουργικότητα της εφαρμογής έχει οργανωθεί σε τρία modules: το \texttt{Administrator}, το \texttt{TaskManager} και το \texttt{UniTask}.

        \subsection{Module \texttt{Administrator}}
            Το \texttt{Administrator} περιλαμβάνει τη λειτουργικότητα που αφορά τη διαχείριση των χρηστών της εφαρμογής. Όλες οι οντότητες του domain model, οι σελίδες και τα microflows του module έχουν δικαιώματα ανάγνωσης και εγγραφής από τον ρόλο \texttt{Administrator}, όπως ορίζεται στο \texttt{Security} της εφαρμογής.

            \subsubsection{Domain model του \texttt{Administrator}}
                \begin{figure}[H] \noindent \centering
                    \includegraphics[width=\textwidth]{UniTask_Mendix/Administrator_DomainModel}
                    \caption{\centering Domain model του \texttt{Administrator}}
                    \label{fig:unitask_Administrator_DomainModel}
                \end{figure}

                Το domain model του \texttt{Administrator} (εικόνα \ref{fig:unitask_Administrator_DomainModel}) περιλαμβάνει την οντότητα \texttt{Student} που κληρονομεί την οντότητα \texttt{System.User} του Mendix. Η οντότητα περιγράφει τον κάθε χρήστη της εφαρμογής και περιλαμβάνει την Boolean ιδιότητα \texttt{buttonQuickDelete} αρχικοποιημένη σε \texttt{False} η οποία χρησιμοποιείται για την ενεργοποίηση της λειτουργίας γρήγορης διαγραφής εργασιών. Η οντότητα \texttt{Student} συσχετίζεται με την οντότητα \texttt{Account} του \texttt{System.User} με σχέση 1-προς-1, η οντότητα \texttt{Task} του \texttt{TaskManager} με σχέση ένα-προς-πολλά (ένα Student συσχετίζεται με πολλά Tasks) και την οντότητα \texttt{AccountPasswordData} με σχέση 1-προς-πολλά (ένα Student συσχετίζεται με πολλά AccountPasswordData).

                Η οντότητα \texttt{AccountPasswordData} είναι μη-διατηρήσιμη οντότητα (δεν αποθηκεύεται στη βάση δεδομένων αλλά μόνο στη μνήμη) και περιλαμβάνει τις ιδιότητες \texttt{OldPassword}, \texttt{NewPassword} και \texttt{ConfirmPassword} και χρησιμοποιείται για την αλλαγή του κωδικού πρόσβασης του χρήστη.

            \subsubsection{Σελίδες του \texttt{Administrator}}
                Στο \texttt{Administrator} περιλαμβάνονται οι εξής σελίδες:

                \begin{figure}[H] \noindent
                    \paragraph{\texttt{Account\_Overview}}
                    \begin{center}
                        \includegraphics[width=\textwidth]{UniTask_Mendix/Account_Overview}
                        \caption{\centering Σελίδα διαχείρισης χρηστών \texttt{Account\_Overview} σε X-Ray Mode}
                        \label{fig:unitask_Account_Overview}
                    \end{center}
                \end{figure}

                    Η σελίδα (εικόνα \ref{fig:unitask_Account_Overview}) χρησιμοποιείται για τη διαχείριση των χρηστών της εφαρμογής από την πλευρά των διαχειριστών.

                    Χρησιμοποιείται το \texttt{UniTask\_SideBar} layout του \texttt{UniTaskDesignSystem} module. Το κύριο μέρος της σελίδας αποτελείται από ένα Data Grid με Data source την οντότητα \texttt{Student} και με στήλες τις ιδιότητες \texttt{Name}, \texttt{Last login}, \texttt{Blocked}, \texttt{Blocked since}, \texttt{Active}, \texttt{Web service user} και \texttt{Is anonymous}.

                \newpage

                \begin{figure}[H] \noindent
                    \paragraph{\texttt{Account\_New}}
                    \begin{center}
                        \includegraphics[width=\textwidth]{UniTask_Mendix/Account_New}
                        \caption{\centering Σελίδα δημιουργίας νέου χρήστη \texttt{Account\_New} σε X-Ray Mode}
                        \label{fig:unitask_Account_New}
                    \end{center}
                \end{figure}

                    Η σελίδα (σελίδα \ref{fig:unitask_Account_New}) χρησιμοποιείται για τη δημιουργία νέων χρηστών της εφαρμογής.

                    Χρησιμοποιείται το \texttt{PopupLayout} layout του \texttt{Atlas\_Core} module. Η σελίδα περιλαμβάνει δύο Parameters, το \texttt{Student} και \texttt{AccountPasswordData} του module \linebreak \texttt{Administrator}. Η σελίδα αποτελείται από δύο εμφωλευμένα Data Views, το εξωτερικό έχει ως Data source το \texttt{AccountPasswordData}, ενώ το εσωτερικό έχει ως Data source τη συσχέτιση του \texttt{AccountPasswordData} με το \texttt{Student}. Η χρήση του \texttt{AccountPasswordData} είναι απαραίτητη καθώς η δημιουργία ενός νέου χρήστη χρειάζεται την αποθήκευση του κωδικού πρόσβασής του.

                    Στο εσωτερικό Data View περιλαμβάνει Text Boxes, Radio Buttons και Input \linebreak Reference Set Selectors όπου εισάγονται τιμές για τα \texttt{Username}, \texttt{Blocked}, \texttt{Active}, \texttt{User role}, \texttt{Language}, \texttt{Time zone}, \texttt{New password} και \texttt{Confirm password}. Έχει σημασία να σημειωθεί πως οι ιδιότητες (γνωρίσματα) που αποθηκεύουμε στην πραγματικότητα δεν είναι ιδιότητες του \texttt{Student} αλλά του \texttt{System.User} του οποίου αποτελεί παιδί. Το Input Reference Set Selector χρησιμοποιείται για την επιλογή του \texttt{UserRole}, που αποτελεί διαφορετική σελίδα που θα αναλυθεί στη συνέχεια.

                    Τέλος, περιλαμβάνεται κουμπί για την αποθήκευση, το οποίο καλεί το microflow \texttt{ACT\_Account\_Save} του \texttt{Administrator} για την αποθήκευση των τιμών, και κουμπί για την ακύρωση της διαδικασίας.

                \begin{figure}[H] \noindent
                    \paragraph{\texttt{Account\_Edit}}
                    \begin{center}
                        \includegraphics[width=\textwidth]{UniTask_Mendix/Account_Edit}
                        \caption{\centering Σελίδα επεξεργασίας χρήστη \texttt{Account\_Edit} σε X-Ray Mode}
                        \label{fig:unitask_Account_Edit}
                    \end{center}
                \end{figure}

                    Η σελίδα (εικόνα \ref{fig:unitask_Account_Edit}) χρησιμοποιείται για την επεξεργασία υπαρχόντων χρηστών της εφαρμογής.

                    Χρησιμοποιείται το \texttt{PopupLayout}. Η σελίδα περιλαμβάνει το Parameter \texttt{Student}. Η σελίδα αποτελείται από ένα Data View με Data source το \texttt{Student} με παρόμοια Text Boxes και Radio Buttons όπως και το \texttt{Account\_New}. Επίσης, περιλαμβάνεται το κουμπί που καλεί το microflow \texttt{ACT\_Password\_Change} για την αλλαγή κωδικού.

                    Τέλος, περιλαμβάνεται κουμπί για την αποθήκευση και κουμπί για την ακύρωση της διαδικασίας. Τα κουμπιά καλούν προεπιλεγμένες ενέργειες του Mendix.

                \newpage

                \begin{figure}[H] \noindent
                    \paragraph{\texttt{Change\_Password}}
                    \begin{center}
                        \includegraphics[width=\textwidth]{UniTask_Mendix/Change_Password}
                        \caption{\centering Σελίδα αλλαγή κωδικού πρόσβασης \texttt{Change\_Password} σε X-Ray Mode}
                        \label{fig:unitask_Change_Password}
                    \end{center}
                \end{figure}

                    Η σελίδα (εικόνα \ref{fig:unitask_Change_Password}) χρησιμοποιείται για την αλλαγή του κωδικού πρόσβασης του υπάρχοντος χρήστη.

                    Χρησιμοποιείται το \texttt{PopupLayout}. Η σελίδα περιλαμβάνει το Parameter \linebreak \texttt{AccountPasswordData}, ένα Data View με Data source το \texttt{Student} με τα απαραίτητα Text Boxes για την αλλαγή των τιμών, κουμπί αποθήκευσης που καλεί το microflow \texttt{ChangePassword} του \texttt{Administrator} και κουμπί για την ακύρωση της διαδικασίας.

                \begin{figure}[H] \noindent
                    \paragraph{\texttt{UserRole\_Select}}
                    \begin{center}
                        \includegraphics[width=\textwidth]{UniTask_Mendix/UserRole_Select}
                        \caption{\centering Σελίδα επιλογής ρόλου χρήστη \texttt{UserRole\_Select} σε X-Ray Mode}
                        \label{fig:unitask_UserRole_Select}
                    \end{center}
                \end{figure}

                    Η σελίδα (εικόνα \ref{fig:unitask_UserRole_Select}) χρησιμοποιείται για την επιλογή του ρόλου του χρήστη κατά τη δημιουργία νέου χρήστη.

                    Χρησιμοποιείται το \texttt{PopupLayout}. Η σελίδα αποτελείται από ένα Data Grid με Data source το \texttt{UserRole} του \texttt{System}.\footnote{Τεχνικά, λόγω των ρόλων χρηστών που έχουν κληρονομηθεί από το Mendix περιλαμβάνεται και ο ρόλος Guest, ο οποίος στην πράξη δε χρησιμοποιείται καθώς έχει πρόσβαση μόνο στη σελίδα σύνδεσης.}

            \subsubsection{Microflows του \texttt{Administrator}}
                Στο \texttt{Administrator} περιλαμβάνονται τα εξής microflows:

                \begin{figure}[H] \noindent
                    \paragraph{\texttt{VAL\_Account}}
                    \begin{center}
                        \includegraphics[width=\textwidth]{UniTask_Mendix/VAL_Account}
                        \caption{\centering Το microflow \texttt{VAL\_Account}}
                        \label{fig:unitask_VAL_Account}
                    \end{center}
                \end{figure}

                    To microflow (εικόνα \ref{fig:unitask_VAL_Account}) καλείται από το microflow \texttt{ACT\_Account\_Save} για να επικυρώσει τον λογαριασμό του χρήστη πριν αποθηκευτούν οι τιμές του.\footnote{Το πρόθεμα \texttt{VAL} χρησιμοποιείται στην ονομασία των microflows για να δηλώσει επικύρωση (validation).}

                    Αρχικά δημιουργείται μια boolean μεταβλητή \texttt{Valid} με αρχική τιμή \texttt{True} η οποία θα επιστραφεί από το microflow. Στη συνέχεια ελέγχεται αν για το αντικείμενο \texttt{Account} τύπου \texttt{Student} ισχύει η συνθήκη (\verb|trim($Account/Name) != ''|). Η έκφραση στη συνθήκη αφού καθαρίσει τα κενά (whitespaces) από το \texttt{Name} του \texttt{Account}, ελέγχει αν είναι διαφορετικό από το κενό string. Αν η συνθήκη δεν ισχύει, τότε η μεταβλητή \texttt{Valid} γίνεται \texttt{False}, η οποία επιστρέφεται μαζί με ένα popout μήνυμα. Αν η συνθήκη ισχύει, δηλαδή αν υπάρχει όνομα, τότε επιστρέφεται \texttt{True}.

                \begin{figure}[H] \noindent
                    \paragraph{\texttt{ACT\_Account\_New}}
                    \begin{center}
                        \includegraphics[width=\textwidth]{UniTask_Mendix/ACT_Account_New}
                        \caption{\centering Το microflow \texttt{ACT\_Account\_New}}
                        \label{fig:unitask_ACT_Account_New}
                    \end{center}
                \end{figure}

                    To microflow (εικόνα \ref{fig:unitask_ACT_Account_New}) καλείται από τη σελίδα \texttt{Account\_Overview} με σκοπό τη δημιουργία ενός νέου χρήστη.\footnote{Το πρόθεμα \texttt{ACT} χρησιμοποιείται στην ονομασία των microflows για να δηλώσει μια ενέργεια (action).}

                    Αρχικά δημιουργούνται δύο στιγμιότυπα τύπου \texttt{Student} και \texttt{AccountPasswordData} με ονόματα \texttt{NewAccount} και \texttt{NewAccountPasswordData} αντίστοιχα. Να σημειωθεί πως το \texttt{NewAccountPasswordData} συσχετίζεται με το \texttt{Student}. Τα αντικείμενα δε γίνονται commit ακόμα στη βάση, καθώς είναι κενά. Στη συνέχεια εμφανίζεται η σελίδα \texttt{Account\_New} με τα αντικείμενα \texttt{NewAccount} και \texttt{NewAccountPasswordData} ως Parameters.

                \begin{figure}[H] \noindent
                    \paragraph{\texttt{ACT\_Account\_Save}}
                    \begin{center}
                        \includegraphics[width=\textwidth]{UniTask_Mendix/ACT_Account_Save}
                        \caption{\centering Το microflow \texttt{ACT\_Account\_Save}}
                        \label{fig:unitask_ACT_Account_Save}
                    \end{center}
                \end{figure}

                    To microflow (εικόνα \ref{fig:unitask_ACT_Account_Save}) καλείται από τη σελίδα \texttt{Account\_New} με σκοπό την αποθήκευση των τιμών του νέου χρήστη.

                    Το microflow έχει ως Parameter το \texttt{AccountPasswordData}. Μαζί με αυτό, ανακτάται το \texttt{Student} αφού συσχετίζονται, και καλείται το microflow \texttt{VAL\_Account} το οποίο ελέγχει αν το \texttt{Name} του \texttt{Student} είναι κενό. Αν το \texttt{Name} είναι κενό, τότε εμφανίζεται ένα popout μήνυμα και το microflow τερματίζεται. Αν το \texttt{Name} δεν είναι κενό, τότε ελέγχεται αν το \texttt{NewPassword} του \texttt{AccountPasswordData} είναι ίσο με το \texttt{ConfirmPassword}, όπως έχουν δοθεί στη φόρμα \texttt{Account\_New}. Αν η συνθήκη δεν ισχύει, τότε εμφανίζεται ένα popout μήνυμα και το microflow τερματίζεται. Αν η συνθήκη ισχύει, τότε το \texttt{NewPassword} γίνεται commit στο \texttt{Account} τύπου \texttt{Student} στο γνώρισμα \texttt{Password} το οποίο είναι Hashed string. Στη συνέχεια το αντικείμενο \texttt{AccountPasswordData} διαγράφεται και κλείνει η σελίδα.

                \begin{figure}[H] \noindent
                    \paragraph{\texttt{ACT\_Account\_Edit}}
                    \begin{center}
                        \includegraphics[width=\textwidth]{UniTask_Mendix/ACT_Account_Edit}
                        \caption{\centering To microflow \texttt{ACT\_Account\_Edit}}
                        \label{fig:unitask_ACT_Account_Edit}
                    \end{center}
                \end{figure}

                    To microflow (εικόνα \ref{fig:unitask_ACT_Account_Edit}) καλείται από τη σελίδα \texttt{Account\_Overview} με σκοπό την επεξεργασία ενός υπάρχοντος χρήστη.

                    Το microflow εμφανίζει τη σελίδα \texttt{Account\_Edit} με το \texttt{Account} τύπου \texttt{Student} ως Parameter.

                \begin{figure}[H] \noindent
                    \paragraph{\texttt{VAL\_Password}}
                    \begin{center}
                        \includegraphics[width=\textwidth]{UniTask_Mendix/VAL_Password}
                        \caption{\centering Το microflow \texttt{VAL\_Password}}
                        \label{fig:unitask_VAL_Password}
                    \end{center}
                \end{figure}

                    To microflow (εικόνα \ref{fig:unitask_VAL_Password}) καλείται από το microflow \texttt{ChangePassword} με σκοπό τον έλεγχο των συνθηκών για την αλλαγή του κωδικού πρόσβασης.

                    Αρχικά δημιουργείται μια boolean μεταβλητή \texttt{Valid} με αρχική τιμή \texttt{True}. Στη συνέχεια ελέγχεται αν για το \texttt{NewPassword} του αντικειμένου \texttt{AccountPasswordData} ισχύει η συνθήκη (\verb|trim($AccountPasswordData/NewPassword) != ''|). Η έκφραση στη συνθήκη ελέγχει αν έχει δοθεί όντως νέος κωδικός στη φόρμα της σελίδας \texttt{Change\_Password}. Αν η συνθήκη ισχύει, ελέγχεται με παρόμοιο τρόπο και το \texttt{ConfirmPassword} όπως επίσης και το αν είναι ίσο με το \texttt{NewPassword}. Αν κάποια συνθήκη από τις προαναφερθείσες δεν ισχύει, η μεταβλητή \texttt{Valid} γίνεται \texttt{False} και εμφανίζεται κατάλληλο popout μήνυμα. Αν όλες οι συνθήκες ισχύουν, τότε η μεταβλητή \texttt{Valid} παραμένει \texttt{True} και επιστρέφεται από το microflow.

                \begin{figure}[H] \noindent
                    \paragraph{\texttt{ChangePassword}}
                    \begin{center}
                        \includegraphics[width=\textwidth]{UniTask_Mendix/ChangePassword}
                        \caption{\centering Tο microflow \texttt{ChangePassword}}
                        \label{fig:unitask_ChangePassword}
                    \end{center}
                \end{figure}

                    To microflow (εικόνα \ref{fig:unitask_ChangePassword}) καλείται από τη σελίδα \texttt{Change\_Password} με σκοπό την αποθήκευση του νέου κωδικού πρόσβασης για έναν υπάρχοντα χρήστη.

                    Το microflow έχει ως Parameter το \texttt{AccountPasswordData}. Στην αρχή καλείται το microflow \texttt{VAL\_Password} για τον έλεγχο των συνθηκών. Αν η μεταβλητή \texttt{Valid} που επιστρέφεται από το microflow είναι \texttt{False}, τότε το microflow τερματίζεται. Αν είναι \texttt{True}, τότε γίνεται retrieve και το αντικείμενο \texttt{Student} ως συσχέτιση, γίνεται commit το \texttt{NewPassword} ως Hashed string \texttt{Password} στο \texttt{Student}, εμφανίζεται κατάλληλο popout μήνυμα, διαγράφεται το \texttt{AccountPasswordData} και κλείνει η σελίδα.

                \newpage

                \begin{figure}[H] \noindent
                    \paragraph{\texttt{ACT\_Password\_Change}}
                    \begin{center}
                        \includegraphics[width=\textwidth]{UniTask_Mendix/ACT_Password_Change}
                        \caption{\centering Το microflow \texttt{ACT\_Password\_Change}}
                        \label{fig:unitask_ACT_Password_Change}
                    \end{center}
                \end{figure}

                    To microflow (εικόνα \ref{fig:unitask_ACT_Password_Change}) καλείται από τη σελίδα \texttt{Account\_Edit} με σκοπό την αλλαγή του κωδικού πρόσβασης ενός υπάρχοντος χρήστη.

                    Με Parameter το \texttt{Account} τύπου \texttt{Student}, αρχικά ελέγχεται αν υπάρχει όντως κάποιο υπαρκτό \texttt{Account}. Αν δεν υπάρχει, εμφανίζεται popout μήνυμα και το microflow τερματίζεται. Αν υπάρχει, τότε δημιουργείται ένα νέο αντικείμενο \texttt{AccountPasswordData} συσχετισμένο με το \texttt{Account} και εμφανίζεται η σελίδα \texttt{Change\_Password} με το \texttt{AccountPasswordData} και \texttt{Account} ως Parameters.

                \begin{figure}[H] \noindent
                    \paragraph{\texttt{ASU\_Administrator\_Create}}
                    \begin{center}
                        \includegraphics[width=\textwidth]{UniTask_Mendix/ASU_Administrator_Create}
                        \caption{\centering Το microflow \texttt{ASU\_Administrator\_Create}}
                        \label{fig:unitask_ASU_Administrator_Create}
                    \end{center}
                \end{figure}

                    Microflow (εικόνα \ref{fig:unitask_ASU_Administrator_Create}) που καλείται κατά την αρχικοποίηση της εφαρμογής για τη δημιουργία του διαχειριστή της εφαρμογής. \footnote{Το πρόθεμα \texttt{ASU} (After Startup) χρησιμοποιείται στην ονομασία των microflows για να δηλώσει ότι καλείται αμέσως μετά την εκκίνηση της εφαρμογής.}

                    Αρχικά δημιουργείται το String \texttt{AccountName} με τιμή \texttt{'admin'} με το username του διαχειριστή. Στη συνέχεια γίνεται retrieve από τη βάση δεδομένων το \texttt{Student} που έχει ως \texttt{Name} το \texttt{AccountName}. Αν δεν υπάρχει τέτοιος λογαριασμός, τότε γίνονται retrieve τα \texttt{System.UserRoles} και δημιουργείται ένα νέο αντικείμενο \texttt{Student} με \texttt{Name} το \texttt{AccountName}, \texttt{Password} η τιμή \texttt{'admin'} και \texttt{UserRole} το \texttt{UserRole}. Αν υπάρχει ήδη λογαριασμός με το \texttt{AccountName}, τότε αλλάζει ο κωδικός σε \texttt{'admin'}. Είναι προφανές ότι τα στοιχεία σύνδεσης του διαχειριστή έχουν οριστεί ως \texttt{'admin'} και \texttt{'admin'}.

        \subsection{Module \texttt{TaskManager}}
            Το \texttt{TaskManager} περιλαμβάνει τη λειτουργικότητα που αφορά τη διαχείριση των εργασιών της εφαρμογής. Όλες οι οντότητες του domain model, οι σελίδες και τα microflows του module έχουν δικαιώματα ανάγνωσης και εγγραφής από τον \texttt{User} ρόλο, όπως ορίζεται στο \texttt{Security} της εφαρμογής, με εξαίρεση το \texttt{Custom\_LogIn\_Page} που έχει δικαίωμα ο \texttt{Guest}.

            \subsubsection{Domain model του \texttt{TaskManager}}
                \begin{figure}[H] \noindent \centering
                    \includegraphics[width=\textwidth]{UniTask_Mendix/TaskManager_DomainModel}
                    \caption{\centering Domain model του \texttt{TaskManager}}
                    \label{fig:unitask_TaskManager_DomainModel}
                \end{figure}

                Το domain model του \texttt{TaskManager} (εικόνα \ref{fig:unitask_TaskManager_DomainModel}) περιλαμβάνει την οντότητα \texttt{Task} και τη μη-διατηρήσιμη οντότητα \texttt{PageHelper}.

                Η οντότητα \texttt{Task} αναπαριστά την εκάστοτε εργασία του χρήστη της εφαρμογής. Περιλαμβάνει τις ιδιότητες \texttt{Title} τύπου String ως 200 χαρακτήρες με το όνομα της εργασίας, \texttt{Notes} τύπου String με απεριόριστους χαρακτήρες όπου μπορούν να προστεθούν σημειώσεις για αυτή και \texttt{DueDate} τύπου Date and time με την ημερομηνία λήξης. Επίσης, περιλαμβάνει τη \texttt{StartDate} τύπου Date and time με την ημερομηνία έναρξης, η οποία αρχικοποιείται με την τιμή \verb|'%CurrentDateTime%]'| (Token που επιστρέφει την τρέχουσα ημερομηνία και ώρα) και τη \texttt{CalendarColor} τύπου String που αποθηκεύει το χρώμα της εργασίας στο ημερολόγιο. Λόγω της φύσης του widget του ημερολογίου, το String θα έχει πάντα τη μορφή \texttt{'rgb(<0-255>, <0-255>, <0-255>)'} και στην οντότητα αρχικοποιείται με την τιμή \texttt{'rgb(38,74,229)'} που αντιστοιχεί στο μπλε χρώμα.

                Επιπλέον, η οντότητα περιλαμβάνει τις ιδιότητες \texttt{TaskStatus} και \texttt{TaskPriority} όπου είναι Enumeration ιδιότητες των Enumeration εγγράφων \texttt{TaskStatus} και \linebreak \texttt{TaskPriority}, αρχικοποιημένες με \texttt{To\_Do} και \texttt{Low} αντίστοιχα. Το \texttt{TaskStatus} καθορίζει την κατάσταση της εργασίας με τις δυνατές καταστάσεις να είναι \texttt{'To\_Do'}, \texttt{'Doing'} και \texttt{'Done'} ενώ το \texttt{TaskPriority} καθορίζει αν η προτεραιότητα της εργασίας είναι χαμηλή, μεσαία ή υψηλή. Επίσης, περιλαμβάνεται η ιδιότητα \texttt{CreationDateBubble} τύπου Boolean που αρχικοποιείται με \texttt{True}. Η ιδιότητα αυτή χρησιμοποιείται για την εμφάνιση ενός επεξηγηματικού μηνύματος στον χρήστη όταν δημιουργεί μια νέα εργασία. Να σημειωθεί επίσης πως το \texttt{DateDue} περιλαμβάνει ένα Validation rule που ελέγχει αν η ημερομηνία λήξης είναι μετά την ημερομηνία έναρξης \texttt{StartDate}, και αν δεν ισχύει, τότε εμφανίζεται κατάλληλο μήνυμα.

                Η οντότητα \texttt{PageHelper} περιλαμβάνει τις ιδιότητες \texttt{CounterTaskDoing}, \linebreak \texttt{CounterTaskDone} και \texttt{CounterTaskToDo} τύπου Integer, των οποίων οι τιμές καθορίζονται από τα microflows \texttt{CounterTaskDoing}, \texttt{CounterTaskDone} και \texttt{CounterTaskToDo} αντίστοιχα. Οι ιδιότητες αυτές χρησιμοποιούνται για την εμφάνιση των τριών μετρητών. Επίσης, περιλαμβάνει την Boolean ιδιότητα \texttt{AreThereTasks} που καθορίζεται από το microflow \texttt{AreThereTasks} και χρησιμεύει για την εμφάνιση του επεξηγηματικού παραθύρου στο Dashboard, την Boolean ιδιότητα \texttt{Confetti} που αρχικοποιείται με \texttt{False} και χρησιμοποιείται για την εμφάνιση του κομφετί (σύστημα επιβράβευσης), και τέλος την ιδιότητα \texttt{openedTab} τύπου Integer αρχικοποιημένη με μηδέν που χρησιμεύει για την αποθήκευση της τρέχουσας καρτέλας του Dashboard που έχει ανοίξει ο χρήστης.

            \subsubsection{Σελίδες του \texttt{TaskManager}}
                Στο \texttt{TaskManager} περιλαμβάνονται οι εξής σελίδες:

                \begin{figure}[H] \noindent
                    \paragraph{\texttt{Custom\_LogIn\_Overview}}
                    \begin{center}
                        \includegraphics[width=\textwidth]{UniTask_Mendix/Custom_LogIn_Page}
                        \caption{\centering Σελίδα σύνδεσης χρηστών \texttt{Custom\_LogIn\_Overview} σε X-Ray Mode}
                        \label{fig:unitask_Custom_LogIn_Page}
                    \end{center}
                \end{figure}

                Η σελίδα (εικόνα \ref{fig:unitask_Custom_LogIn_Page}) χρησιμοποιείται για την είσοδο του χρήστη στην εφαρμογή. Πρόσβαση στη σελίδα έχουν μόνο οι \texttt{Guest} χρήστες.

                Χρησιμοποιείται το \texttt{Layout\_LogIn} layout του \texttt{UniTaskDesignSystem} module, με μια φόρμα που περιλαμβάνει τα Text Boxes για το \texttt{Username} και \texttt{Password} του χρήστη και το κουμπί για την είσοδο. Τα κουμπιά καλούν προεπιλεγμένες ενέργειες του Mendix.

                \begin{figure}[H] \noindent
                    \paragraph{\texttt{PAGE\_Home\_Page}}
                    \begin{center}
                        \includegraphics[width=\textwidth]{UniTask_Mendix/PAGE_Home_Page}
                        \caption{\centering Η αρχική σελίδα \texttt{PAGE\_Home\_Page} σε X-Ray Mode}
                        \label{fig:unitask_PAGE_Home_Page}
                    \end{center}
                \end{figure}

                Η αρχική σελίδα της εφαρμογής που εμφανίζεται μετά την είσοδο του χρήστη (εικόνα \ref{fig:unitask_PAGE_Home_Page}).

                Η σελίδα χρησιμοποιεί το \texttt{UniTask\_TopBar} layout του \texttt{UniTaskDesignSystem} module, το οποίο εμφανίζει το κεντρικό μενού της εφαρμογής στο πάνω μέρος της σελίδας. Περιλαμβάνει το τίτλο {\ZonaSB UniTask} με ένα call to action κουμπί που καλεί το microflow \texttt{ShowPage\_TasksOverview}.

                \begin{figure}[H] \noindent
                    \paragraph{\texttt{PAGE\_Tasks\_Overview}}
                    \begin{center}
                        \includegraphics[width=\textwidth]{UniTask_Mendix/PAGE_Tasks_Overview}
                        \caption{\centering Η κεντρική σελίδα εργασιών \texttt{PAGE\_Tasks\_Overview} σε X-Ray Mode}
                        \label{fig:unitask_PAGE_Tasks_Overview}
                    \end{center}
                \end{figure}

                Η σελίδα (εικόνα \ref{fig:unitask_PAGE_Tasks_Overview}) χρησιμοποιείται για την εμφάνιση και διαχείριση των εργασιών του χρήστη. Χρησιμοποιεί το \texttt{UniTask\_TopBar} layout του \texttt{UniTaskDesignSystem} module και έχει ως Parameters το \texttt{Student} και το \texttt{PageHelper}.

                Η σελίδα αποτελείται από ένα Layout Grid με διαφορετικά rows. To πρώτο row χρησιμοποιείται για το καλωσόρισμα του χρήστη. Για την εμφάνιση του ονόματός του έχει χρησιμοποιηθεί ένα Data View με Data source το \texttt{Student}. Πίσω από το text widget που εμφανίζει το μήνυμα καλωσορίσματος υπάρχει στην πραγματικότητα η έκφραση \verb|καλώς όρισες, {1}|. Τα \verb|{X}| αποτελούν placeholders για μεταβλητές· στη συγκεκριμένη περίπτωση το \verb|{1}| αντιστοιχεί στο \texttt{Name} του \texttt{Student}.

                Το δεύτερο row του Layout Grid περιλαμβάνει ένα εσωτερικό Layout Grid που δημιουργεί δύο στήλες (9 και 3)\footnote{Χρησιμοποιώντας παρόμοια λογική όπως το Bootstrap, το Mendix χρησιμοποιεί ένα σύστημα 12 στηλών ώστε να καθορίσουμε το πλάτος των στηλών.}. Στην αριστερή στήλη περιλαμβάνονται οι κάρτες με τις εργασίες. Για να επιτευχθεί αυτό έχουν δημιουργηθεί δύο εμφωλευμένα Data Views με Data sources τα \texttt{Student} και \texttt{PageHelper}. Μέσα στο Data View περιλαμβάνεται ένα Tab Container με τρία tabs που αντιστοιχούν στις τρεις καταστάσεις των εργασιών. Σε κάθε tab αντιστοιχίζεται ένα ξεχωριστό πράσινο κουμπί ({\Zona Νέα εργασία}) που, ανάλογα με το tab που είναι ενεργό, καλεί το microflow \texttt{CreateNewTask\_ToDo}, \texttt{CreateNewTask\_Doing} ή \texttt{CreateNewTask\_Done}. Το περιεχόμενο της καρτέλας είναι ένα List View με τις κάρτες εργασιών. Το περιεχόμενο των καρτών γίνεται populate (έχει δηλαδή Data source) από το microflow \texttt{DSL\_TaskToDo}, \texttt{DSL\_TaskDoing} ή \texttt{DSL\_TaskDone} αντίστοιχα. Για την καλύτερη οργάνωση, η κάρτα ουσιαστικά αποτελεί ένα Snippet που επαναχρησιμοποιείται. Έχουν δημιουργηθεί τα Snippets \texttt{TaskCard\_Overview\_Doing}, \texttt{TaskCard\_Overview\_ToDo} και \texttt{TaskCard\_Overview\_Done} που θα αναλυθούν στη συνέχεια.

                Η δεξιά στήλη με τους μετρητές περιλαμβάνει ένα Data View με Data source το \texttt{PageHelper}. Εσωτερικά περιλαμβάνονται containers με text widgets με τις μεταβλητές \texttt{CounterTaskToDo}, \texttt{CounterTaskDoing} και \texttt{CounterTaskDone}.

                Όσον αφορά για το σύστημα επιβράβευσης, το κομφετί αποτελεί ένα add-on widget από το Marketplace του Mendix. Το widget αυτό βρίσκεται τοποθετημένο μέσα σε ένα Data View με Data source το \texttt{PageHelper} και εμφανίζεται μόνο όταν η μεταβλητή \texttt{Confetti} γίνεται \texttt{True}.

                Στο κάτω μέρος του Tab container υπάρχει ένα επεξηγηματικό μπλε παράθυρο (εικόνα \ref{fig:unitask_TaskDashboard}) που εμφανίζεται μόνο όταν δεν έχουν δημιουργηθεί εργασίες. Αυτό επιτυγχάνεται με την έκφραση \verb|not($PageHelper/AreThereTasks)| στην συνθήκη Visibility του container που αντιπροσωπεύει το παράθυρο. Αξίζει επίσης να σημειωθεί πως τα call-to-action κουμπιά για τη νέα εργασία και τη σελίδα Kanban είναι και αυτά clickable και έχουν χρησιμοποιηθεί custom CSS κλάσεις για την εμφάνισή τους.

                \begin{figure}[H] \noindent
                    \paragraph{\texttt{TaskCard\_Overview\_Doing}}
                    \begin{center}
                        \includegraphics[width=\textwidth]{UniTask_Mendix/TaskCard_Overview_Doing}
                        \caption{\centering Η κάρτα \texttt{PAGE\_Tasks\_Overview} σε X-Ray Mode}
                        \label{fig:unitask_TaskCard_Overview_Doing}
                    \end{center}
                \end{figure}

                Πρόκειται για ένα Snippet (εικόνα \ref{fig:unitask_TaskCard_Overview_Doing}) που χρησιμοποιείται για την εμφάνιση των καρτών με τις εργασίες που βρίσκονται στην κατάσταση \texttt{Doing}. Έχει ως Parameters τα \texttt{Task}, \texttt{PageHelper} και \texttt{Student}.

                Αποτελείται από ένα Layout Grid με τρεις στήλες. Η πρώτη στήλη περιλαμβάνει τον τίτλο, το Progress Bar widget και ένα εσωτερικό Layout Grid με την προτεραιότητα της εργασίας και τις ημερομηνίες έναρξης και λήξης.

                Το Progress Bar χρησιμοποιείται για την εμφάνιση της προόδου της εργασίας και ολοκληρώνεται όσο πλησιάζει η ημερομηνία λήξης. Αυτό επιτυγχάνεται με τον καθορισμό τριών τιμών: Current, Minimum και Maximum value. Στις τιμές Minimum και Maximum value καθορίζονται οι εκφράσεις \verb|dateTimeToEpoch($Task/StartDate)| και \verb|dateTimeToEpoch($Task/DueDate)| αντίστοιχα, στις οποίες ουσιαστικά μετατρέπεται η ημερομηνία και ώρα σε ακέραιο αριθμό. Αυτό βοηθάει στο να είναι καθορισμένο ένα άνω και κάτω αριθμητικό όριο στο οποίο θα κινείται το Current value. Το Current value καθορίζεται από την έκφραση\footnote{Η έκφραση επιστρέφει την τρέχουσα ημερομηνία σε Epoch μορφή, ή --σε περίπτωση που βρισκόμαστε πριν την ημερομηνία έναρξης ή μετά την ημερομηνία λήξης-- αυτές τις ημερομηνίες πάλι σε Epoch μορφή. Η αναπαράσταση μιας χρονικής στιγμής σε Epoch βοηθάει στη σύγκριση μεταξύ ακεραίων για το Progress Bar. Να σημειωθεί πως λέγοντας Epoch μορφή εννοούμε τα δευτερόλεπτα που έχουν περάσει από τη 1η Ιανουαρίου 1970 μέχρι τη χρονική στιγμή που καθορίζουμε.}:

                \begin{lstlisting}[mystyle]
if [%CurrentDateTime%] > $Task/DueDate then dateTimeToEpoch($Task/DueDate)
else if [%CurrentDateTime%] < $Task/StartDate then dateTimeToEpoch($Task/StartDate)
else dateTimeToEpoch([%CurrentDateTime%]) \end{lstlisting}

                \noindent η οποία καταφέρνει τη συγκράτηση της τιμής μέσα σε αυτά τα όρια.

                Η προτεραιότητα εμφανίζεται ως ένα Badge widget. Στην κάρτα βρίσκονται τοποθετημένα και τα τρία, και ανάλογα με το ποιο είναι το \texttt{TaskPriority} εμφανίζεται και εξαφανίζονται τα αντίστοιχα Badges. Τέλος, η ημερομηνία έναρξης και λήξης έχει επιλεχθεί να εμφανίζεται με τη μορφή \texttt{dd MMM yy, h:mm a}, που αντιστοιχεί σε \say{23 Απρ 18, 1:37 μ.μ.} για παράδειγμα.

                Η δεύτερη στήλη περιλαμβάνει ένα Layout Grid με διαφορετικά rows, το πρώτο αφορά δύο containers που εμφανίζονται υπό συνθήκη και περιλαμβάνουν ενημερώσεις για το αν λήγει μια εργασία σε λιγότερο από μια εβδομάδα ή αν έχει ήδη λήξει. Το πρώτο container εμφανίζεται από την έκφραση\footnote{Η έκφραση συγκρίνει τις ημέρες ανάμεσα στην τρέχουσα ημέρα και της ημερομηνίας λήξης της εργασίας, και επιστέφει \texttt{True} όταν η διαφορά είναι μικρότερη από μια εβδομάδα και η ημερομηνία λήξης ακόμη είναι στο μέλλον.}:
                \begin{lstlisting}[mystyle]
if daysBetween($Task/DueDate, [%CurrentDateTime%]) < 7 and daysBetween($Task/DueDate, [%CurrentDateTime%]) > 0 and $Task/DueDate > [%CurrentDateTime%]
then true else false \end{lstlisting}

                \noindent και εμφανίζει \say{Η προθεσμία λήγει σε \{1\} ημέρες!} όπου το \{1\} αντιστοιχεί σε:

                \begin{lstlisting}[mystyle]
toString(floor(daysBetween($Task/DueDate, [%CurrentDateTime%]))) \end{lstlisting}

                To δεύτερο container εμφανίζεται το \verb|[%CurrentDateTime]| είναι μικρότερο ή ίσο από το \texttt{DueDate} και το κουμπί του καλεί το microflow \texttt{ChangeTaskStatus\_TaskDone}.

                Κάτω από το container υπάρχει ένα Data View με Data source το \texttt{Task} και το Bootstrap RTE widget. Το Rich Text Editor παρέχει στους χρήστες τη δυνατότητα δημιουργίας εμπλουτισμένου περιεχομένου, όπως έντονο (bold) και πλάγιο (italic) κείμενο. Το widget αποθηκεύει το κείμενο σε String μορφή.

                Στη δεξιά στήλη υπάρχουν τα κουμπιά επεξεργασίας και διαγραφής που καλούν τα microflows \texttt{EditTask} και \texttt{DeleteTask} (μετά από επιβεβαίωση) αντίστοιχα. Να σημειωθεί πως το κουμπί διαγραφής εμφανίζεται ανάλογα με την τιμή της Boolean ιδιότητας \texttt{buttonQuickDelete} του \texttt{Student}.

                \begin{figure}[H] \noindent
                    \paragraph{\texttt{TaskCard\_Overview\_ToDo}}
                    \begin{center}
                        \includegraphics[width=\textwidth]{UniTask_Mendix/TaskCard_Overview_ToDo}
                        \caption{\centering Η κάρτα \texttt{TaskCard\_Overview\_ToDo} σε X-Ray Mode}
                        \label{fig:unitask_TaskCard_Overview_ToDo}
                    \end{center}

                    \paragraph{\texttt{TaskCard\_Overview\_Done}}
                    \begin{center}
                        \includegraphics[width=\textwidth]{UniTask_Mendix/TaskCard_Overview_Done}
                        \caption{\centering Η κάρτα \texttt{TaskCard\_Overview\_Done} σε X-Ray Mode}
                        \label{fig:unitask_TaskCard_Overview_Done}
                    \end{center}
                \end{figure}

                Τα Snippets \texttt{TaskCard\_Overview\_ToDo} και \texttt{TaskCard\_Overview\_Done} (εικόνες \ref{fig:unitask_TaskCard_Overview_ToDo} και \ref{fig:unitask_TaskCard_Overview_Done}) είναι παρόμοια με το \texttt{TaskCard\_Overview\_Doing} με τις απαραίτητες διαφοροποιήσεις.

                Να σημειωθεί πως στο \texttt{TaskCard\_Overview\_Done} το Progress Bar πάντα είναι ολοκληρωμένο, ενώ το \texttt{TaskCard\_Overview\_ToDo} πάντα κενό, και επίσης πως στο δεύτερο δεν υπάρχει call-to-action κουμπί για την αλλαγή κατάστασης της εργασίας μετά την ημερομηνία λήξης, καθώς σε ένα σενάριο που ο χρήστης έχει προσθέσει μια εργασία ως To-Do και έχει τελειώσει η προθεσμία της πριν περάσει στην Doing κατάσταση, η προβλεπόμενη κίνηση θα είναι να τη διαγράψει.

                \begin{figure}[H] \noindent
                    \paragraph{\texttt{POPOUT\_Task\_NewEdit}}
                    \begin{center}
                        \includegraphics[width=\textwidth]{UniTask_Mendix/POPOUT_Task_NewEdit}
                        \caption{\centering Η σελίδα δημιουργίας εργασίας \texttt{POPOUT\_Task\_NewEdit} σε X-Ray Mode}
                        \label{fig:unitask_POPOUT_Task_NewEdit}
                    \end{center}
                \end{figure}

                Η σελίδα (εικόνα \ref{fig:unitask_POPOUT_Task_NewEdit}) καλείται στα microflows \texttt{CreateNewTask\_ToDo}, \texttt{CreateNewTask\_Doing} και \texttt{CreateNewTask\_Done} και χρησιμοποιείται για τη δημιουργία νέων εργασιών. Χρησιμοποιεί το \texttt{Popout\_Layout} layout του \texttt{Atlas\_Core} και έχει ως Parameters το \texttt{Task} και το \texttt{PageHelper}.

                Περιλαμβάνει ένα Data View με ένα Layout Grid με τα απαραίτητα Text Boxes και Date Pickers για την εισαγωγή του τίτλου και των ημερομηνιών έναρξης και λήξης. Για την επιλογή του χρώματος χρησιμοποιείται το widget ColorPicker που αποθηκεύει το επιλεγμένο χρώμα στη μεταβλητή \texttt{CalendarColor}.

                Για την επιλογή της κατάστασης και της προτεραιότητας έχει δημιουργηθεί ένα σύνολο από ζεύγη κουμπιών, το ένα άχρωμο και το άλλο έγχρωμο. Ανάλογα με το ποιο είναι το \texttt{TaskStatus} και το \texttt{TaskPriority} εμφανίζεται και εξαφανίζεται το αντίστοιχο σύνολο κουμπιών. Ταυτόχρονα, οι λεζάντες κάτω από τα κουμπιά περιέχουν τη δυναμική κλάση που καθορίζεται από την έκφραση:

                \begin{lstlisting}[mystyle]
if $Task/TaskPriority = TaskManager.TaskPriority.Low then 'labelSelected'
else ''         \end{lstlisting}

Αντίστοιχα στο αρχείο \verb|Styling/web/custom-variables.scss| έχει δημιουργηθεί η κλάση:
                \begin{lstlisting}[mystyle]
.labelSelected {
    font-weight: bold;
}                \end{lstlisting}

                Με αυτόν τον τρόπο επιτυγχάνεται η εμφάνιση της επιλεγμένης κατάστασης και προτεραιότητας με έντονη γραφή. Επιπλέον, το κάθε σύνολο από τα ζεύγη κουμπιών βρίσκεται τοποθετημένο σε ένα container, το οποίο αν πατηθεί καλεί τα microflow \texttt{ChangeTaskStatus\_TaskToDo}, \texttt{ChangeTaskStatus\_TaskDoing}, \texttt{ChangeTaskStatus\_}\linebreak\texttt{TaskDone} και \texttt{ChangeTaskPriority\_Low}, \texttt{ChangeTaskPriority\_Medium} και \linebreak \texttt{ChangeTaskPriority\_High} αντίστοιχα.

                Τέλος, η εμφάνιση του μπλε παραθύρου που ενημερώνει ότι η ημερομηνία λήξης έχει προκαθοριστεί αυτόματα βασίζεται στην Boolean μεταβλητή \texttt{CreationDateBubble} του \texttt{Task}, και επίσης όταν γίνεται κλικ στο Date Picker για την ημερομηνία λήξης, καλείται το microflow  \texttt{DisableCreationDateBubble}.

                Για να αποθηκευτεί η νέα εργασία καλείται το microflow \texttt{SaveTask}.

                \begin{figure}[H] \noindent
                    \paragraph{\texttt{POPOUT\_Task\_NewEdit\_Edit}}
                    \begin{center}
                        \includegraphics[width=\textwidth]{UniTask_Mendix/POPOUT_Task_NewEdit_Edit}
                        \caption{\centering Η σελίδα επεξεργασίας εργασίας \texttt{POPOUT\_Task\_NewEdit\_Edit} \\ σε X-Ray Mode}
                        \label{fig:unitask_POPOUT_Task_NewEdit_Edit}
                    \end{center}
                \end{figure}

                Η σελίδα (εικόνα \ref{fig:unitask_POPOUT_Task_NewEdit_Edit}) καλείται από το microflow \texttt{Edit\_Task}. Χρησιμοποιείται αντίστοιχη λογική με την \texttt{POPOUT\_Task\_NewEdit} με τη διαφορά ότι δεν υπάρχει επεξηγηματικό παράθυρο για την ημερομηνία λήξης, υπάρχει το widget Bootstrap RTE για την επεξεργασία της ιδιότητας \texttt{Notes} και επιπλέον υπάρχει το κουμπί διαγραφής της εργασίας που καλεί το microflow \texttt{DeleteTask} μετά από επιβεβαίωση.

                \begin{figure}[H] \noindent
                    \paragraph{\texttt{PAGE\_Tasks\_Kanban}}
                    \begin{center}
                        \includegraphics[width=\textwidth]{UniTask_Mendix/PAGE_Tasks_Kanban}
                        \caption{\centering Η σελίδα Kanban \texttt{PAGE\_Tasks\_Kanban} σε X-Ray Mode}
                        \label{fig:unitask_PAGE_Tasks_Kanban}
                    \end{center}
                \end{figure}

                Η σελίδα χρησιμοποιείται ως ένας εναλλακτικός τρόπος εμφάνισης των εργασιών του χρήστη σε έναν Kanban πίνακα. Χρησιμοποιεί το \texttt{UniTask\_TopBar} layout του \texttt{UniTaskDesignSystem} module και έχει ως Parameter το \texttt{PageHelper}.

                Η σελίδα αποτελείται από ένα Layout Grid με 3 στήλες, με τις δύο ακραίες να αποτελούν negative space. Η κεντρική στήλη περιέχει ένα Layout Grid με 3 στήλες. Το πρώτο row του περιλαμβάνει τις επικεφαλίδες του πίνακα μαζί με κουμπιά για την προσθήκη νέας εργασίας. Κάθε κουμπί είναι προσαρμοσμένο να δημιουργεί μια εργασία που αντιστοιχεί στην κατάσταση της επικεφαλίδας, καλώντας τα microflows \texttt{CreateNewTask\_ToDo}, \texttt{CreateNewTask\_Doing} και \texttt{CreateNewTask\_Done} αντίστοιχα. Το δεύτερο row περιλαμβάνει Galleries με Data sources τα \texttt{DSL\_TaskToDo}, \texttt{DSL\_TaskDoing} και \texttt{DSL\_TaskDone} που δημιουργούν τις κάρτες, που αν πατηθούν καλούν το microflow \texttt{EditTask}. Τέλος, περιλαμβάνεται το Confetti widget, όπως και στην Dashboard σελίδα.

                \newpage

                \begin{figure}[H] \noindent
                    \paragraph{\texttt{TaskCard\_Kanban}}
                    \begin{center}
                        \includegraphics[width=\textwidth]{UniTask_Mendix/TaskCard_Kanban}
                        \caption{\centering Η κάρτα επεξεργασίας εργασίας \texttt{TaskCard\_Kanban} σε X-Ray Mode}
                        \label{fig:unitask_TaskCard_Kanban}
                    \end{center}
                \end{figure}

                Πρόκειται για ένα Snippet (εικόνα \ref{fig:unitask_TaskCard_Kanban}) που χρησιμοποιείται για την εμφάνιση των καρτών με τις εργασίες στον πίνακα Kanban. Έχει ως Parameters τα \texttt{Task} και \texttt{Student}.

                Αποτελείται από ένα Layout Grid με τον τίτλο, τις ημερομηνίες έναρξης και λήξης και Badges για την προτεραιότητα της εργασίας, με παρόμοιο τρόπο όπως στα Snippets του Dashboard.

                \begin{figure}[H] \noindent
                    \paragraph{\texttt{PAGE\_Calendar}}
                    \begin{center}
                        \includegraphics[width=\textwidth]{UniTask_Mendix/PAGE_Calendar}
                        \caption{\centering Η σελίδα ημερολογίου \texttt{PAGE\_Calendar} σε X-Ray Mode}
                        \label{fig:unitask_PAGE_Calendar}
                    \end{center}
                \end{figure}

                Η σελίδα (εικόνα \ref{fig:unitask_PAGE_Calendar}) χρησιμοποιείται για την εμφάνιση των εργασιών του χρήστη σε έναν ημερολόγιο. Χρησιμοποιεί το \texttt{UniTask\_TopBar} layout του \texttt{UniTaskDesignSystem} module και έχει ως Parameter το \texttt{Task} και \texttt{PageHelper}.

                Το Calendar widget είναι τοποθετημένο μέσα σε ένα Data View με Data source το \texttt{Task}. Έτσι καθορίζεται πως το Event entity, δηλαδή ο τύπος αντικειμένου που θα εμφανίζεται στο ημερολόγιο, είναι το \texttt{Task}. Το ημερολόγιο γίνεται populate μέσω του microflow \texttt{ShowTasks\_Calendar} και στα Properties του Widget επιλέγονται οι ιδιότητες \texttt{Title}, \texttt{StartDate}, \texttt{DueDate} και \texttt{CalendarColor} του \texttt{Task} για να καθοριστεί η εμφάνισή του στο ημερολόγιο. Όσον αφορά το Timeline στα δεξιά του, έχει επιλεχθεί να εμφανίζονται οι εργασίες ομαδοποιημένες βάσει της ημερομηνίας έναρξής τους και μάλιστα να μπορεί να γίνει επεξεργασία τους αν γίνει κλικ πάνω τους.

                \begin{figure}[H] \noindent
                    \paragraph{\texttt{PAGE\_Settings}}
                    \begin{center}
                        \includegraphics[width=\textwidth]{UniTask_Mendix/PAGE_Settings}
                        \caption{\centering Η σελίδα ρυθμίσεων \texttt{PAGE\_Settings} σε X-Ray Mode}
                        \label{fig:unitask_PAGE_Settings}
                    \end{center}
                \end{figure}

                Η σελίδα (εικόνα \ref{fig:unitask_PAGE_Settings}) χρησιμοποιείται για την εμφάνιση των ρυθμίσεων του χρήστη. Χρησιμοποιεί το \texttt{UniTask\_TopBar} layout του \texttt{UniTaskDesignSystem} module και έχει ως Parameters το \texttt{Student} και το \texttt{PageHelper}.

                Οι ρυθμίσεις βρίσκονται τοποθετημένες σε ένα Accordion widget με δύο ομάδες: διαγραφή και αρχικοποίηση. Στη διαγραφή περιλαμβάνεται ένα Data View με Data source το \texttt{Student}. Εσωτερικά του υπάρχει ένα Switch widget που κάνει toggle την Boolean ιδιότητα \texttt{buttonQuickDelete} του \texttt{Student} και ένα κουμπί ({\Zona διαγραφή όλων των εργασιών}) που καλεί το microflow \texttt{DeleteAllTasks} μετά από επιβεβαίωση. Στην αρχικοποίηση περιλαμβάνεται το κουμπί ({\Zona αρχικοποίηση}) που καλεί το microflow \texttt{InitializeTasks} μετά από επιβεβαίωση.

            \subsubsection{Microflows του \texttt{TaskManager}}
                Στο \texttt{TaskManager} περιλαμβάνονται τα εξής microflows\footnote{Στα ονόματα των microflows περιλαμβάνεται ο parent φάκελος όπου βρίσκονται για τον πιο εμφανή διαχωρισμό τους.}:

                \begin{figure}[H] \noindent
                    \paragraph{\texttt{RET\_Student}}
                    \begin{center}
                        \includegraphics[width=\textwidth]{UniTask_Mendix/RET_Student}
                        \caption{\centering Το microflow \texttt{RET\_Student}}
                        \label{fig:unitask_RET_Student}
                    \end{center}
                \end{figure}

                Το microflow (εικόνα \ref{fig:unitask_RET_Student}) χρησιμοποιείται για την ανάκτηση του \texttt{Student} με βάση τον τρέχοντα \texttt{System.User}. Αν δεν υπάρχει τέτοιος χρήστης, τότε το microflow επιστρέφει ένα κενό αντικείμενο, αλλιώς επιστρέφεται το \texttt{Student}.

                \begin{figure}[H] \noindent
                    \paragraph{\texttt{Auxiliary/AreThereTasks}}
                    \begin{center}
                        \includegraphics[width=\textwidth]{UniTask_Mendix/AreThereTasks}
                        \caption{\centering Το microflow \texttt{AreThereTasks}}
                        \label{fig:unitask_AreThereTasks}
                    \end{center}
                \end{figure}

                To microflow χρησιμοποιείται από το domain model για τον υπολογισμό της τιμής της Boolean ιδιότητας \texttt{AreThereTasks} του \texttt{PageHelper}.

                Το microflow καλεί το microflow \texttt{CounterTotalTasks}, το οποίο επιστρέφει ως Integer τον συνολικό αριθμό των εργασιών του χρήστη. Αν ο αριθμός είναι μεγαλύτερος ή ίσος του 1, τότε το microflow επιστρέφει \texttt{true}, αλλιώς \texttt{false}.

                \begin{figure}[H] \noindent
                    \paragraph{\texttt{Auxiliary/DisableCreationDateBubble}}
                    \begin{center}
                        \includegraphics[width=\textwidth]{UniTask_Mendix/DisableCreationDateBubble}
                        \caption{\centering Το microflow \texttt{DisableCreationDateBubble}}
                        \label{fig:unitask_DisableCreationDateBubble}
                    \end{center}
                \end{figure}

                Το microflow (εικόνα \ref{fig:unitask_DisableCreationDateBubble}) χρησιμοποιείται από τη σελίδα \texttt{POPOUT\_Task\_NewEdit} για να απενεργοποιήσει το μπλε παράθυρο που ενημερώνει ότι η ημερομηνία λήξης έχει προκαθοριστεί αυτόματα. Καλείται όταν ο χρήστης αλλάζει την ημερομηνία λήξης της εργασίας.

                Το microflow αλλάζει την τιμή της Boolean ιδιότητας \texttt{CreationDateBubble} του \texttt{Task} σε \texttt{false}.

                \begin{figure}[H] \noindent
                    \paragraph{\texttt{ChangeTaskPriorities/ChangeTaskPriority\_TaskLow} \\ \texttt{ChangeTaskPriorities/ChangeTaskPriority\_TaskMedium} \\ \texttt{ChangeTaskPriorities/ChangeTaskPriority\_TaskHigh}}
                    \begin{center}
                        \includegraphics[width=\textwidth]{UniTask_Mendix/ChangeTaskPriority}
                        \caption{\centering Τα microflow \texttt{ChangeTaskPriority\_TaskLow}, \texttt{ChangeTaskPriority\_TaskMedium} και \texttt{ChangeTaskPriority\_TaskHigh}}
                        \label{fig:unitask_ChangeTaskPriority}
                    \end{center}
                \end{figure}

                Τα microflows (εικόνα \ref{fig:unitask_ChangeTaskPriority}) χρησιμοποιούνται για την αλλαγή της προτεραιότητας κάποιας εργασίας. Καλούνται από τα κουμπιά της σελίδας \texttt{POPOUT\_Task\_NewEdit} και \texttt{POPOUT\_Task\_NewEdit\_Edit}.

                Το κάθε microflow αλλάζει την τιμή της Enumeration ιδιότητας \texttt{TaskPriority} του \texttt{Task} σε \texttt{Low}, \texttt{Medium} και \texttt{High} αντίστοιχα.

                \begin{figure}[H] \noindent
                    \paragraph{\texttt{ChangeTaskStatuses/ChangeTaskStatus\_TaskDoing} \\ \texttt{ChangeTaskStatuses/ChangeTaskStatus\_TaskToDo} \\ \texttt{ChangeTaskStatuses/ChangeTaskStatus\_TaskDone}}
                    \begin{center}
                        \includegraphics[width=\textwidth]{UniTask_Mendix/ChangeTaskStatus}
                        \includegraphics[width=\textwidth]{UniTask_Mendix/ChangeTaskStatus_TaskDone}
                        \caption{\centering Τα microflow \texttt{ChangeTaskStatus\_TaskDoing}, \texttt{ChangeTaskStatus\_TaskToDo} και \texttt{ChangeTaskStatus\_TaskDone}}
                        \label{fig:unitask_ChangeTaskStatus}
                    \end{center}
                \end{figure}

                Τα microflows (εικόνες \ref{fig:unitask_ChangeTaskStatus}) χρησιμοποιούνται για την αλλαγή της κατάστασης κάποιας εργασίας. Καλούνται από τα κουμπιά της σελίδας \texttt{POPOUT\_Task\_NewEdit} και \texttt{POPOUT\_Task\_NewEdit\_Edit}.

                Το κάθε microflow αλλάζει την τιμή της Enumeration ιδιότητας \texttt{TaskStatus} του \texttt{Task} σε \texttt{TaskStatus.Doing}, \texttt{TaskStatus.ToDo} και \texttt{TaskStatus.Done} αντίστοιχα. Επιπλέον, στην περίπτωση του microflow \texttt{ChangeTaskStatus\_TaskDone} καλεί το microflow \texttt{ConfettiTrigger} για το εφέ του κομφετί.

                \begin{figure}[H] \noindent
                    \paragraph{\texttt{Confetti/ConfettiTrigger}}
                    \begin{center}
                        \includegraphics[width=\textwidth]{UniTask_Mendix/ConfettiTrigger}
                        \caption{\centering Το microflow \texttt{ConfettiTrigger}}
                        \label{fig:unitask_ConfettiTrigger}
                    \end{center}
                \end{figure}

                Το microflow (εικόνα \ref{fig:unitask_Confetti}) καλείται από το microflow \texttt{ChangeTaskStatus\_TaskDone} και χρησιμοποιείται για την εμφάνιση του εφέ κομφετί στην οθόνη του χρήστη. Αλλάζει την τιμή της Boolean ιδιότητας \texttt{Confetti} του \texttt{PageHelper} σε \texttt{true}.

                \begin{figure}[H] \noindent
                    \paragraph{\texttt{Confetti/ConfettiNotTriggering}}
                    \begin{center}
                        \includegraphics[width=\textwidth]{UniTask_Mendix/ConfettiNotTriggering}
                        \caption{\centering Το microflow \texttt{ConfettiNotTriggering}}
                        \label{fig:unitask_ConfettiNotTriggering}
                    \end{center}
                \end{figure}

                Το microflow καλείται από το microflow \texttt{SaveTask} και χρησιμοποιείται για να επαναφέρει την τιμή της Boolean ιδιότητας \texttt{Confetti} του \texttt{PageHelper} σε \texttt{false}.

                \newpage

                \begin{figure}[H] \noindent
                    \paragraph{\texttt{ShowTasks/DSL\_TaskDoing} \\ \texttt{ShowTasks/DSL\_TaskDone} \\ \texttt{ShowTasks/DSL\_TaskToDo}}
                    \begin{center}
                        \includegraphics[width=\textwidth]{UniTask_Mendix/DSL_TaskDoingDoneToDo}
                        \caption{\centering Τα microflow \texttt{DSL\_TaskDoing}, \texttt{DSL\_TaskDone} και \texttt{DSL\_TaskToDo}}
                        \label{fig:unitask_DSLTaskDoingDoneToDo}
                    \end{center}
                \end{figure}

                Τα microflows (εικόνα \ref{fig:unitask_DSLTaskDoingDoneToDo}) φιλτράρουν και επιστρέφουν τις εργασίες του χρήστη ανάλογα με την κατάστασή τους. Χρησιμοποιούνται από τις σελίδες \texttt{PAGE\_Tasks\_Overview} και \texttt{PAGE\_Tasks\_Kanban} και τα microflows \texttt{ShowTasks\_Calendar} και \texttt{CounterTaskDoing}.

               Με Parameter το \texttt{Student}, το microflow κάνει retrieve τη λίστα των Tasks (αφού συσχετίζονται). Η λίστα φιλτράρεται βάσει του \texttt{TaskStatus}, ταξινομείται βάσει δύο ιδιοτήτων σε αύξουσα σειρά, τα \texttt{TaskPriority} και \texttt{DueDate}, και επιστρέφεται.

                \begin{figure}[H] \noindent
                    \paragraph{\texttt{ShowTasks/ShowTasks\_Calendar}}
                    \begin{center}
                        \includegraphics[width=\textwidth]{UniTask_Mendix/ShowTasks_Calendar}
                        \caption{\centering Το microflow \texttt{ShowTasks\_Calendar}}
                        \label{fig:unitask_ShowTasks_Calendar}
                    \end{center}
                \end{figure}

                Το microflow (εικόνα \ref{fig:unitask_ShowTasks_Calendar}) επιστρέφει το σύνολο των εργασιών του χρήστη για την εμφάνισή τους στο ημερολόγιο.

                Το \texttt{RET\_Student} επιστρέφει το \texttt{Student}, ώστε να χρησιμοποιηθεί ως παράμετρος στα microflows \texttt{DSL\_TaskDoing}, \texttt{DSL\_TaskDone} και \texttt{DSL\_TaskToDo}, τα οποία καλούνται και οι λίστες τους επιστρέφονται. Στη συνέχεια με List Operations οι λίστες συνενώνονται, ταξινομούνται βάσει της ημερομηνίας έναρξης και επιστρέφονται.

                Ο λόγος που δε χρησιμοποιείται κάποια παράμετρος \texttt{Student} απευθείας και καλείται η \texttt{RET\_Student} είναι γιατί από τη φύση του Calendar widget είναι απαραίτητο να βρίσκεται σε ένα Data View με Data source το \texttt{Task}, άρα κατά συνέπεια θα έχει παράμετρο το \texttt{Task}.

                \begin{figure}[H] \noindent
                    \paragraph{\texttt{CounterGenerators/CounterTaskDoing} \\ \texttt{CounterGenerators/CounterTaskDone} \\ \texttt{CounterGenerators/CounterTaskToDo}}
                    \begin{center}
                        \includegraphics[width=\textwidth]{UniTask_Mendix/CounterTask}
                        \caption{\centering Τα microflow \texttt{CounterTaskDoing}, \texttt{CounterTaskDone} \\ και \texttt{CounterTaskToDo}}
                        \label{fig:unitask_CounterTask}
                    \end{center}
                \end{figure}

                Τα microflows (εικόνα \ref{fig:unitask_CounterTask}) χρησιμοποιούνται για τον υπολογισμό του συνολικού αριθμού των εργασιών του χρήστη ανάλογα με την κατάστασή τους, και καλούνται από το domain model για τον υπολογισμό των τιμών των ιδιοτήτων \texttt{TaskToDoCount}, \texttt{TaskDoingCount} και \texttt{TaskDoneCount} του \texttt{PageHelper}.

                Το κάθε microflow καλεί το \texttt{RET\_Student} και τα microflows \texttt{DSL\_TaskToDo}, \texttt{DSL\_TaskDoing} και \texttt{DSL\_TaskDone} αντίστοιχα. Έπειτα μέσω του Function Count του Aggregate List Action του Mendix, υπολογίζει τον αριθμό των εργασιών και τον επιστρέφει.

                \newpage

                \begin{figure}[H] \noindent
                    \paragraph{\texttt{CounterGenerators/CounterTotalTasks}}
                    \begin{center}
                        \includegraphics[width=\textwidth]{UniTask_Mendix/CounterTotalTasks}
                        \caption{\centering Το microflow \texttt{CounterTotalTasks}}
                        \label{fig:unitask_CounterTotalTasks}
                    \end{center}
                \end{figure}

                Το microflow (εικόνα \ref{fig:unitask_CounterTotalTasks}) καλείται από το microflow \texttt{AreThereTasks} και χρησιμοποιείται για τον υπολογισμό του συνολικού αριθμού των εργασιών του χρήστη. Χρησιμοποιείται παρόμοια λογική με τα προηγούμενα microflows: γίνεται retrieve το σύνολο των εργασιών για έναν συγκεκριμένο χρήστη και καταμετρούνται.

                \begin{figure}[H] \noindent
                    \paragraph{\texttt{CreateNewTask/CreateNewTask\_Doing} \\ \texttt{CreateNewTask/CreateNewTask\_Done} \\ \texttt{CreateNewTask/CreateNewTask\_ToDo}}
                    \begin{center}
                        \includegraphics[width=\textwidth]{UniTask_Mendix/CreateNewTask}
                        \caption{\centering Τα microflow \texttt{CreateNewTask\_Doing}, \texttt{CreateNewTask\_Done} \\ και \texttt{CreateNewTask\_ToDo}}
                        \label{fig:unitask_CreateNewTask}
                    \end{center}
                \end{figure}

                To microflow (εικόνα \ref{fig:unitask_CreateNewTask}) καλείται από τα κουμπιά {\Zona νέα εργασία} των σελίδων \texttt{PAGE\_Tasks\_Kanban} και \texttt{PAGE\_Tasks\_Overview} με σκοπό τη δημιουργία μιας νέας εργασίας.

                Αρχικά καλείται το microflow \texttt{RET\_Student} για την ανάκτηση του \texttt{Student}. Στη συνέχεια δημιουργείται ένα νέο αντικείμενο τύπου \texttt{Task} όπου ορίζονται κάποιες αρχικές τιμές. Συγκεκριμένα ορίζεται η ημερομηνία λήξης \texttt{DueDate} ως \linebreak \verb|addDays([%CurrentDateTime%], 7)| (μια εβδομάδα μετά την τρέχουσα ημερομηνία), ο τίτλος \texttt{Title} ως \texttt{'Νέα εργασία'}, και το \texttt{TaskStatus} ανάλογα με το τύπο του microflow. Έπειτα, εμφανίζεται η αναδυόμενη σελίδα \texttt{POPOUT\_Task\_NewEdit} για την επεξεργασία των ιδιοτήτων της εργασίας και την τελική αποθήκευσή της.

                \begin{figure}[H] \noindent
                    \paragraph{\texttt{ShowPages/ShowPage\_Calendar} \\ \texttt{ShowPages/ShowPage\_Homepage} \\ \texttt{ShowPages/ShowPage\_Kanban} \\ \texttt{ShowPages/ShowPage\_Settings} \\ \texttt{ShowPages/ShowPage\_TasksOverview}}
                    \begin{center}
                        \includegraphics[width=\textwidth]{UniTask_Mendix/ShowPage}
                        \caption{\centering Τα microflow \texttt{ShowPage\_Calendar}, \texttt{ShowPage\_Homepage}, \texttt{ShowPage\_Kanban}, \texttt{ShowPage\_Settings} και \texttt{ShowPage\_TasksOverview}}
                        \label{fig:unitask_ShowPage}
                    \end{center}
                \end{figure}

                Τα microflows (εικόνα \ref{fig:unitask_ShowPage}) χρησιμοποιούνται για την ανακατεύθυνση του χρήστη σε μια συγκεκριμένη σελίδα. Καλούνται όποτε χρειάζεται η εμφάνιση μιας σελίδας, κυρίως από το Navigation μενού.

                Αφού ανακτηθεί το \texttt{Student} μέσω του microflow \texttt{RET\_Student}, το εκάστοτε microflow δημιουργεί ένα \texttt{PageHelper} αντικείμενο και εμφανίζει την αντίστοιχη σελίδα. Η δημιουργία του \texttt{PageHelper} είναι απαραίτητη καθώς χρησιμοποιείται ως Parameter στις σελίδες.

                \begin{figure}[H] \noindent
                    \paragraph{\texttt{SaveTask}}
                    \begin{center}
                        \includegraphics[width=\textwidth]{UniTask_Mendix/SaveTask}
                        \caption{\centering Το microflow \texttt{SaveTask}}
                        \label{fig:unitask_SaveTask}
                    \end{center}
                \end{figure}

                Το microflow (εικόνα \ref{fig:unitask_SaveTask}) καλείται από τις αναδυόμενες σελίδες \texttt{POPOUT\_Task\_NewEdit} και \texttt{POPOUT\_Task\_NewEdit\_Edit} για την αποθήκευση των αλλαγών που έγιναν στις ιδιότητες της εργασίας.

                Γίνεται commit στο αντικείμενο \texttt{Task}, κλείνει η σελίδα, γίνεται ένα Change Object στο \texttt{PageHelper} για να ανανεωθούν οι τιμές του, όπως υπολογίζονται από τα microflows και καλείται το microflow \texttt{ConfettiNotTriggering} για την επαναφορά της τιμής της Boolean ιδιότητας \texttt{Confetti} του \texttt{PageHelper} σε \texttt{false}.

                \begin{figure}[H] \noindent
                    \paragraph{\texttt{EditTask}}
                    \begin{center}
                        \includegraphics[width=\textwidth]{UniTask_Mendix/EditTask}
                        \caption{\centering Το microflow \texttt{EditTask}}
                        \label{fig:unitask_EditTask}
                    \end{center}
                \end{figure}

                Το microflow (εικόνα \ref{fig:unitask_EditTask}) καλείται από τα κουμπιά των καρτών στην Dashboard σελίδα και τις κάρτες οτυ πίνακα Kanban και του ημερολογίου για την επεξεργασία μιας εργασίας.

                Το microflow εμφανίζει την αναδυόμενη σελίδα \texttt{POPOUT\_Task\_NewEdit\_Edit}, κάνει commit στο αντικείμενο \texttt{Task} και ανανεώνει το \texttt{PageHelper}.

                \begin{figure}[H] \noindent
                    \paragraph{\texttt{DeleteTask}}
                    \begin{center}
                        \includegraphics[width=\textwidth]{UniTask_Mendix/DeleteTask}
                        \caption{\centering Το microflow \texttt{DeleteTask}}
                        \label{fig:unitask_DeleteTask}
                    \end{center}
                \end{figure}

                Το microflow (εικόνα \ref{fig:unitask_DeleteTask}) καλείται από το κουμπί της αναδυόμενης σελίδας \texttt{POPOUT\_Task\_NewEdit\_Edit} για τη διαγραφή μιας εργασίας.

                Γίνεται delete το αντικείμενο \texttt{Task}, κλείνει η σελίδα και ανανεώνεται το \texttt{PageHelper}.

                \begin{figure}[H] \noindent
                    \paragraph{\texttt{DeleteTask\_Snippet}}
                    \begin{center}
                        \includegraphics[width=\textwidth]{UniTask_Mendix/DeleteTask_Snippet}
                        \caption{\centering Το microflow \texttt{DeleteTask\_Snippet}}
                        \label{fig:unitask_DeleteTask_Snippet}
                    \end{center}
                \end{figure}

                Το microflow (εικόνα \ref{fig:unitask_DeleteTask_Snippet}) καλείται από το κουμπί της κάρτας στην Dashboard σελίδα για τη διαγραφή μιας εργασίας, στην περίπτωση που ο χρήστης έχει ενεργοποιήσει την επιλογή γρήγορης διαγραφής.

                Γίνεται delete το αντικείμενο \texttt{Task} και ανανεώνεται το \texttt{PageHelper}.

                \begin{figure}[H] \noindent
                    \paragraph{\texttt{DeleteAllTasks}}
                    \begin{center}
                        \includegraphics[width=\textwidth]{UniTask_Mendix/DeleteAllTasks}
                        \caption{\centering Το microflow \texttt{DeleteAllTasks}}
                        \label{fig:unitask_DeleteAllTasks}
                    \end{center}
                \end{figure}

                Το microflow (εικόνα \ref{fig:unitask_DeleteAllTasks}) καλείται από το κουμπί της σελίδας \texttt{POPOUT\_Settings} για τη διαγραφή όλων των εργασιών του χρήστη.

                Αρχικά γίνεται retrieve το σύνολο των εργασιών του χρήστη ως μια λίστα και στη συνέχεια γίνεται διαγραφή της λίστας. Τέλος, ανανεώνεται το \texttt{PageHelper} και στέλνεται επιβεβαιωτικό μήνυμα.

                \begin{figure}[H] \noindent
                    \paragraph{\texttt{InitializeTasks}}
                    \begin{center}
                        \includegraphics[width=\textwidth]{UniTask_Mendix/InitializeTasks}
                        \caption{\centering Το microflow \texttt{InitializeTasks}}
                        \label{fig:unitask_InitializeTasks}
                    \end{center}
                \end{figure}

                Το microflow (εικόνα \ref{fig:unitask_InitializeTasks}) καλείται από το κουμπί της σελίδας \texttt{POPOUT\_Settings} για την αρχικοποίηση των εργασιών του χρήστη.

                Αρχικά επιστρέφεται το \texttt{Student} μέσω του \texttt{RET\_Student} και δημιουργείται ένα \texttt{PageHelper} που θα χρησιμοποιηθεί ως όρισμα. Πριν δημιουργηθούν οι νέες εργασίες, γίνονται retrieve οι υπάρχουσες, διαγράφονται και ανανεώνεται το \texttt{PageHelper} Στη συνέχεια δημιουργούνται τέσσερις εργασίες με hard-coded προκαθορισμένες τιμές και στέλνεται επιβεβαιωτικό μήνυμα.

        \subsection{Module \texttt{UniTaskDesignSystem}}
            Το \texttt{UniTaskDesignSystem} περιλαμβάνει τα layouts της εφαρμογής όπως επίσης παρεμβάσεις που αφορούν το styling της εφαρμογής.

            \subsubsection{Layouts του \texttt{UniTaskDesignSystem}}
                Το module \texttt{UniTaskDesignSystem} περιλαμβάνει τα εξής layouts:

                \begin{figure}[H] \noindent
                    \paragraph{\texttt{UniTask\_TopBar}}
                    \begin{center}
                        \includegraphics[width=\textwidth]{UniTask_Mendix/UniTask_TopBar}
                        \caption{\centering Το layout \texttt{UniTask\_TopBar}}
                        \label{fig:unitask_UniTask_TopBar}
                    \end{center}
                \end{figure}

                Το layout (εικόνα \ref{fig:unitask_UniTask_TopBar}) χρησιμοποιείται για την εμφάνιση της μπάρας πλοήγησης στην κορυφή της σελίδας. Η μπάρα πλοήγησης αποτελείται από δύο διαφορετικά Navigation μενού, το αριστερό (το κύριο Project navigation) και το δεξί (ένα Menu document του \texttt{UniTaskDesignSystem}) τοποθετημένα σε ένα Layout Grid με δύο στήλες. Τα Navigation μενού καλούν τα αντίστοιχα \texttt{ShowPage} microflows για την ανακατεύθυνση του χρήστη στην επιλεγμένη σελίδα. Εξαίρεση αποτελεί το κουμπί {\Zona Home} που καλεί τη σελίδα \texttt{PAGE\_Home\_Page} απευθείας και το κουμπί {\Zonaa αποσύνδεση} που αποσυνδέει τον χρήστη.

                Ο τίτλος της εφαρμογής {\ZonaSB UniTask} που εμφανίζεται στα αριστερά χρησιμοποιεί custom CSS και περιλαμβάνεται ένα hamburger μενού στα αριστερά για μεγαλύτερη προσβασιμότητα. Το {\Zone Admin} περιλαμβάνει το υπομενού {\Zona Account Overview} και είναι προσβάσιμα μόνο από τον Administrator.

                \begin{figure}[H] \noindent
                    \paragraph{\texttt{UniTask\_SideBar}}
                    \begin{center}
                        \includegraphics[width=\textwidth]{UniTask_Mendix/UniTask_SideBar}
                        \caption{\centering Το layout \texttt{UniTask\_SideBar}}
                        \label{fig:unitask_UniTask_SideBar}
                    \end{center}
                \end{figure}\textbf{}

                Ισχύουν τα αντίστοιχα με το προηγούμενο layout με τη διαφορά ότι δεν εμφανίζονται οι Ρυθμίσεις και το κύριο Project navigation εμφανίζεται στα αριστερά (εικόνα \ref{fig:unitask_UniTask_SideBar}).

            \subsubsection{Styling του \texttt{UniTaskDesignSystem}}
                Στα αρχεία \texttt{custom-variables.scss} και \texttt{design.scss} των \texttt{UniTaskDesignSystem} και \texttt{App} modules περιλαμβάνεται custom CSS κώδικας για το styling της εφαρμογής. Εκεί ορίζεται για παράδειγμα το πορτοκαλί χρώμα που έχουν οι μπάρες πλοήγησης, η custom γραμματοσειρά Zona Pro που χρησιμοποιείται ή κάποιες custom δυναμικές κλάσεις.
%    \chapter{Πειραματική διαδικασία και αποτελέσματα}
    Με στόχο τη διερεύνηση της ευχρηστίας της εφαρμογής, πραγματοποιήθηκε πειραματική διαδικασία με τη συμμετοχή φοιτητών του Πανεπιστημίου Πατρών. Ο στόχος της μελέτης ήταν να αξιολογηθεί η εμπειρία χρήσης, να εντοπιστούν πιθανά προβλήματα χρηστικότητας και να συλλεχθεί ανατροφοδότηση σχετικά με τις λειτουργίες της εφαρμογής. Μέσω αυτής της διαδικασίας, επιδιώξαμε να κατανοήσουμε τον τρόπο με τον οποίο οι φοιτητές αλληλεπιδρούν με την εφαρμογή και πώς αυτή μπορεί να βελτιώσει τη διαχείριση των καθημερινών τους εργασιών.

    \section{Πειραματική διαδικασία}
        Το πείραμα σχεδιάστηκε ώστε να προσομοιώσει ένα ρεαλιστικό σενάριο χρήσης της εφαρμογής. Οι συμμετέχοντες κλήθηκαν να ολοκληρώσουν συγκεκριμένες εργασίες μέσα στην εφαρμογή, ενώ στο τέλος της διεξαγωγής του πειράματος καταγράφτηκαν οι εντυπώσεις τους.

        Συνολικά συμμετείχαν 12 φοιτητές. Πριν τη διεξαγωγή της πειραματικής διαδικασίας, οι συμμετέχοντες ενημερώθηκαν πλήρως για τους σκοπούς του πειράματος και υπέγραψαν έντυπο συναίνεσης, το οποίο τόνιζε τον εθελοντικό χαρακτήρα της συμμετοχής τους, διασφαλίζοντας ότι οι συμμετέχοντες κατανοούσαν το δικαίωμά τους να αποχωρήσουν από τη μελέτη οποιαδήποτε στιγμή χωρίς συνέπειες. Η φόρμα ενημέρωνε πλήρως τους συμμετέχοντες σχετικά με τον σκοπό, τις διαδικασίες, και τα οφέλη της μελέτης και εξασφαλιζόταν η εμπιστευτικότητα και η προστασία των προσωπικών δεδομένων των συμμετεχόντων.

        Για την προσέλκυση συμμετεχόντων, χρησιμοποιήθηκαν διάφορα μέσα επικοινωνίας, όπως ανακοινώσεις μέσω κοινωνικών δικτύων και πανεπιστημιακά κανάλια επικοινωνίας. Οι ενδιαφερόμενοι συμμετέχοντες έλαβαν email με λεπτομερείς οδηγίες και το έντυπο συναίνεσης όπως επίσης τον σύνδεσμο της εφαρμογής και κωδικούς πρόσβασης.

        \subsection{Οδηγίες}
            Οι οδηγίες που δόθηκαν στους συμμετέχοντες περιελάμβαναν τα εξής:

            \textit{Ο σκοπός αυτής της μελέτης είναι να αξιολογήσουμε την ευχρηστία της εφαρμογής και να εντοπίσουμε πιθανά σημεία που μπορούν να βελτιωθούν. Θα σας ζητηθεί να ολοκληρώσετε μια σειρά από βασικές ενέργειες στην εφαρμογή και στη συνέχεια να συμπληρώσετε ένα ερωτηματολόγιο SUS (System Usability Scale), εκφράζοντας την εμπειρία σας.}

            \textit{Η συμμετοχή σας εκτιμάται να διαρκέσει περίπου 15--20 λεπτά και ευχαριστούμε εκ των προτέρων για τον χρόνο που θα αφιερώσετε για τη συμμετοχή σας. Η συμμετοχή σας είναι εθελοντική και μπορείτε να αποσυρθείτε ανά πάσα στιγμή χωρίς καμία συνέπεια.}

            \textit{Η συμμετοχή σας περιλαμβάνει τα εξής βήματα:}

            \paragraph{\textit{Ανάγνωση και αποδοχή των όρων συμμετοχής.}}
                \textit{Επισυνάπτεται το Έντυπο Συναίνεσης. Παρακαλούμε διαβάστε το προσεκτικά και, εφόσον συμφωνείτε, επιβεβαιώστε τη συμμετοχή σας απαντώντας σε αυτό το email με τη φράση: \say{Αποδέχομαι τη συμμετοχή μου στην έρευνα}.}

            \paragraph{\textit{Πρόσβαση στην εφαρμογή.}}
                \textit{Μπορείτε να αποκτήσετε πρόσβαση στην εφαρμογή μέσω του ακόλουθου συνδέσμου:} \texttt{https://unitask-sandbox.mxapps.io/login.html} \textit{Τα στοιχεία σύνδεσής σας είναι τα εξής:}

            \begin{table}[H] \noindent
                \begin{tabular}{ll}
                    \textit{Όνομα χρήστη:} & \texttt{foithtisX} \\
                    \textit{Κωδικός πρόσβασης}: & \texttt{<password>}
                \end{tabular}
            \end{table}

            \noindent \textit{όπου} \texttt{X} \textit{ένας αριθμός μεταξύ 1 και 12, και} \texttt{<password>} \textit{ένα εξαψήφιο αλφαριθμητικό.}

            \paragraph{\textit{Σενάριο χρήσης.}}
                \textit{Αφού αποκτήσετε πρόσβαση στην εφαρμογή, θα σας ζητηθεί να εκτελέσετε μια σειρά από συγκεκριμένες ενέργειες που έχουν σχεδιαστεί για την αξιολόγηση της ευχρηστίας και της συνολικής εμπειρίας χρήστη.}

            \begin{enumerate}[label=\textit{\arabic*}.]
                \setlength\itemsep{-0.25em}
                \item \textit{Δοκιμάστε να πλοηγηθείτε στο γραφικό περιβάλλον της εφαρμογής: δείτε τις διαφορετικές σελίδες, τη δομή του μενού και τις διαθέσιμες λειτουργίες.}
                \item \textit{Ο στόχος είναι να δημιουργήσετε ένα σύνολο από εργασίες που αντιστοιχούν σε δραστηριότητες που κάνατε ή θα κάνετε στην καθημερινότητά σας. Δημιουργήστε μια ολοκληρωμένη εργασία, μια εργασία σε εξέλιξη και μια επόμενη εργασία. Για την καθεμία θέστε την κατάλληλη προτεραιότητα, χρώμα και δοκιμάστε να αφήσετε μια σύντομη σημείωση.}
                \item \textit{Επεξεργαστείτε τις εργασίες που δημιουργήσατε τροποποιώντας την ημερομηνία λήξης και την κατηγορία τους.}
                \item \textit{Διαγράψτε κάποια από τις εργασίες που δημιουργήσατε.}
                \item \textit{Δοκιμάστε τα παραπάνω και στη σελίδα} {\ZonaSB kanban}.
                \item \textit{Παρατηρήστε τις εργασίες που δημιουργήσατε στη σελίδα} {\ZonaSB ημερολόγιο}.
                \item \textit{Διαγράψτε με μια ενέργεια το σύνολο των εργασιών που δημιουργήσατε.}
            \end{enumerate}

            \textit{Μετά την ολοκλήρωση παρακαλούμε συμπληρώστε το σύντομο ερωτηματολόγιο αξιολόγησης ευχρηστίας.}

        \subsection{Ερωτηματολόγιο}
            Το System Usability Scale (SUS) είναι ένα από τα πιο διαδεδομένα και αξιόπιστα εργαλεία για την αξιολόγηση της χρηστικότητας ενός συστήματος, εφαρμογής ή διεπαφής χρήστη.

            Δημιουργήθηκε από τον John Brooke το 1986 και έχει χρησιμοποιηθεί ευρέως σε έρευνες και βιομηχανικές εφαρμογές για την αξιολόγηση της εμπειρίας χρήστη. Αποτελείται από 10 προτάσεις με τις οποίες οι συμμμετέχοντες καλούνται να τις βαθμολογήσουν σε μια κλίμακα 1 -- 5 με το 1 να αντιστοιχεί σε \say{Διαφωνώ απόλυτα} και το 5 σε \say{Συμφωνώ απόλυτα}. Οι προτάσεις έχουν σχεδιαστεί ώστε να καλύπτουν διάφορες πτυχές της χρηστικότητας, όπως η ευκολία χρήσης, η πολυπλοκότητα, η συνέπεια και η αυτοπεποίθηση των χρηστών κατά τη χρήση του συστήματος \cite{SUS, SUSXenos}.

            Ακολουθεί η λίστα των ερωτήσεων που κλήθηκαν να απαντήσουν οι συμμετέχοντες.

            \begin{table}[H] \noindent\centering\small
                    \begin{tabular}{l|l}
                        \# & \textbf{Ερωτήσεις} \\
                        \midrule
                        01 & Νομίζω ότι θα ήθελα να χρησιμοποιώ αυτή την εφαρμογή συχνά. \\
                        \midrule
                        02 & Βρήκα αυτή την εφαρμογή αδικαιολόγητα περίπλοκη. \\
                        \midrule
                        03 & Σκέφτηκα πως αυτή η εφαρμογή ήταν εύκολη στη χρήση. \\
                        \midrule
                        04 & Νομίζω ότι θα χρειαστώ βοήθεια από κάποιον τεχνικό \\
                          & για να είμαι σε θέση να χρησιμοποίησω αυτή την εφαρμογή. \\
                        \midrule
                        05 & Βρήκα τις διάφορες λειτουργίες σε αυτή την εφαρμογή καλά ολοκληρωμένες. \\
                        \midrule
                        06 & Σκέφτηκα ότι υπήρχε μεγάλη ασυνέπεια σε αυτή την εφαρμογή. \\
                        \midrule
                        07 & Φαντάζομαι ότι οι περισσότεροι άνθρωποι θα μάθουν \\
                          & να χρησιμοποιούν αυτή την εφαρμογή πολύ γρήγορα. \\
                        \midrule
                        08 & Βρήκα αυτή την εφαρμογή πολύ περίπλοκη/δύσκολη στη χρήση. \\
                        \midrule
                        09 & Ένιωσα πολύ σίγουρος/η χρησιμοποιώντας αυτή την εφαρμογή. \\
                        \midrule
                        10 & Χρειάστηκε να μάθω πολλά πράγματα πριν μπορέσω \\
                           & να ξεκινήσω με αυτή την εφαρμογή. \\
                    \end{tabular}
            \end{table}


    \section{Αποτελέσματα}
        Στον πίνακα που ακολουθεί παρουσιάζονται οι μέσες τιμές και οι τυπικές αποκλίσεις των απαντήσεων των συμμετεχόντων στο ερωτηματολόγιο.
        \begin{table}[H] \noindent\centering \small
                \begin{tabular}{c|c|c}
                   \textbf{Ερωτήσεις} & \textbf{Μέση τιμή} & \textbf{Τυπική απόκλιση} \\
                    \midrule
                    Ερώτηση 01  & 4.58 & 0.51 \\
                    Ερώτηση 02  & 1.83 & 0.57 \\
                    Ερώτηση 03  & 4.33 & 0.88 \\
                    Ερώτηση 04  & 1.41 & 0.79 \\
                    Ερώτηση 05  & 4.33 & 0.77 \\
                    Ερώτηση 06  & 1.08 & 0.29 \\
                    Ερώτηση 07  & 4.41 & 0.51 \\
                    Ερώτηση 08  & 1.66 & 0.77 \\
                    Ερώτηση 09  & 4.16 & 0.57 \\
                    Ερώτηση 10 & 1.50 & 0.67 \\
                \end{tabular}
        \end{table}

        Οι ερωτήσεις του SUS διακρίνονται σε θετικές και αρνητικές. Στις θετικές ερωτήσεις, που βρίσκονται σε περιττές θέσεις (1, 3, 5, 7, 9), η βαθμολογία κάθε συμμετέχοντα προκύπτει αφαιρώντας 1 από την αρχική του απάντηση. Για παράδειγμα, αν ένας χρήστης απαντήσει 5, η τιμή μετατρέπεται σε 4, αν απαντήσει 3, μετατρέπεται σε 2 κ.ο.κ. Αντίστοιχα, στις αρνητικές ερωτήσεις, που βρίσκονται σε ζυγές θέσεις (2, 4, 6, 8, 10), το σκορ υπολογίζεται αφαιρώντας την απάντηση από 5. Αυτό σημαίνει πως αν κάποιος απαντήσει 5, η τιμή μετατρέπεται σε 0, αν απαντήσει 3, μετατρέπεται σε 2 κ.ο.κ.

        Μετά την προσαρμογή των απαντήσεων, τα σκορ όλων των ερωτήσεων αθροίζονται και πολλαπλασιάζονται με τον συντελεστή 2.5. Με αυτόν τον τρόπο, το τελικό SUS σκορ κυμαίνεται από 0 έως 100, επιτρέποντας την εύκολη σύγκριση μεταξύ διαφορετικών προϊόντων ή εφαρμογών. Όσο υψηλότερο είναι το σκορ, τόσο καλύτερη θεωρείται η ευχρηστία της υπό εξέταση εφαρμογής.

        Στην παρούσα αξιολόγηση, οι απαντήσεις των χρηστών έδειξαν ότι η εφαρμογή ήταν ιδιαίτερα ευχάριστη στη χρήση, με μέση τιμή 85.82. Αυτή η τιμή υποδηλώνει ότι οι περισσότεροι χρήστες είχαν θετική εμπειρία χρήσης και αξιολόγησαν την εφαρμογή ως εύχρηστη και καλά σχεδιασμένη.
%    \chapter{Συμπεράσματα και μελλοντικές προεκτάσεις}
    Έχοντας αναφερθεί σε έννοιες που αφορούν τη διαχείριση εργασιών, τον χαμηλό κώδικα, τις πλατφόρμες χαμηλού κώδικα όπως το Mendix, και μετά από την πραγματοποίηση έρευνας προτίμησης και της πειραματικής διαδικασίας, φτάνουμε στον επίλογο της συγκεκριμένης διπλωματικής εργασίας, όπου πλέον θα αναλυθούν τα συμπεράσματα που βγάλαμε από ολόκληρη την υλοποίηση της διπλωματικής όπως επίσης και μελλοντικές προεκτάσεις της.

    \section{Σύνοψη και συμπεράσματα}
        Ο στόχος αυτής της διπλωματικής εργασίας είναι η ανάπτυξη μιας εφαρμογής σε Low-Code περιβάλλον για τον προγραμματισμό και την παρακολούθηση εργασιών. Για να επιτευχθεί αυτό, ήταν αναγκαίο να αποσαφηνίσουμε τι είναι η διαχείριση εργασιών. Πρόκειται για μια διαδικασία που, ιστορικά, αποτελεί ένα ζωτικό στοιχείο της κοινωνίας μας και του πολιτισμού μας. Από τα αρχαία χρόνια έως σήμερα, οι άνθρωποι χρησιμοποιούσαν διάφορα μέσα για να οργανώνουν τις εργασίες τους, από ημερολόγια και ρολόγια μέχρι σύγχρονες ψηφιακές εφαρμογές. Μάλιστα, πέρα από την οργάνωση, λόγω της βαρύτητας που έχει η διαχείριση εργασιών στην καθημερινότητά των ανθρώπων υπήρξε η ανάγκη για την περαιτέρω βελτιστοποίησή της, με μεθοδολογίες όπως η Kanban και τα διαγράμματα Gantt και PERT. Αυτές οι μεθοδολογίες επιτρέπουν την αποτελεσματική οργάνωση των εργασιών, την εκτίμηση του χρόνου και των πόρων που απαιτούνται για την ολοκλήρωσή τους, και την παρακολούθηση της προόδου τους.

        Όσον αφορά το θεματικό πλαίσιο της εφαρμογής, επιλέχθηκε το ακαδημαϊκό περιβάλλον καθώς πρόκειται για ένα περιβάλλον αγχώδες και απαιτητικό, με πολλαπλές υποχρεώσεις και ανάγκη για ιεράρχηση χρόνου, άρα με κατεξοχήν ανάγκη για συστηματική οργάνωση και διότι έτσι ήταν ευκολότερη η διεξαγωγή των ερευνητικών και πειραματικών διαδικασιών.

        Η ίδια η εφαρμογή αναπτύχθηκε στην πλατφόρμα χαμηλού κώδικα, Mendix. Ο χαμηλός κώδικας θεωρείται το μέλλον της ανάπτυξης λογισμικού, καθώς ενοποιεί τους δύο πιο σημαντικούς πυλώνες της: την ανάγκη αυτοματοποίησης επαναλαμβανόμενων ενεργειών (όπως τα CASE περιβάλλοντα και η MDA αρχιτεκτονική), και την αύξηση του επιπέδου αφαίρεσης των γλωσσών προγραμματισμού διευκολύνοντας τη χρήση τους από μη εξειδικευμένους χρήστες. Οι πλατφόρμες ανάπτυξης λογισμικού σε χαμηλό κώδικα επιτρέπουν την ανάπτυξη εφαρμογών με τη χρήση γραφικού περιβάλλοντος και drag-and-drop στοιχείων. Παρέχουν γρήγορη μοντελοποίηση των δεδομένων, διαγράμματα ροής για τη λογική, χρήση προκατασκευασμένων στοιχείων και άλλα.

        Για τη δημιουργία της εφαρμογής χρησιμοποιήθηκαν υπάρχουσες βιβλιογραφικές έρευνες όπου εξερευνούσαν τα προβλήματα διαχείρισης των φοιτητών και τα χαρακτηριστικά που αυτοί θα επιθυμούσαν σε μια εφαρμογή, αλλά επιπλέον διεξήχθη και μια πρωτογενής έρευνα προτίμησης μεταξύ 14 φοιτητών. Η έρευνα έδειξε πως οι φοιτητές αντιμετωπίζουν προβλήματα στη διαχείριση των υποχρεώσεών τους με προβλήματα στην τήρηση προθεσμιών, ενώ θεωρούν ανεπαρκή τον τρόπο οργάνωσης που ακολουθούν.

        Η ανάπτυξη της εφαρμογής βασίστηκε στα αποτελέσματα της έρευνας, με έμφαση στα χαρακτηριστικά που οι φοιτητές θεώρησαν πιο χρήσιμα. Στόχος ήταν η δημιουργία μιας εφαρμογής που θα είναι αποτελεσματική, εύχρηστη και φιλική προς τον χρήστη και μπορεί όντως να αποτελέσει μια επιλογή διαχείρισης εργασιών και πέρα από τα πλαίσια αυτής της διπλωματικής εργασίας. Η εφαρμογή περιλαμβάνει τη δυνατότητα δημιουργίας εργασιών, την οργάνωσή τους σε διάφορες κατηγορίες και προτεραιότητες, την παρακολούθηση της προόδου τους, τη χρήση μηνυμάτων και ειδοποιήσεων για την ενημέρωση του χρήστη μέχρι τη λήξη τους, ενώ περιλαμβάνονται διαφορετικές σελίδες και προβολές (όπως πίνακας Kanban και ημερολόγιο) για την ευκολότερη παρακολούθηση των εργασιών.

        Τέλος, το έργο της διπλωματικής αξιολογήθηκε μέσω πειράματος συμμετοχής χρηστών σε ένα SUS ημερολόγιο όπου 23 συμμετέχοντες διαφορετικών ακαδημαϊκών επιπέδων και εξειδικεύσεων εκτέλεσαν ένα τυπικό σενάριο χρήσης και αξιολόγησαν την εφαρμογή με βάση την ευχρηστία της. Τα αποτελέσματα ανέδειξαν την ευχρηστία της εφαρμογής, με χαμηλές τυπικές αποκλίσεις και με μέσο όρο 83.59\% στις μέσες τιμές. Αυτό αποδεικνύει ότι η εφαρμογή υλοποιήθηκε με επιτυχία, είναι αποτελεσματική και μπορεί να αποτελέσει μια επιλογή για τη διαχείριση εργασιών σε ακαδημαϊκό περιβάλλον.


    \section{Μελλοντικές προεκτάσεις}
        Βάσει της ανάγκης για μια πιο ολοκληρωμένη λύση, η εφαρμογή μπορεί να επεκταθεί με νέες λειτουργίες και βελτιώσεις στις υπάρχουσες:

        Αρχικά, η εφαρμογή μπορεί να ενσωματώσει τη δυνατότητα συγχρονισμού με εξωτερικές εφαρμογές και υπηρεσίες όπως ημερολόγια (Google Calendar) ή άλλες task-management εφαρμογές (Trello, Notion) με τη χρήση REST APIs. Η αξιοποίηση των Java actions και των microflows στο Mendix μπορεί να χρησιμοποιηθεί για την υλοποίηση διαδικασιών εξουσιοδότησης μέσω OAuth2, εξασφαλίζοντας την ασφαλή ανταλλαγή δεδομένων μεταξύ της εφαρμογής και των εξωτερικών συστημάτων. Έτσι, ο χρήστης θα μπορεί να διαχειρίζεται τις εργασίες του από μία κεντρική εφαρμογή, ενώ θα μπορεί να ενημερώνεται για τις εργασίες του από διάφορες πηγές.

        Επίσης, μπορούν να προστεθούν collaboration εργαλεία που θα επιτρέπουν τη δυνατότητα συνεργασίας μεταξύ χρηστών/φοιτητών μέσω κοινών εργασιών, σχολίων ή ανταλλαγής μηνυμάτων. Πατώντας πάνω σε αυτή την ιδέα μπορεί η εφαρμογή να εξελιχθεί σε μία πλατφόρμα κοινωνικής δικτύωσης περιλαμβάνοντας προσωπικά προφίλ, δημόσιες και ιδιωτικές συζητήσεις της πανεπιστημιακής κοινότητας όπου οι φοιτητές θα μπορούν να ανταλλάσσουν πληροφορίες και μηνύματα σε πραγματικό χρόνο και να συνεργάζονται σε κοινά projects.

        Επιπλέον, για την ενοποίηση της εφαρμογής με το πανεπιστημιακό περιβάλλον, μπορεί να προστεθεί η δυνατότητα ενσωμάτωσης δεδομένων από το Ψηφιακό Άλμα / e-class, όπως τα μαθήματα, οι βαθμοί και οι ανακοινώσεις, κάτι που θα επιτρέψει την απεικόνιση ακριβών και ενημερωμένων ακαδημαϊκών πληροφοριών, δημιουργώντας έτσι έναν ολοκληρωμένο κόμβο πληροφοριών για τους φοιτητές.

        Επίσης, η αξιοποίηση αλγορίθμων μηχανικής μάθησης θα μπορούσε να επιτρέψει την αυτόματη δημιουργία εργασιών κάνοντας scrape τις ανακοινώσεις των μαθημάτων από το API του e-class. Κάτι τέτοιο μπορεί να επιτευχθεί απευθείας χρησιμοποιώντας Python και NLP τεχνικές και τεχνικές κατηγοριοποίησης δεδομένων χρησιμοποιώντας βιβλιοθήκες όπως Sklearn ή Tensorflow. Ο Python κώδικας μπορεί να συνδεθεί στο Mendix με modules στο Marketplace όπως το AWS Lambda Connector ή σε server ξεχωριστά (Flask, FastAPI κ.α.), δημιουργώντας endpoints σε REST APIs που καλεί το Mendix.

        Έτσι, η εφαρμογή θα μπορεί να αναλύει νέες ανακοινώσεις χρησιμοποιώντας προηγούμενα δεδομένα, να τις κατηγοριοποιεί και δημιουργεί αυτοματοποιημένα νέες επόμενες εργασίες για τους χρήστες, παρέχοντας εξατομικευμένες προτάσεις για τον προγραμματισμό των υποχρεώσεών τους. Με αυτό τον τρόπο, η εφαρμογή θα μπορούσε να λειτουργεί ως ένας προσωπικός βοηθός για τους φοιτητές, προτείνοντάς τους εργασίες που πρέπει να ολοκληρώσουν και προτείνοντας τους τρόπους για την εκτέλεσή τους, κάτι πολύ σημαντικό δεδομένου ότι οι φοιτητές συχνά αντιμετωπίζουν προβλήματα με την οργάνωση του χρόνου τους και την προθεσμία των εργασιών τους.

        Επιπλέον, το σύστημα επιβράβευσης μπορεί να εξελιχθεί περαιτέρω σε ένα gamification σύστημα, με τη δημιουργία διαφορετικών επιπέδων, badges, πόντων και επιτευγμάτων ως ανταμοιβή για την ολοκλήρωση εργασιών, με παρόμοια λογική όπως η εφαρμογή Duolingo. Οι φοιτητές θα μπορούν να ανταγωνίζονται μεταξύ τους για την απόκτηση badges και την αύξηση των πόντων τους, ενθαρρύνοντάς τους να ολοκληρώνουν τις εργασίες τους εγκαίρως και να βελτιώνουν την απόδοσή τους.

        Τέλος, η εφαρμογή είναι κατασκευασμένη χρησιμοποιώντας Web-based εργαλεία στο Mendix. Μπορεί ταυτόχρονα, μέσω του Mendix Native Mobile να δημιουργηθεί μια καθαρά mobile εφαρμογή, θα εξασφαλίσει την απρόσκοπτη πρόσβαση και χρήση των λειτουργιών της εφαρμογής από κινητές συσκευές, επιτρέποντας την υλοποίηση push notifications και offline λειτουργικότητας.

        Έτσι, η εφαρμογή θα μπορούσε να μετατραπεί σε ένα ενοποιημένο σύστημα πληροφόρησης και οργάνωσης της ακαδημαϊκής ζωής των φοιτητών, ώστε να μπορούν να αντιμετωπίσουν τις προκλήσεις της ακαδημαϊκής ζωής με περισσότερη αυτοπεποίθηση και αυτονομία, ενώ ταυτόχρονα θα μπορούν να απολαμβάνουν την εμπειρία της μάθησης με περισσότερη οργάνωση και αποτελεσματικότητα.



    \printbibliography
\end{document}