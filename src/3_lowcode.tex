\chapter{Low-Code}
    Στο κεφάλαιο αυτό, περιγράφεται η έννοια του low-code (χαμηλός κώδικας). Αναφερόμαστε στην έννοια του Low-Code, στην ιστορική εξέλιξη της μηχανικής λογισμικού που επέτρεψε την ύπαρξη πλατφόρμων ανάπτυξης λογισμικού σε low-code...
%    Τα πετρογραφικά και οι ζωγραφιές στους τοίχους παλαιών σπηλαίων είναι ένα δείγμα από το πόσο ουσιώδης είναι η οπτική επικοινωνία για τον άνθρωπο. \cite{CASEKuhn}


    \section{Τι είναι ο χαμηλός κώδικας}
        Ένας τρόπος για να αντιληφθούμε καλύτερα την έννοια του χαμηλού κώδικα είναι να κοιτάξουμε την εξέλιξη των γλωσσών προγραμματισμού. Για παράδειγμα, η Python, μια γλώσσα υψηλού επιπέδου, θα μπορούσε να χαρακτηριστεί ως χαμηλός κώδικας σε σύγκριση με τη C++. Αντίστοιχα η C θα μπορούσε να χαρακτηριστεί και αυτή χαμηλός κώδικας συγκριτικά με την Assembly, όπως και η Assembly είναι χαμηλός κώδικας αν τη συγκρίνουμε με το να αλλάζουμε χειροκίνητα μηδενικά και άσσους στους καταχωρητές. Πρακτικά, η εξέλιξη του προγραμματισμού, είναι και η εξέλιξη του χαμηλού κώδικα. Ο χαμηλός κώδικας επιτρέπει σε κόσμο με λιγότερη προγραμματιστική εμπειρία\footnote{Συχνά αποκαλούμενοι ως \textbf{citizen developers} (προγραμματιστές πολίτες), πρόκειται για χρήστες με λίγη ή μηδενική προγραμματιστική εμπειρία που χρησιμοποιούν προγραμματιστικά εργαλεία χαμηλού κώδικα για τον σχεδιασμό εφαρμογών.} και γνώσεις να προγραμματίσουν, όπως η Python είναι πιο προσβάσιμη σε σχέση με τη Java ή την Assembly.

        Επομένως, αν και ο όρος \say{low-code} είναι σχετικά πρόσφατος, αναφερόμαστε σε μια έννοια αφαιρετικότητας που δεν είναι καθόλου καινούρια.
        Στο παρελθόν οι χρήστες αξιοποιούν τεχνολογίες όπως η γλώσσα Visual Basic της Microsoft παλαιότερα, ή πλατφόρμες όπως η Outsystems ή Mendix.
        Ο προγραμματισμός σε χαμηλό κώδικα περιλαμβάνει γραφικό περιβάλλον με drag-and-drop επαναχρησιμοποιούμενα στοιχεί, επιτρέποντας την πιο γρήγορη και ενστικτώδη κατασκευή εφαρμογών. Παρακάτω παρατίθενται δύο παραδείγματα από την παραδοσιακή ανάπτυξη εφαρμογών και την ανάπτυξη εφαρμογών σε low-code στην πλατφόρμα Mendix. \cite{LowCodeMendix} \cite{LowCodeDemocratization}

        \begin{figure}[H] \noindent \centering
                \includegraphics[trim={0 15cm 0 3cm}, clip, width=0.8\textwidth]{ClassicProgramming}
                \includegraphics[trim={0 0 0 0}, clip, width=0.8\textwidth]{MendixLowCodeDevelopment}
                \caption{Παραδοσιακός κώδικας και το γραφικό περιβάλλον του Mendix.}
        \end{figure}

        Στο γραφικό περιβάλλον ο προγραμματισμός γίνεται σε ένα διάγραμμα ροής, σε αντίθεση με τον παραδοσιακό κώδικα όπου γράφουμε γραμμή-γραμμή, κάτι που καθιστά την ανάπτυξη εφαρμογών εξαιρετικά γρήγορη και προσβάσιμη από περισσοτέρους. Παρόλα αυτά, προτού αναφερθούμε εκτενέστερα στις πλατφόρμες ανάπτυξης λογισμικού σε low-code, αξίζει να κάνουμε μια αναφορά στις τεχνολογικές εξελίξεις που μας οδήγησαν σήμερα σε αυτές τις πλατφόρμες.

    \section{Πώς φτάσαμε στον χαμηλό κώδικα}
        Η μηχανική λογισμικού\footnote{Θα ορίζαμε τη μηχανική λογισμικού ως μια πειθαρχημένη και αυστηρή εφαρμογή μεθόδων, διαδικασιών και εργαλείων στη διαχείριση και ανάπτυξη υπολογιστικών συστημάτων. Πρόκειται για ένα εννοιλογικό πλαίσιο που περιγράφει τη διαχείριση συστημάτων.} έχει περάσει πολλά στάδια στην ιστορία της μέχρι να αρχίσουμε να αναφερόμαστε και σε χρήση χαμηλού κώδικα. Μέχρι τη δεκαετία του 1970, η ανάπτυξη πληροφοριακών συστημάτων έμοιαζε παραπάνω ως μια τέχνη παρά ως επιστήμη, καθώς δεν ακολουθούσε κάποια δόμηση. Αντιθέτως, ο προγραμματιστής λάμβανε μια σειρά από απαιτήσεις και ανάγκες από την πλευρά του χρήστη, και μετά από ένα διάστημα παρέδιδε ένα σύστημα που συνήθως δεν κάλυπτε εξ' ολοκλήρου όλες τις απαιτήσεις του χρήστη αλλά ήταν σίγουρα καλύτερο από το τίποτα. Η συγκεκριμένη μέθοδος, αποκαλούμενη ως \textbf{κλασική μέθοδος}, χαρακτηρίζεται από ανεπίσημες οδηγίες, έλλειψη τυποποίησης και αναφορών (documentation).

        Η ανάγκη για αύξηση της παραγωγικότητας στον κύκλο ζωής έκδοσης λογισμικού (software release life cycle)\footnote{Πρόκειται μια έννοια που αναφέρεται στις φάσεις ανάπτυξης και ύπαρξης ενός λογισμικού. Ξεκινάει από τη σύλληψη της ιδέας, τη μελέτη για τις απαιτήσεις του, την υλοποίηση του, τη διάθεση του προϊόντος στο πελάτη, τη υποστήριξή του με ενημερώσεις και τέλος την απόσυρσή του.} οδήγησε στη δημιουργία \textbf{επίσημων μεθόδων} στα τέλη της δεκαετίας του 1970. Σκοπός τους ήταν η τυποποίηση του σχεδιασμού με στόχο τη βελτίωση της ποιότητάς του. Παράδειγμα αυτών των τεχνικών είναι η χρήση δομημένης ανάλυσης (structured analysis). Έτσι, οι μηχανικοί μπορούσαν να φτιάξουν διαγράμματα ροών δεδομένων (data flow diagrams)\footnote{Είναι μια οπτική αναπαράσταση της ροής των δεδομένων σε ένα σύστημα. Αναγράφονται οι διεργασίες (κύκλοι), οι είσοδοι και έξοδοι (τετράγωνα) και η αποθήκευση των δεδομένων (παράλληλες γραμμές).}, μοντέλα οντοτήτων-συσχετίσεων (ER -- entity-relationship models)\footnote{Περιγράφει ένα σύνολο αντικειμένων (οντότητες) και τις \textit{σχέσεις} μεταξύ αυτών των αντικειμένων.}, δημιουργώντας μια συστημική περιγραφή του λογικού και φυσικού μέρους του πληροφοριακού συστήματος που ανέπτυσσαν.

        Με την περαιτέρω ανάπτυξη των προσωπικών υπολογιστών στα μέσα της δεκαετίας του 1980, η χρήση επίσημων μεθόδων ήταν μονόδρομος, και μάλιστα δημιούργησε μια νέα τάση, την ανάγκη για αυτοματοποίηση της ανάπτυξης κώδικα και την ελαχιστοποίηση της χειροκίνητης γραφής του.

        Απόρροια αυτής της τάσης ήταν α) τη δεκαετία του 1980 οι \textbf{γλώσσες προγραμματισμού τέταρτης γενιάς} (4GLs)\footnote{Σε αντίθεση με τις γλώσσες προγραμματισμού τρίτης γενιάς (C, Java κτλ), οι γλώσσες προγραμματισμού τέταρτης γενιάς σχεδιάστηκαν με γνώμονα την απλοποίηση του προγραμματισμού ειδικά για χρήστες χωρίς προγραμματιστική εμπειρία. Χαρακτηρίζονται από υψηλού επιπέδου αφαιρέσεις, πλησιάζοντας στην ανθρώπινη γλώσσα, κάνοντάς τες πιο εύκολα κατανοητές.}, τα \textbf{Computer-Aided Software Engineering} (CASE) περιβάλλοντα, β) η \textbf{Ταχεία Ανάπτυξη Εφαρμογών} (Rapid Application Development -- RAD)\footnote{Πρόκειται για μια μεθοδολογία που δίνει βάση στη δημιουργία πρωτοτύπων. Έτσι οι προγραμματιστές δε χρειάζεται να ξεκινάνε από το μηδέν την ανάπτυξη κάθε νέου λογισμικού, ούτε να ξοδεύουν πολύτιμο χρόνο για να περιγράψουν λεπτομερώς όλες τις προδιαγραφές του. Χρησιμοποιώντας έτοιμα (μάλιστα και modular) πρωτότυπα, ο χρόνος ανάπτυξης μειώνεται. Παραδείγματα RAD εργαλείων είναι οι GUI builders· πρόκειται για WYSIWYG (What-You-See-Is-What-You-Get) επεξεργαστές για τη γρήγορη ανάπτυξη λογισμικών με διεπαφή χρήστη (user interface).} τη δεκαετία του 1990, γ) ο \textbf{Προγραμματισμός τελικού χρήστη} (End-User Development -- EUD)\footnote{Περιγράφει εργαλεία που επιτρέπουν τον προγραμματισμό από τους απλούς τελικούς χρήστες. Παραδείγματα είναι λογιστικά φύλλα όπως το Microsoft Excel ή εκπαιδευτικά εργαλεία όπως το Scratch.} τη δεκαετία του 2000, και οι \textbf{αρχιτεκτονικές που βασίζονται σε μοντέλα} (model-driven architecture -- MDA) τις τελευταίες δύο δεκαετίες. \cite{Case1985, CASEChikofsky, MDELow}

        Στις επόμενες υποενότητες θα αναφερθούμε πιο εκτεταμένα στα CASE και MDA περιβάλλοντα, τα οποία υπήρξαν και τα κύρια πρωταίτια των Low-Code περιβαλλόντων.

        \subsection{Computer-Aided Software Enginnering (CASE)}
            Τα \textbf{Computer-Aided Software Engineering} (CASE -- Μηχανική Λογισμικού Υποβοηθούμενη από Υπολογιστή) περιβάλλοντα επέτρεπαν τους μηχανικούς να καταγράφουν και να μοντελοποιούν με συστηματικό τρόπο ένα πληροφοριακό σύστημα από τις αρχικές περιγραφές του χρήστη ως τη σχεδίαση και την υλοποίηση και να εκτελούν δοκιμές για τη συνέπειά του.

            Μέχρι πρότινος, τα εργαλεία που αφορούν την ανάπτυξη λογισμικού εστιάζουν κυρίως στην επεξεργασία πηγαίου κώδικα και την αποσφαλμάτωση του. Σε αντίθεση λοιπόν με τα υπάρχοντα εργαλεία, τα CASE περιβάλλοντα βοηθούν στη \textit{μεθοδολογία} της ανάπτυξης λογισμικού: στην ανάλυση απαιτήσεων, στο λογικό σχεδιασμό, στον έλεγχο εγκυρότητας, την επαναχρησιμοποίηση και εξάλειψη πλεονασμών.

            Είναι το δεξί χέρι ενός μηχανικού λογισμικού, βοηθώντας σε πολλές εργασίες που τον δυσκολεύουν, αυξάνοντας την παραγωγικότητα και την ποιότητα της δουλειάς του. Τα εργαλεία που περιλαμβάνουν ποικίλλουν: κάποια επιτρέπουν τη δημιουργία διαγραμμάτων ενώ άλλα μπορούν να αυτοματοποιούν όλα τα στάδια του κύκλου ζωής έκδοσης λογισμικού. Ένα ολοκληρωμένο CASE περιβάλλον περιλαμβάνει:
            \vspace{-0.5em}
            \begin{itemize}[label={\tiny \blacksquare}]
                \setlength\itemsep{-0.25em}
                \item ένα διαδραστικό, φιλικό για το χρήστη, γραφικό περιβάλλον διαχείρισης
                \item ένα σύνολο από εργαλεία ανάπτυξης (επεξεργαστές κειμένου, λεξικά, αναλυτές σχεδιασμού κ.α.)
                \item ένα σύνολο από εργαλεία για τον έλεγχο της διαδικασίας (για τον χρονοπρογραμματισμό, τη διασφάλιση ποιότητας κ.α.)
                \item ένα περιβάλλον βοήθειας με το documentation των εργαλείων
                \item ένα σύστημα διαχείρισης βάσεων δεδομένων
            \end{itemize}
            \vspace{-0.5em}

            Τα συστήματα που δημιουργούνται από το CASE είναι εφαρμογές της πειθαρχημένης εφαρμογής μεθόδων της μηχανικής λογισμικού.
            Τα CASE περιβάλλοντα έθεσαν τα θεμέλια για τη δημιουργία νέων προτύπων, όπως ο οπτικός προγραμματισμός (visual programming) και ο προγραμματισμός που βασίζεται σε μοντέλα. \cite{CASEChikofsky, Case1985, CASEKuhn, AdoptionCASE}

        \subsection{Model-driven Architecture (MDA)}
            Τα μοντέλα προσφέρουν τη δυνατότητα αφαίρεσης σε ένα σύστημα, κάτι που επιτρέπει τους μηχανικούς να επικεντρώνονται μόνο στο πρόβλημα που προσπαθούν να λύσουν, αγνοώντας τις υπόλοιπες λεπτομέρειες. Η χρήση μοντέλων είναι ζωτικής σημασίας για την κατανόηση και επεξεργασία πολύπλοκων συστημάτων. Ένα παράδειγμα μοντελοποίησης, για παράδειγμα, θα μπορούσε να είναι τα διαφορετικά επίπεδα εμφάνισης ενός συστήματος (δομικό επίπεδο, επίπεδο συμπεριφοράς κα).

            Η ιδέα των μοντέλων αποτέλεσε τη βάση για μια νέα προσέγγιση ανάπτυξης λογισμικού, τη \textit{μηχανική που βασίζονται σε μοντέλα} (model-driven engineering -- MDE). Μπορούμε να περιγράψουμε με σαφήνεια και με κανόνες τα μοντέλα (για παράδειγμα το πως θα μετατραπεί ένα μοντέλο σε ένα άλλο, την παρακολούθηση μεταξύ στοιχείων ενός μοντέλου κτλ), συντελώντας πλέον στην \textbf{αρχιτεκτονική που βασίζεται σε μοντέλα} (model-driven architecture -- MDA).

            Η MDA υποστηρίζεται από την Object Management Group (OMG) \cite{OMG_MDA}, βασίζεται σε ένα σύνολο προτύπων για τον ορισμών μοντέλων, συμβολισμών και κανόνων μετασχηματισμού και προσφέρει μια βάση για τη λειτουργία μοντέλων όπως το UML. Τα μοντέλα χρησιμοποιούνται για τον προσδιορισμό, την προσομοίωση, την επαλήθευση, τον εκσυγχρονισμό, τη συντήρηση, την κατανόηση και τη δημιουργία κώδικα. Στόχος παραμένει η αυτοματοποίηση διαφόρων βημάτων στην ανάπτυξη λογισμικού, αυξάνοντας παράλληλα την ποιότητά του. Επιπλέον, χρησιμοποιώντας μοντέλα είναι εφικτός ο διαχωρισμός της λειτουργικότητας των εφαρμογών από την υλοποίησή της σε μια συγκεκριμένη πλατφόρμα. Ως αποτέλεσμα, οι προγραμματιστές μπορούν να εστιάζουν περισσότερο στον σχεδιασμό και λιγότερο στο να λύνουν θέματα που αφορούν την πλατφόρμα υλοποίησης.

            Εν τέλει, η αρχιτεκτονική που βασίζεται σε μοντέλα άλλαξε τον τρόπο σκέψης των προγραμματιστών, καθώς πλέον επικεντρώνονταν παραπάνω στον σωστό διαχωρισμό των χαρακτηριστικών, στην αφαιρετικότητα και στην αυτοματοποίηση.
            \cite{Bucaioni2022, MDELow, MDSDSpringer}

    \section{Πλατφόρμες Ανάπτυξης Εφαρμογών σε Low-Code (LCDP)}
        Όσο και αν η έννοια των μοντέλων άλλαξε τη μηχανική λογισμικού και έθεσε νέες προδιαγραφές στον σχεδιασμό του, η εξάρτησή της από δύσχρηστα πρότυπα όπως το UML περιόριζε την ευρεία υιοθέτησή της στη βιομηχανία. Τα τελευταία χρόνια έχουν εμφανιστεί πλατφόρμες χτισμένες πάνω στις αρχές της αφαίρεσης των μοντέλων, που απλοποιούν ακόμη παραπάνω τη διαδικασία της ανάπτυξης. Οι συγκεκριμένες πλατφόρμες ονομάζονται \textbf{Πλατφόρμες Ανάπτυξης Εφαρμογών σε Low-Code} (Low-Code Development Platforms -- LCDPs)\footnote{Εναλλακτικές ονομασίες είναι Low-Code Platforms (LCP), Low-Code Development Platforms (LCDP), ενώ η διαδικασία ανάπτυξης αναφέρεται ως Low-Code Software Development (LCSD). Η έννοια του Low-Code συχνά αναφέρεται και ως Χαμηλός Κώδικας.}.\cite{Bock2021}

        \begin{figure}[H] \noindent \centering
                \includegraphics[width=0.6\textwidth]{MDE_LCSD}
                \caption{\centering Σύγκριση μεταξύ της μηχανικής που βασίζεται σε μοντέλα και των Low-Code πλατφορμών. \cite{MDELow}}
        \end{figure}

        \begin{displayquote}
            \small Στην περιοχή 1 χρησιμοποιούνται μοντέλα χωρίς να γίνονται προσπάθειες για μείωση του κώδικα, σε αντίθεση με τις περιοχές 2 και 3 που στοχεύουν στην μείωση του κώδικα. Από την άλλη, οι πλατφόρμες ανάπτυξης κώδικα σε low-code δεν βασίζονται πάντα σε μοντέλα (περιοχές 4 και 5), αλλά για παράδειγμα χρησιμοποιούν δεδομένα σε σχεσιακές βάσεις ή σε XML αρχεία. Οι διαφορές μεταξύ low-code application platforms και low-code software development έγγειται στο ότι οι πλατφόρμες προσφέρουν επιπλέον την δυνατότητα διάθεσης του λογισμικού στο ευρύ κοινό, όπως επίσης και την διαχείρισή του καθ' όλο το κύκλο ζωής του.
        \end{displayquote}

        Μια \textbf{Πλατφόρμα Ανάπτυξης Εφαρμογών σε Low-Code} (LCAP) είναι μια πλατφόρμα ανάπτυξης λογισμικού που υποστηρίζει την ταχεία ανάπτυξη και διαχείριση εφαρμογών. Συνήθως είναι Platform-as-a-service (PaaS) cloud μοντέλα, και χρησιμοποιείται ελάχιστος ή και μηδενικός δομημένος προγραμματισμός (structured \linebreak programming).

        Για τον προγραμματισμό παρέχεται γραφικό περιβάλλον με οπτικές αφαιρέσεις (visual abstractions). Έτσι οι προγραμματιστές ή οι citizen developers εστιάζουν παραπάνω στη \textit{σχεδίαση} της εφαρμογής, χωρίς να ξοδεύουν χρόνο άσκοπα σε λεπτομέρειες.

        Με λίγα λόγια, σκοπός είναι η παραγωγική ανάπτυξη λογισμικού με τη λιγότερη δυνατή προσπάθεια και με χαμηλότερο κόστος, και η εύκολη προσαρμογή του λογισμικού στις ταχέως μεταβαλλόμενες συνθήκες των σημερινών λειτουργικών συστημάτων. Η χρήση των LCAP έχει τύχει θετικής αποδοχής από τη βιομηχανία και η υιοθέτησή τους αυξάνεται συνεχώς. \cite{Bock2021,Bucaioni2022,Sahay2020}

        \begin{figure}[h!] \noindent \centering
                \includegraphics[width=0.9\textwidth]{LowCodePlatforms}
                \caption{LCDPs από αριστερά προς τα δεξιά: Mendix, OutSystems, Appian \cite{LowCodeMendix}}
        \end{figure}

        \begin{displayquote} \justifying
            \say{When you can visually create new business applications with minimal \linebreak hand-coding --when your developers can do more of greater value, faster-- \linebreak that’s low-code.} \cite{Ibm_2024}
        \end{displayquote}

        \subsection{Χαρακτηριστικά των LCDP}
            Παραθέτονται κάποια από τα χαρακτηριστικά που διακρίνουν τις πλατφόρμες ανάπτυξης σε Low-Code:

            \vspace{-0.5em}
            \begin{itemize}[label={\tiny \blacksquare}]
                \setlength\itemsep{-0.25em}
                \item \textbf{Γραφικό περιβάλλον χρήστη}: περιέρχονται εργαλεία και γραφικά στοιχεία, drag-and-drop ευελιξίες, μηχανές αποφάσεων για τη μοντελοποίηση σύνθετης λογικής, κατασκευαστές φορμών (form builder)
                \item \textbf{Συνεργατική ανάπτυξη}: επιτρέπει (απομακρυσμένους) χρήστες να εργαστούν στο πρότζεκτ
                \item \textbf{Επαναχρησιμοποίηση} προκατασκευασμένων στοιχείων, ύπαρξη marketplace,
                \item \textbf{Domain Model}: κατασκευή domain model για την αναπαράσταση εννοιών και σχέσεων μεταξύ τους. \cite{MDELow}
            \end{itemize}
            \vspace{-0.5em}

