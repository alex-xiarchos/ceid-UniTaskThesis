\chapter{Low-Code}
    Τα πετρογραφικά και οι ζωγραφιές στους τοίχους παλαιών σπηλαίων είναι ένα δείγμα από το πόσο ουσιώδης είναι η οπτική επικοινωνία για τον άνθρωπο. \cite{CASEKuhn}


    %    Από το 2014 και μετά, αρχίζει και εδραιώνεται μια νέα προσέγγιση ανάπτυξης λογισμικού, που έχει ως άμεσο στόχο την αύξηση της παραγωγικότητας των προγραμματιστών. Οι πλατφόρμες αυτές ονομάζονται \textbf{Πλατφόρμες Ανάπτυξης Εφαρμογών σε Low-Code} (Low-Code Application Platforms - LCDP)\footnote{Εναλλακτικές ονομασίες είναι Low-Code Platforms (LCP), Low-Code Development Platforms (LCDP), ενώ η διαδικασία ανάπτυξης αναφέρεται ως Low-Code Software Development (LCSD). Η έννοια του Low-Code συχνά αναφέρεται και ως Χαμηλός Κώδικας.}. \cite{Bock2021}

    \section{Ορισμός}
        Μια \textbf{Πλατφόρμα Ανάπτυξης Εφαρμογών σε Low-Code} (LCAP) είναι μια πλατφόρμα ανάπτυξης λογισμικού που υποστηρίζει την ταχεία ανάπτυξη και διαχείριση εφαρμογών. Συνήθως είναι Platform-as-a-service (PaaS) cloud μοντέλα, και χρησιμοποιείται ελάχιστος ή και μηδενικός δομημένος προγραμματισμός (structured \linebreak programming).

        Για τον προγραμματισμό παρέχεται γραφικό περιβάλλον με οπτικές αφαιρέσεις (visual abstractions), μάλιστα επιτρέποντας χρήστες χωρίς προγραμματιστική εμπειρία να συνεισφέρουν στην ανάπτυξη του λογισμικού χωρίς είναι αναγκαία η βοήθεια των προγραμματιστών. Έτσι οι προγραμματιστές εστιάζουν παραπάνω στη \textit{σχεδίαση} της εφαρμογής, χωρίς να ξοδεύουν χρόνο άσκοπα σε λεπτομέρειες.

        Με λίγα λόγια, σκοπός είναι η παραγωγική ανάπτυξη λογισμικού με τη λιγότερη δυνατή προσπάθεια και με χαμηλότερο κόστος, και η εύκολη προσαρμογή του λογισμικού στις ταχέως μεταβαλλόμενες συνθήκες των σημερινών λειτουργικών συστημάτων. Η χρήση των LCAP έχει τύχει θετικής αποδοχής από τη βιομηχανία και η υιοθέτησή τους αυξάνεται συνεχώς. \cite{Bock2021,Bucaioni2022,Sahay2020}

        \begin{displayquote} \justifying
            \say{When you can visually create new business applications with minimal \linebreak hand-coding --when your developers can do more of greater value, faster-- \linebreak that’s low-code.} \cite{Ibm_2024}
        \end{displayquote}

    \section{Πριν το Low-Code}
        Φυσικά η μηχανική λογισμικού\footnote{Θα ορίζαμε τη μηχανική λογισμικού ως μια πειθαρχημένη και αυστηρή εφαρμογή μεθόδων, διαδικασιών και εργαλείων στη διαχείριση και ανάπτυξη υπολογιστικών συστημάτων. Πρόκειται για ένα εννοιλογικό πλαίσιο που περιγράφει τη διαχείριση συστημάτων.} έχει περάσει πολλά στάδια στην ιστορία της μέχρι να αρχίσουμε να αναφερόμαστε και σε χρήση αφαιρέσεων. Μέχρι τη δεκαετία του 1970, η ανάπτυξη πληροφοριακών συστημάτων έμοιαζε παραπάνω ως μια τέχνη παρά ως επιστήμη, καθώς δεν ακολουθούσε κάποια δόμηση. Αντιθέτως, ο προγραμματιστής λάμβανε μια σειρά από απαιτήσεις και ανάγκες από την πλευρά του χρήστη, και μετά από ένα διάστημα παρέδιδε ένα σύστημα που συνήθως δεν κάλυπτε εξ' ολοκλήρου όλες τις απαιτήσεις του χρήστη αλλά ήταν σίγουρα καλύτερο από το τίποτα. Η συγκεκριμένη μέθοδος, αποκαλούμενη ως \textbf{κλασική μέθοδος}, χαρακτηρίζεται από ανεπίσημες οδηγίες, έλλειψη τυποποίησης και αναφορών (documentation).

        Η ανάγκη για αύξηση της παραγωγικότητας στον κύκλο ζωής έκδοσης λογισμικού (software release life cycle)\footnote{Πρόκειται μια έννοια που αναφέρεται στις φάσεις ανάπτυξης και ύπαρξης ενός λογισμικού. Ξεκινάει από τη σύλληψη της ιδέας, τη μελέτη για τις απαιτήσεις του, την υλοποίηση του, τη διάθεση του προϊόντος στο πελάτη, τη υποστήριξή του με ενημερώσεις και τέλος την απόσυρσή του.} οδήγησε στη δημιουργία \textbf{επίσημων μεθόδων} στα τέλη της δεκαετίας του 1970. Σκοπός τους ήταν η τυποποίηση του σχεδιασμού με στόχο τη βελτίωση της ποιότητάς του. Παράδειγμα αυτών των τεχνικών είναι η χρήση δομημένης ανάλυσης (structured analysis). Έτσι, οι μηχανικοί μπορούσαν να φτιάξουν διαγράμματα ροών δεδομένων (data flow diagrams)\footnote{Είναι μια οπτική αναπαράσταση της ροής των δεδομένων σε ένα σύστημα. Αναγράφονται οι διεργασίες (κύκλοι), οι είσοδοι και έξοδοι (τετράγωνα) και η αποθήκευση των δεδομένων (παράλληλες γραμμές).}, μοντέλα οντοτήτων-συσχετίσεων (ER -- entity-relationship models)\footnote{Περιγράφει ένα σύνολο αντικειμένων (οντότητες) και τις \textit{σχέσεις} μεταξύ αυτών των αντικειμένων.}, δημιουργώντας μια συστημική περιγραφή του λογικού και φυσικού μέρους του πληροφοριακού συστήματος που ανέπτυσσαν.

        Με την περαιτέρω ανάπτυξη των προσωπικών υπολογιστών στα μέσα της δεκαετίας του 1980, η χρήση επίσημων μεθόδων ήταν μονόδρομος, και μάλιστα οδήγησε στην άνθηση αυτόματων περιβαλλόντων και εργαλείων. Αυτά τα περιβάλλοντα, γνωστά ως CASE, επέτρεπαν τους μηχανικούς να καταγράφουν και να μοντελοποιούν με συστηματικό τρόπο ένα πληροφοριακό σύστημα από τις αρχικές περιγραφές του χρήστη ως τη σχεδίαση και την υλοποίηση και να εκτελούν δοκιμές για τη συνέπειά του. \cite{CASEChikofsky, Case1985}

        \subsection{Computer-Aided Software Enginnering (CASE)}
            Μέχρι πρότινος, τα εργαλεία που αφορούν την ανάπτυξη λογισμικού εστιάζουν κυρίως στην επεξεργασία πηγαίου κώδικα και την αποσφαλμάτωση του. Σε αντίθεση λοιπόν με τα υπάρχοντα εργαλεία, τα CASE περιβάλλοντα βοηθούν στη \textit{μεθοδολογία} της ανάπτυξης λογισμικού: στην ανάλυση απαιτήσεων, στο λογικό σχεδιασμό, στον έλεγχο εγκυρότητας, την επαναχρησιμοποίηση και εξάλειψη πλεονασμών.

            Είναι το δεξί χέρι ενός μηχανικού λογισμικού, βοηθώντας σε πολλές εργασίες που τον δυσκολεύουν, αυξάνοντας την παραγωγικότητα και την ποιότητα της δουλειάς του. Τα εργαλεία που περιλαμβάνουν ποικίλλουν: κάποια επιτρέπουν τη δημιουργία διαγραμμάτων ενώ άλλα μπορούν να αυτοματοποιούν όλα τα στάδια του κύκλου ζωής έκδοσης λογισμικού. Ένα ολοκληρωμένο CASE περιβάλλον περιλαμβάνει:
            \vspace{-0.5em}
            \begin{itemize}[label={\tiny \blacksquare}]
                \setlength\itemsep{-0.25em}
                \item ένα διαδραστικό, φιλικό για το χρήστη, οπτικό περιβάλλον διαχείρισης
                \item ένα σύνολο από εργαλεία ανάπτυξης (επεξεργαστές κειμένου, λεξικά, αναλυτές σχεδιασμού κ.α.)
                \item ένα σύνολο από εργαλεία για τον έλεγχο της διαδικασίας (για τον χρονοπρογραμματισμό, τη διασφάλιση ποιότητας κ.α.)
                \item ένα περιβάλλον βοήθειας με το documentation των εργαλείων
                \item ένα σύστημα διαχείρισης βάσεων δεδομένων
            \end{itemize}
            \vspace{-0.5em}

            Τα συστήματα που δημιουργούνται από το CASE είναι εφαρμογές της πειθαρχημένης εφαρμογής μεθόδων της μηχανικής λογισμικού.
            Τα CASE περιβάλλοντα έθεσαν τα θεμέλια για τη δημιουργία νέων προτύπων, όπως ο οπτικός προγραμματισμός (visual programming) και ο προγραμματισμός που βασίζεται σε μοντέλα. \cite{Case1985, CASEKuhn, AdoptionCASE}



        \subsection{Model-driven Architecture}
            Οι αρχιτεκτονικές που βασίζονται σε μοντέλα (model-driven architecture - MDA) διαχωρίζουν τη λειτουργικότητα μιας εφαρμογής από την υλοποίησή της σε μια συγκεκριμένη πλατφόρμα, προσφέροντας ένα υψηλότερο επίπεδο αφαίρεσης. Στόχος είναι οι προγραμματιστές να εστιάζουν περισσότερο στον σχεδιασμό και λιγότερο στο να λύνουν θέματα που αφορούν την πλατφόρμα υλοποίησης.

            Μέχρι πρότινος η ανάπτυξη λογισμικού αφορούσε καθαρά δομικό κώδικα. Η MDA άλλαξε τον τρόπο σκέψης των προγραμματιστών, καθώς πλέον επικεντρώνονταν παραπάνω στον σωστό διαχωρισμό των χαρακτηριστικών, στην αφαιρετικότητα και στην αυτοματοποίηση. \cite{Bucaioni2022, MDAFAQ}
